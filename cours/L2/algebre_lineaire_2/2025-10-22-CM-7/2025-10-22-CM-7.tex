\documentclass[a4paper, 12pt]{article}

\usepackage{utils}

\renewcommand*{\today}{22 October 2025}

\begin{document}

\hotbox{Algèbre linéaire 2}{CM 7}{\today}

\begin{proposition}{}{}
    Pour tout n-uplet $(x_1, \cdots, x_n)$ et tout $\varphi \in \wedge^{*n}(E)$, on a
    $$
    \forall u \in \mathcal{L}(E), \quad \varphi(u(x_1), \cdots, u(x_n)) = \det(u) \varphi(x_1, \cdots, x_n).
    $$
\end{proposition}

\begin{demonstration}
    On a nécessairement $\varphi = \lambda \Delta, k \in \K$ d'après le chapitre précédent.

    On a alors $\varphi(u(x_1), \cdots, u(x_n)) = \lambda \Delta(u(x_1), \cdots, u(x_n))$.

    Il suffit de démontrer la propriété pour $\varphi = \Delta$.

    Dans la base canonique, on a
    $$
    \forall j \in \llbracket 1, n \rrbracket, \quad x_j = \sum\limits_{i = 1}^n x_{i,j} e^i
    $$

    $X$ est la matrice des coordonnées de la famille $(x_1, \cdots, x_n)$ dans la base canonique.
    De même, $U = Mat_{Bcan}(u)$

    \begin{hotwarn}
        à finir
    \end{hotwarn}
\end{demonstration}

\begin{proposition}{}{}
    Le déterminant de l'application identité vaut 1.
\end{proposition}

\begin{hotwarn}
    à finir
\end{hotwarn}

\section{Réduction des endomorphismes}

\begin{definition}
    \begin{hotwarn}
        Def de valeur propre
    \end{hotwarn}
\end{definition}

\end{document}