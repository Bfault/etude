\documentclass[a4paper, 12pt]{article}

\usepackage{utils}

\renewcommand*{\today}{02 October 2025}

\begin{document}

\hotbox{Algèbre linéaire 2}{CM 4}{\today}

\section{Multi-linéarité}

\begin{hotwarn}
    À faire
\end{hotwarn}

\begin{demonstration}
    Soit $\varphi \in \mathcal{L}_p(E)$ et $\sigma \in \mathcal{S}_p, p \geq 1$

    $\forall \lambda \in \K, \forall (x, x') \in E^2$, on note $\sigma(i) = j, i \in \llbracket 1, n \rrbracket \text{ fixé}$

    $\sigma^\star (\varphi)(x_1, \ldots, x_{i-1}, \lambda x + x', x_{i+1}, \ldots, x_p) = \varphi(x_{\sigma(1)}, \ldots, x_{\sigma(i-1)}, \lambda x + x', x_{\sigma(i+1)}, \ldots, x_{\sigma(p)})$
    \begin{hotwarn}
        À faire
    \end{hotwarn}
\end{demonstration}

\begin{definition}
    Soit $E$ un EV et $\varphi \in \mathcal{L}_p(E, \K)$. On dit que $\varphi$ est
    \begin{itemize}
        \item \textbf{symétrique} si, pour tout $\sigma \in \mathcal{S}_p$ on a $\sigma^\star(\varphi) = \varphi$
        \item \textbf{alternée} si, $\exists i \neq j \text{ avec } x_i = x_j \implies \varphi(x_1, \ldots, x_p) = 0$
    \end{itemize}

    On note $\wedge^{\star p}(E)$ l'ensemble des formes alternées de $\mathcal{L}_p(E, \K)$
\end{definition}

\begin{remarque}
    Si $\varphi \wedge^{\star 2}(E)$ est une forme 
\end{remarque}

\begin{theoreme}{}{}
    Si $\varphi$ est une forme p-linéaire alternée, alors pour tout transposition $\tau \in \mathcal{S}_p$, on a
    $$
        \tau^\star(\varphi) = -\varphi
    $$
\end{theoreme}

\begin{demonstration}
    Soit $\varphi \in \wedge^{\star p}(E)$ une forme alternée et $\sigma = \tau_{i, j}$
    une transposition : tel que $\sigma(i) = j, \sigma(j) = i$ et $\forall k \neq i, j, \sigma(k) = k$
    \begin{hotwarn}
        À faire
    \end{hotwarn}
    Soit $(x_1, \ldots, x_p) \in E^p$,\n on a $\sigma^\star(\varphi)(x_1, \ldots, x_p) = \varphi(x_1, \ldots, x_{i-1}, x_j, x_{i+1}, \ldots, x_{j-1}, x_i, x_{j+1}, \ldots, x_p)$
    On ajoute $\varphi(x_1, \ldots, x_p)$ et $\sigma^\star(\varphi)(x_1, \ldots, x_p)$
    \begin{align*}
        & \varphi(x_1, \ldots, x_{i-1}, x_j, x_{i+1}, \ldots, x_{j-1}, x_i, x_{j+1}, \ldots, x_p) + \varphi(x_1, \ldots, x_p) \\
        = & \varphi(x_1, \ldots, x_{i-1}, x_j + x_i, x_{i+1}, \ldots, x_{j-1}, x_i, x_{j+1}, \ldots, x_p) \\
        = & 0 \text{ car } x_i \text{ et } x_j + x_i \text{ sont égaux}
    \end{align*}
    Donc $\sigma^\star(\varphi)(x_1, \ldots, x_p) = -\varphi(x_1, \ldots, x_p)$
\end{demonstration}

\begin{corollaire}{}{}
    Si $\varphi$ est une forme p-linéaire alternée de $E$, pour toute permutation $\sigma \in \mathcal{S}_p$, on a
    $$
        \sigma^\star(\varphi) = \eps(\sigma) \varphi
    $$

    En particulier
\end{corollaire}

\end{document}