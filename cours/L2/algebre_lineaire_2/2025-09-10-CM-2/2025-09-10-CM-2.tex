\documentclass[a4paper, 12pt]{article}

\usepackage{utils}

\renewcommand*{\today}{10 septembre 2025}

\begin{document}

\hotbox{Algèbre linéaire 2}{CM 2}{\today}

\section{Permutations}

\begin{definition}
    Soit $n \in \N^*$.
    On note $\S_n$ l'ensemble des bijections de $\llbracket 1, n \rrbracket$ dans lui-même.\n
    On appelle les éléments de $\S_n$ des \textbf{permutations}.    
\end{definition}

\begin{remarque}
    On retrouve aussi souvent $\mathfrak{S}_n$ (S gothique).
\end{remarque}

\noindent Il existe plusieurs façons de représenter une permutation.

$$
\sigma := 
\begin{pmatrix}
    1 & 2 & 3 & 4 & 5\\
    \sigma(1) & \sigma(2) & \sigma(3) & \sigma(4) & \sigma(5)
\end{pmatrix}
$$

\begin{exemple}
    Soit $n = 5$ et la permutation $\sigma \in \S_5$ définie par
    $$
    \sigma := 
    \begin{pmatrix}
        1 & 2 & 3 & 4 & 5\\
        3 & 5 & 4 & 2 & 1
    \end{pmatrix}
    $$
    C'est à dire que $\sigma(1) = 3, \sigma(2) = 5, \sigma(3) = 4, \sigma(4) = 2, \sigma(5) = 1$.
\end{exemple}

\begin{remarque}
    L'identité de $\S_n$ est notée $Id$ ou $e_n$

    $$
    e_n := 
    \begin{pmatrix}
        1 & 2 & \cdots & n\\
        1 & 2 & \cdots & n
    \end{pmatrix}
    $$
\end{remarque}

\noindent On peut aussi représenter une permutation par une matrice de permutation définie par
$$
    p_{ij} =
    \begin{cases}
        1 & \text{si } j = \sigma(i)\\
        0 & \text{sinon}
    \end{cases}
$$

\begin{exemple}
    $$
    M_\sigma = 
    \begin{pmatrix}
        0 & 0 & 1 & 0 & 0\\
        0 & 0 & 0 & 0 & 1\\
        0 & 0 & 0 & 1 & 0\\
        0 & 1 & 0 & 0 & 0\\
        1 & 0 & 0 & 0 & 0
    \end{pmatrix}
    $$
    c'est à dire que $\sigma(1) = 5, \sigma(2) = 4, \sigma(3) = 1, \sigma(4) = 3, \sigma(5) = 2$.
\end{exemple}

\begin{proposition}{}{}
    $\S_n$ est un ensemble fini de cardinal $n!$.
\end{proposition}

\begin{demonstration}
    Il y a $n$ choix pour $\sigma(1)$, $n-1$ choix pour $\sigma(2)$, ..., $1$ choix pour $\sigma(n)$.
    Donc il y a $n \times (n-1) \times \cdots \times 1 = n!$ permutations dans $\S_n$.
\end{demonstration}

\begin{proposition}{}{}
    $(\S_n, \circ)$ est un groupe.
\end{proposition}

\begin{proposition}{}{}
    Pour $n \geq 3$, le groupe $(\S_n, \circ)$ n'est pas commutatif.
\end{proposition}

\begin{demonstration}
    Pour $n = 3$, on considère les permutations
    $$
    \sigma_1 = 
    \begin{pmatrix}
        1 & 2 & 3\\
        2 & 3 & 1
    \end{pmatrix}
    \quad\text{et}\quad
    \sigma_2 = 
    \begin{pmatrix}
        1 & 2 & 3\\
        1 & 3 & 2
    \end{pmatrix}
    $$
    On a
    $$
    \sigma_1 \circ \sigma_2 =
    \begin{pmatrix}
        1 & 2 & 3\\
        2 & 1 & 3
    \end{pmatrix}
    \quad\text{et}\quad
    \sigma_2 \circ \sigma_1 =
    \begin{pmatrix}
        1 & 2 & 3\\
        3 & 2 & 1
    \end{pmatrix}
    $$
    Donc $\sigma_1 \circ \sigma_2 \neq \sigma_2 \circ \sigma_1$.

    \noindent Soient $n > 3$, les permutations
    $$
    \tau_1 = \left( \sigma_1 \quad\bigg| \begin{array}{ccc}
        4 & \cdots & n\\
        4 & \cdots & n
    \end{array} \right)
    \quad\text{et}\quad
    \tau_2 = \left( \sigma_2 \quad\bigg| \begin{array}{ccc}
        4 & \cdots & n\\
        4 & \cdots & n
    \end{array} \right)
    $$

    \begin{align*}
        &\forall i > 3,\quad \tau_1 \circ \tau_2 (i) = \tau_1(i) = i = \tau_2 \circ \tau_1 (i)\\
        &\forall i \leq 3,\quad \tau_1 \circ \tau_2 (i) = \sigma_1 \circ \sigma_2 (i) \quad\text{ et }\quad \tau_2 \circ \tau_1 (i) = \sigma_2 \circ \sigma_1 (i)
    \end{align*}

    Donc $\tau_1 \circ \tau_2 \neq \tau_2 \circ \tau_1$, et $(\S_n, \circ)$ n'est pas commutatif.
\end{demonstration}

\begin{definition}
    Soient $0 < i < j \leq n$. On appelle \textbf{transposition} la permutation $\tau_{ij} \in \S_n$ définie par
    $$
    \tau_{ij} :=
    \begin{pmatrix}
        1 & \cdots & i & \cdots & j & \cdots & n\\
        1 & \cdots & j & \cdots & i & \cdots & n
    \end{pmatrix}
    $$
\end{definition}

\begin{remarque}
    $e_n = \tau_{ij} \circ \tau_{ij}$ avec $1 \leq i \lt j \leq n$.
\end{remarque}

\begin{theoreme}{}{}
    Soit $n \geq 2$, toute permutation $\sigma \in S_n$ peut s'écrire comme composition
    finie de transpositions $\tau_1 \circ \cdots \circ \tau_p$.

    On dit que $\S_n$ est engendré par les transpositions $\langle \tau_1, \cdots, \tau_p \rangle$.
\end{theoreme}

\begin{demonstration}
    Par récurrence
\end{demonstration}

\begin{remarque}
    La décomposition en transpositions n'est pas unique.
\end{remarque}

\begin{remarque}
    soit $\sigma \in S_n$, décomposer en $\sigma = \tau_1 \circ \cdots \circ \tau_p$ où les $\tau_i$ sont des transpositions.
    Comme $(\tau_k)^{-1} = \tau_k$, pour tout $k$, on a $\sigma^{-1} = \tau_p \circ \cdots \circ \tau_1$.
\end{remarque}

\end{document}
