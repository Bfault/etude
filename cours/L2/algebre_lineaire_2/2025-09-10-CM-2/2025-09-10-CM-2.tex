\documentclass[a4paper, 12pt]{article}

\usepackage{utils}

\renewcommand*{\today}{10 septembre 2025}

\begin{document}

\hotbox{Algèbre linéaire 2}{CM 2}{\today}

\section{Permutations}

\begin{definition}
    Soit $n \in \N^*$, on note $S_n$ l'ensemble des bijections de $\llbracket 1, n \rrbracket$ dans lui-même.
    On appelle les éléments de $S_n$ des \textbf{permutations}.    
\end{definition}

\begin{theoreme}{}{}
    Pour $n \geq 3$, le groupe $(S_n, \circ)$ n'est pas commutatif.
\end{theoreme}

\begin{demonstration}
    Soit $n \gt 3$, soit les permutations $\tau_1 = \left( \sigma_1 | ... \right)$

    $\forall i \gt 3, \tau_1 \circ \tau_2 (i) = \tau_1(i) = i = \tau_2 \circ \tau_1 (i)$

    $\forall i \geq 3, \tau_1 \circ \tau_2 (i), \tau_2 \circ \tau_1 = $
\end{demonstration}

\begin{theoreme}{}{}
    Soit $n \geq 2$, toute permutation $\sigma \in S_n$ peuut s'écrire comme composition
    finir de transpositions $\tau_1 \circ \cdots \circ \tau_p$.

    On dit que $S_n$ est engendré par les transpositions $\langle \tau_1, \cdots, \tau_p \rangle$.
\end{theoreme}

\begin{demonstration}
    Par récurrence
\end{demonstration}

\begin{remarque}
    soit $\sigma \in S_n$, décomposer en $\sigma = \tau_1 \circ \cdots \circ \tau_p$ où les $\tau_i$ sont des transpositions.
    Comme $(\tau_k)^{-1} = \tau_k$, pour tout $k$, on a $\sigma^{-1} = \tau_p \circ \cdots \circ \tau_1$.
\end{remarque}

\begin{remarque}
    $e_n$ l'identité de $S_n$ s'écrit comme produit de n'importe quel transposition avec elle-même.
    $e_n = \tau_{ij} \circ \tau_{ij}$ avec $1 \leq i \lt j \leq n$.
\end{remarque}

\end{document}
