\documentclass[a4paper, 12pt]{article}

\usepackage{utils}

\renewcommand*{\today}{11 septembre 2025}

\begin{document}

\hotbox{Algèbre linéaire 2}{CM 3}{\today}

\begin{definition}
    Soit $n \geq 2$. On appelle \textbf{orbite} d'un élément $a \in \llbracket 1, n \rrbracket$ par $\sigma \in S_n$ l'ensemble
    des images itérées de $a$ par $\sigma$, c'est-à-dire
    $$
    \mathcal{O}_\sigma(a) = \{a, \sigma(a), \sigma^2(a), \ldots\}
    $$
    Lorsque $\mathcal{O}_\sigma(a) = \{a\}$ on dit que $a$ est un point fixe de $\sigma$.
\end{definition}

\begin{definition}
    On appelle \textbf{cycle} un élément $\sigma \in S_n$ de la forme
    $$
    \begin{pmatrix}
        a_1 & a_2 & \cdots & a_k \\
        a_2 & a_3 & \cdots & a_1
    \end{pmatrix}
    $$
    où $k \geq 2$

    On note ce cycle $(a_1\;\; a_2\;\; \cdots\;\; a_k)$.
\end{definition}

\begin{definition}
    On appelle \textbf{support} du cycle $\sigma = (a_1\; a_2\; \cdots\; a_k)$ l'ensemble
    $\{a_1, a_2, \ldots, a_k\}$.
\end{definition}

\begin{definition}
    Deux cycles sont dits \textbf{disjoints} si leurs supports sont disjoints.
\end{definition}

\begin{definition}
    On appelle \textbf{longueur} d'un cycle le nombre d'éléments de son support.
\end{definition}

\begin{definition}
    Soit $n \geq 2$ et $\sigma \in \S_n$. On appelle \textbf{support} de $\sigma$ l'ensemble de
    $\llbracket 1, n \rrbracket$ qui n'est pas invariant par $\sigma$, on le note $supp(\sigma)$.
\end{definition}

\begin{remarque}
    C'est une généralisation de la notion de support d'un cycle.
\end{remarque}

\begin{lemme}{}{orbite-finite}
    Soit $\sigma \in \S_n$ pour $n \geq 2$ une permutation non triviale et $a \in supp(\sigma)$.
    Alors il existe $p \in \N^*$ tel que
    $$
    \mathcal{O}_\sigma(a) = \{a, \sigma(a), \ldots, \sigma^{p-1}(a)\},\quad \sigma^p(a) = a
    $$

    C'est-à-dire que l'orbite de $a$ est finie.
\end{lemme}

\begin{demonstration}
    Soit $\sigma \in \S_n$ pour $n \geq 2$ une permutation non triviale et $a \in supp(\sigma)$.

    On sait que $\mathcal{O}_\sigma(a) \subset \llbracket 1, n \rrbracket$ est un ensemble fini.
    Donc il existe $k < l$ tels que
    \begin{align*}
        \sigma^k(a) = \sigma^l(a) &\implies \sigma^{k} \circ \sigma^{-k}(a) = \sigma^{l} \circ \sigma^{-k}(a) \\
        &\implies Id(a) = \sigma^{l-k}(a) \\
        &\implies a = \sigma^{l-k}(a)
    \end{align*}

    D'où l'ensemble $\{r \in \N^* \mid \sigma^r(a) = a\}$ est non vide, il admet donc un plus petit élément $p \in \N^*$
    (et même $p \geq 2$ car $\sigma(a) \neq a$).

    Donc $\mathcal{O}_\sigma(a) = \{a, \sigma(a), \ldots, \sigma^{p-1}(a)\}$ car $\sigma^p(a) = a$.
\end{demonstration}

\begin{lemme}{}{}
    Soient $\sigma \in S_n$ pour $n \geq 2$ une permutation non triviale et $a, b \in supp(\sigma)$ tels que $a \neq b$.
    Alors on se retrouve dans l'un des deux cas suivants :
    \begin{itemize}
        \item $\mathcal{O}_\sigma(a) = \mathcal{O}_\sigma(b)$
        \item $\mathcal{O}_\sigma(a) \cap \mathcal{O}_\sigma(b) = \emptyset$
    \end{itemize}

    Autrement dit, $\mathcal{O}_\sigma(a) \cap \mathcal{O}_\sigma(b) \neq \emptyset \iff \mathcal{O}_\sigma(a) = \mathcal{O}_\sigma(b)$.
\end{lemme}

\begin{demonstration}
    Sens $\impliedby$
    \vspace{4mm}

    C'est évident.

    \vspace{4mm}
    Sens $\implies$
    \vspace{4mm}
    Soit $c \in \mathcal{O}_\sigma(a) \cap \mathcal{O}_\sigma(b)$. Alors il existe $k, l \in \N$ tels que
    $$
    c = \sigma^k(a) = \sigma^l(b)
    $$
    Il existe $p \geq l$ tel que $\sigma^p(b) = b$ (d'après le lemme~\ref{lemme:orbite-finite}).
    Donc
    \begin{align*}
        &\sigma^{p-l}(c) = \sigma^{p - l} \circ \sigma^l(b) = \sigma^{p - k} \circ \sigma^k(a) \\
        \implies &\sigma^p(b) = \sigma^{p - k + l}(a) \\
        \implies &b = \sigma^{p - k + l}(a) \\
        \implies &b \in \mathcal{O}_\sigma(a)
    \end{align*}

    On fait de même pour $a$ et on conclut que $\mathcal{O}_\sigma(a) = \mathcal{O}_\sigma(b)$.
\end{demonstration}

\begin{theoreme}{}{}
    Soit $n \geq 2$ et $\sigma \in S_n$, avec $\sigma \neq Id$. Il existe une
    famille finie unique (aux commutations près) de cycles $(c_i)_{1 \leq i \leq p}$ à supports disjoints tels que
    $$
    \sigma = \prod\limits_{i=1}^p c_i = c_1 \circ \cdots \circ c_p
    $$
\end{theoreme}

\begin{lemme}{}{}
    On a $(a_1\; \cdots\; a_p) = \tau_1 \circ \cdots \circ \tau_{p-1}$ où $\tau_i = (a_i\; a_{i+1})$.
\end{lemme}

\begin{definition}
    Pour tout $\sigma \in \S_n$ on appelle signature de $\sigma$, notée $\varepsilon(\sigma)$ le nombre
    $$
    \prod_{1 \leq i \lt j \leq n} \text{sgn}(\sigma(j) - \sigma(i))
    $$

    $\sigma$ est dite paire si $\varepsilon(\sigma) = 1$ et impaire si $\varepsilon(\sigma) = -1$.
\end{definition}

\begin{proposition}{}{}
    Toute transposition est impaire.
\end{proposition}

\begin{theoreme}{}{}
    Si $\sigma_1, \sigma_2 \in \S_n$ alors $\varepsilon(\sigma_1 \circ \sigma_2) = \varepsilon(\sigma_1) \varepsilon(\sigma_2)$.
\end{theoreme}

\begin{theoreme}{}{}
    Si $\sigma = \tau_1 \circ \cdots \circ \tau_p$ où chaque $\tau_i$ est une transposition, alors
    $\varepsilon(\sigma) = (-1)^p$.
\end{theoreme}

On note $A_n = \{\sigma \in \S_n \mid \varepsilon(\sigma) = 1\}$.
C'est-à-dire l'ensemble des permutations paires.

\begin{proposition}{}{}
    Si $\rho \in \S_n \setminus A_n$ alors $\rho$ induit une bijection $\funcdef{}{\S_n \setminus A_n}{A_n}{\sigma}{\sigma \circ \rho}$
\end{proposition}

\begin{remarque}
    $A_n$ est un groupe, qu'on appelle le groupe alterné sur $n$ éléments.
\end{remarque}

\end{document}
