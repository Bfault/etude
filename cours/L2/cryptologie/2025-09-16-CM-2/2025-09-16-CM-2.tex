\documentclass[a4paper, 12pt]{article}

\usepackage{utils}

\renewcommand*{\today}{16 septembre 2025}

\begin{document}

\hotbox{Cryptologie}{CM 2}{\today}

\begin{definition}[(Principe de Kerckhoffs)]
  Un système de chiffrement doit être sécurisé même si tout est connu, sauf la clé.
\end{definition}

\begin{remarque}
    Cela signifie que la sécurité ne doit pas reposer sur le secret de l'algorithme, mais uniquement sur la clé.
\end{remarque}

L'espace des clés doit être suffisamment grand pour résister aux attaques par force brute.

On dit qu'un cryptosystème est \textbf{cassé} si un attaquant peut retrouver la clé (ou le texte clair) en un temps raisonnable.

\subsection{Différents cadres d'attaques à considérer}

\begin{itemize}
    \item Un texte crypté connu
    \item Un texte clair connu
    \item Un texte clair choisi (accés à la machine de chiffrement)
    \item Un texte crypté choisi (accés à la machine de déchiffrement)
\end{itemize}

\end{document}
