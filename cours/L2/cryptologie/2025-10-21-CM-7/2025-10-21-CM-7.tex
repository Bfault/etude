\documentclass[a4paper, 12pt]{article}

\usepackage{utils}

\renewcommand*{\today}{21 October 2025}

\begin{document}

\hotbox{Cryptologie}{CM 7}{\today}

\begin{theoreme}{Théorème d'Euler}{}
    Soit $a \in (\Z /n\Z)^\times$ alors $a^{\phi(n)} = 1_{\Z/n\Z}$
\end{theoreme}

\begin{demonstration}
    $((\Z/n\Z)^\times, \times)$ est un groupe de cardinal $\phi(n)$
\end{demonstration}

\begin{theoreme}{Petit théorème de Fermat}{}
    Soit $p$ un nombre premier et $a$ non-multiple de $p$ alors $a^{p-1} \equiv 1 [p]$
\end{theoreme}

\begin{demonstration}
    On applique le théorème d'Euler avec $n = p$ et $\overline{a}^{p-1} = \overline{1}$ dans $\Z/p\Z$
    c'est-à-dire $a^{p-1} \equiv 1 [p]$
\end{demonstration}

\begin{theoreme}{}{}
    Si $r, s \in \N^*$ vérifient $r \equiv s [p-1]$ alors pour tout $a \in \Z$ on a $a^r \equiv a^s [p]$
\end{theoreme}

\begin{demonstration}
    Supposons $r \geq s$ c'est-à-dire $r = s + k(p - 1)$ avec $k \in \N$
    
    Si $a$ est non multiple de $p$ alors
    \begin{align*}
        a^r &= a^{s + k(p-1)} \\
        &= (a^{p-1})^k \times a^s \\
        &\equiv 1^k \times a^s [p] \\
        &\equiv a^s [p]
    \end{align*}

    Si $a$ est multiple de $p$ alors $a^r$ et $a^s$ sont multiples de $p$ donc $a^r \equiv a^s [p]$
\end{demonstration}

\subsection{Le système RSA}

\begin{definition}[Primitive RSA]
    Soient $p, q$ deux nombres premiers distincts. On pose $n = p \times q$ et on a $\phi = \phi(n) = (p-1)(q-1)$.

    On prend $e \in \llbracket 1, \phi - 1 \rrbracket$ tel que $e \wedge \phi = 1$.

    On définit alors
    $$
    \funcdef{RSA_{(n, e)}}{\Z/n\Z}{\Z/n\Z}{m}{m^e}
    $$
\end{definition}

\begin{proposition}{}{}
    Soient $p, q, n, \phi, e$ comme ci-dessus.

    Soit $d \in \llbracket 1, \phi - 1 \rrbracket$ tel que $ed \equiv 1 [\phi]$
    
    Alors $RSA_{n,d} \circ RSA_{n, e} = Id_{\Z/n\Z}$ et $RSA_{n, e} \circ RSA_{n, d} = Id_{\Z/n\Z}$
\end{proposition}

\begin{demonstration}
    Par le théorème précédent, pour $x \in \Z$

    On a $x^{ed} \equiv x [p]$ et de même $x^{ed} \equiv x [q]$

    Or $p, q$ sont premiers entre eux donc $x^{ed} \equiv x [pq]$

    Or $RSA_{n, d} \circ RSA_{n, e} (\overline{x}) = (\overline{x}^e)^d = \overline{x}^{ed} = \overline{x}$
    De même pour $RSA_{n,e} \circ RSA_{n, d}$
\end{demonstration}

\begin{definition}[Cryptosystème RSA]
    $\mathcal{M} = \mathcal{E} = \mathcal{L}(\mathcal{A})$ où $\mathcal{K} = \{((n, e), (n, d))\}$ avec $n = p \times q$ avec $p, q$ deux nombres premiers distincts,
    $e \in \llbracket 1, \phi(n) - 1 \rrbracket$ tel que $e \wedge \phi(n) = 1$ et $d \in \llbracket 1, \phi(n) - 1 \rrbracket$ tel que $ed \equiv 1 [\phi(n)]$.

    On définit des fonctions de conversions
    $$
    \mathcal{L}(\mathcal{A}) \underset{\psi_2}{\overset{\psi_1}{\rightleftarrows}} \mathcal{L}(\mathcal{B})
    $$

    On pose alors $E_{(n, e)} = \psi_2 \circ RSA_{(n, e)} \circ \psi_1$ et avec d'autres fonctions de conversions

    $$
    \mathcal{L}(\mathcal{A}) \underset{\psi_2'}{\overset{\psi_1'}{\rightleftarrows}} \mathcal{L}(\mathcal{B})
    $$

    tels que $\psi_1' \circ \psi_1 = Id_{\mathcal{L}(\mathcal{A})}$ et $\psi_2' \circ \psi_2 = Id_{\mathcal{L}(\mathcal{B})}$ on pose $D_{(n, d)} = \psi_1' \circ RSA_{(n, d)} \circ \psi_2'$
\end{definition}

\end{document}