\documentclass[a4paper, 12pt]{article}

\usepackage{utils}

\renewcommand*{\today}{04 November 2025}

\begin{document}

\hotbox{Cryptologie}{CM 8}{\today}

\begin{proposition}
    La connaissance de $\phi$ et $n$ permet de retrouver $p$ et $q$.
\end{proposition}

\begin{demonstration}
    \begin{align*}
        n - \phi + 1 &= pq - (p-1)(q-1) + 1 \\
        &= pq - (pq - p - q + 1) + 1 \\
        &= p + q
    \end{align*}

    On a aussi $n = pq$.

    On peut donc retrouver $p$ et $q$ comme racines de l'équation
    $$
    (X - p)(X - q) = X^2 - (p + q)X + pq
    $$
    
\end{demonstration}

\end{document}