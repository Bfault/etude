\documentclass[a4paper, 12pt]{article}

\usepackage{utils}

\renewcommand*{\today}{09 septembre 2025}

\begin{document}

\hotbox{Cryptologie}{CM: 1}{\today}

\subsection{Références}

\begin{itemize}
    \item Wikipédia
    \item Le livre "Handbook of Applied Cryptography" de Menezes
\end{itemize}

\section{Histoire et contexte}

\subsection{Schéma d'une situation cryptographie}

Alice veut envoyer un message \textit{m} en clair à Bob sur un canal \textbf{non sécurisé}.
\vspace{2mm}

Elle utilise un algorithme de chiffrement E pour chiffrer son message $m' = E(m)$.
Elle envoie \textit{m'} à Bob qui utilise un algorithme de déchiffrement D pour retrouver le message initial $m = D(m')$.
\vspace{2mm}

Un attaquant, Eve, peut intercepter \textit{m'} voire le modifier et peut aussi essayer de retrouver \textit{m} à partir de \textit{m'}.

\vspace{3mm}

\begin{definition}
    La \textbf{cryptographie} est l'étude des techniques mathématiques liée à la sécurité de l'information.
    Les buts sont
    \begin{itemize}
        \item Inaccessibilité de l'information à un tiers.
        \item Authentification de l'origine du message.
        \item Inaltération du message.
    \end{itemize}
\end{definition}

\begin{definition}
    La \textbf{cryptanalyse} est l'étude des techniques mathématiques mettant en défaut les techniques de cryptographie.
\end{definition}

\begin{remarque}
    L'ensemble des techniques de cryptographie et de cryptanalyse est appelé la \textbf{cryptologie}.
\end{remarque}

\subsection{Techniques cryptographie (avec date)}

\begin{itemize}
    \item (-200 avant JC) Substitution monoalphabétique (ex: César)
    \item (1585) Vigenère (substitution polyalphabétique)
    \item (1919) Enigma (utilisé pour la seconde guerre mondiale par l'armée allemande)
\end{itemize}

\section{Formalisation de la cryptographie}

\begin{definition}
    Un \textbf{alphabet} est un ensemble fini de symboles.
\end{definition}

\begin{exemple}
    $\mathcal{A} = \{A, B, C, \ldots, Z\}$ est un alphabet de 26 symboles.
\end{exemple}

\begin{exemple}
    $\mathcal{A} = \{\heartsuit, \spadesuit, \diamondsuit, \clubsuit\}$ est un alphabet de 4 symboles.
\end{exemple}

\begin{definition}
    Un \textbf{message} dans un alphabet $\mathcal{A}$ est une suite finie à valeurs dans $\mathcal{A}$.\n
    noté $m = m_1 m_2 \ldots m_n$ où $m_i \in \mathcal{A}$ et l'ensemble des messages est noté
    $$
    \mathcal{L}(\mathcal{A}) = \bigcup\limits_{n \in \N} \mathcal{A}^n
    $$
\end{definition}

\begin{definition}
    Soient deux messages $m = m_1 \ldots m_n, m' = m'_1 \ldots m'_p \in \mathcal{L}(\mathcal{A})$\n
    La concaténation de $m$ et $m'$ est définie par
    $$
    m'' = m \| m' = m_1'' \ldots m_{n+p}'' \text{ où } \begin{cases}
        m_i'' = m_i & \text{si } i \leq n\\
        m_i'' = m'_{i-n} & \text{si } i > n
    \end{cases}
    $$
\end{definition}

\begin{definition}
    Un chiffrement est une fonction $E: \mathcal{M} \rightarrow \mathcal{C}$
    où $\mathcal{M}, \mathcal{C} \subset \mathcal{L}(\mathcal{A})$
    \begin{itemize}
        \item $\mathcal{M}$ est l'ensemble des messages admissibles (pouvant être cryptés)
        \item $\mathcal{C}$ est l'ensemble des messages cryptés
    \end{itemize}
\end{definition}

\begin{definition}
    Un déchiffrement pour $E$ est une fonction $D: \mathcal{C} \rightarrow \mathcal{M}$
    tel que $D \circ E = Id_\mathcal{M}$
\end{definition}

\begin{remarque}
    Dans la plupart des cas, on a $\mathcal{M} = \mathcal{C} = \mathcal{L}(\mathcal{A})$
\end{remarque}

\begin{proposition}{}{}
    $E$ est injective
\end{proposition}

\begin{demonstration}
    Soient $m, m' \in \mathcal{M}$\n
    $E(m) = E(m') \implies D(E(m)) = D(E(m')) \implies m = m'$
\end{demonstration}

\begin{proposition}{}{}
    $D$ est surjective
\end{proposition}

\begin{demonstration}
    Soit $m \in \mathcal{M}$\n
    $E(m) \in \mathcal{C}$ et $D(E(m)) = m$
\end{demonstration}

\begin{definition}
    Un \textbf{cryptosystème} est un quadruplet $(\mathcal{M}, \mathcal{C}, \mathcal{K}, (E_e, D_d)_{(e, d) \in \mathcal{K}})$
    où 
    \begin{itemize}
        \item $\mathcal{M}, \mathcal{C} \subset \mathcal{L}(A)$
        \item $\mathcal{K}$ est un ensemble appelé l'espace des clefs. Il est constitué de paires de clefs $(e, d)$
        où $e$ est la clef de cryptage et $d$ la clef de décryptage.
        \item Pour $(e, d) \in \mathcal{K}$, $E_e: \mathcal{M} \rightarrow \mathcal{C}$
        est une fonction de cryptage et $D_d: \mathcal{C} \rightarrow \mathcal{M}$ pour fonction
        de décryptage de $E_e$.
    \end{itemize}
\end{definition}

\noindent Soit $A = \{A, \ldots, 0\} \simeq \llbracket 0, 25 \rrbracket$

\begin{exemple}[César]
    \begin{align*}
        \mathcal{M} &= \mathcal{C} = \mathcal{L}(A) \\
        \mathcal{K} &=\{(e, d) \mid e \in \llbracket 0, 25 \rrbracket \text{ et } d = -e\} \\
        &=\{(e, -e) \mid e \in \llbracket 0, 25 \rrbracket\} \\
        E_\alpha &= \begin{cases}
            E_\alpha(m_i) = m_i + \alpha \mod 26\\
            E_\alpha(m_1 \ldots m_n) = E_\alpha(m_1) \ldots E_\alpha(m_n)
        \end{cases} \\
        D_\alpha &= E_\alpha \\
    \end{align*}
\end{exemple}

\begin{exemple}[Par permutation]\n
    Soit $\ell$: longueur de la permutation.
    \begin{align*}
        \mathcal{M} &= \mathcal{C} = \bigcup\limits_{n \in \N} A^{n\ell} \\
        \mathcal{K} &= \{(\sigma, \sigma^{-1}) | \sigma \in \mathfrak{S}_\ell\} \\
        E_\sigma &= \begin{cases}
            E_\sigma(m_1 \ldots m_\ell) = m_{\sigma(1)} \ldots m_{\sigma(\ell)} \\
            E_\sigma(M_1 || \ldots || M_k) = E_\sigma(M_1) || \ldots || E_\sigma(M_k) \text{ où } M_i \in A^\ell \\
        \end{cases} \\
        D_\tau &= E_\tau \\
    \end{align*}
\end{exemple}

\begin{exemple}[\textbf{monoalphabétique}]
    \begin{align*}
        \mathcal{M} &= \mathcal{C} = \mathcal{L}(A) \\
        \mathcal{K} &= \{(\sigma, \sigma^{-1}) | \sigma \in \mathfrak{S}(\mathcal{A}) \cong \mathfrak{S}_{26}\} \\
        E_\sigma &= \begin{cases}
            E_\sigma(m_i) = \sigma(m_i) \\
            E_\sigma(m_1 \ldots m_n) = E_\sigma(m_1) \ldots E_\sigma(m_n)
        \end{cases} \\
        D_\sigma &= E_{\sigma^{-1}} \\
    \end{align*}
\end{exemple}

\begin{remarque}
    César et la permutation sont des cas particuliers de la monoalphabétique.
\end{remarque}

\begin{exemple}[Vigenère]\n
    Soit $\ell$: longueur de la clef.
    \begin{align*}
        \mathcal{M} &= \mathcal{C} = \mathcal{L}(A) \\
        \mathcal{K} &= \{(k = k_0 \ldots k_{\ell-1}, k) | k_i \in A\} \\
        E_k &= \begin{cases}
            E_k(m_i) = m_i + k_{i \mod \ell} \mod 26 \\
            E_k(m_0 \ldots m_{n-1}) = E_k(m_0) \ldots E_k(m_{n-1})
        \end{cases} \\
        D_k &= \begin{cases}
            D_k(c_i) = c_i - k_{i \mod \ell} \mod 26 \\
            D_k(c_0 \ldots c_{n-1}) = D_k(c_0) \ldots D_k(c_{n-1})
        \end{cases} \\
    \end{align*}
\end{exemple}

\begin{exemple}[\textbf{polyalphabétique}]\n
    Soit $\ell$: longueur de la clef.
    \begin{align*}
        \mathcal{M} &= \mathcal{C} = \mathcal{L}(A) \\
        \mathcal{K} &= \{(f: \llbracket 0, \ell-1 \rrbracket \rightarrow \mathfrak{S}(A), f^{-1})\} \\
        E_f &= \begin{cases}
            E_f(m_i) = f(i \mod \ell)(m_i) \\
            E_f(m_0 \ldots m_{n-1}) = E_f(m_0) \ldots E_f(m_{n-1})
        \end{cases} \\
        D_f &= \begin{cases}
            D_f(c_i) = f^{-1}(i \mod \ell)(c_i) \\
            D_f(c_0 \ldots c_{n-1}) = D_f(c_0) \ldots D_f(c_{n-1})
        \end{cases} \\
    \end{align*}
\end{exemple}

\begin{remarque}
    Vigenère est un cas particulier de la polyalphabétique.
\end{remarque}

\end{document}
