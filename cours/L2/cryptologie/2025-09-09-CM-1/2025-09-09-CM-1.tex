\documentclass[a4paper, 12pt]{article}

\usepackage{utils}

\renewcommand*{\today}{09 septembre 2025}

\begin{document}

\hotbox{Cryptologie}{CM: 1}{\today}

\subsection{Références}

\begin{itemize}
    \item Wikipédia
    \item Le livre "Handbook of Applied Cryptography" de Menezes
\end{itemize}

\section{Histoire et contexte}

\subsection{Schéma d'une situation cryptographie}

Alice veut envoyer un message m "en clair" à Bob.
Elle doit utiliser un canal non sécurisé.
Elle utilise un algorithme de chiffrement E pour chiffrer son message m en m'.
Elle envoie m' à Bob.
Souvent un clef k peut être utilisée.
Sur le canal non sécurisé, Eve peut intercepter m' voire le modifier.
Eve peut aussi essayer de retrouver m à partir de m'.

\begin{definition}
    La \textbf{cryptographie} est l'étude des techniques mathématiques liée à la sécurité de l'information.
    Les buts sont:
    \begin{enumerate}
        \item Inaccessibilité de l'information à un tiers.
        \item Authentification de l'origine du message.
        \item Inaltération du message.
    \end{enumerate}
\end{definition}

\begin{definition}
    La \textbf{cryptanalyse} est l'étude des techniques mathématiques mettant en défaut les techniques de cryptographie.
\end{definition}

\begin{remarque}
    L'ensemble des techniques de cryptographie et de cryptanalyse est appelé la \textbf{cryptologie}.
\end{remarque}

\begin{exemple}
    \begin{enumerate}
        \item (X - VII siècle avant JC) Scytale (bâton roulé)
        \item (V siècle avant JC) Passage de la bible en hébreu
        \item (-50 avant JC) César (décalage de 3 dans l'alphabet)
    \end{enumerate}
\end{exemple}

Contextes d'utilisation historique pré-informatique:
\begin{itemize}
    \item Guerres (communication entre les troupes)
    \item Association de personnes suceptible d'être menacée
    \item Diplomatie (négociation)
    \item Commerce (négociation)
    \item Infidélité (lettres)
\end{itemize}

Les cannaux de communication:
\begin{itemize}
    \item Messager
    \item Pigeon voyageur
    \item Journeaux
    \item Internet
    \item Télégraphe
    \item Radio
\end{itemize}

\subsection{Techniques cryptographie (avec date)}

\begin{itemize}
    \item (-200 avant JC) Substitution monoalphabétique (ex: César)\n
    cassé en 800
    \item (1585) Vigenère (substitution polyalphabétique)\n
    cassé en 1863 par Kasiski
    \item (1919) Enigma (utilisé pour la seconde guerre mondiale par l'armée allemande)\n
    cassé en 1941 par Alan Turing
\end{itemize}

\subsection{Rupture du numérique}

\begin{itemize}
    \item Explosion de la puissance de calcul
    \item Automatisation du cryptage et du décryptage
    \item Nécessité d'échanger des clefs à distance (sur le canal non sécurisé)
    \item Systèmes plus global (beaucoup de Eve)
    \item Facilité de dupliquer l'information
\end{itemize}

utilisations modernes:

\begin{itemize}
    \item Messages privés (sms, mail, whatsapp, telegram, signal)
    \item Authentification (mdp, carte bancaire, biométrie)
    \item Signature électronique (contrat, logiciel)
\end{itemize}

Futur:
\begin{itemize}
    \item Développement de l'informatique quantique
    il y a donc un besoin de nouveaux systèmes cryptographiques "post-quantique"
\end{itemize}

\section{Formalisation de la cryptographie}

\begin{definition}
    Un \textbf{alphabet} est un ensemble fini de symboles.
\end{definition}

\begin{definition}
    Un \textbf{message} dans un alphabet A est une suite finie à valeurs dans A.\n
    noté $m = m_1 m_2 \cdots m_n$ où $m_i \in A$ et l'ensemble des messages est noté $\mathscr{L}(A) = \bigcup\limits_{n \in \N} A^n$.
\end{definition}

\begin{definition}
    Soient deux messages $m = m_1 \cdots m_n, m' = m'_1 \cdots m'_p \in \mathscr{L}(a)$\n
    La concaténation de $m$ et $m'$ est définie par $m'' = m \| m' = m_1 \cdots m_n m'_1 \cdots m'_p$
\end{definition}

\begin{definition}
    Une fonction de chiffrement est une fonction $E: \mathscr{M} \rightarrow \mathscr{C}$
    où $\mathscr{M}, \mathscr{C} \subset \mathscr{L}(A)$
    \begin{itemize}
        \item $\mathscr{M}$ est l'ensemble des messages pouvant être crypté
        \item $\mathscr{C}$ est l'ensemble des messages cryptés
    \end{itemize}
\end{definition}

\begin{definition}
    Une fonction de déchiffrement pour $E$ est une fonction $D: \mathscr{C} \rightarrow \mathscr{M}$
    tel que $\forall m \in \mathscr{M}, D(E(m)) = m$ i.e. $D \circ E = Id$
\end{definition}

\begin{proposition}{}{}
    $E$ est injective
\end{proposition}

\begin{demonstration}
    Soient $m, m' \in \mathscr{M}$\n
    $E(m) = E(m') \implies D(E(m)) = D(E(m')) \implies m = m'$
\end{demonstration}

\begin{proposition}{}{}
    $D$ est surjective
\end{proposition}

\begin{demonstration}
    \begin{hotwarn}
        à faire
    \end{hotwarn}
\end{demonstration}

\begin{definition}
    Un \textbf{cryptosystème} est un quadruplet $(\mathscr{M}, \mathscr{C}, \mathscr{K}, (E_e, D_d)_{(e, d) \in \mathscr{K}}$
    où $\mathscr{M}, \mathscr{C} \in \mathscr{L}(A)$,
    $\mathscr{K}$ est un ensemble de paires de ("clef de cryptage", "clef de décriptage").\n
    Pour chaque $(e, d) \in \mathscr{K}$ on a $E_e: \mathscr{M} \rightarrow \mathscr{C}$
    est une fonction de cryptage ayant $D_d: \mathscr{C} \rightarrow \mathscr{M}$ pour une fonction
    de décryptage.
\end{definition}

Soit $A = \{A, \cdots, 0\} \simeq \llbracket 0, 25 \rrbracket$

\begin{exemple}[César]
    $\mathscr{M} = \mathscr{C} = \mathscr{L}(A)$\n
    \begin{align*}
        \mathscr{K}&=\{(e, d) | e \in \llbracket 0, 25 \rrbracket \text{ et } d = -e\}
        &=\{(e, -e) | e \in \llbracket 0, 25 \rrbracket\}
    \end{align*}

    $D_\alpha = E_\alpha$\n
    $\forall m_i \in A, D_\alpha(m_i) = m_i + \alpha \mod 26$
\end{exemple}

\begin{exemple}[Par permutation]
    Soit $l$: longueur des permutations considérées.

    $\mathscr{M} = \mathscr{C} = \bigcup\limits_{n \in \N}A^{nl}$

    $\mathscr{K} = \{(\sigma, \sigma^{-1}) | \sigma \in \mathfrak{S}_l\}$

    Si $m = m_1 \cdots m_l$ est de longueur $l$\n
    $E_\sigma(m) = m_{\sigma(1) \cdots m_{\sigma(l)}}$
    et $D_\tau = E_\tau$
\end{exemple}

\begin{exemple}[Monoalphabétique]
    $\mathscr{M} = \mathscr{C} = \mathscr{L}(A)$

    $\mathscr{K}$ indexé par $\big\mathfrak{S}(A)$
\end{exemple}

\end{document}
