\documentclass[a4paper, 12pt]{article}

\usepackage{utils}

\renewcommand*{\today}{30 September 2025}

\begin{document}

\hotbox{Analyse 3}{CM 8}{\today}

\subsection{Critère d'intégrabilité}

\begin{proposition}{}{}
    Soient $a, b \in \R, a < b$ et $f: [a, b] \to \R$ telle que $f \in \mathcal{R}(a, b)$

    $$
    \forall \eps > 0, \exists \sigma(\eps) \in \sum(a, b), 0 \leq S(f, \sigma) - s(f, \sigma) \leq \eps
    $$
\end{proposition}

\begin{demonstration}
    Si $f \in \mathcal{R}(a, b)$, soit $\eps > 0$.
    Par caractérisation de la borne inférieure,
    \begin{align}
        \exists \sigma(\eps), S(f) \leq S(f, \sigma) \leq S(f) + \dfrac{\eps}{2}
    \end{align}

    De même
    \begin{align}
        \exists \sigma'(\eps), s(f) + \dfrac{\eps}{2} \leq s(f, \sigma'(\eps)) \leq s(f)
    \end{align}

    Soit $\sigma''(\eps) = \sigma(\eps) \cup \sigma'(\eps)$, (1) reste vraie avec $\sigma''(\eps)$ au lieu de $\sigma(\eps)$ car $\sigma''(\eps)$ est plus fine.

    De même (2) reste vraie avec $\sigma''(\eps)$ au lieu de $\sigma'(\eps)$.

    D'où en soustrayant (1) - (2)
    \begin{align*}
        S(f) - s(f) &\leq (S - s)(f, \sigma(\eps))\\
        &\leq S(f) - s(f) + \eps
    \end{align*}

    Comme $f \in \mathcal{R}(a, b)$, $S(f) - s(f) = 0$ d'où le résultat.

    Supposons \begin{hotwarn}
        FLEMME
    \end{hotwarn}
\end{demonstration}

\subsection{Exemples fondamentaux}

\begin{remarque}
    Monotone sur un segment implique bornée.
\end{remarque}

\begin{lemme}{}{}
    Soient $a, b \in \R, a < b$ et $f: [a, b] \to \R$.

    $f \in \mathcal{R}(a, b) \iff -f \in \mathcal{R}(a, b)$ et alors
    $\int_a^b -f(x)dx = -\int_a^b f(x)dx$
\end{lemme}

\begin{demonstration}
    $A \subset \R, A \neq \emptyset$, $sup(-A) = -inf(A) \text{ et } inf(-A) = -sup(A)$
\end{demonstration}

\begin{proposition}{}{}
    Soient $a, b \in \R, a < b$ et $f: [a, b] \to \R$ on a

    $$
    f \text{ monotone } \implies f \in \mathcal{R}(a, b)
    $$
\end{proposition}

\begin{demonstration}
    D'après le lemme, il suffit de faire $f$ croissant, on utilise le critère
    $$
    \forall \sigma \in \sum(a, b), S(f, \sigma) - s(f, \sigma) = \sum\limits_{k=1}^n (M_k - m_k)(x_k - x_{k-1}) \leq 
    $$
    \begin{hotwarn}
        FLEMME
    \end{hotwarn}
\end{demonstration}

\begin{proposition}{}{}
    Soient $a, b \in \R, a < b$ et $f: [a, b] \to \R$ continue. Alors

    $$
    f \in \mathcal{R}(a, b)
    $$
\end{proposition}

\end{document}