\documentclass[a4paper, 12pt]{article}

\usepackage{utils}

\renewcommand*{\today}{10 septembre 2025}

\begin{document}

\hotbox{Analyse 3}{CM 5}{\today}

\subsection{Série à termes de signe non fixé}

\begin{definition}[(Convergence absolue)]
    Soit $\seriedef{u_n}{0}$ une série.

    On dit qu'elle converge \textbf{absolument} si la série $\seriedef{|u_n|}{0}$ converge.
\end{definition}

\begin{proposition}{}{}
    Soit $\seriedef{u_n}{0}$ une série qui converge absolument, alors elle converge.
\end{proposition}

\begin{demonstration}
    Soit $S_n = \sum\limits_{k=0}^{n} u_k$ et $S_n' = \sum\limits_{k=0}^{n} |u_k|$.
    Comme $\sum\limits_{n=0}^{+\infty} |u_n|$ converge, la suite $(S_n')_{n \in
    \mathbb{N}}$ est de Cauchy. Donc, pour tout $\epsilon > 0$, il existe $N \in
    \mathbb{N}$ tel que pour tous $m,n \geq N$, on a
    \[
        |S_n' - S_m'| = \sum_{k=m+1}^{n} |u_k| < \epsilon.
    \]
    Or, par inégalité triangulaire, on a
    \[
        |S_n - S_m| = \left|\sum_{k=m+1}^{n} u_k\right| \leq \sum_{k=m+1}^{n} |u_k| = |S_n' - S_m'|.
    \]
    Donc, pour tous $m,n \geq N$, on a
    \[
        |S_n - S_m| < \epsilon.
    \]
    Ainsi, la suite $(S_n)_{n \in \mathbb{N}}$ est de Cauchy et donc converge. Par
    conséquent, la série $\sum\limits_{n=0}^{+\infty} u_n$ converge.
    \begin{hotwarn}
        À vérifier
    \end{hotwarn}
\end{demonstration}

\begin{definition}
    On dit qu'une série $\seriedef{u_n}{0}$ est alternée si
    $$
    \exists N \in \N, \forall n \geq N, u_n u_{n+1} \leq 0
    $$
\end{definition}

\begin{theoreme}{Critère de Leibniz}{}
    Soit $\seriedef{u_n}{0}$ une série alternée telle que
    \begin{itemize}
        \item $u_n \rightarrow 0$
        \item $|u_n|$ décroissante
    \end{itemize}
    Alors la série $\seriedef{u_n}{0}$ converge et si $S$ est sa somme, on a
    $$
    |S - S_n| \leq |u_{n+1}|
    $$
\end{theoreme}

\begin{demonstration}
    Soit $S_n$ la suite des sommes partielles. On va démontrer que $(S_{2n}, S_{2n+1})$ sont adjacentes.

    Sans perte de généralité la série $\sum\limits_{n=0}^{+\infty} (-1)^n a_n$ avec $a_n \geq 0$ et $u_n = (-1)^n a_n$.

    Pour tout $n \in \N$ on a
    \begin{align*}  
        S_{2n + 1} - S_{2n} = a_{2n+1} \rightarrow 0\\
        S_{2n + 2} - S_{2n} = a_{2n+1} - a_{2n+2} \geq 0\\
        S_{2n + 3} - S_{2n+1} = a_{2n+2} - a_{2n+3} \geq 0
    \end{align*}
    Ce qui démontre que les deux suites sont effectivement adjacentes.
    Donc la suite des sommes partielles converge, et la série converge.

    On a alors $S_{2n + 1} \leq S \leq S_{2n}$ et donc $|S - S_n| \leq a_{n+1} = |u_{n+1}|$.
    \begin{hotwarn}
        À vérifier
    \end{hotwarn}
\end{demonstration}

\begin{exemple}
    Étudier la convergence de la série $\seriedef{\dfrac{(-1)^n}{\sqrt{n}}}{1}$.
    \begin{itemize}
        \item La série est alternée et $|u_n| = \dfrac{1}{\sqrt{n}} \rightarrow 0$
        \item $|u_n|' = -\dfrac{1}{2n^{3/2}} < 0$ donc $|u_n|$ est décroissante pour $n \geq 1$
    \end{itemize}
    Donc la série converge.
\end{exemple}

\end{document}
