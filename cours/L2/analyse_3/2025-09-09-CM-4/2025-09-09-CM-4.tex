\documentclass[a4paper, 12pt]{article}

\usepackage{utils}

\renewcommand*{\today}{09 septembre 2025}

\begin{document}

\hotbox{Analyse 3}{CM 4}{\today}

\subsection{Critères de d'Alembert et de Cauchy}

\begin{proposition}{Critère de d'Alembert}{}
    Soit $\seriedef{u_n}{0}$ une série atp telle que $\dfrac{u_{n+1}}{u_n} \rightarrow l$:
    \begin{enumerate}
        \item Si $l \lt 1$ la série converge.
        \item Si $l \gt 1$ la série diverge.
        \item Si $l = 1$ le critère est non concluant.
    \end{enumerate}
\end{proposition}

\begin{demonstration}(Pour 2)\n
    $\exists M, n \geq M, \dfrac{u_{n+1}}{u_n} \geq 1$ ainsi $u_{n+1} \geq u_n$
    donc $(u_n)$ ne converge pas vers 0, la série diverge.

    (Pour 1)\n
    Par la définition de la limite,
    $$
    \forall \eps \gt 0, \exists M \geq N, n \geq M \implies l - \eps \lt \dfrac{u_{n+1}}{u_n} \lt \eps + l
    $$
    en particulier pour $\eps = \dfrac{1 - l}{2} \gt 0$
    $$
    \exists M, n \geq M \implies \dfrac{u_{n+1}}{u_n} \lt \dfrac{l + 1}{2}
    $$
    Sans perte de généralité puisqu'on ne change pas la convergence en changeant un nombre
    fini de termes $M = 0$ et donc
    $$
    \forall n, u_{n+1} \lt \dfrac{l + 1}{2}u_n
    $$
    puis par récurence
    $$
    \forall n, u_n \lt \left( \dfrac{l + 1}{2} \right)^n u_0
    $$
    Comme $\dfrac{l+1}{2} \in [0, 1[$ la série de terme général $\left( \dfrac{l+1}{2} \right)^n u_0$
    converge (c'est une série géométrique convergente), d'après la proposition de comparaison ci-dessus
    la série de terme général $u_n$ converge également
    \begin{hotwarn}
        démonstration à revoir
    \end{hotwarn}
\end{demonstration}

\begin{exemple}
    Étudier la convergence de la série $\seriedef{\binom{2n}{n}}{0}$.
    On a $\dfrac{u_{n+1}}{u_n} = \dfrac{\binom{2(n+1)}{n+1}}{\binom{2n}{n}} = \dfrac{(2n+2)(2n+1)}{(n+1)^2} \sim 4$
    donc la série diverge.
\end{exemple}

\begin{proposition}{Critère de Cauchy}{}
    Soit $\seriedef{u_n}{0}$ une série atp telle que $\sqrt[n]{u_n} \rightarrow l$:
    \begin{enumerate}
        \item Si $l \lt 1$ la série converge.
        \item Si $l \gt 1$ la série diverge.
        \item Si $l = 1$ le critère est non concluant.
    \end{enumerate}
\end{proposition}

\begin{demonstration}
    \begin{hotwarn}
        à faire
    \end{hotwarn}
\end{demonstration}

\begin{exemple}
    Étudier la convergence de la série $\seriedef{\left(1 - \dfrac{1}{n}\right)^{n^2}}{1}$.
    On a $\sqrt[n]{u_n} = \left(1 - \dfrac{1}{n}\right)^n \rightarrow \dfrac{1}{e} \lt 1$
    donc la série converge.
\end{exemple}

\begin{proposition}{}{}
    Soit $\seriedef{u_n}{0}$ série numérique à terme \textbf{strictement} positif à partir d'un certain rang.\n
    $\exists L \in [0, +\infty[, \left( \dfrac{u_{n+1}}{u_n} \right) \rightarrow L \implies \left( \sqrt[n]{u_n} \right) \rightarrow L$
\end{proposition}

\begin{demonstration}
    \begin{hotwarn}
        à faire
    \end{hotwarn}
\end{demonstration}

\end{document}
