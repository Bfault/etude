\documentclass[a4paper, 12pt]{article}

\usepackage{utils}

\renewcommand*{\today}{14 October 2025}

\begin{document}

\hotbox{Analyse 3}{CM 11}{\today}

\section{Théorème fondamental de l'analyse}

\begin{theoreme}{Fondamental de l'analyse}{}
    Soit $f: [a,b] \to \mathbb{R}$ une fonction continue et intégrable sur $[a, b]$.
    Soit $\funcdef{F}{[a,b]}{\mathbb{R}}{x}{\int_a^x f(t) \, dt}$

    Alors $F$ est dérivable sur $]a,b[$ et $F'(x) = f(x)$ pour tout $x \in ]a,b[$.
\end{theoreme}

\begin{remarque}
    Si $x = a$ (resp. $x = b$), il s'agit d'une demi-dérivée à droite (resp. à gauche).
\end{remarque}

\begin{corollaire}{}{}
    Soit $I \subset \R, f: [a, b] \to \R$ continue.
    Alors $f$ admet une primitive $F$ sur $[a, b]$.
\end{corollaire}

\begin{demonstration}
    Soit $h$ tel que $x_0 + l \in [a, b], h > 0$.
    On considère le taux d'accroissement de $F$ en $x_0$:
    \begin{align*}
        T_h &= \dfrac{F(x_0 + h) - F(x_0)}{h} - f(x_0) \\
        &= \dfrac{1}{h} \int\limits_{x_0}^{x_0 + h} f(t) \, dt - f(x_0) \\
        &= \dfrac{1}{h} \int\limits_{x_0}^{x_0 + h} (f(t) - f(x_0)) \, dt \\
        \implies \left| T_h \right| &= \left| \dfrac{1}{h} \int\limits_{x_0}^{x_0 + h} (f(t) - f(x_0)) \, dt \right| \\
        &\leq \dfrac{1}{h} \int\limits_{x_0}^{x_0 + h} \left| f(t) - f(x_0) \right| \, dt \\
        &\leq supp\{\left| f(t) - f(x_0) \right|, t \in [x_0, x_0 + h]\}
    \end{align*}
    Or $f$ est continue en $x_0$, donc $\forall \varepsilon > 0, \exists \eta > 0$ tel que si $|t - x_0| < \eta$, alors $|f(t) - f(x_0)| < \varepsilon$.
    Donc si $|h| < \eta$, on a $|f(x_0) - f(t)| < \varepsilon \implies supp\{\left| f(x_0) - f(t) \right|\} \leq \varepsilon$.
    On fait pareil pour $h < 0$.

    Finalement $\forall \varepsilon > 0, \exists \eta > 0$ tel que si $|h| < \eta$, alors $|T_h| < \varepsilon$ ce qui est la définition de 
    $$
    \lim\limits_{h \to 0} \dfrac{F(x_0 + h) - F(x_0)}{h} = f(x_0), \quad \text{c'est à dire } F'(x_0) = f(x_0).
    $$
\end{demonstration}

\begin{corollaire}{}{}
    Soit $f: [a, b] \to \R$ continue, alors pour tout $F$ primitive de $f$

    $$
    \int\limits_a^b f(t) \, dt = F(b) - F(a)
    $$
\end{corollaire}

\begin{demonstration}
    Soit $\funcdef{F_a}{[a, b]}{\R}{x}{\int\limits_a^x f(t) \, dt}$. On a bien
    Soit $F$ une autre primitive qui diffère de $F_a$ à une constante près
    \begin{align*}
        F_a(b) &= F_a(b) - F_a(a) \\
        &= \int\limits_a^b f(t) \, dt
    \end{align*}
\end{demonstration}

\begin{theoreme}{Intégration par parties}{}
    $f, g: [a, b] \to \R$ de classe $\mathcal{C}^1$. Alors
    $$
    \int\limits_a^b(f'g)(t) \, dt =  (fg)(b) (fg)(a) - \int\limits_a^b (fg')(t) \, dt
    $$
\end{theoreme}

\begin{demonstration}
    On connait $(fg)' = f'g + fg'$ d'où $f'g = (fg)' - fg'$. Comme $f, g$ sont de classe $\mathcal{C}^1$ donc $f, g$ sont continues
    , donc $f, g$ sont intégrables.
    Donc
    \begin{align*}
        \int\limits_a^b (f'g)(t) \, dt &= \int\limits_a^b (fg)'(t) \, dt - \int\limits_a^b (fg')(t) \, dt \\
        &= (fg)(b) - (fg)(a) - \int\limits_a^b (fg')(t) \, dt \quad \text{car (fg) est une primitive de (fg)'}
    \end{align*}
\end{demonstration}

\begin{theoreme}{Changement de variable}{}
    Soit $\varphi: [a, b] \to \R$ de classe $\mathcal{C}^1$ et $f:\varphi([a, b]) \to \R$ continue. Alors
    $$
    \int\limits_a^b f(\varphi(t)) \varphi'(t) \, dt = \int\limits_{\varphi(a)}^{\varphi(b)} f(u) \, du
    $$
\end{theoreme}

\begin{demonstration}
    $f$ est continue sur $[a, b]$, donc admet une primitive $F$, d'où $\int\limits_\varphi(a)^\varphi(b) f(u) \, du = F(\varphi(b)) - F(\varphi(a))$.
    D'autre part, $f \circ \varphi \varphi' = (F \circ \varphi)'$ qui est une fonction continue
    car $f, \varphi, \varphi'$ sont continues.
    Donc
    $$
    \int\limits_a^b f \circ \varphi(t) \varphi'(t) \, dt = \int\limits_a^b(F \circ \varphi)'(t) \, dt = (F \circ \varphi)(b) - (F \circ \varphi)(a)
    $$
\end{demonstration}

\begin{proposition}{première formule de la moyenne}{}
    Soit $f: [a, b] \to \R$ continue. Alors il existe $c \in ]a, b[$ tel que
    $$
    \int\limits_a^b f(t) \, dt = (b - a) f(c)
    $$
\end{proposition}

\begin{demonstration}
    On applique le TAF à $F$ une primitive de $f$ et voilà.
\end{demonstration}

\end{document}