\documentclass[a4paper, 12pt]{article}
\usepackage{utils}

\renewcommand*{\today}{02 septembre 2025}

\begin{document}

\hotbox{Analyse 3}{CM 1}{\today}

\section{Rappels}

\begin{definition}
    Soit $\xn{u}$ une suite réelle (ou complexe).\n
    $(u_n)$ est une \textbf{suite de Cauchy} si:
    $$
    \forall \epsilon > 0, \exists N \in \N, \forall p, q \geq N, |u_p - u_q| < \eps
    $$
\end{definition}

\begin{proposition}{}{}
    Dans $\R$ (ou $\C$), une suite converge si et seulement si elle est de Cauchy.
\end{proposition}

\begin{demonstration}
    \begin{itemize}
        \item $(\implies)$\n
            Si $\xn{u}$ converge vers $l \in \R$ (ou $\C$), alors:
            $$
            \forall \eps > 0, \exists N \in \N, \forall n \geq N, |u_n - l| < \eps
            $$
            Soit $\eta \geq 0$ on pose $\eps = \dfrac{\eta}{2}$, on a $|u_n - l| \lt \dfrac{\eta}{2}$\n
            Donc pour $p, q \geq N$:
            $$
            |u_p - u_q| = |u_p - l + l - u_q| \leq |u_p - l| + |u_q - l| < \dfrac{\eta}{2} + \dfrac{\eta}{2} = \eta
            $$

        \item
            \begin{hotwarn}
                Ce sens est plus compliqué à démontrer.
            \end{hotwarn}
    \end{itemize}
\end{demonstration}

\section{Chapitre 1 - Série numériques et complexes}

\subsection{Définitions et premières propriétés}

\begin{definition}
    On appelle série numérique (ou complexe) la donnée de deux suite réelles (ou complexes)
    $(u_n, S_n)_{n \in \N}$ telles que $S_n = \sum\limits_{k=0}^{n}u_k$.\n
    On dit que la série converge si il existe $S \in \R$ (ou $\mathbb{C}$) tel que $S_n \rightarrow S$.\n
    Sinon la série est dite divergente.

    La suite $(S_n)$ est appelée \textbf{suite des sommes partielles}.\n
    $u_n$ le \textbf{terme général}.\n
    En cas de convergence $S$ est appelée \textbf{somme de la série}.\n
    $(S - S_n)_{n \in \N}$ est appelée \textbf{suite des restes}.
\end{definition}

\begin{remarque}
    On note $(\sum\limits_{k=0}^{+\infty} u_k)$ la série (convergente ou non) et $\sum\limits_{k=0}^{+\infty} u_k$ la somme de la série (uniquement si la série converge).
\end{remarque}

\begin{remarque}
    Comme une série est définie par une suite, on ne peut pas commuter un nombre infini de termes
    (mais commuter un nombre fini de termes ne change pas la nature convergente de la série).
\end{remarque}

\begin{exemple}
    $(\sum\limits_{n=0}^{+\infty}n)$\n
    On a $S_n = \dfrac{n(n+1)}{2} \rightarrow +\infty$ donc la série diverge.
\end{exemple}

\begin{exemple}
    $(\sum\limits_{n=0}^{+\infty}(-1)^n)$\n
    On a $S_n = 1$ si $n$ est pair et $S_n = 0$ si $n$ est impair donc la série diverge.
\end{exemple}

\begin{exemple}
    $(\sum\limits_{n=0}^{+\infty}\dfrac{1}{n(n-1)})$\n
    On a $\dfrac{1}{n(n-1)} = \dfrac{1}{n-1} - \dfrac{1}{n}$ (par décomposition en éléments simples)\n
    ainsi
    \begin{align*}
        S_n &= \sum\limits_{n=2}^{N}\dfrac{1}{n(n-1)} = \sum\limits_{n=2}^{N}\dfrac{1}{n-1} - \sum\limits_{n=2}^{N}\dfrac{1}{n}\\
        &= \sum\limits_{n=1}^{N-1}\dfrac{1}{n} - \sum\limits_{n=2}^{N}\dfrac{1}{n} = 1 - \dfrac{1}{N} \rightarrow 1
    \end{align*}
    Donc la série converge.
\end{exemple}

\begin{methode}
    La décomposition en éléments simples est souvent utile.
    On a la formule:
    $$
    \dfrac{1}{(n + a)(n + b)} = \dfrac{1}{b - a}\left(\dfrac{1}{n + a} - \dfrac{1}{n + b}\right)
    $$
\end{methode}

\begin{exemple}
    La \textbf{série harmonique} $(\sum\limits_{n=1}^{+\infty}\dfrac{1}{n})$\n
    $S_{2N} - S_N = \sum\limits_{n=N+1}^{2N}\dfrac{1}{n} \geq N \times \dfrac{1}{2N} = \dfrac{1}{2}$\n
    (La minoration est obtenue en prenant le plus petit terme de la somme $\dfrac{1}{2N}$ et en le multipliant par le nombre de termes $N$)\n
    Donc la suite des sommes partielles n'est pas de Cauchy et la série diverge.
\end{exemple}

\begin{proposition}{}{}
    Pour tout $\alpha \gt 2$, la série $(\sum\limits_{n=1}^{+\infty}\dfrac{1}{n^\alpha})$ converge.
\end{proposition}

\begin{remarque}
    On verra par la suite que ce type de série s'appelle une \textbf{série de Riemann} et que la série converge aussi pour $\alpha \gt 1$.
    Mais la démonstration est plus simple pour $\alpha \gt 2$.
\end{remarque}

\begin{demonstration}
    Soit $\alpha \geq 2$.\n
    On a:
    $$
    \forall N \geq 2, S_N = \sum\limits_{n=2}^{N}\dfrac{1}{n^\alpha} \leq \sum\limits_{n=2}^{N}\dfrac{1}{n(n-1)} \leq 1
    $$
    Aussi
    \begin{align*}
    S_{N+1} - S_N &= \sum\limits_{n=2}^{N+1}\dfrac{1}{n^\alpha} - \sum\limits_{n=2}^{N}\dfrac{1}{n^\alpha}\\
    &= \sum\limits_{n=N+1}^{N+1}\dfrac{1}{n^\alpha}\\
    &= \dfrac{1}{(N+1)^\alpha} \gt 0
    \end{align*}
    Ainsi $(S_N)$ est croissante et majorée par 1, donc elle converge.
\end{demonstration}

\begin{proposition}{}{}
    Pour tout $\alpha \in [0, 1]$, la série $(\sum\limits_{n=1}^{+\infty}\dfrac{1}{n^\alpha})$ diverge.
\end{proposition}

\begin{demonstration}
    Soit $\alpha \in [0, 1]$.\n
    On a pour tout $n$: $\dfrac{1}{n^\alpha} \geq \dfrac{1}{n}$.\n
    Ainsi pour tout $N$: $S_N^\alpha \geq S_N^1$.\n
    Or on a vu que la série harmonique diverge (vers $+\infty$) et est croissante, donc $S_N^\alpha \rightarrow +\infty$.\n
\end{demonstration}

\begin{proposition}{}{}
    Si une série converge, alors son terme général tend vers 0.
\end{proposition}

\begin{methode}
    C'est la contraposée qui est utile\n
    $u_n \not\rightarrow 0 \implies \left(\sum\limits_{n=0}^{+\infty} u_n\right) \text{ diverge}$.

    C'est le premier critère qu'on utilise pour étudier la nature d'une série.
\end{methode}

\begin{demonstration}
    Si la série $\left( \sum\limits_{n=0}^{+\infty} u_n \right)$ converge,
    alors la suite des sommes partielles $(S_n)$ converge donc est de Cauchy.\n
    D'où $S_{n+1} - S_n = u_{n+1} \rightarrow 0$.
\end{demonstration}

\begin{remarque}
    La réciproque est fausse.\n
    Par exemple la série harmonique a un terme général qui tend vers 0 mais diverge.
\end{remarque}

\begin{exemple}
    On peut maintenant étudier la série $(\sum\limits_{n=0}^{+\infty}(-1)^n)$.\n
    Le terme général $u_n = (-1)^n$ est soit 1 soit -1, donc ne tend pas vers 0.\n
    Ainsi la série diverge.
\end{exemple}

\subsection{Série géométrique et produit dérivés}

On connait : pour tout $x \in \C, x \neq 1$ et tout $n \in \N$:
$$
\sum\limits_{k=0}^{n} x^k = \dfrac{1 - x^{n+1}}{1 - x} = \dfrac{1}{1 - x} - \dfrac{x^{n+1}}{1 - x}
$$

Si $|x| \geq 1$, alors $x^{n+1} \not\rightarrow 0$ donc la série diverge.\n
Si $|x| \lt 1$, alors $x^{n+1} \rightarrow 0$ donc la série converge et $\sum\limits_{n=0}^{N} = \dfrac{1}{1 - x}$.

\begin{proposition}{Série géométrique dérivée}{}
    La série $(\sum\limits_{n=0}^{+\infty} nx^n)$ converge pour $|x| \lt 1$ et sa somme est
    $$
    \sum\limits_{n=0}^{+\infty}nx^n = \dfrac{x}{(1 - x)^2}
    $$
\end{proposition}

\begin{demonstration}
    En dérivant la somme de la série géométrique, on trouve:
    \begin{align*}
        \sum\limits_{n=1}^{+\infty}nx^{n-1} &= \dfrac{1}{(1 - x)^2} - \dfrac{(n+1)x^n(1 - x) + x^{n + 1}}{(1 - x)^2}\\
        &= \dfrac{1}{(1 - x)^2} - \dfrac{(n+1)x^n}{1 - x} - \dfrac{x^{n + 1}}{(1 - x)^2}\\
        \implies \sum\limits_{n=1}^{+\infty}nx^{n} &= \dfrac{x}{(1 - x)^2} - \dfrac{(n+1)x^{n+1}}{1 - x} - \dfrac{x^{n + 2}}{(1 - x)^2}
    \end{align*}
    Si $|x| \lt 1$, alors $x^{n+1} \rightarrow 0$ et $x^{n+2} \rightarrow 0$ donc la série converge et
    $$
    \sum\limits_{n=1}^{+\infty}nx^{n} = \dfrac{x}{(1 - x)^2}
    $$
    Dans l'autre cas ($|x| \geq 1$), la série diverge.
\end{demonstration}

\begin{proposition}{Primitive de série géométrique}{}
    La série $(\sum\limits_{n=1}^{+\infty}\dfrac{1}{n}x^n)$ converge pour $|x| \in [-1, 1[$ et sa somme est
    $$
    \sum\limits_{n=1}^{+\infty}\dfrac{1}{n}x^n = -\ln(1 - x)
    $$
\end{proposition}

\begin{demonstration}
    En intégrant la somme de la série géométrique, on trouve:
    \begin{align*}
        \sum\limits_{n=0}^{+\infty}\dfrac{1}{n+1}x^{n+1} &= -\ln(1 - x) - \dfrac{x^{n+1}}{n+1}\dfrac{1}{1 - x}\\
        = \sum\limits_{n=1}^{+\infty}\dfrac{1}{n}x^{n} &= -\ln(1 - x) - \dfrac{x^{n+1}}{n+1}\dfrac{1}{1 - x}
    \end{align*}
    Si $|x| \in [-1, 1[$, alors $\dfrac{x^n}{n} \rightarrow 0$ donc la série converge et
    $$
    \sum\limits_{n=1}^{+\infty}\dfrac{1}{n}x^{n} = -\ln(1 - x)
    $$
    Dans l'autre cas, la série diverge.
\end{demonstration}

\end{document}
