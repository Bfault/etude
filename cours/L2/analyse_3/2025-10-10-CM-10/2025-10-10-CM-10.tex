\documentclass[a4paper, 12pt]{article}

\usepackage{utils}

\renewcommand*{\today}{10 October 2025}

\begin{document}

\hotbox{Analyse 3}{CM 10}{\today}

\subsection{Parties positives, négatives, valeurs absolue, inf, sup}

\begin{proposition}{}{}
    Soit $f \in \mathcal{R}(a,b)$ avec $a < b$. Alors $f^+, f^⁻ \in \mathcal{R}(a, b)$
\end{proposition}

\begin{demonstration}
    Il suffit de démontrer que $f^+ \in \mathcal{R}(a, b)$ et alors $f^- = f^+ - f \in \mathcal{R}(a, b)$ puisque c'est un EV.
    On utilise le critère.

    Soit $\sigma = (x_i)_{0 \leq i \leq n} \in \sum(a, b)$. On note $M_i$ et $m_i$ comme d'habitude, et
    $M_i^+$ et $m_i^+$ de même pour $f^+$.

    Pour tout $i \in \llbracket 1, n \rrbracket$ on a

    \begin{itemize}
        \item si $M_i \geq m_i \geq 0$ alors $M_i = M_i^+$ et $m_i = m_i^+$
        et donc $M_i^+ - m_i^+ = M_i - m_i \leq M_i - m_i$
        \item Si $m_i \geq M_i \geq 0$ alors $M_i^+ - m_i^+ = 0 - 0 = 0 \leq M_i - m_i$
        \item Si $m_i < 0 \leq M_i$ alors $M_i^+ = M_i$ alors $m_i^+ = 0$ et donc $M_i^+ - m_i^+ = M_i \leq M_i - m_i$
    \end{itemize}

    En résumé, pour tout $i$,
    $$
    M_i^+ - m_i^+ = M_i \leq M_i - m_i
    $$
    En multipliant par $x_i - x_{i-1} \geq 0$ et en sommant sur $i$, il vient
    $$
    (S - s)(f^+, \sigma) \leq (S - s)(f, \sigma)
    $$
    Et le résulat avec le critère.
\end{demonstration}

\begin{demonstration}
    On connait
    $$
    \sup\{f, g\} = \dfrac{f + g + |f - g|}{2}
    $$
    Comme $\mathcal{R}(a, b)$ est un EV $f - g \in \mathcal{R}(a, b)$ et donc d'après le corollaire 1,
    $|f - g| \in \mathcal{R}(a, b)$ et donc $\sup\{f, g\} \in \mathcal{R}(a, b)$ puisque c'est un EV.
\end{demonstration}

\begin{demonstration}
    Si $f \geq 0$ pour tout $\sigma \in \sum(a, b), s(f, \sigma) \geq 0$ et comme
    $\int_a^b f(x)dx \geq s(f, \sigma)$ le résultat suit.
\end{demonstration}

\begin{demonstration}
    En effet $f - g \geq 0$ d'après la proposition précédente
    \begin{hotwarn}
        à faire.
    \end{hotwarn}
\end{demonstration}

\begin{demonstration}
    En effet
    $$
    f \leq |f| \text{ et } -f \leq |f|
    $$
    D'où avec le corollaire précédent
    \begin{align*}
        \int_a^bf(x)dx \leq \int_a^b |f(x)|dx \\
        \int_a^bf(x)dx \leq
    \end{align*}
    \begin{hotwarn}
        À faire.
    \end{hotwarn}
\end{demonstration}

\begin{demonstration}
    C'est vrai pour tout $s(f, \sigma)$ et $S(f, \sigma)$ (voir propriétés des sommes de Darboux)
\end{demonstration}

\begin{demonstration}
    Par définition de la continuité en $x_0$
    $$
    \forall \varepsilon > 0, \exists \delta > 0, |x_0 - x| < \delta \Rightarrow |f(x_0) - f(x)| < \varepsilon
    $$
    En particulier avec $\varepsilon = f(\dfrac{x_0}{2})$
    $$
    \exists \delta > 0, |x_0 - x| < \delta, x \in [a, b] \implies f(x) > \dfrac{1}{2}f(x_0)
    $$
    Soit $[c, d] = [a, b] \cap [x - \delta, x + \delta]$ on a donc $f(x) > \dfrac{1}{2}f(x_0)$ Soit $g$ la fonction
    en escalier définie par
    $$
    g(x) = \begin{cases}
    \dfrac{1}{2} f(x_0) \text{ si } x \in [c, d] \\
    0 \text{ sinon}
    \end{cases}
    $$
\end{demonstration}

\end{document}