\documentclass[a4paper, 12pt]{article}

\usepackage{utils}

\renewcommand*{\today}{24 October 2025}

\begin{document}

\hotbox{Géometrie 1}{CM 6}{\today}

\section{Applications affines}

\begin{hotwarn}
    unicité
\end{hotwarn}

\begin{demonstration}
Soit $a \in \E$ et $\funcdef{\theta_a}{\E}{E}{m}{\overline{am}}$ une bijection.
    
Donc si $\vec{u} \in E, \exists! m \in \E, \vec{u} = \overline{am}$ et
$\vec{f}(\vec{u}) = \vec{f}(\overline{am}) = \overline{f(a)f(m)}$ qui est unique
dans $E$ donc $\vec{f}: E \to F$ est unique.

De plus soit $b \in \E$

\begin{align*}
    \vec{f}(\overline{bm}) &= \vec{f}(\overline{ba} + \overline{am}) \\
    &= \vec{f}(\overline{am} - \overline{ba}) \\
    &= \vec{f}(\overline{am}) - \vec{f}(\overline{ba}) \\
    &= \overline{f(a)f(m)} - \overline{f(a)f(b)} \\
    &= \overline{f(b)f(m)}
\end{align*}
\end{demonstration}

\end{document}