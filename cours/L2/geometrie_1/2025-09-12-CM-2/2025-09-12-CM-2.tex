\documentclass[a4paper, 12pt]{article}

\usepackage{utils}

\renewcommand*{\today}{12 septembre 2025}

\begin{document}

\hotbox{Géométrie 1}{CM 2}{\today}

\begin{demonstration}
    Soit $F$ un sev de $E$ et $m_0 \in \mathscr{E}$.

    On pose $\mathscr{F} = \{ m \in \mathscr{E} \mid \vec{m_0 m} \in F\}$
    c'est à dire $\mathscr{F}$ est le sous-espace affine passant par $m_0$ et de direction $F$.

    Montrons que $\mathscr{F}$ est un espace affine de direction $F$.

    Soit $\theta'$ la restriction de $\theta: \mathscr{E} \times \mathscr{E} \rightarrow E$ à $\mathscr{F} \times \mathscr{F}$.
    c'est-à-dire $\funcdef{\theta'}{\mathscr{F} \times \mathscr{F}}{E}{(m, m')}{\theta(m, m') = \vec{mm'}}$.

    Montrons que: $\forall (m, m') \in \mathscr{F} \times \mathscr{F}, \theta'(m, m') \in F$.

    Soit $(m, m') \in \mathscr{F} \times \mathscr{F}$.

    \begin{align*}
        \vec{mm'} = \vec{m, m_0} + \vec{m_0, m'} \in F\\
        &= -\vec{m_0, m} + \vec{m_0, m'} \in F \text{ car $F$ est un sev de $E$}\\
    \end{align*}

    D'où $\funcdef{\theta'}{\mathscr{F} \times \mathscr{F}}{F}{(m, m')}{\theta'(m, m') = \vec{mm'}}$ est une application et
    vérifie la relation de Chasles car $\theta$ la vérifie.

    Soit $a \in \mathscr{F}$\n
    On a $\funcdef{\theta'_a}{\mathscr{F}}{F}{m}{\theta'(a, m) = \vec{am}}$

    Montrons que $\theta'_a$ est bijective.

    Soit $\vec{v} \in F$, donc $\vec{v} \in E$.

    Comme $\theta_a$ est bijective, il existe un unique $m \in \mathscr{E}$ tel que $\theta_a(m) = \vec{v}$.

    Reste à montrer que $m \in \mathscr{F}$.

    En utilisant Chasles, on a:
    
    $$
    \vec{m_0 m} = \vec{m_0 a} + \vec{am} = \vec{m_0 a} + \theta_a(m) = \vec{m_0 a} + \vec{v} \in F
    $$
\end{demonstration}

\begin{demonstration}
    Avec $\mathscr{F} = \{m \in \mathscr{E} \mid \vec{m_0 m} \in F\}$

    Soit $m_1 \in \mathscr{F}$.

    Montrons que $\mathscr{F} = \{m \in \mathscr{E} \mid \vec{m_1 m} \in F\}$

    Soit $m \in \mathscr{F}$

    \begin{align*}
        \vec{m_1 m} &= \vec{m_1 m_0} + \vec{m_0 m} \in F\\
        &= -\vec{m_0 m_1} + \vec{m_0 m} \in F \text{ car $F$ est un sev de $E$}
    \end{align*}

    Donc $\mathscr{F} \subset \{m \in \mathscr{E} \mid \vec{m_1 m} \in F\}$

    Réciproquement, soit $m \in \mathscr{E}$ tel que $\vec{m_1 m} \in F$.

    \begin{align*}
        \vec{m_0 m} &= \vec{m_0 m_1} + \vec{m_1 m} \in F\\
        &= \vec{m_0 m_1} + \vec{m_1 m} \in F \text{ car $F$ est un sev de $E$}
    \end{align*}

    Ainsi $\mathscr{F} = \{m \in \mathscr{E} \mid \vec{m_1 m} \in F\}$
\end{demonstration}

\begin{demonstration}
    Soient $m_0 \in \mathscr{E}$, $F$ un sev de $E$.

    Soit $\mathscr{F} = \{m \in \mathscr{E} \mid \vec{m_0 m} \in F\}$

    Alors $\mathscr{F}$ est un sous espace de $\mathscr{E}$ de direction $F$ passant par $m_0$.
    d'où l'existence.

    Pour l'unicité, si $\mathscr{F'}$ est un sous-espace affine de $\mathscr{E}$ de direction $F$ passant par $m_0$.

    Ainsi $\mathscr{F'} = m_0 + F = \mathscr{F}$
\end{demonstration}

\end{document}
