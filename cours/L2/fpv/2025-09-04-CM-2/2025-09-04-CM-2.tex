\documentclass[a4paper, 12pt]{article}

\usepackage{utils}
\usepackage{pgfplots}

\renewcommand*{\today}{04 septembre 2025}

\begin{document}

\hotbox{Fonction à plusieurs variables}{CM 2}{\today}

\begin{definition}[(Caractérisation séquentielle de la limite)]
    Soient $A \subset E$, $f : A \to F$,\n
    $a \in \overline{A}$ et $l \in \overline{F}$.

    $$
    \lim_{x \to a}f(x) = l \iff \left(\, \forall (x_n) \in A^\N,\, x_n \rightarrow a \implies f(x_n) \rightarrow l \right)
    $$
\end{definition}

\begin{definition}[(Caractérisation de la limite avec la norme)]
    Soient $A \subset E$ et $V \subset F$, $f: A \rightarrow B$.\n
    Soient $a \in \overline{A}$ et $l \in \overline{V}$.

    \begin{align*}
        &\lim_{x \to a}f(x) = l\\
        &\iff \forall \eps \gt 0,\, \exists \delta \gt 0,\, \forall x \in A,\, ||x - a||_E \lt \delta \implies ||f(x) - l||_F \lt \eps\\
        &\iff \forall \eps \gt 0,\, \exists \delta \gt 0,\, \forall x \in (B(a, \delta) \cap A) \implies f(x) \in B(l, \eps)
    \end{align*}

    \begin{hotwarn}
        Mettre le dessin
    \end{hotwarn}
\end{definition}

\begin{exemple}
    Soient $E = (\R^2, ||.||_2)$, $F = (\R, |.|)$, $U = \R^2\backslash\{(0,0)\}$ et $f: U \rightarrow F$ définie par
    $f(x,y) = \frac{xy}{x^2 + y^2}$.

    Supposons que $\lim_{(x,y) \to (0,0)}f(x,y) = l$.

    \n

    Soit $(x_n) := (\frac{1}{n}, 0)_n$, $\lim_{n \to +\infty}x_n = (0,0)$,
    
    de plus $l = \lim_{n \to +\infty}f(\frac{1}{n}, 0) = \lim_{n \to +\infty}\frac{\frac{1}{n} \times 0}{(\frac{1}{n})^2 + 0^2} = 0$.

    \n

    Soit $(y_n) := (\frac{1}{n}, \frac{1}{n})_n$, $\lim_{n \to +\infty}y_n = (0,0)$,
    
    de ce côté $l = \lim_{n \to +\infty}f(\frac{1}{n}, \frac{1}{n}) = \lim_{n \to +\infty}\frac{\frac{1}{n} \times \frac{1}{n}}{(\frac{1}{n})^2 + (\frac{1}{n})^2} = \frac{1}{2}$.

    \n

    Absurde $0 \neq \frac{1}{2}$.
    Donc $f$ n'admet pas de limite en $(0,0)$.
\end{exemple}

\begin{definition}
    Soient $A, B \subset E, f: A \rightarrow F$ et $a \in \overline{A \cap B}$

    On dit que \textbf{$f$ tend vers $b \in F$ en $a$ le long de $B$}, et l'on note $\lim_{x \to a, x \in B}f(x) = b$ si
    $$
    \forall \eps \gt 0,\, \exists \delta \gt 0,\, \forall x \in A \cap B,\, \|x - a\|_E \lt \delta \implies \|f(x) - b\|_F \lt \eps
    $$

    ou plus simplement si $\lim_{x \to a}f|_{A \cap B}(x) = b$.
\end{definition}

\begin{remarque}
    Si $F = \R$ on peut aussi considérer $b \in F \cup \{-\infty, +\infty\}$.
\end{remarque}

\begin{definition}[(Continuité)]
    Soient $A \subset E, B \subset F$ et $f: A \rightarrow B$.

    Pour $a \in A$, on dit que $f$ est \textbf{continue en $a$} si $\lim_{x \to a}f(x) = f(a)$.
\end{definition}

\begin{theoreme}{Théorème des gendarmes}{}
    Soient $A \subset E, a \in \overline{A}$ et $f,g,h: A \to F$.

    Si $\exists \eps \gt 0, \forall x \in A \cap B(a, \delta)$ tel que $f(x) \leq g(x) \leq h(x)$\n et
    $\lim_{x \to a}f(x) = \lim_{x \to a}h(x) = b$ alors $\lim_{x \to a}g(x) = b$.
\end{theoreme}

\begin{remarque}
    Au vu de la restriction par des evn ordonnés, ce théorème n'est pas très utile quand $F \neq \R^n$.
\end{remarque}

\begin{demonstration}
    \begin{hotwarn}
        À faire
    \end{hotwarn}
\end{demonstration}

\begin{proposition}{Composition des limites}{}
    Soient $A \subset E, B \subset F, C \subset G$ et
    $f: A \rightarrow B$, $g: B \rightarrow C$.

    Soient $a \in \overline{A}, b \in \overline{B}, c \in \overline{C}$.

    $$
    \lim\limits_{x \to a}f(x) = b \text{ et } \lim\limits_{x \to b}g(x) = c \implies \lim_{x \to a}g(f(x)) = c
    $$

    En partie si $f$ est continue en $a$, $g$ en $b=f(a)$ alors $g \circ f$ est continue en $a$.
\end{proposition}

\begin{demonstration}
    \begin{hotwarn}
        À faire
    \end{hotwarn}
\end{demonstration}

\begin{definition}[(Prolongement par continuité)]
    Soient $A \subset E$ et $f: A \rightarrow F$.

    Si $f$ possède une limite $b \in F$ en un point $a \in \overline{A}\backslash A$
    , alors l'application:

    $$
    \funcdef{\tilde{f}}{A \cup \{a\}}{F}{x}{\begin{cases}
        f(x) & \text{si } x \in A\\
        b & \text{si } x = a
    \end{cases}}
    $$
    est appelée le \textbf{prolongement par continuité} de $f$ en $a$.
\end{definition}

\begin{remarque}
    Si $f$ admet un prolongement par continuité en $a$, alors celui-ci est unique.
\end{remarque}

\end{document}