\documentclass[a4paper, 12pt]{article}

\usepackage{utils}

\renewcommand*{\today}{10 septembre 2025}

\begin{document}

\hotbox{Fonctions à plusieurs variables}{CM 3}{\today}

\subsection{Compacité et équivalence des normes en dimension finie}

Dans la suite on considère les evn $(E, ||.||), (F, ||.||')$.

\vspace{3mm}
\noindent
En essayant de généraliser le théorème des borns atteintes, c'est à dire

Soit $f: [a, b] \rightarrow \R$ continue, alors $f$ est bornée et atteint ses bornes.

\noindent
On peut définir la notion suivante

\begin{definition}
    Soit $K \subset E$.

    On dit que $K$ est \textbf{compacte} si
    $$
    \forall (x_n) \in K^\N, \exists \phi: \N \to \N, \phi \text{ strictement croissante}, \lim_{n \to +\infty}x_{\phi(n)} \in K
    $$
    C'est à dire que toute suite à valeur dans $K$ admet une sous-suite convergente vers un élément de $K$.
\end{definition}

\begin{remarque}
    Une reformulation plus simple est:

    Une partie $K$ de $E$ est \textbf{compacte} si toute suite d'éléments de $K$ possède au moins une valeur d'adhérence dans $K$.
\end{remarque}

\begin{definition}
    Une partie $V \subset E$ est \textbf{bornée} si $\exists M \gt 0, \forall x \in V, ||x|| \leq M$.
\end{definition}

\begin{proposition}{}{}
    Si $K$ est une partie compacte de $\R$, alors $\exists a, b \in K, a = \inf K, b = \sup K$.

    On peut donc écrire $a = \min K, b = \max K$.
\end{proposition}

\begin{hotwarn}
    Démonstration à faire
\end{hotwarn}

\begin{proposition}
    Soient $U \subset E, V \subset F$ et $f: U \rightarrow V$ continue.
    Si $K \subset U$ est compacte, alors $f(K)$ est compacte.
\end{proposition}

\begin{hotwarn}
    Démonstration à faire
\end{hotwarn}

\begin{proposition}{}{}
    Un ensemble compacte $K$ d'un env est fermé et borné.
\end{proposition}

\begin{theoreme}{Bolzano-Weierstrass}{}
    Soit $V$ une partie fermée et bornée de $(\R, \|.\|)$ alors $V$ est compacte.
\end{theoreme}

\begin{exemple}
    $\xn{x} \text{ avec } x_n := \dfrac{1}{n} + (-1)^n$ suite dans $[-2, 2]$
\end{exemple}

\end{document}
