\documentclass[a4paper, 12pt]{article}

\usepackage{utils}

\renewcommand*{\today}{10 septembre 2025}

\begin{document}

\hotbox{Fonctions à plusieurs variables}{CM 3}{\today}

\subsection{Compacité et équivalence des normes en dimension finie}

En essayant de généraliser le théorème des borns atteintes, c'est à dire

Soit $f: [a, b] \rightarrow \R$ continue, alors $f$ est bornée et atteint ses bornes.

\noindent
On peut définir la notion suivante

\begin{definition}[(Définition séquentielle de la compacité)]
    Soit $K \subset E$.

    On dit que $K$ est \textbf{compacte} si
    $$
    \forall (x_n) \in K^\N, \exists \phi: \N \to \N, \phi \text{ strictement croissante}, \lim_{n \to +\infty}x_{\phi(n)} \in K
    $$
    C'est à dire que toute suite à valeur dans $K$ admet une sous-suite convergente vers un élément de $K$.
\end{definition}

\begin{remarque}
    Une reformulation plus simple est:

    Une partie $K$ de $E$ est \textbf{compacte} si toute suite d'éléments de $K$ possède au moins une valeur d'adhérence dans $K$.
\end{remarque}

\begin{definition}
    Une partie $V \subset E$ est \textbf{bornée} si $\exists M \gt 0, \forall x \in V, \|x\| \leq M$.
\end{definition}

\begin{proposition}{}{}
    Si $K$ est une partie compacte de $\R$, alors $\exists a, b \in K, a = \inf K, b = \sup K$.

    On peut donc écrire $a = \min K, b = \max K$.
\end{proposition}

\begin{hotwarn}
    Démonstration à faire
\end{hotwarn}

\begin{proposition}{}{}
    Soient $U \subset E, V \subset F$ et $f: U \rightarrow V$ continue.
    Si $K \subset U$ est compacte, alors $f(K)$ est compacte.
\end{proposition}

\begin{hotwarn}
    Démonstration à faire
\end{hotwarn}

\begin{proposition}{}{}
    Un ensemble compact $K$ d'un env est fermé et borné.
\end{proposition}

\begin{remarque}
    La réciproque est fausse en dimension infinie mais utile en dimension finie.
\end{remarque}

\begin{demonstration}
    \begin{hotwarn}
        Démonstration à faire
    \end{hotwarn}
\end{demonstration}

\begin{theoreme}{Bolzano-Weierstrass}{}
    Soit $V$ une partie fermée et bornée de $(\R, \|.\|)$ alors $V$ est compacte.
\end{theoreme}

\begin{demonstration}
    \begin{hotwarn}
        Démonstration à faire
    \end{hotwarn}
\end{demonstration}

\begin{lemme}{}{}
    Soient $n \in \N^*$ et $A$ une partie fermée bornée de $\R^n$ munie de la norme $l^\infty$.
    Alors $A$ est compacte.
\end{lemme}

\begin{demonstration}
    \begin{hotwarn}
        Démonstration à faire
    \end{hotwarn}
\end{demonstration}

\begin{definition}
    Soient $N_1$ et $N_2$ deux normes sur $E$. On dit que $N_1$ et $N_2$ sont \textbf{équivalentes}
    si et seulement si il existe $\alpha, \beta \gt 0$ tels que
    $$
    \forall x \in E, \alpha N_1(x) \leq N_2(x) \leq \beta N_1(x)
    $$
\end{definition}

\begin{remarque}
    L'équivalence des normes est une relation d'équivalence.
\end{remarque}

\begin{proposition}{}{}
    Soient $N_1$ et $N_2$ deux normes équivalentes sur $E$.
    Alors une suite $(x_n) \in E^\N$ converge pour $N_1$ si et seulement si elle converge pour $N_2$.
\end{proposition}

\begin{demonstration}
    \begin{hotwarn}
        Démonstration à faire
    \end{hotwarn}
\end{demonstration}

\begin{corollaire}{}{}
    Soient $N_1$ et $N_2$ deux normes équivalentes sur $E$.
    Si $(x_n) \in E^\N$ telle que $(x_n)$ converge pour $N_1$ vers $l \in E$, alors $(x_n)$ converge pour $N_2$ vers $l$.
\end{corollaire}

\begin{demonstration}
    \begin{hotwarn}
        Démonstration à faire
    \end{hotwarn}
\end{demonstration}

\begin{proposition}
    Soit $N$ une norme sur $\R^n$.
    Alors $N$ est équivalente à la norme $l^\infty$.
\end{proposition}

\begin{demonstration}
    \begin{hotwarn}
        Démonstration à faire
    \end{hotwarn}
\end{demonstration}

\begin{proposition}
    Soient $N_1$ et $N_2$ deux normes sur un ev de dimension finie $E$.
    Alors $N_1$ et $N_2$ sont équivalentes.
\end{proposition}

\begin{demonstration}
    \begin{hotwarn}
        Démonstration à faire
    \end{hotwarn}
\end{demonstration}

\begin{proposition}{}{}
    Soient $N$ une norme sur $R^n$ et $((x_{i,k})_{1 \leq i \leq n})_k \in (\R^n)^\N$ une suite.
    On a $\lim\limits_{k \to +\infty} (x_{i,k})_{1 \leq i \leq n} = (l_i)_{1 \leq i \leq n}$ si et seulement si
    $$
    \forall i \in \llbracket 1, n \rrbracket, \lim\limits_{k \to +\infty} x_{i,k} = l_i
    $$
\end{proposition}

\begin{demonstration}
    \begin{hotwarn}
        Démonstration à faire
    \end{hotwarn}
\end{demonstration}

\begin{proposition}{}{}
    Soient $A \subset E, a \in \overline{A}$ et $f_1, f_2, \ldots, f_n: A \rightarrow F$ des fonctions.

    La fonction $\funcdef{g}{A}{\R^n}{x}{(f_1(x), f_2(x), \ldots, f_n(x))}$ converge vers $l = (l_1, l_2, \ldots, l_n) \in F^n$ en $a$ si et seulement si
    $$
    \forall i \in \llbracket 1, n \rrbracket, \lim\limits_{x \to a} f_i(x) = l_i
    $$
\end{proposition}

\begin{demonstration}
    \begin{hotwarn}
        Démonstration à faire
    \end{hotwarn}
\end{demonstration}

\subsection{Norme matricielle}

On note $\mathcal{L}_c(E, F)$ l'ensemble des applications linéaires continues de $E$ dans $F$.

\begin{definition}
    Soit $u \in \mathcal{L}_c(E, F)$.
    L'ensemble des réels $C \geq 0$ tels que
    $$
    \forall x \in E, \norm{u(x)} \leq C \norm{x}
    $$
    admet un plus petit élément, appelé \textbf{norme subordonnée} de $u$ et notée $\mnorm{u}$.
    On a de plus
    $$
    \mnorm{u} = \sup\limits_{x \neq 0} \frac{\norm{u(x)}}{\norm{x}} = \sup\limits_{\norm{x} \leq 1} \norm{u(x)} = \sup\limits_{\norm{x} = 1} \norm{u(x)}
    $$
\end{definition}

\begin{definition}
    Soient $N_n$ et $N_m$ deux normes sur $\K^n$ et $\K^m$ respectivement.
    Étant donné $A \in \mathcal{M}_{m,n}(\K)$, on définit la \textbf{norme subordonnée} de $A$,
    et l'on note $\mnorm{A}$ la norme subordonnée de l'application linéaire canonique associée à $A$.
\end{definition}

\end{document}
