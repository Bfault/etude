\documentclass[a4paper, 12pt]{article}
\usepackage{amsmath, amssymb, amsthm, stmaryrd}
\usepackage{geometry}
\usepackage{pgfplots}
\usepackage{tcolorbox}
\geometry{hmargin=2.5cm, vmargin=2.5cm}

\renewcommand*{\today}{06 février 2025}

\title{Analyse2 | CM: 3}
\author{Par Lorenzo}
\date{\today}

\newtheorem{theorem}{Théorème}[section]
\newtheorem{definition}{Définition}[section]
\newtheorem{example}{Example}[section]
\newtheorem{remark}{Remarques}[section]
\newtheorem{lemme}{Lemme}[section]
\newtheorem{corollaire}{Corollaire}[section]

\newtheorem{_proposition}{Proposition}[section]
\newenvironment{proposition}[1][]{
    \begin{_proposition}[#1]~\par
    \vspace*{0.5em}
}{%
    \end{_proposition}%
}

\newtheorem{_proprietes}{Propriétés}[section]
\newenvironment{proprietes}[1][]{
        \begin{_proprietes}[#1]~\par
        \vspace*{0.5em}
}{%
        \end{_proprietes}%
}

\newenvironment{rdem}[1][]{
    \begin{tcolorbox}[colframe=black, colback=white!10, sharp corners]
        #1
}{%
    \end{tcolorbox}
     
}

\newtheorem{_demonstration}{Démonstration}[section]
\newenvironment{demonstration}[1][]{
    \begin{_demonstration}[#1]~\par
    \vspace*{0.5em}
}{%
    \end{_demonstration}%
    \qed%
}

\newtheorem*{_demonstration*}{Démonstration}
\newenvironment{demonstration*}[1][]{
    \begin{_demonstration*}[#1]~\par
    \vspace*{0.5em}
}{%
    \end{_demonstration*}%
    \qed%
}

\newenvironment{ldefinition}{
    \begin{definition}~\par
    \vspace*{0.5em}
    \begin{enumerate}
}{
        \end{enumerate}
        \end{definition}
}

\newenvironment{lexample}{
    \begin{example}~\par
    \vspace*{0.5em}
    \begin{enumerate}
}{
        \end{enumerate}
        \end{example}
}

\newtheorem{_methode}{Méthode}[section]
\newenvironment{methode}{
    \begin{_methode}~\par
    \vspace*{0.5em}
}{
        \end{_methode}
}

\def\N{\mathbb{N}}
\def\Z{\mathbb{Z}}
\def\Q{\mathbb{Q}}
\def\R{\mathbb{R}}
\def\C{\mathbb{C}}
\def\K{\mathbb{K}}
\def\k{\Bbbk}

\def\un{(u_n)_{n \in \N}}
\def\xn#1{(#1_n)_{n \in \N}}

\def\o{\overline}
\def\eps{\varepsilon}

% \funcdef{name}{domain}{codomain}{variable}{expression}
% name: Name of the function (e.g. f)
% domain: Domain of the function (e.g. \mathbb{R})
% codomain: Codomain of the function (e.g. \mathbb{R})
% variable: Variables of the function (e.g. x)
% expression: Expression of the function (e.g. x^2)
\newcommand{\funcdef}[5]{%
    #1 :
    \begin{cases}
        #2 \rightarrow #3 \\
        #4 \mapsto #5
    \end{cases}
}

\newcommand{\lt}{\ensuremath <}
\newcommand{\gt}{\ensuremath >}

\begin{document}

\maketitle

\begin{demonstration}
    Si $f: I \rightarrow f(I)$ est continue et bijective, alors elle est stricement monotone.

    L'application $f: I \rightarrow f(I)$ est surjective par définition, il suffit de montrer que $f$ est injective.

    Par l'absurde on suppose qu'il existe $x_1, x_2 \in I$ vérifiant $f(x_1) = f(x_2)$ et $x_1 \neq x_2$.

    Ainsi $x_1 < x_2 \lor x_2 < x_1$ qui veut dire $\exists c \in ]x_1, x_2[, f'(c) = 0$ par le théorème de Rolle mais par hypothèse $f'$ ne s'annule pas sur $\stackrel{\circ}{I}$.
    Il y a donc contradiction, D'ou $f$ est injective.

    \begin{rdem}
        Donc $f$ est bijective.
    \end{rdem}
\end{demonstration}

\begin{demonstration}
    Si $f$ est constante, il est clair que $f$ est dérivable de dérivée nulle.

    Maintenant supposons que $f$ est de dérivée nulle. Par l'inégalité des accroissements finis avec $k = 0$,
    on a pour tout $x, y \in I, |f(x) - f(y)| \leq 0 \times |x - y| = 0$ et donc $f(x) = f(y)$.

    D'où $f$ est constante sur $I$.
\end{demonstration}

\begin{demonstration}
    Supposons que $f: I \rightarrow \R$ est k-lipschitzienne pour $k \geq 0$.

    \begin{itemize}
        \item Si $k = 0$ alors $f$ est constante donc continue.
        \item Supposons que $k \gt 0$.
        
        Soit $a \in I$, on va montrer que $f$ est continue en a:

        Soit $\varepsilon \gt 0$. En posant $\eta = \frac{\varepsilon}{k}$ on a pour tout $x \in I: |x - a| \lt \eta \implies |f(x) - f(a)| \leq k|x - a| \leq k \eta = k \frac{\varepsilon}{k} = \varepsilon$.
    \end{itemize}
\end{demonstration}

\end{document}
