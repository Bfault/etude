\documentclass[a4paper, 12pt]{article}
\usepackage{amsmath, amssymb, amsthm, stmaryrd}
\usepackage{geometry}
\usepackage{pgfplots}
\usepackage{tcolorbox}
\geometry{hmargin=2.5cm, vmargin=2.5cm}

\renewcommand*{\today}{23 janvier 2025}

\title{Analyse2 | CM: 2}
\author{Par Lorenzo}
\date{\today}

\newtheorem{theorem}{Théorème}[section]
\newtheorem{definition}{Définition}[section]
\newtheorem{example}{Example}[section]
\newtheorem{remark}{Remarques}[section]
\newtheorem{lemme}{Lemme}[section]
\newtheorem{corollaire}{Corollaire}[section]

\newtheorem{_proposition}{Proposition}[section]
\newenvironment{proposition}[1][]{
    \begin{_proposition}[#1]~\par
    \vspace*{0.5em}
}{%
    \end{_proposition}%
}

\newtheorem{_proprietes}{Propriétés}[section]
\newenvironment{proprietes}[1][]{
        \begin{_proprietes}[#1]~\par
        \vspace*{0.5em}
}{%
        \end{_proprietes}%
}

\newenvironment{rdem}[1][]{
    \begin{tcolorbox}[colframe=black, colback=white!10, sharp corners]
        #1
}{%
    \end{tcolorbox}
     
}

\newtheorem{_demonstration}{Démonstration}[section]
\newenvironment{demonstration}[1][]{
    \begin{_demonstration}[#1]~\par
    \vspace*{0.5em}
}{%
    \end{_demonstration}%
    \qed%
}

\newtheorem*{_demonstration*}{Démonstration}
\newenvironment{demonstration*}[1][]{
    \begin{_demonstration*}[#1]~\par
    \vspace*{0.5em}
}{%
    \end{_demonstration*}%
    \qed%
}

\newenvironment{ldefinition}{
    \begin{definition}~\par
    \vspace*{0.5em}
    \begin{enumerate}
}{
        \end{enumerate}
        \end{definition}
}

\newenvironment{lexample}{
    \begin{example}~\par
    \vspace*{0.5em}
    \begin{enumerate}
}{
        \end{enumerate}
        \end{example}
}

\newtheorem{_methode}{Méthode}[section]
\newenvironment{methode}{
    \begin{_methode}~\par
    \vspace*{0.5em}
}{
        \end{_methode}
}

\def\N{\mathbb{N}}
\def\Z{\mathbb{Z}}
\def\Q{\mathbb{Q}}
\def\R{\mathbb{R}}
\def\C{\mathbb{C}}
\def\K{\mathbb{K}}
\def\k{\Bbbk}

\def\un{(u_n)_{n \in \N}}
\def\xn#1{(#1_n)_{n \in \N}}

\def\o{\overline}
\def\eps{\varepsilon}

% \funcdef{name}{domain}{codomain}{variable}{expression}
% name: Name of the function (e.g. f)
% domain: Domain of the function (e.g. \mathbb{R})
% codomain: Codomain of the function (e.g. \mathbb{R})
% variable: Variables of the function (e.g. x)
% expression: Expression of the function (e.g. x^2)
\newcommand{\funcdef}[5]{%
    #1 :
    \begin{cases}
        #2 \rightarrow #3 \\
        #4 \mapsto #5
    \end{cases}
}

\newcommand{\lt}{\ensuremath <}
\newcommand{\gt}{\ensuremath >}

\begin{document}

\maketitle

\subsection{Quelques applications}

\subsubsection{Comparaison de moyennes}

Soit $n \in \N^*$ et $x_1, x_2, \ldots, x_n \gt 0$.

On définit leurs moyennes:

\begin{itemize}
    \item arithmétique : $m_a = \dfrac{1}{n} \sum_{i=1}^{n}x_i$
    \item harmonique : $m_h = \dfrac{n}{\sum_{i=1}^{n}\frac{1}{x_i}}$
    \item géométrique : $m_g = (\Pi_{i=1}^{n}x_i)^{\frac{1}{n}}$
    \item quadratique : $m_q = \sqrt{\dfrac{1}{n} \sum_{i=1}^{n}x_i^2}$
\end{itemize}

\begin{proprietes}
    On a les inégalités: $m_h \leq m_g \leq m_a \leq m_q$.
\end{proprietes}

\begin{demonstration}
    \begin{enumerate}
        \item Montrons que $m_h \leq m_g$.
        
            Comme la fonction $x \mapsto \ln x$ est concave sur $]0, +\infty[$ et par l'inégalité de Jensen, %mettre une ref
            on a
            \begin{flalign*}
                \ln \frac{1}{m_h} &= \ln(\frac{1}{n} \sum_{i=1}^{n}\frac{1}{x_i}) \\
                &\geq \frac{1}{n} \sum_{i=1}^{n} \ln \frac{1}{x_i}
            \end{flalign*}
            Ainsi
            \begin{flalign*}
                \ln(m_h) &\leq \frac{1}{n}\sum_{i=1}^{n}\ln x_i \\
                &= \frac{1}{n} \ln (\Pi_{i=1}^{n}x_i) \\
                &= \ln((\Pi_{i=1}^{n}x_i)^{\frac{1}{n}}) \\
                &= \ln(m_g)
            \end{flalign*}
        
        \item Montrons que $m_g \leq m_a$.
        
            Par la concavité de la fonction $x \mapsto \ln(x)$ et l'inégalité de Jensen, %mettre une ref
            on a

            \begin{flalign*}
                \ln(m_a) &= \ln(\frac{1}{n} \sum_{i=1}^{n}x_i) \\
                &\geq \frac{1}{n} \sum_{i=1}^{n} \ln x_i \\
                &= \ln(m_g)
            \end{flalign*}

            Par croissance de la fonction $x \mapsto \ln(x)$ on a $m_g \leq m_a$.
        
        \item Montrons que $m_a \leq m_q$
        
            Par la convexité de la fonction $x \mapsto x^2$ et l'inégalité de Jensen, %mettre une ref
            \begin{flalign*}
                m_a^2 &= (\frac{1}{n} \sum_{i=1}^{n}x_i)^2 \\
                &\leq \frac{1}{n} \sum_{i=1}^{n}x_i^2 \\
                &= m_q
            \end{flalign*}

            En prenant les racines carrées, on obtient $m_a \leq m_q$.
    \end{enumerate}
\end{demonstration}

\begin{proprietes}
    $\forall a \geq 1, \forall x_1, x_2 \gt 0, (x_1 + x_2)^\alpha \geq 2^{\alpha - 1}(x_1^{\alpha} + x_2^{\alpha})$.
\end{proprietes}

\begin{demonstration}
    Soit $\alpha \geq 1$. La fonction $\funcdef{}{]0, +\infty[}{\R}{x}{x^\alpha}$ est deux fois dérivable
    $f'(x) = \alpha x^{\alpha - 1}$ et $f''(x) = \alpha(\alpha - 1)x^{\alpha - 2}$ est positive sur $I = ]0, +\infty[$.

    Donc f est convexe et pour $\lambda = \frac{1}{2}$, on a pour tout $x_1, x_2 \gt 0$

    \begin{flalign*}
        (\frac{x_1 + x_2}{2})^\alpha &= f(\frac{1}{2} x_1 + \frac{1}{2} x_2) \\
        &\leq \frac{1}{2} f(x_1) + \frac{1}{2} f(x_2) \\
        &\leq \frac{1}{2} x_1^\alpha + \frac{1}{2} x_2^\alpha \\
        &= \frac{x_1^\alpha + x_2^\alpha}{2}
    \end{flalign*}

    Donc $(x_1 + x_2)^\alpha \leq 2^{1 - \alpha}(x_1^\alpha + x_2^\alpha)$.
\end{demonstration}

\subsubsection{Inégalité de Hölder}

\begin{proposition}[Inégalité de Young]
    Soient $p, q \in ]1, +\infty[$ tels que $\frac{1}{p} + \frac{1}{q} = 1$ et $a, b \gt 0$, on a
    $$
    ab \leq \frac{a^p}{p} + \frac{b^q}{q}
    $$
\end{proposition}

\begin{demonstration}
    À faire.
\end{demonstration}

\begin{proposition}[Inégalité de Hölder]
    Soient $p, q \in ]1, +\infty[$ tels que $\frac{1}{p} + \frac{1}{q} = 1$. Pour tout $n \in \N^*, x_1, x_2, \ldots, x_n, y_1, y_2, \ldots, y_n \in \R$, on a
    $$
    \sum_{i=1}^{n}|x_iy_i| \leq (\sum_{i=1}^{n}|x_i|^p)^{\frac{1}{p}}(\sum_{i=1}^{n}|y_i|^q)^{\frac{1}{q}}
    $$
\end{proposition}

\begin{demonstration}
    À faire.
\end{demonstration}

\section{Complément sur la dérivation}


\end{document}
