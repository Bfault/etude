\documentclass[a4paper, 12pt]{article}
\usepackage{amsmath, amssymb, amsthm, stmaryrd}
\usepackage{geometry}
\usepackage{pgfplots}
\usepackage{tcolorbox}
\geometry{hmargin=2.5cm, vmargin=2.5cm}

\renewcommand*{\today}{20 février 2025}

\title{Analyse2 | CM: 5}
\author{Par Lorenzo}
\date{\today}

\newtheorem{theorem}{Théorème}[section]
\newtheorem{definition}{Définition}[section]
\newtheorem{example}{Example}[section]
\newtheorem{remark}{Remarques}[section]
\newtheorem{lemme}{Lemme}[section]
\newtheorem{corollaire}{Corollaire}[section]

\newtheorem{_proposition}{Proposition}[section]
\newenvironment{proposition}[1][]{
    \begin{_proposition}[#1]~\par
    \vspace*{0.5em}
}{%
    \end{_proposition}%
}

\newtheorem{_proprietes}{Propriétés}[section]
\newenvironment{proprietes}[1][]{
        \begin{_proprietes}[#1]~\par
        \vspace*{0.5em}
}{%
        \end{_proprietes}%
}

\newenvironment{rdem}[1][]{
    \begin{tcolorbox}[colframe=black, colback=white!10, sharp corners]
        #1
}{%
    \end{tcolorbox}
     
}

\newtheorem{_demonstration}{Démonstration}[section]
\newenvironment{demonstration}[1][]{
    \begin{_demonstration}[#1]~\par
    \vspace*{0.5em}
}{%
    \end{_demonstration}%
    \qed%
}

\newtheorem*{_demonstration*}{Démonstration}
\newenvironment{demonstration*}[1][]{
    \begin{_demonstration*}[#1]~\par
    \vspace*{0.5em}
}{%
    \end{_demonstration*}%
    \qed%
}

\newenvironment{ldefinition}{
    \begin{definition}~\par
    \vspace*{0.5em}
    \begin{enumerate}
}{
        \end{enumerate}
        \end{definition}
}

\newenvironment{lexample}{
    \begin{example}~\par
    \vspace*{0.5em}
    \begin{enumerate}
}{
        \end{enumerate}
        \end{example}
}

\newtheorem{_methode}{Méthode}[section]
\newenvironment{methode}{
    \begin{_methode}~\par
    \vspace*{0.5em}
}{
        \end{_methode}
}

\def\N{\mathbb{N}}
\def\Z{\mathbb{Z}}
\def\Q{\mathbb{Q}}
\def\R{\mathbb{R}}
\def\C{\mathbb{C}}
\def\K{\mathbb{K}}
\def\k{\Bbbk}

\def\un{(u_n)_{n \in \N}}
\def\xn#1{(#1_n)_{n \in \N}}

\def\o{\overline}
\def\eps{\varepsilon}

% \funcdef{name}{domain}{codomain}{variable}{expression}
% name: Name of the function (e.g. f)
% domain: Domain of the function (e.g. \mathbb{R})
% codomain: Codomain of the function (e.g. \mathbb{R})
% variable: Variables of the function (e.g. x)
% expression: Expression of the function (e.g. x^2)
\newcommand{\funcdef}[5]{%
    #1 :
    \begin{cases}
        #2 \rightarrow #3 \\
        #4 \mapsto #5
    \end{cases}
}

\newcommand{\lt}{\ensuremath <}
\newcommand{\gt}{\ensuremath >}

\begin{document}

\maketitle

% propo 3.3.2

\begin{demonstration}
    Comme $g$ ne s'annule pas sur $(I \cap V)\backslash\{a\}$, par la proposition 3.1.1, on a:
    \begin{align*}
        f(x) \underset{x \to a}{\sim} g(x) &\iff f(x) - g(x) \underset{x \to a}{\sim} o(g(x)) \\
        &\iff \lim\limits_{x \to a}\frac{f(x) - g(x)}{g(x)} = 0 \\
        &\iff \lim\limits_{x \to a}\frac{f(x)}{g(x)} - 1 = 0 \\
        &\iff \lim\limits_{x \to a}\frac{f(x)}{g(x)} = 1
    \end{align*}
\end{demonstration}

\begin{example}
    \begin{enumerate}
        \item On veut calculer $\lim\limits_{x \to 0}\frac{\sin(x\ln(x))}{x}$\\
        Comme $\lim\limits_{x \to 0}x\ln(x) = 0$ et $\sin(x) \underset{x \to 0}{\sim} x$, on a:
        $\sin(x\ln(x)) \underset{x \to 0}{\sim} x\ln(x)$

        Ainsi 
        \begin{align*}
            \frac{\sin(x\ln(x))}{x} &\underset{x \to 0}{\sim} \ln(x)
        \end{align*}
        D'où:
        \begin{align*}
            \lim\limits_{x \to 0}\frac{\sin(x\ln(x))}{x} &= \lim\limits_{x \to 0}\ln(x) = -\infty
        \end{align*}
        
        \item On veut calculer $\lim\limits_{x \to 0}\frac{\ln(1 + 2\tan(x))}{\sin(x)}$\\
        Comme $\lim\limits_{x \to 0}\tan(x) = 0$, $\ln(1 + x) \underset{x \to 0}{\sim} x$, $\sin(x) \underset{x \to 0}{\sim} x$ et $\tan(x) \underset{x \to 0}{\sim} x$, on a:
        
        \begin{align*}
            \frac{\ln(1 + 2\tan(x))}{\sin(x)} &\underset{x \to 0}{\sim} \frac{2\tan(x)}{\sin(x)} \\
            &\underset{x \to 0}{\sim} \frac{2x}{x} \\
            &\underset{x \to 0}{\sim} 2
        \end{align*}
        D'où $\lim_{x \to 0}\dfrac{1 + 2\tan(x)}{\sin(x)} = 2$
    \end{enumerate}
\end{example}

\end{document}
