\documentclass[a4paper, 12pt]{article}
\usepackage{amsmath, amssymb, amsthm, stmaryrd}
\usepackage{geometry}
\usepackage{pgfplots}
\usepackage{tcolorbox}
\geometry{hmargin=2.5cm, vmargin=2.5cm}

\renewcommand*{\today}{13 février 2025}

\title{Analyse2 | CM: 4}
\author{Par Lorenzo}
\date{\today}

\newtheorem{theorem}{Théorème}[section]
\newtheorem{definition}{Définition}[section]
\newtheorem{example}{Example}[section]
\newtheorem{remark}{Remarques}[section]
\newtheorem{lemme}{Lemme}[section]
\newtheorem{corollaire}{Corollaire}[section]

\newtheorem{_proposition}{Proposition}[section]
\newenvironment{proposition}[1][]{
    \begin{_proposition}[#1]~\par
    \vspace*{0.5em}
}{%
    \end{_proposition}%
}

\newtheorem{_proprietes}{Propriétés}[section]
\newenvironment{proprietes}[1][]{
        \begin{_proprietes}[#1]~\par
        \vspace*{0.5em}
}{%
        \end{_proprietes}%
}

\newenvironment{rdem}[1][]{
    \begin{tcolorbox}[colframe=black, colback=white!10, sharp corners]
        #1
}{%
    \end{tcolorbox}
     
}

\newtheorem{_demonstration}{Démonstration}[section]
\newenvironment{demonstration}[1][]{
    \begin{_demonstration}[#1]~\par
    \vspace*{0.5em}
}{%
    \end{_demonstration}%
    \qed%
}

\newtheorem*{_demonstration*}{Démonstration}
\newenvironment{demonstration*}[1][]{
    \begin{_demonstration*}[#1]~\par
    \vspace*{0.5em}
}{%
    \end{_demonstration*}%
    \qed%
}

\newenvironment{ldefinition}{
    \begin{definition}~\par
    \vspace*{0.5em}
    \begin{enumerate}
}{
        \end{enumerate}
        \end{definition}
}

\newenvironment{lexample}{
    \begin{example}~\par
    \vspace*{0.5em}
    \begin{enumerate}
}{
        \end{enumerate}
        \end{example}
}

\newtheorem{_methode}{Méthode}[section]
\newenvironment{methode}{
    \begin{_methode}~\par
    \vspace*{0.5em}
}{
        \end{_methode}
}

\def\N{\mathbb{N}}
\def\Z{\mathbb{Z}}
\def\Q{\mathbb{Q}}
\def\R{\mathbb{R}}
\def\C{\mathbb{C}}
\def\K{\mathbb{K}}
\def\k{\Bbbk}

\def\un{(u_n)_{n \in \N}}
\def\xn#1{(#1_n)_{n \in \N}}

\def\o{\overline}
\def\eps{\varepsilon}

% \funcdef{name}{domain}{codomain}{variable}{expression}
% name: Name of the function (e.g. f)
% domain: Domain of the function (e.g. \mathbb{R})
% codomain: Codomain of the function (e.g. \mathbb{R})
% variable: Variables of the function (e.g. x)
% expression: Expression of the function (e.g. x^2)
\newcommand{\funcdef}[5]{%
    #1 :
    \begin{cases}
        #2 \rightarrow #3 \\
        #4 \mapsto #5
    \end{cases}
}

\newcommand{\lt}{\ensuremath <}
\newcommand{\gt}{\ensuremath >}

\begin{document}

\maketitle

\section{Relations de comparaisons de fonctions}

\dots

\begin{demonstration}
    \begin{itemize}
        \item Le resultat est immédiat par le fait qu'une fonction ayant une limite finie en a est bornée au voisinage de a.
        \item Soient V, W deux voisinages de a et $\varepsilon_0, B$ deux fonctions telles que:
        \begin{itemize}
            \item $\lim_{x \to a}\varepsilon_0(x) = 0$
            \item B est bornée sur V
        \end{itemize}
        Soient $f, g, h$ trois fonctions telles que:
        \begin{itemize}
            \item $f(x) = B(x)g(x)$ pour tout $x \in I \cap V$
            \item $g(x) = \varepsilon_0(x)h(x)$ pour tout $x \in I \cap W$
        \end{itemize}

        Ainsi pour tout $x \in I \cap (V \cap W)$ on a $f(x) = B(x)\varepsilon_0(x)h(x)$, d'où le résultat avec $\varepsilon = B \times \varepsilon_0 \underset{a}{\rightarrow} 0$.

        Finalement on a $f(x) = h(x)\varepsilon \iff f(x) \underset{x \to a}{=} o(h(x))$.
    \end{itemize}
\end{demonstration}

\dots

\begin{demonstration}
    On a $f(x) \underset{x \to a}{\sim} g(x) \iff f(x) \underset{x \to a}{=} g(x) + o(g(x))$

    i.e. il existe une fonction $\varepsilon$ définie sur un voisinage V de a tel que $\lim_{x \to a}\varepsilon(x) = 0$ et $f(x) = g(x) + \varepsilon(x)g(x) \forall x \in I \cap V$ ou encore $f(x) = (1 + \varepsilon(x))g(x) \forall x \in I \cap V$. Prendre $\theta = 1 + \varepsilon$.
\end{demonstration}

\end{document}
