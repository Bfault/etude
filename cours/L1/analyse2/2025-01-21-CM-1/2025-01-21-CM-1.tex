\documentclass[a4paper, 12pt]{article}
\usepackage{amsmath, amssymb, amsthm, stmaryrd, mathrsfs}
\usepackage{geometry}
\usepackage{pgfplots}
\usepackage{tcolorbox}
\geometry{hmargin=2.5cm, vmargin=2.5cm}

\renewcommand*{\today}{21 janvier 2025}

\title{Analyse2 | CM: 1}
\author{Par Lorenzo}
\date{\today}

\newtheorem{theorem}{Théorème}[section]
\newtheorem{definition}{Définition}[section]
\newtheorem{example}{Example}[section]
\newtheorem{remark}{Remarques}[section]
\newtheorem{lemme}{Lemme}[section]
\newtheorem{corollaire}{Corollaire}[section]

\newtheorem{_proposition}{Proposition}[section]
\newenvironment{proposition}[1][]{
    \begin{_proposition}[#1]~\par
    \vspace*{0.5em}
}{%
    \end{_proposition}%
}

\newtheorem{_proprietes}{Propriétés}[section]
\newenvironment{proprietes}[1][]{
        \begin{_proprietes}[#1]~\par
        \vspace*{0.5em}
}{%
        \end{_proprietes}%
}

\newenvironment{rdem}[1][]{
    \begin{tcolorbox}[colframe=black, colback=white!10, sharp corners]
        #1
}{%
    \end{tcolorbox}
     
}

\newtheorem{_demonstration}{Démonstration}[section]
\newenvironment{demonstration}[1][]{
    \begin{_demonstration}[#1]~\par
    \vspace*{0.5em}
}{%
    \end{_demonstration}%
    \qed%
}

\newtheorem*{_demonstration*}{Démonstration}
\newenvironment{demonstration*}[1][]{
    \begin{_demonstration*}[#1]~\par
    \vspace*{0.5em}
}{%
    \end{_demonstration*}%
    \qed%
}

\newenvironment{ldefinition}{
    \begin{definition}~\par
    \vspace*{0.5em}
    \begin{enumerate}
}{
        \end{enumerate}
        \end{definition}
}

\newenvironment{lexample}{
    \begin{example}~\par
    \vspace*{0.5em}
    \begin{enumerate}
}{
        \end{enumerate}
        \end{example}
}

\newtheorem{_methode}{Méthode}[section]
\newenvironment{methode}{
    \begin{_methode}~\par
    \vspace*{0.5em}
}{
        \end{_methode}
}

\def\N{\mathbb{N}}
\def\Z{\mathbb{Z}}
\def\Q{\mathbb{Q}}
\def\R{\mathbb{R}}
\def\C{\mathbb{C}}
\def\K{\mathbb{K}}
\def\k{\Bbbk}

\def\un{(u_n)_{n \in \N}}
\def\xn#1{(#1_n)_{n \in \N}}

\def\o{\overline}
\def\eps{\varepsilon}

% \funcdef{name}{domain}{codomain}{variable}{expression}
% name: Name of the function (e.g. f)
% domain: Domain of the function (e.g. \mathbb{R})
% codomain: Codomain of the function (e.g. \mathbb{R})
% variable: Variables of the function (e.g. x)
% expression: Expression of the function (e.g. x^2)
\newcommand{\funcdef}[5]{%
    #1 :
    \begin{cases}
        #2 \rightarrow #3 \\
        #4 \mapsto #5
    \end{cases}
}

\newcommand{\lt}{\ensuremath <}
\newcommand{\gt}{\ensuremath >}

\begin{document}

\maketitle

\section{Notion de convexité et application}

\begin{definition}.
    \begin{itemize}
        \item Soit $n \in \N^*$. On dit qu'une partie $A \subset \R^n$ est \textbf{convexe} si:
        $$
        \forall x, y \in A, \forall \lambda \in [0, 1], \lambda x + (1 - \lambda)y \in A
        $$
        \item On dit qu'une fonction $f: I \rightarrow \R$ est \textbf{convexe} si:
        $$
        \forall (x, y) \in I, \forall \lambda \in [0, 1], f(\lambda x + (1 - \lambda)y) \leq \lambda f(x) + (1 - \lambda)f(y)
        $$
        \item On dit qu'une fonction $f: I \rightarrow \R$ est \textbf{strictement convexe} si:
        $$
        \forall (x, y) \in I, x \neq y, \forall \lambda \in ]0, 1[, f(\lambda x + (1 - \lambda)y) < \lambda f(x) + (1 - \lambda)f(y)
        $$
        \item On dit qu'une fonction $f: I \rightarrow \R$ est \textbf{concave} si $-f$ est convexe.
        $$
        \forall (x, y) \in I, \forall \lambda \in [0, 1], f(\lambda x + (1 - \lambda)y) \geq \lambda f(x) + (1 - \lambda)f(y)
        $$
        \item On dit qu'une fonction $f: I \rightarrow \R$ est \textbf{strictement concave} si $-f$ est strictement convexe.
        $$
        \forall (x, y) \in I, x \neq y, \forall \lambda \in ]0, 1[, f(\lambda x + (1 - \lambda)y) > \lambda f(x) + (1 - \lambda)f(y)
        $$
    \end{itemize}
\end{definition}

\begin{remark}
    Les fonctions affines sont convexes et concaves.
\end{remark}

\begin{proposition}
    Une partie $A \subset \R$ est convexe si et seulement si $A$ est un intervalle.
\end{proposition}

\begin{demonstration}
    À faire.
\end{demonstration}

\begin{corollaire}
    \begin{itemize}
        \item Soit $x, y \in \R$, $x < y$. Alors $[x, y] = \{\lambda x + (1 - \lambda)y \in \R; \lambda \in [0, 1]\}$.
        \item Pour tout $n \in \N, n \geq 2, x_1, ..., x_n \in I$ et $\lambda_1, ..., \lambda_n \in \R_+$ tel que $\sum_{i=1}^{n}\lambda_i = 1$, alors $\sum_{i=1}^{n}\lambda_ix_i \in I$.
    \end{itemize}
\end{corollaire}

\begin{remark}
    $\sum_{i=1}^{n}\lambda_i x_i$ est appelé \textbf{une combinaison convexe} d'élements de $I$.
\end{remark}

\begin{demonstration}
    À faire.
\end{demonstration}

\begin{definition}
    Soit $f: I \rightarrow \R$ une fonction.
    \begin{itemize}
        \item On appelle \textbf{graphe} (ou \textbf{courbe représentative}) de la fonction f, l'ensemble de points:
        $$
        \mathscr{C}_f = \{(x, f(x)); x \in I\}
        $$
        \item L'équation $y = f(x)$ est appelée \text{équation cartésienne} du graphe de f.
        \item Soit $a, b \in I$ avec $a \lt b$. Soient A le point du plan de coordonnées $(a, f(a))$ et B celui de coordonnées $(b, f(b))$.
        La droite passant par A et B est appelée \textbf{corde} du graphe de f.
        La portion de $\mathscr{C}_f$ comprise entre A et B est appelée \textbf{arc} de f entre A et B.
    \end{itemize}
\end{definition}

\begin{proposition}
    Soient $f: I \rightarrow \R$ une fonction, $a, b \in I$ avec $a \lt b$.
    A le point de coordonnées $(a, f(a))$ et B celui de coordonnées $(b, f(b))$ et $\lambda \in [0, 1]$.

    \begin{enumerate}
        \item L'équation de la corde $[A, B]$ est:
        $$
        y = \frac{f(b) - f(a)}{b - a}(x - a) + f(a)
        $$
        \item Le point G du segment $[A, B]$ ayant pour abscisse $\lambda a + (1 - \lambda)b$ a pour ordonnée $\lambda f(a) + (1 - \lambda) f(b)$.
    \end{enumerate}
\end{proposition}

\begin{demonstration}
    À faire.
\end{demonstration}

\textbf{Interprétation graphique de la convexité.}
Soit $f: I \rightarrow \R$ une fonction convexe, de courbe représentative $\mathscr{C}_f$.

\begin{itemize}
    \item Soient $a, b \in I$ avec $a \lt b$ et $A$ le point de coordonnées $(a, f(a))$ et $B$ celui de coordonnées $(b, f(b))$.
    \item Soit G le point du segment [A, B] ayant pour abscisse $\lambda a + (1 - \lambda)b$ et pour ordonnée $\lambda f(a) + (1 - \lambda)f(b)$.
    \item Le point P de $\mathscr{C}_f$ d'abscisse $\lambda a + (1 - \lambda)b$ a pour ordonnée $f(\lambda a + (1 - \lambda)b)$.
\end{itemize}

Ainsi la définition de convexité de f exprime que le point G est situé au-dessus de P.

%mettre image si pas flemme

\subsection{Inégalité de Jensen}

\begin{proposition}
    Soit $f: I \rightarrow \R$ une fonction convexe, $n \in \N^*$, $x_1, ..., x_n \in I$ et $\lambda_1, ..., \lambda_n \in \R_+$ tels que $\sum_{i=1}^{n}\lambda_i = 1$.
    Alors:
    $$
    f(\sum_{i=1}^{n}\lambda_ix_i) \leq \sum_{i=1}^{n}\lambda_if(x_i)
    $$
\end{proposition}

\begin{demonstration}
    À faire.
\end{demonstration}

\subsection{Convexité et dérivabilité}

\subsubsection{Inégalité des pentes}

\begin{proposition}
    Soit $f: I \rightarrow \R$ une fonction définie sur I. f est convexe sur I si et seulement si pour tout $a \in I$, la fonction:
    $$
    \funcdef{\tau_a}{I\backslash\{a\}}{\R}{x}{\frac{f(x)-f(a)}{x - a}}
    $$
    est croissante.
\end{proposition}

\begin{demonstration}
    À faire.
\end{demonstration}

\begin{corollaire}
    Soit $f : I \rightarrow \R$ une fonction. f est convexe si et seulement si, pour tout $x_1, x_2, x_3 \in I$ avec $x_1 \lt x_2 \lt x_3$, on a:
    $$
    \frac{f(x_2) - f(x_1)}{x_2 - x_1} \leq \frac{f(x_3) - f(x_2)}{x_3 - x_2}
    $$
\end{corollaire}

%pour la suite
\begin{demonstration}
    Montrons ($\implies$)

        À faire.
    
    Montrons ($\impliedby$)

    Supposons aue $\forall x_1, x_2, x_3 \in I, x_1 \lt x_2 \lt x_3$.

    $$
    \frac{f(x_2) - f(x_1)}{x_2 - x_1} \leq \frac{f(x_3) - f(x_2)}{x_3 - x_2}
    $$

    
    Montrons que f est convexe, c'est à dire

    $\forall x, y \in I, \forall \lambda \in [0, 1], f(\lambda x + (1 - \lambda)y) \leq \lambda f(x) + (1 - \lambda) f(y)$

    Posons $a = \lambda x + (1 - \lambda) y \in [x, y]$ donc $x \lt a \lt y$.

    Par hypothèse, on a:

    $$
    \frac{f(a) - f(x)}{a - x} \leq \frac{f(y) - f(a)}{y - a}
    $$

    On a $a - x = \lambda x + (1 - \lambda)y - x = (1 - \lambda)(y - x)$
    et $y - a = y - \lambda x - (1 - \lambda)y = \lambda(y - x)$.

    On a alors:

    \begin{flalign*}
        \frac{f(a) - f(x)}{(1 - \lambda)(y - x)} \leq \frac{f(y) - f(a)}{\lambda(y - x)}
        &\implies \lambda(f(a) - f(x)) \leq (1 - \lambda)(f(y) - f(a)) \\
        &\implies f(a) \leq \lambda f(x) + (1 - \lambda)f(y) \\
        &\implies f(\lambda x + (1 - \lambda)y) \leq \lambda f(x) + (1 - \lambda)f(y)
    \end{flalign*}

    Donc f est convexe.
\end{demonstration}

\begin{corollaire}[Inégalité des trois pentes]
    Soit $f: I \rightarrow \R$ une fonction convexe. Alors pour tout $x_1, x_2, x_3 \in I$ avec $x_1 \lt x_2 \lt x_3$, on a:
    $$
    \frac{f(x_2) - f(x_1)}{x_2 - x_1} \leq \frac{f(x_3) - f(x_1)}{x_3 - x_1} \leq \frac{f(x_3) - f(x_2)}{x_3 - x_2}
    $$
\end{corollaire}

\begin{demonstration}
    À faire.
\end{demonstration}

%mettre image si pas flemme

\subsubsection{Caractérisation des fonctions convexes dérivables}

\begin{definition}
    Soit $f: I \rightarrow \R$ une fonction. On dit que $f$ est deux fois dérivable sur I, si $f$ est dérivable et $f'$ est aussi dérivable.
    On note alors $f''$ la dérivé de $f'$. $f''$ est appelée dérivée seconde de f.
\end{definition}

\begin{theorem}[Convexité des fonctions dérivables]
    Soit $f: I \rightarrow \R$ une fonction dérivable. f est convexe si et seulement si sa dérivée est croissante.

    En particulier une fonction deux fois dérivable est convexe si et seulement si sa dérivée seconde est positive sur I.
\end{theorem}

\begin{demonstration}
    À faire.
\end{demonstration}

\begin{definition}
    Soit $f: I \rightarrow \R$ un fonction. En un point où la dérivée seconde de f s'annule en changeant de signe, la courbe $\mathscr{C}_f$ change de concavité : on dit que c'est \textbf{un point d'inflexion}.
\end{definition}

\begin{remark}
    Grâce à ce théorème il est très rapide de déterminer la convexité d'une fonction dérivable deux fois.
\end{remark}

\subsubsection{Position par rapport à une tangente}

Soit $f: I \rightarrow \R$ une fonction. On suppose que f est dérivable en $a \in I$. Alors la courbe $\mathscr{C}_f$ possède une tangente au point de coordonnées $(a, f(a))$ d'équation $y = f'(a)(x - a) + f(a)$.

\begin{proposition}
    Soit $f: I \rightarrow \R$ une fonction dérivable sur I. Alors f est convexe si et seulement si elle est au-dessus de chacune de ses tangentes.
\end{proposition}

\begin{demonstration}
    À faire.
\end{demonstration}

\end{document}
