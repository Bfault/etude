\documentclass[a4paper, 12pt]{article}
\usepackage{amsmath, amssymb, amsthm, stmaryrd}
\usepackage{geometry}
\usepackage{pgfplots}
\usepackage{tcolorbox}
\geometry{hmargin=2.5cm, vmargin=2.5cm}

\renewcommand*{\today}{27 février 2025}

\title{Analyse2 | CM: 6}
\author{Par Lorenzo}
\date{\today}

\newtheorem{theorem}{Théorème}[section]
\newtheorem{definition}{Définition}[section]
\newtheorem{example}{Example}[section]
\newtheorem{remark}{Remarques}[section]
\newtheorem{lemme}{Lemme}[section]
\newtheorem{corollaire}{Corollaire}[section]

\newtheorem{_proposition}{Proposition}[section]
\newenvironment{proposition}[1][]{
    \begin{_proposition}[#1]~\par
    \vspace*{0.5em}
}{%
    \end{_proposition}%
}

\newtheorem{_proprietes}{Propriétés}[section]
\newenvironment{proprietes}[1][]{
        \begin{_proprietes}[#1]~\par
        \vspace*{0.5em}
}{%
        \end{_proprietes}%
}

\newenvironment{rdem}[1][]{
    \begin{tcolorbox}[colframe=black, colback=white!10, sharp corners]
        #1
}{%
    \end{tcolorbox}
     
}

\newtheorem{_demonstration}{Démonstration}[section]
\newenvironment{demonstration}[1][]{
    \begin{_demonstration}[#1]~\par
    \vspace*{0.5em}
}{%
    \end{_demonstration}%
    \qed%
}

\newtheorem*{_demonstration*}{Démonstration}
\newenvironment{demonstration*}[1][]{
    \begin{_demonstration*}[#1]~\par
    \vspace*{0.5em}
}{%
    \end{_demonstration*}%
    \qed%
}

\newenvironment{ldefinition}{
    \begin{definition}~\par
    \vspace*{0.5em}
    \begin{enumerate}
}{
        \end{enumerate}
        \end{definition}
}

\newenvironment{lexample}{
    \begin{example}~\par
    \vspace*{0.5em}
    \begin{enumerate}
}{
        \end{enumerate}
        \end{example}
}

\newtheorem{_methode}{Méthode}[section]
\newenvironment{methode}{
    \begin{_methode}~\par
    \vspace*{0.5em}
}{
        \end{_methode}
}

\def\N{\mathbb{N}}
\def\Z{\mathbb{Z}}
\def\Q{\mathbb{Q}}
\def\R{\mathbb{R}}
\def\C{\mathbb{C}}
\def\K{\mathbb{K}}
\def\k{\Bbbk}

\def\un{(u_n)_{n \in \N}}
\def\xn#1{(#1_n)_{n \in \N}}

\def\o{\overline}
\def\eps{\varepsilon}

% \funcdef{name}{domain}{codomain}{variable}{expression}
% name: Name of the function (e.g. f)
% domain: Domain of the function (e.g. \mathbb{R})
% codomain: Codomain of the function (e.g. \mathbb{R})
% variable: Variables of the function (e.g. x)
% expression: Expression of the function (e.g. x^2)
\newcommand{\funcdef}[5]{%
    #1 :
    \begin{cases}
        #2 \rightarrow #3 \\
        #4 \mapsto #5
    \end{cases}
}

\newcommand{\lt}{\ensuremath <}
\newcommand{\gt}{\ensuremath >}

\begin{document}

\maketitle

\begin{proposition}[Unicité d'un développement limité]
    Soient $n \in \N$, $f$ une fonction définie sur $I$ et $a \in I$.

    Si $f$ admet un $DL_n(a)$, alors ce DL est unique. Autrement dit:

    \begin{align*}
        f(x) = \sum_{k=0}^{n} c_k(x-a)^k + o((x-a)^n) = \sum_{k=0}^{n} b_k(x-a)^k + o((x-a)^n) \implies \forall k \in \llbracket 0, n \rrbracket, a_k = b_k
    \end{align*}
\end{proposition}

\begin{demonstration}
    Par l'absurde, supposons que $a_k \neq b_k$ pour un certain $k \in \llbracket 0, n \rrbracket$. Soit $k$ le plus petit tel que $a_k \neq b_k$.

    On a alors:

    \begin{align*}
        f(x) &= \sum_{k=0}^{n} a_k(x-a)^k + o((x-a)^n) \\
        &= \sum_{k=0}^{n} b_k(x-a)^k + o((x-a)^n)
    \end{align*}

    En prenant le plus petit $k_0$ dans $\llbracket 0, n \rrbracket$ tel que $a_{k_0} \neq b_{k_0}$, on a:

    \begin{align*}
        f(x) &= c_{k_0}(x-a)^k_0 + o((x-a)^{k_0}) \\
        &= b_{k_0}(x-a)^k_0 + o((x-a)^{k_0}) \\
        &\implies (c_{k_0} - b_{k_0})(x - a)^{k_0} \underset{x \to a}{=} o((x - a)^{k_0})
    \end{align*}

    Ainsi on a $c_{k_0} - b_{k_0} \underset{x \to a}{=} o(1) \implies c_{k_0} - b_{k_0} = 0 \implies a_{k_0} = b_{k_0}$, ce qui est absurde.

\end{demonstration}

\begin{proposition}
\end{proposition}

\begin{demonstration}
    Soit $f$ une fonction paire admettant un $DL_n(0)$ tel qu'au voisinage de 0,

    $f(x) = c_0 + c_1x + \ldots + c_nx^n + o(x^n)$ alors au voisinage de 0,

    $f(-x) = c_0 - c_1x + \ldots + (-1)^nc_nx^n + o(x^n)$

    Ainsi par la parité de $f$ et l'unicité d'un DL on peut identifier les coefficients correspondants:

    $c_1 = -c1, c_3 = -c_3, \ldots, c_n = (-1)^nc_n$ ou encore $c_1 = c_3 = \ldots = c_n = 0$.

    Ainsi tous les coefficients impairs du DL de $f$ sont nuls. D'où la partie principale est paire.

    Faire le même raisonnement pour une fonction impaire.
\end{demonstration}

\end{document}
