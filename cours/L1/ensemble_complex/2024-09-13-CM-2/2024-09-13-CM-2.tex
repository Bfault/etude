\documentclass[a4paper, 12pt]{article}
\usepackage{amsmath, amssymb, amsthm}
\usepackage{geometry}
\usepackage{tcolorbox}
\geometry{hmargin=2.5cm, vmargin=2.5cm}

\renewcommand*{\today}{13 septembre 2024}

\title{Ensemble Complex | CM: 2}
\author{Par Lorenzo}
\date{\today}

\newtheorem{theorem}{Théorème}[section]
\newtheorem{definition}{Définition}[section]
\newtheorem{example}{Example}[section]
\newtheorem{remark}{Remarques}[section]
\newtheorem{lemme}{Lemme}[section]
\newtheorem{corollaire}{Corollaire}[section]

\newtheorem{_proposition}{Proposition}[section]
\newenvironment{proposition}[1][]{
    \begin{_proposition}[#1]~\par
    \vspace*{0.5em}
}{%
    \end{_proposition}%
}

\newtheorem{_proprietes}{Propriétés}[section]
\newenvironment{proprietes}[1][]{
        \begin{_proprietes}[#1]~\par
        \vspace*{0.5em}
}{%
        \end{_proprietes}%
}

\newenvironment{rdem}[1][]{
    \begin{tcolorbox}[colframe=black, colback=white!10, sharp corners]
        #1
}{%
    \end{tcolorbox}
     
}

\newtheorem{_demonstration}{Démonstration}[section]
\newenvironment{demonstration}[1][]{
    \begin{_demonstration}[#1]~\par
    \vspace*{0.5em}
}{%
    \end{_demonstration}%
    \qed%
}

\newtheorem*{_demonstration*}{Démonstration}
\newenvironment{demonstration*}[1][]{
    \begin{_demonstration*}[#1]~\par
    \vspace*{0.5em}
}{%
    \end{_demonstration*}%
    \qed%
}

\newenvironment{ldefinition}{
    \begin{definition}~\par
    \vspace*{0.5em}
    \begin{enumerate}
}{
        \end{enumerate}
        \end{definition}
}

\newenvironment{lexample}{
    \begin{example}~\par
    \vspace*{0.5em}
    \begin{enumerate}
}{
        \end{enumerate}
        \end{example}
}

\newtheorem{_methode}{Méthode}[section]
\newenvironment{methode}{
    \begin{_methode}~\par
    \vspace*{0.5em}
}{
        \end{_methode}
}

\def\N{\mathbb{N}}
\def\Z{\mathbb{Z}}
\def\Q{\mathbb{Q}}
\def\R{\mathbb{R}}
\def\C{\mathbb{C}}
\def\K{\mathbb{K}}
\def\k{\Bbbk}

\def\un{(u_n)_{n \in \N}}
\def\xn#1{(#1_n)_{n \in \N}}

\def\o{\overline}
\def\eps{\varepsilon}

% \funcdef{name}{domain}{codomain}{variable}{expression}
% name: Name of the function (e.g. f)
% domain: Domain of the function (e.g. \mathbb{R})
% codomain: Codomain of the function (e.g. \mathbb{R})
% variable: Variables of the function (e.g. x)
% expression: Expression of the function (e.g. x^2)
\newcommand{\funcdef}[5]{%
    #1 :
    \begin{cases}
        #2 \rightarrow #3 \\
        #4 \mapsto #5
    \end{cases}
}

\newcommand{\lt}{\ensuremath <}
\newcommand{\gt}{\ensuremath >}

\begin{document}

\maketitle

\section{Méthodes de démonstration 1}

\subsection{Implication}

Pour démontrer un énoncé du type $P \implies Q$ (Si P alors Q)

\begin{methode}
    On suppose P.
    
    raisonnement profond.

    on en conclut Q.
\end{methode}


\begin{example}
    \begin{align*}
        \text{Démontrons que }x \in \mathbb{R} \implies x + 1 \in \mathbb{R}_+
    \end{align*}
    On suppose $x \in \mathbb{R}$
    \begin{flalign*}
        &x \geq 0&& \\
        &x + 1 \geq 0 + 1 \geq 0&&
    \end{flalign*}
    Donc $x + 1 \in \mathbb{R}_+$
\end{example}

\begin{remark}
    $A \subset B$ ($A, B \subset E$) par définition se traduit par $\forall x \in E, x \in A \implies x \in B$
\end{remark}

\subsection{Règle d'inférence ("Déduction naturelle")}

Supposons A, B deux formules logiques dépendant d'énoncés élémentaires P, Q, R, ...
Imaginons que pour chaque ligne de la table de vérité où A est vrai, B l'est également.
Ainsi lorsqu'on a A on pourra déduire B

\begin{example}
    Supposons $P \implies P \lor Q$

    Quand P est vrai $P \lor Q$ est vrai.
\end{example}

\subsection{Disjonction de cas}

De même une tautologie est une règle qui est toujours vraie. Un exemple ($P \lor \neg Q$).

\begin{example}
    Théorème si $x \in \mathbb{N}$ alors $x(x + 1)$ pair.

    \vspace{1em}

    Si x est pair alors $x = 2k, k \in \mathbb{N}$.

    $x(x + 1) = 2(k(x + 1))$ est pair.

    \vspace{1em}

    Si x est impair $x + 1$ est pair implique $\exists k \in \mathbb{N}, (x + 1) = 2k$.

    $x(x + 1) = x + 2k = 2(kx)$ implique $x(x + 1)$ pair.

    \vspace{1em}

    Conclusion: Dans tous les cas $x(x + 1)$ est pair.
    
    C'est un raisonnement par disjonction de cas.
\end{example}

\begin{methode}
    Plus généralement si $A \lor B \lor C \lor ...$ est une tautologie alors la méthode de la démonstration

    \vspace{0.5em}

    \item \textbf{1) Suppose A}
    \item \textbf{raisonnement profond}
    \item \textbf{conclusion}
    \item \textbf{2) Suppose B}
    \item \textbf{raisonnement profond}
    \item \textbf{même conclusion}
    \item \textbf{3) \ldots}
    \item \textbf{Donc la conclusion est vrai dans tous les cas}
\end{methode}

\begin{remark}
    Pour trouver des synonyme en regardant toute les valeurs dans une table de vérités, on utilise une disjonction de cas.
\end{remark}

\begin{example}
    Soit un entier n, n est impair ou $n^2$ est un multiple de 4.

    \vspace{1em}

    Supposons que n ne soit pas impair.

    Alors n est pair, donc il existe $k \in \Z$ tel que $n = 2k$

    Ainsi $n^2 = (2k)^2 = 4k^2$.

    $n^2$ est un multiple de 4.
\end{example}

\subsection{La double implication}

Si vous souhaitez montrer que $P \iff Q$, il suffit de montrer $P \implies Q$ et $Q \implies P$.

\begin{example}
    $x \in \N \iff x + 1 \in \N^*$

    1) Supposons que $x \in \N$, donc $x \geq 0 \implies x + 1 \geq 1$

    Ainsi $x + 1 \in \N^*$

    2) Supposons que $x + 1 \in \N^*$, donc $x + 1 \geq 1 \implies x + 1 - 1 \geq 1 - 1 \implies x \geq 0$.

        La soustraction est stable sur $\Z$, de plus $x \geq 0$ donc $x \in \Z_+ \equiv \N$
\end{example}

\begin{remark}
    Régulièrement utilisé pour montre que deux ensembles sont égaux, on montre $A \subset B$ et $B \subset A$.
    Cela s'appelle la \textbf{double inclusion}.
\end{remark}

\subsection{Raisonnement par contraposée}

$P \implies Q$ est synonyme à $\neg Q \implies \neg P$, ça se prouve facilement en comparant leurs tables de vérités.

\begin{example}
    Soit $n \in \N$, montrer que si $n^2$ alors n est pair.

    \vspace{0.5em}

    Supposons que n est impair, alors il existe $k \in \Z, n = 2k + 1$

    Ainsi $n^2 = (2k + 1)^2 = 4k^2 + 4k + 1 = 2(2k^2 + 2k) + 1$ est impair.

    n impair implique $n^2$ impair donc $n^2$ pair implque n pair (la proposition de base).
\end{example}

\subsection{Raisonnement par l'absurde}

Pour montrer que P est vrai, on peut supposer P faux et arriver à une contradiction.

\begin{example}
    Montrons que $\sqrt{2}$ est irrationnel.

    \begin{demonstration*}
        On suppose que $\sqrt{2}$ est rationnel, i.e. on peut écrire $\sqrt{2} = \dfrac{a}{b}$ avec $a \in \Z, b \in \N$ (et $pgcd(a, b) = 1$).

        $\sqrt{2} = \dfrac{a}{b} \implies 2 = \dfrac{a^2}{b^2} \implies 2b^2 = a^2$

        Ainsi $a^2$ est pair et avec le raisonnement précédent a est aussi pair (donc $a = 2k$ avec $k \in \Z$).

        $2b^2 = (2k)^2 \implies 2b^2 = 4k^2 \implies b^2 = 2k^2$

        Ainsi $b^2$ est pair et b aussi.

        \begin{rdem}
            Absurde a et b sont premiers entre eux, ils ne peuvent pas être tous les deux multiple de 2.
        \end{rdem}
    \end{demonstration*}
\end{example}

\end{document}
