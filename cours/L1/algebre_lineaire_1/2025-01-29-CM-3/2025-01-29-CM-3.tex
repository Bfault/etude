\documentclass[a4paper, 12pt]{article}

\usepackage{utils}

\renewcommand*{\today}{29 janvier 2025}

\begin{document}

\hotbox{Algèbre linéaire 1}{CM 3}{\today}

\begin{definition}
    Une matrice $A \in \mathcal{M}_{n,p}(\K)$ est dite sous forme échelonnée si:
    \begin{enumerate}
        \item toutes ses lignes non identiquement nulles sont situées au dessus de ses lignes identiquement nulles.
        \item chaque élément de tête d'une ligne (élément non nul le plus ç gauche d'une ligne non identiquement nulle) se trouve dans une colonne à droite de l'élément de tête de la ligne précédente.
    \end{enumerate}
\end{definition}

\begin{remarque}
    La condition (2) implique que tous les éléments en dessous d'un élément de tête sont nuls.
\end{remarque}

\begin{definition}[Échelonnement d'une matrice]
    Soit $A \in \mathcal{M}_{n,p}(\K)$. Il existe une matrice $E \in \mathcal{M}_{n}(\K)$, produit de matrice élémentaires, telle que la matrice $E \times A$ est échelonnée.
    Autrement dit, toute matrice est équivalente par rapport aux lignes à une matrice échelonnée.
\end{definition}

\begin{lemme}
    Soit $A \in \mathcal{M}_{n,1}(\K), n \geq 2$ et $i \in \{1, \cdots, n-1\}$, tels que l'un des coefficients $a_j$ pour indice $j \in \{i, \cdots, n\}$ soit non nul, alors
    il existe $E_A \in \mathcal{M}_n(\K)$ produit de matrice élémentaires.
    $a \in \K, a \neq 0$ tel que si:
    $$
    A = \left(
        \begin{array}{c}
            a_1 \\
            \vdots \\
            a_{i-1} \\
            a_i \\
            a_{i+1} \\
            \vdots \\
            a_n
        \end{array}
    \right)
    $$
    alors 
    $$
    E_A \times A = \left(
        \begin{array}{c}
            a_1 \\
            \vdots \\
            a_{i-1} \\
            a_i \\
            0 \\
            \vdots \\
            0
        \end{array}
    \right)
    $$

    %TODO
\end{lemme}

\begin{demonstration}
    Quitte à échanger la ligne i avec une ligne j en dessius (ce qui revient à faire le produit $E_{i,j} \times A$),
    on se ramène au cas $a = a_i \neq 0$, on fait les opérations:
    $I_j \leftarrow I_j - \frac{a_j}{a_i}I_i, j \in \{i + 1, \cdots, n\}$.

    %TODO
\end{demonstration}

%TODO jusqu'au rang

\end{document}
