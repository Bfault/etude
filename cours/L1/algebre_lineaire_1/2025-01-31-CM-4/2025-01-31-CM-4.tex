\documentclass[a4paper, 12pt]{article}

\usepackage{utils}

\renewcommand*{\today}{31 janvier 2025}

\begin{document}

\hotbox{Algèbre linéaire 1}{CM 4}{\today}

\begin{theoreme}
    Soit $A \in \mathcal{M}_n(\K)$. Les conditions suivantes sont équivalentes:
    \begin{enumerate}
        \item $A$ est inversible.
        \item Pour toute matrice colonne $B \in \mathcal{M}_{n, 1}(\K)$, l'équation $AX = B$ admet une unique solution.
        \item Le système $AX = 0_{n,1}$ possède comme unique solution $0_{n,1}$ dans $\mathcal{M}_{n, 1}(\K)$.
        \item La forme échelonée réduite de $A$ est $Id_n$
        \item $A$ est un produit de matrices élémentaires.
    \end{enumerate}
\end{theoreme}

%TODO

\subsection{Système linéaire: resultats généraux}

\begin{definition}
    Dans un système linéaire échelonée, les variables qui correspondent au colonne de la matrice où il y a des positions pivots sont dites
    liées ou principales, les autres variables sont dites libres ou secondaires.
\end{definition}

%TODO

\end{document}
