\documentclass[a4paper, 12pt]{article}

\usepackage{utils}

\renewcommand*{\today}{17 janvier 2025}

\begin{document}

\hotbox{Algèbre linéaire 1}{CM 1}{\today}

\section{Matrices}

\begin{definition}
    Soit $\K$ un corps. Soit $n, p \in \N^*$.
    Une matrice de taille $n \times p$ à coefficients dans $\K$ est une famille
    d'éléments de $\K$ indexée par $I = \{1, ..., n\} \times \{1, ..., p\}$.
    On la représente par un tableau rectangulaire :
    $$
        A = \begin{pmatrix}
            a_{1, 1} & \cdots & a_{1, p} \\
            \vdots & \ddots & \vdots \\
            a_{n, 1} & \cdots & a_{n, p}
        \end{pmatrix}
    $$
\end{definition}

\begin{definition}
    On note $\mathcal{M}_{n, p}(\K)$ l'ensemble des matrices de taille $n \times p$ à coefficients dans $\K$.
    ou encore $\mathcal{M}_{n}(\K)$ si $n = p$.
\end{definition}

\begin{definition}
    Une matrice A de $\mathcal{M}_{n}(\K)$ est dite carrée d'ordre $n$.
    \begin{enumerate}
        \item les termes $a_{i, i}$ sont les éléments diagonaux de A.
        \item A est dite diagonale si $a_{i, j} = 0$ pour $i \neq j$.
        \item A est dite triangulaire supérieure si $a_{i, j} = 0$ pour $i > j$.
        \item A est dite triangulaire inférieure si $a_{i, j} = 0$ pour $i < j$.
    \end{enumerate}
\end{definition}

\begin{definition}
    La matrice nulle de taille $n \times p$ est la matrice dont tous les coefficients sont nuls.
\end{definition}

\begin{definition}
    La matrice identité de taille $n \times n$ est la matrice diagonale dont les éléments diagonaux sont tous égaux à 1.
    $$
        Id_n = \begin{pmatrix}
            1 & 0 & \cdots & 0 \\
            0 & 1 & \cdots & 0 \\
            \vdots & \vdots & \ddots & \vdots \\
            0 & 0 & \cdots & 1
        \end{pmatrix}
    $$
\end{definition}

\begin{definition}
    La matrice transposée de A est la matrice obtenue en échangeant les lignes et les colonnes de A.
    $$
        A = \begin{pmatrix}
            a_{1, 1} & \cdots & a_{1, p} \\
            \vdots & \ddots & \vdots \\
            a_{n, 1} & \cdots & a_{n, p}
        \end{pmatrix}
        \quad \text{et} \quad
        ^tA = \begin{pmatrix}
            a_{1, 1} & \cdots & a_{n, 1} \\
            \vdots & \ddots & \vdots \\
            a_{1, p} & \cdots & a_{n, p}
        \end{pmatrix}
    $$
    Notée aussi $A^t$ ou $^tA$.
\end{definition}

\begin{definition}
    Une matrice $\mathcal{M}_{n}(\K)$ est dite symétrique si $A = ^tA$.
\end{definition}

\begin{remarque}
    Soit $A \in \mathcal{M}_{n, p}(\K)$.
    \begin{enumerate}
        \item $^t(^tA) = A$.
        \item $^t(A + B) = ^tA + ^tB$.
        \item $^t(\lambda A) = \lambda ^tA$.
        \item $^t(AB) = ^tB \cdot ^tA$.
    \end{enumerate}
\end{remarque}

\begin{definition} On a les opérations
    \begin{enumerate}
        \item 
        Deux matrice $A, B \in \mathcal{M}_{n, p}(\K)$ sont égales si elles ont les mêmes coefficients.
        $A = B \iff \forall i \in \{1, ..., n\}, \forall j \in \{1, ..., p\}, a_{i, j} = b_{i, j}$
        \item Si $\lambda \in \K$, on note $\lambda A$ la matrice obtenue en multipliant chaque coefficient de A par $\lambda$.
        \item Si $A, B \in \mathcal{M}_{n, p}(\K)$, on note $A + B$ la matrice obtenue en additionnant les coefficients correspondants.
        $A + B = (a_{i, j} + b_{i, j})$
    \end{enumerate}
\end{definition}

\begin{propriete}{}{}
    Soit $\lambda, \mu \in \K$ et $A, B, C \in \mathcal{M}_{n, p}(\K)$.
    \begin{enumerate}
        \item A + (B + C) = (A + B) + C
        \item A + B = B + A
        \item A + 0 = A
        \item A + (-A) = 0
        \item $\lambda(A + B) = \lambda A + \lambda B$ et $(\lambda + \mu)A = \lambda A + \mu A$
        \item $\lambda(\mu A) = (\lambda \mu)A$
    \end{enumerate}
\end{propriete}

\begin{definition}
    Soit $m$ matrices $A_1, ..., A_m \in \mathcal{M}_{n, p}(\K)$,
    soit $\lambda_1, ..., \lambda_m \in \K$.
    On appelle combinaison linéaire de $A_1, ..., A_m$ pondérée par $\lambda_1, ..., \lambda_m$ la matrice
    $\lambda_1 A_1 + ... + \lambda_m A_m$.
\end{definition}

\begin{definition}
    Soit $A, B \in \mathcal{M}_{n, p}(\K)$.
    Le produit de A par B est la matrice $C \in \mathcal{M}_{n, p}(\K)$ définie par
    $$
        c_{i, j} = \sum_{k = 1}^{p} a_{i, k} b_{k, j}
    $$
\end{definition}

\begin{remarque}
    On peut utilise l'aide mémoire suivante pour se rappeler de la formule du produit de deux matrices.
    $$
    \begin{array}{ccc}
        & \begin{matrix}
            b_{1, 1} & \cdots & b_{1, q} \\
            \vdots & \ddots & \vdots \\
            b_{p, 1} & \cdots & b_{p, q}
        \end{matrix} & \\
        \begin{matrix}
            a_{1, 1} & \cdots & a_{1, p} \\
            \vdots & \ddots & \vdots \\
            a_{n, 1} & \cdots & a_{n, p}
        \end{matrix} &
        \begin{matrix}
            c_{1, 1} & \cdots & c_{1, q} \\
            \vdots & \ddots & \vdots \\
            c_{n, 1} & \cdots & c_{n, q}
        \end{matrix} &
    \end{array}
    $$
\end{remarque}

\begin{definition}
    On appelle sous-matrice de A la matrice obtenue en supprimant une ou plusieurs lignes et/ou colonnes de A.
    Ainsi on peut décomposer A en blocs.
    Soit en lignes :
    $$
        A = \begin{pmatrix}
            A_1 \\
            \vdots \\
            A_p
        \end{pmatrix}
    $$
    Soit en colonnes :
    $$
        A = \begin{pmatrix}
            A_1 & \cdots & A_p
        \end{pmatrix}
    $$
\end{definition}

\begin{propriete}{}{}
    Pour toute matrice $A \in \mathcal{M}_{n, p}(\K)$, on a
    $A \times Id_p = A$ et $Id_n \times A = A$.
\end{propriete}

\begin{remarque}
    On voit que l'identité n'est pas unique si $n \neq p$ (la matrice n'est pas carrée).
\end{remarque}

\begin{propriete}{}{}
    Soit $\lambda \in \K$ et $A, B \in \mathcal{M}_{n, p}(\K)$.
    \begin{enumerate}
        \item \lambda(AB) = (\lambda A)B = A(\lambda B)
        \item $A(B + C) = AB + AC$ et $(A + B)C = AC + BC$
        \item $AB \neq BA$ en général
        \item $A(BC) = (AB)C$
    \end{enumerate}
\end{propriete}

\begin{remarque}
    \begin{itemize}.
        \item Le produit de deux matrices n'est pas commutatif.
        \item On peut multiplier deux matrices non nulls et obtenir une matrice nulle.
    \end{itemize}
    Ainsi on dit que l'ensemble $\mathcal{M}(\K)$ possède des diviseurs de zéro.
\end{remarque}

\end{document}
