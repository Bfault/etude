\documentclass[a4paper, 12pt]{article}

\usepackage{utils}

\renewcommand*{\today}{07 février 2025}

\begin{document}

\hotbox{Algèbre linéaire 1}{CM 5}{\today}

\section{Espaces vectoriels}

\subsection{Espace vectoriel}

\begin{definition}
    Étant donné deux ensembles $E$ et $\K$, toute application de $\K \times E$ dans $E$ s'appelle loi de composition externe sur $E$ (à domaine opérateur $\K$).
\end{definition}

\begin{definition}
    On dit qu'un ensemble $E$ est un espace vectoriel sur un corps $\K$ s'il est muni d'une loi interne notée $+$
    et d'une loi externe notée $\bullet$ de $\K \times E$ dans $E$.
    
    $(\lambda, u) \mapsto \lambda \cdot u$ telles que:
    
    \begin{enumerate}
        \item $(E, +)$ est un groupe commutatif.
        \item $\forall (\lambda, \mu) \in \K^2, \forall (u, v) \in E^2$, on a:
        \begin{enumerate}
            \item $(\lambda + \mu) \bullet u = \lambda \bullet u + \mu u$
            \item $\lambda \bullet (u + v) = \lambda \bullet u + \lambda \bullet v$
            \item $\lambda \bullet (\mu \bullet u) = (\lambda \mu) \bullet u$
            \item $1 \bullet u = u$ ($1 \in \K$)
        \end{enumerate}
    \end{enumerate}

    Les éléments de $E$ sont appelés vecteurs et les éléments de $\K$ sont appelés scalaires.
    E est $\K$-espace vectoriel.
\end{definition}

\begin{definition}
    \begin{enumerate}
        \item La commutativé de $(E, +)$ découle des autres axiomes d'espace vectoriel.
        \item L'espace vectoriel nul est $E = \{0_e\}$.
    \end{enumerate}
\end{definition}

\end{document}
