\documentclass[a4paper, 12pt]{article}

\usepackage{utils}

\renewcommand*{\today}{24 janvier 2025}

\begin{document}

\hotbox{Algèbre linéaire 1}{CM 2}{\today}

\subsection{Partitionnement par blocs}

\begin{definition}
    Les sous matrices permettent un partitionnement par blocs.
    Soit $A \in \mathcal{M}_{n,p}(\K)$, on peut écrire $A$ sous la forme:
    $$
    A = \left(
    \begin{array}{c|c|c}
        A_{1,1} & \cdots & A_{1,m} \\
        \hline
        \vdots & \ddots & \vdots \\
        \hline
        A_{r,1} & \cdots & A_{r,m}
    \end{array}
    \right)
    $$
    où pour chaque $i \in \{1, \cdots r\}$, les matrice $A_{ik}$ ont le même nombre de lignes pour tout $k \in \{1, \cdots r\}$,
    et où pour chaque $j \in \{1, \cdots, m\}$, les matrices $A_{kj}$ ont le même nombre de colonnes pour tout $k \in \{1, \cdots r\}$.
\end{definition}

\begin{definition}
    On peut faire un \textbf{produit par blocs}.

    Soit
    $$
    A = \left(
    \begin{array}{c|c|c}
        A_{1,1} & \cdots & A_{1,m} \\
        \hline
        \vdots & \ddots & \vdots \\
        \hline
        A_{r,1} & \cdots & A_{r,m}
    \end{array}
    \right)
    \in \mathcal{M}_{n,p}(\K)
    $$
    et
    $$
    B = \left(
    \begin{array}{c|c|c}
        B_{1,1} & \cdots & B_{1,s} \\
        \hline
        \vdots & \ddots & \vdots \\
        \hline
        B_{m,1} & \cdots & B_{m,s}
    \end{array}
    \right)
    \in \mathcal{M}_{p,q}(\K)
    $$
    Avec le nombre de colonnes de $A_{i,k}$ égal au nombre de lignes de $B_{k,j}$ pour tout
    $i \in \{1, \cdots, r\}, j \in \{1, \cdots, s\}, k \in \{1, \cdots, m\}$ Alors 
    $$
    AB = \left(
    \begin{array}{c|c|c}
        \sum_{k=1}^{m}A_{1,k}B_{k,1} & \cdots & \sum_{k=1}^{m}A_{1,k}B_{k,s} \\
        \hline
        \vdots & \ddots & \vdots \\
        \hline
        \sum_{k=1}^{m}A_{r,k}B_{k,1} & \cdots & \sum_{k=1}^{m}A_{r,k}B_{k,s}
    \end{array}
    \right)
    $$
\end{definition}

\begin{remarque}
    Avec $A \in \mathcal{M}_{n,p}(\K)$ partitionnée par lignes et $B \in \mathcal{M}_{p,q}(\K)$ partitionnée par colonnes, alors on a:

    $r = n, m = 1$ et $s = q$.
    ainsi $AB = (c_{i,j})$ avec $c_{i,j} = A_{i,1}B_{1,j} = l_i(A)c_j(B)$ pour $i \in \{1, \cdots, n\}$ et $j \in \{1, \cdots, q\}$.
\end{remarque}

\subsection{Inverse et puissance}

\begin{definition}
    Soit $A \in \mathcal{M}_n(\K)$, on note $A^0 = Id_n$ et pour $k \in \N^*$, $A^k$ est défini par récurrence par : $A^k = AA^{k-1}$
\end{definition}

\begin{propriete}{}{}
    Soit $A \in \mathcal{M}_n(\K)$ et $k \in \N^*$, alors $A^k = A \times A \times \cdots \times A$ ($k$ fois).
\end{propriete}

\begin{definition}
    Soit $A \in \mathcal{M}_n(\K)$, on dit que $A$ est inversible s'il existe $B \in \mathcal{M}_n(\K)$ tel que $AB = BA = Id_n$.
    On note $\mathcal{GL}_n(\K)$ l'ensemble des matrices inversibles de $\mathcal{M}_n(\K)$.
\end{definition}

\begin{remarque}
    La notation $\mathcal{GL}$ veut dire "Groupe Linéaire" et provient du fait que l'ensemble des matrices inversibles muni de la multiplication des matrices est un groupe.
\end{remarque}

\begin{proposition}{}{}
    Soit $A \in \mathcal{M}_n(\K)$, si $A$ est inversible, il existe une unique matrice $B \in \mathcal{M}_n(\K)$ telle que $AB = BA = Id_n$.
    Cette matrice s'appelle l'inverse de $A$ et est notée $A^{-1}$.
\end{proposition}

\begin{demonstration}
    S'il existe deux matrices $B$ et $B'$ telles que $AB = BA = Id_n = AB' = B'A$ alors $B' = B'Id_n = B'(AB) = (B'A)B = Id_nB = B$
\end{demonstration}

\begin{proposition}{}{}
    Soit $A \in \mathcal{M}_n(\K)$, s'il existe une matrice $B \in \mathcal{M}_n(\K)$ telle que $AB = Id_n$, alors $A$ est inversible et $B = A^{-1}$.
    Il suffit donc de vérifier le produit d'un seul coté.
\end{proposition}

\begin{proposition}{}{}
    Soit $A, B \in \mathcal{M}_n(\K)$, si $AB$ est inversible alors $(AB)^{-1} = B^{-1}A^{-1}$.
\end{proposition}

\begin{demonstration}
    $(AB)(B^{-1}A^{-1}) = A(BB^{-1}A^{-1}) = AId_nA^{-1} = AA^{-1} = Id_n$
\end{demonstration}

\begin{proposition}{}{}
    Soit $A \in \mathcal{M}_n(\K), B \in \mathcal{M}_{n,p}(\K)$ possède une unique solution $X = A^{-1}B$
\end{proposition}

\begin{demonstration}
    $AX = B \implies A^{-1}(AX) = A^{-1}B \implies (A^{-1}A)X = A^{-1}B \implies X = A^{-1}B$
    
    et réciproquement
    $X = A^{-1}B \implies A(A^{-1}B) = B \implies (AA^{-1})B = B$
\end{demonstration}

\subsection{Système linéaire}

\begin{definition}
    Soit $n, p \in \N^*$. Un système linéaire de $n$ équations linéaire à $p$ inconnues (à coefficients dans $\K$) s'écrit:
    $$
    \begin{cases}
        a_{1,1}x_1 + a_{1,2}x_2 + \cdots + a_{1,p}x_p = b_1 \\
        a_{2,1}x_1 + a_{2,2}x_2 + \cdots + a_{2,p}x_p = b_2 \\
        \vdots \\
        a_{n,1}x_1 + a_{n,2}x_2 + \cdots + a_{n,p}x_p = b_n \\
    \end{cases}
    $$
    Les coefficients $a_{i,j}$ et $b_i$ pour $i \in \{1, \cdots, n\}$ et $j \in \{1, \cdots, p\}$ sont des éléments de $\K$.
    Les coefficients $x_1, \cdots, x_p$ sont les inconnues du système.

    On appelle solution du système linéaire tout $p$-uplet $(x_1, \cdots, x_p) \in \K^p$ tel que les équations du système sont vérifiées.
    $(b_1, \cdots, b_n)$ s'appelle le second membre du système linéaire.
    On note $S(L1)$ l'ensemble des solutions du système linéaire $L1$.
    Lorsque $b_i = 0$ pour tout $i \in \{1, \cdots, n\}$, le système linéaire est dit homogène.
\end{definition}

\begin{definition}
    On appelle matrice augmentée la matrice
    $$
    (A|B) = \left(
    \begin{array}{cccc|c}
        a_{1,1} & a_{1,2} & \cdots & a_{1,p} & b_1 \\
        a_{2,1} & a_{2,2} & \cdots & a_{2,p} & b_2 \\
        \vdots & \vdots & \ddots & \vdots & \vdots \\
        a_{n,1} & a_{n,2} & \cdots & a_{n,p} & b_n \\
    \end{array}
    \right)
    \in \mathcal{M}_{n,p+1}(\K)
    $$
\end{definition}

\begin{definition}
    Un système linéaire de n équations à n équations est régulier ou de Cramer, s'il possède une unique solution.
    Un système linéaire de n équations linéaires à p inconnues est dit compatible quand il a au moins une solution.
\end{definition}

\begin{definition}
    Deux système linéaire (L1) et (L2) à p inconnues $x_1, \cdots, x_p$ sont équivalents si $S(L1) = S(L2)$.
\end{definition}

\begin{definition}
    Soit $n, p \in \N^*$, $A \in \mathcal{M}_{n,p}(\K)$ et $\alpha, \lambda \in \K, \alpha \neq 0$
    \begin{enumerate}
        \item Notons A' la matrice obtenue à partir de A en multipliant par la $\alpha$ la qième ligne de A, $q \in \{1, \cdots, n\}$. Les autres lignes restant inchangées.
        \item Notons A'' (resp A''') la matrice obtenue à partir de A en échangeant les lignes q et k de A (resp. en ajotant à la qième ligne de A le produit de $\lambda$ de la kième ligne de A).
        $q, k \in \{1, \cdots, n\}$, $q \neq k, n \geq 2$, les autres lignes restant inchangées.
        On dit que les matrices A', A'', A''' se déduisent de A par opération élémentaire sur les lignes.
    \end{enumerate}
\end{definition}

%TODO

\end{document}
