\documentclass[a4paper, 12pt]{article}
\usepackage{amsmath, amssymb, amsthm, stmaryrd}
\usepackage{geometry}
\usepackage{pgfplots}
\usepackage{tcolorbox}
\geometry{hmargin=2.5cm, vmargin=2.5cm}

\renewcommand*{\today}{18 février 2025}

\title{Discrete Et Geometrique | CM: 4}
\author{Par Lorenzo}
\date{\today}

\newtheorem{theorem}{Théorème}[section]
\newtheorem{definition}{Définition}[section]
\newtheorem{example}{Example}[section]
\newtheorem{remark}{Remarques}[section]
\newtheorem{lemme}{Lemme}[section]
\newtheorem{corollaire}{Corollaire}[section]

\newtheorem{_proposition}{Proposition}[section]
\newenvironment{proposition}[1][]{
    \begin{_proposition}[#1]~\par
    \vspace*{0.5em}
}{%
    \end{_proposition}%
}

\newtheorem{_proprietes}{Propriétés}[section]
\newenvironment{proprietes}[1][]{
        \begin{_proprietes}[#1]~\par
        \vspace*{0.5em}
}{%
        \end{_proprietes}%
}

\newenvironment{rdem}[1][]{
    \begin{tcolorbox}[colframe=black, colback=white!10, sharp corners]
        #1
}{%
    \end{tcolorbox}
     
}

\newtheorem{_demonstration}{Démonstration}[section]
\newenvironment{demonstration}[1][]{
    \begin{_demonstration}[#1]~\par
    \vspace*{0.5em}
}{%
    \end{_demonstration}%
    \qed%
}

\newtheorem*{_demonstration*}{Démonstration}
\newenvironment{demonstration*}[1][]{
    \begin{_demonstration*}[#1]~\par
    \vspace*{0.5em}
}{%
    \end{_demonstration*}%
    \qed%
}

\newenvironment{ldefinition}{
    \begin{definition}~\par
    \vspace*{0.5em}
    \begin{enumerate}
}{
        \end{enumerate}
        \end{definition}
}

\newenvironment{lexample}{
    \begin{example}~\par
    \vspace*{0.5em}
    \begin{enumerate}
}{
        \end{enumerate}
        \end{example}
}

\newtheorem{_methode}{Méthode}[section]
\newenvironment{methode}{
    \begin{_methode}~\par
    \vspace*{0.5em}
}{
        \end{_methode}
}

\def\N{\mathbb{N}}
\def\Z{\mathbb{Z}}
\def\Q{\mathbb{Q}}
\def\R{\mathbb{R}}
\def\C{\mathbb{C}}
\def\K{\mathbb{K}}
\def\k{\Bbbk}

\def\un{(u_n)_{n \in \N}}
\def\xn#1{(#1_n)_{n \in \N}}

\def\o{\overline}
\def\eps{\varepsilon}

% \funcdef{name}{domain}{codomain}{variable}{expression}
% name: Name of the function (e.g. f)
% domain: Domain of the function (e.g. \mathbb{R})
% codomain: Codomain of the function (e.g. \mathbb{R})
% variable: Variables of the function (e.g. x)
% expression: Expression of the function (e.g. x^2)
\newcommand{\funcdef}[5]{%
    #1 :
    \begin{cases}
        #2 \rightarrow #3 \\
        #4 \mapsto #5
    \end{cases}
}

\newcommand{\lt}{\ensuremath <}
\newcommand{\gt}{\ensuremath >}

\begin{document}

\maketitle

\section{Fondement de la théorie des probabilités - cas de l'univers fini}

La théorie des probabilités est une science appliquée, qui doit être modélisé et se servir d'appareil mathématique.

Un modèle probabiliste introduit:
\begin{itemize}
    \item $\Omega$ - l'univers
    \item événements
    \item probabilité
\end{itemize}

\begin{definition}
    $\Omega$ "l'univers" est l'ensemble des résultats possibles d'une expérience aléatoire.
\end{definition}

\begin{remark}
    L'introduction de $\Omega$ est un choix du modélisateur.
\end{remark}

\begin{example}
    \begin{enumerate}
        \item On lance un dé: $\Omega = \{1, 2, 3, 4, 5, 6\}$.
        \item On lance deux dés distinguables: $\Omega = \{(1, 1), (1, 2), \ldots, (6, 6)\}$.
    \end{enumerate}
\end{example}

\begin{definition}
    Un événement est un sous-ensemble de $\Omega$. On note $\mathcal{P}(\Omega)$ l'ensemble des parties de $\Omega$.
\end{definition}

\begin{remark}
    \begin{itemize}
        \item $\Omega \subset \Omega$ est appelé l'événement certain.
        \item $\emptyset \subset \Omega$ est appelé l'événement impossible.
    \end{itemize}
\end{remark}

\begin{example}
    \begin{itemize}
        \item On lance un dé: $\Omega = \{1, 2, 3, 4, 5, 6\}$.
        $A = \{2, 4, 6\} \subset \Omega$ est la modélisation est "le dé a montré un nombre pair de points".
        \item On lance deux dés distinguables: $\Omega = \{(1, 1), (1, 2), \ldots, (6, 6)\}$.
        $A = \{(1, 6), (2, 5), (3, 4), (4, 3), (5, 2), (6, 1)\} \subset \Omega$ est la modélisation est "la somme des points est égale à 7".
        $B = \{(2, 1), (2, 2), \dots (4, 1), (4, 2), \dots, (6, 1), (6, 2)\} \subset \Omega$ est la modélisation est "le premier dé a montré un nombre pair de points".
    \end{itemize}
\end{example}

\begin{definition}
    Une probabilité est une fonction $P: \mathcal{P}(\Omega) \to [0, 1]$ telle que:
    \begin{enumerate}
        \item $P(\Omega) = 1$
        \item $\forall (A, B) \subset \Omega^2, A \cap B = \emptyset \implies P(A \cup B) = P(A) + P(B)$
        \item $A \neq \emptyset \implies P(A) \neq 0$
    \end{enumerate}
\end{definition}

\begin{proposition}
    Dans les conditions de la définition $P(\emptyset) = 0$.
\end{proposition}

\begin{demonstration}
    Posons $A = \Omega$ et $B = \emptyset$ alors $A, B \subset \Omega, A \cap B = \emptyset$. Donc $P(\Omega \cup \emptyset) = P(\Omega) + P(\emptyset) = 1 + 0 = 1$. Donc $P(\emptyset) = 0$.
\end{demonstration}

\end{document}
