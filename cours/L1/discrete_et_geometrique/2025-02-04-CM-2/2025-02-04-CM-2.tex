\documentclass[a4paper, 12pt]{article}
\usepackage{amsmath, amssymb, amsthm, stmaryrd}
\usepackage{geometry}
\usepackage{pgfplots}
\usepackage{tcolorbox}
\geometry{hmargin=2.5cm, vmargin=2.5cm}

\renewcommand*{\today}{04 février 2025}

\title{Discrete Et Geometrique | CM: 2}
\author{Par Lorenzo}
\date{\today}

\newtheorem{theorem}{Théorème}[section]
\newtheorem{definition}{Définition}[section]
\newtheorem{example}{Example}[section]
\newtheorem{remark}{Remarques}[section]
\newtheorem{lemme}{Lemme}[section]
\newtheorem{corollaire}{Corollaire}[section]

\newtheorem{_proposition}{Proposition}[section]
\newenvironment{proposition}[1][]{
    \begin{_proposition}[#1]~\par
    \vspace*{0.5em}
}{%
    \end{_proposition}%
}

\newtheorem{_proprietes}{Propriétés}[section]
\newenvironment{proprietes}[1][]{
        \begin{_proprietes}[#1]~\par
        \vspace*{0.5em}
}{%
        \end{_proprietes}%
}

\newenvironment{rdem}[1][]{
    \begin{tcolorbox}[colframe=black, colback=white!10, sharp corners]
        #1
}{%
    \end{tcolorbox}
     
}

\newtheorem{_demonstration}{Démonstration}[section]
\newenvironment{demonstration}[1][]{
    \begin{_demonstration}[#1]~\par
    \vspace*{0.5em}
}{%
    \end{_demonstration}%
    \qed%
}

\newtheorem*{_demonstration*}{Démonstration}
\newenvironment{demonstration*}[1][]{
    \begin{_demonstration*}[#1]~\par
    \vspace*{0.5em}
}{%
    \end{_demonstration*}%
    \qed%
}

\newenvironment{ldefinition}{
    \begin{definition}~\par
    \vspace*{0.5em}
    \begin{enumerate}
}{
        \end{enumerate}
        \end{definition}
}

\newenvironment{lexample}{
    \begin{example}~\par
    \vspace*{0.5em}
    \begin{enumerate}
}{
        \end{enumerate}
        \end{example}
}

\newtheorem{_methode}{Méthode}[section]
\newenvironment{methode}{
    \begin{_methode}~\par
    \vspace*{0.5em}
}{
        \end{_methode}
}

\def\N{\mathbb{N}}
\def\Z{\mathbb{Z}}
\def\Q{\mathbb{Q}}
\def\R{\mathbb{R}}
\def\C{\mathbb{C}}
\def\K{\mathbb{K}}
\def\k{\Bbbk}

\def\un{(u_n)_{n \in \N}}
\def\xn#1{(#1_n)_{n \in \N}}

\def\o{\overline}
\def\eps{\varepsilon}

% \funcdef{name}{domain}{codomain}{variable}{expression}
% name: Name of the function (e.g. f)
% domain: Domain of the function (e.g. \mathbb{R})
% codomain: Codomain of the function (e.g. \mathbb{R})
% variable: Variables of the function (e.g. x)
% expression: Expression of the function (e.g. x^2)
\newcommand{\funcdef}[5]{%
    #1 :
    \begin{cases}
        #2 \rightarrow #3 \\
        #4 \mapsto #5
    \end{cases}
}

\newcommand{\lt}{\ensuremath <}
\newcommand{\gt}{\ensuremath >}

\begin{document}

\maketitle

\begin{definition}
    Le nombre de combinaisons de $m$ parmi $n$ ($n \in \N, m \in \Z$) noté $\binom{n}{m}$ (ou $C^m_n$) est le nombre de sous-ensembles de $m$ éléments d'un ensemble à $n$ éléments.
\end{definition}

\begin{remark}
    \begin{itemize}.
        \item $C^m_n = 0 \text{ si } m \lt 0 \text{ ou } m \gt n$.
        \item $C^0_n = 1$.
        \item $C^n_n = n$.
        \item $C^2_n = \frac{n(n-1)}{2}$ (si $n \geq 2$).
    \end{itemize}
\end{remark}

\begin{proposition}
    Soit $n \in \N, m \in \Z$. Alors $C^m_n = 0$ si $m \lt 0$ ou $m \gt n$.
    Et $C^m_n = \frac{n!}{m!(n-m)!}$ si $0 \leq m \leq n$.    
\end{proposition}

\begin{demonstration}
    Supposons $0 \lt m \lt n$ (sinon le résultat est évident).

    $A^m_n$ est le nombre de façons de choisir $m$ éléments parmi $n$ éléments.

    et aussi $A^m_n =$ (nombre de façons de choisir un sous-ensemble de $m$ éléments dans $\{1, \cdots, n$\}) $\times$ (nombre de façons d'ordonner $m$ éléments).

    Donc $A^m_n = C^m_n \times m!$. Donc $C^m_n = \frac{A^m_n}{m!}$.
\end{demonstration}

\begin{remark}
    à $m = 0$

    $C^0_n = \frac{n!}{0!(n-0)!} = \frac{n!}{n!} = 1$.
\end{remark}

\begin{remark}
    à $m = n$

    $C^n_n = \frac{n!}{n!(n-n)!} = \frac{n!}{n!} = 1$.
\end{remark}

\begin{proposition}
    Soit $n \in \N, m \in \Z$. Alors $C^m_n = C^{n-m}_n$.
\end{proposition}

\begin{demonstration}[1]
    $C^m_n = \frac{n!}{m!(n-m)!}$ et $C^{n-m}_n = \frac{n!}{(n-m)!(n-(n-m))!} = \frac{n!}{(n-m)!m!}$.

    Donc $C^m_n = C^{n-m}_n$.
\end{demonstration}

\begin{demonstration}[2]
    $C^m_n =$ nombre de façons de choisir $m$ éléments parmi $n$ = nombre de façons de choisir les $n - m$ éléments à laisser = $C^{n-m}_n$.
\end{demonstration}

\begin{proposition}[identité de Pascal]
    Soit $n \in \N, n \geq 1$ et $0 \leq m \leq n$.
    Alors $C^m_{n+1} = C^{m-1}_{n} + C^m_{n}$. 
\end{proposition}

\begin{remark}
    Le triangle de Pascal est une représentation graphique des coefficients binomiaux. Chaque nombre est la somme des deux nombres situés au-dessus de lui.
    \begin{center}
    \begin{tikzpicture}[every node/.style={font=\footnotesize}]
        % Ligne 0
        \node at (0,0) {$C^{0}_{0}$};
        % Ligne 1
        \node at (-1,-1) {$C^{0}_{1}$};
        \node at (1,-1) {$C^{1}_{1}$};
        % Ligne 2
        \node at (-2,-2) {$C^{0}_{2}$};
        \node at (0,-2) {$C^{1}_{2}$};
        \node at (2,-2) {$C^{2}_{2}$};
        % Ligne 3
        \node at (-3,-3) {$C^{0}_{3}$};
        \node at (-1,-3) {$C^{1}_{3}$};
        \node at (1,-3) {$C^{2}_{3}$};
        \node at (3,-3) {$C^{3}_{3}$};
    \end{tikzpicture}
    \end{center}
\end{remark}

\begin{demonstration}[1]
    \begin{align*}    
        C^m_n + C^{m - 1}_n &= \frac{n!}{m!(n-m)!} + \frac{n!}{(m-1)!(n-m+1)!} \\
        &= \frac{n!}{m(m - 1)!(n - m)!} + \frac{n!}{(m - 1)!(n - m + 1)(n - m)!} \\
        &= \frac{n!}{(m - 1)!(n - m)!} \left( \frac{1}{m} + \frac{1}{n - m + 1} \right) \\
        &= \frac{n!}{(m - 1)!(n - m)!} \left( \frac{n + 1}{n(n - m + 1)} \right) \\
        &= \frac{n!(n+1)}{(m - 1)!(n - m)!m(n - m + 1)} \\
        &= \frac{(n + 1)!}{m!((n + 1) - m)!} = C^m_{n+1}
    \end{align*}
\end{demonstration}

\begin{demonstration}[2]
    $C^m_{n+1} =$ nombre de façons de choisir $m$ éléments dans $\{1, \cdots, n+1\} =$ Nombre de façons de sous ensembles contenant l'éléments $n + 1$ + Nombre de façons de sous ensembles ne contenant pas l'éléments $n + 1$ = \#dfc $m-1$ éléments parmi $\{1, \cdots, n\}$ = $C^{m-1}_n + C^m_n$.
\end{demonstration}


\begin{theorem}[Binôme de Newton]\

    $\forall n \in \N^*, \forall x, y \in \mathbb{C}, (x + y)^n = \sum_{k = 0}^{n} C^k_n x^k y^{n-k}$.
\end{theorem}

\begin{demonstration}
    À faire.
\end{demonstration}

\begin{proposition}
    Soit $n \in \N^*$
    Alors le nombre de tous les sous-ensembles de $\{1, \cdots, n\}$ noté $B_n$ est $2^n$.
\end{proposition}

\begin{demonstration}[1]
    $B_n = C^0_n + C^1_n + \cdots + C^n_n = \sum_{k = 0}^{n} C^k_n = (1 + 1)^n = 2^n$.
\end{demonstration}

\end{document}
