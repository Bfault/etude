\documentclass[a4paper, 12pt]{article}
\usepackage{amsmath, amssymb, amsthm, stmaryrd}
\usepackage{geometry}
\usepackage{pgfplots}
\usepackage{tcolorbox}
\geometry{hmargin=2.5cm, vmargin=2.5cm}

\renewcommand*{\today}{28 janvier 2025}

\title{Discrete Et Geometrique | CM: 1}
\author{Par Lorenzo}
\date{\today}

\newtheorem{theorem}{Théorème}[section]
\newtheorem{definition}{Définition}[section]
\newtheorem{example}{Example}[section]
\newtheorem{remark}{Remarques}[section]
\newtheorem{lemme}{Lemme}[section]
\newtheorem{corollaire}{Corollaire}[section]

\newtheorem{_proposition}{Proposition}[section]
\newenvironment{proposition}[1][]{
    \begin{_proposition}[#1]~\par
    \vspace*{0.5em}
}{%
    \end{_proposition}%
}

\newtheorem{_proprietes}{Propriétés}[section]
\newenvironment{proprietes}[1][]{
        \begin{_proprietes}[#1]~\par
        \vspace*{0.5em}
}{%
        \end{_proprietes}%
}

\newenvironment{rdem}[1][]{
    \begin{tcolorbox}[colframe=black, colback=white!10, sharp corners]
        #1
}{%
    \end{tcolorbox}
     
}

\newtheorem{_demonstration}{Démonstration}[section]
\newenvironment{demonstration}[1][]{
    \begin{_demonstration}[#1]~\par
    \vspace*{0.5em}
}{%
    \end{_demonstration}%
    \qed%
}

\newtheorem*{_demonstration*}{Démonstration}
\newenvironment{demonstration*}[1][]{
    \begin{_demonstration*}[#1]~\par
    \vspace*{0.5em}
}{%
    \end{_demonstration*}%
    \qed%
}

\newenvironment{ldefinition}{
    \begin{definition}~\par
    \vspace*{0.5em}
    \begin{enumerate}
}{
        \end{enumerate}
        \end{definition}
}

\newenvironment{lexample}{
    \begin{example}~\par
    \vspace*{0.5em}
    \begin{enumerate}
}{
        \end{enumerate}
        \end{example}
}

\newtheorem{_methode}{Méthode}[section]
\newenvironment{methode}{
    \begin{_methode}~\par
    \vspace*{0.5em}
}{
        \end{_methode}
}

\def\N{\mathbb{N}}
\def\Z{\mathbb{Z}}
\def\Q{\mathbb{Q}}
\def\R{\mathbb{R}}
\def\C{\mathbb{C}}
\def\K{\mathbb{K}}
\def\k{\Bbbk}

\def\un{(u_n)_{n \in \N}}
\def\xn#1{(#1_n)_{n \in \N}}

\def\o{\overline}
\def\eps{\varepsilon}

% \funcdef{name}{domain}{codomain}{variable}{expression}
% name: Name of the function (e.g. f)
% domain: Domain of the function (e.g. \mathbb{R})
% codomain: Codomain of the function (e.g. \mathbb{R})
% variable: Variables of the function (e.g. x)
% expression: Expression of the function (e.g. x^2)
\newcommand{\funcdef}[5]{%
    #1 :
    \begin{cases}
        #2 \rightarrow #3 \\
        #4 \mapsto #5
    \end{cases}
}

\newcommand{\lt}{\ensuremath <}
\newcommand{\gt}{\ensuremath >}

\begin{document}

\maketitle

\section{Maths discrètes}

\subsection{Dénombrement}

\begin{definition}
    Soit $A$ un ensemble de $N$ éléments ($N \in \N$).
    Une permutation de $A$ est une application bijective entre $A$ et $A$.
\end{definition}

\begin{remark}
    Ici $N$ sert à que $A$ soit fini.
\end{remark}

%Rappel
\begin{definition}
    Soit $A, B$ deux ensembles.
    $f$ est une fonction de $A$ dans $B$ si et seulement si:
    \begin{itemize}
        \item $f \subset A \times B$
        \item $(a, b) \in f \land (a, b') \in f \implies b = b'$
    \end{itemize}
\end{definition}

\begin{proposition}
    Soit $A$ un ensemble de $N$ éléments.
    Soit $f: A \to A$ une application alors,
    $f$ est bijective si et seulement si $f$ injective si et seulement si $f$ surjective.
\end{proposition}

\begin{demonstration}
    \begin{itemize}
        \item \textbf{$f$ bijective $\implies$ $f$ injective.} Par définition de la bijection.
        \item \textbf{$f$ surjective $\implies$ $f$ injective.} On suppose $f$ n'est pas injective.
        Alors $\exists a, a' \in A, f(a) = f(a') \text{ et } a \neq a'$
        On sépare A en deux parties: $\{a, a'\}$ et $A \setminus \{a, a'\}$
        On a $f(A) = f(\{a, a'\}) \cup f(A \setminus \{a, a'\}) = f(a) \cup f(A \setminus \{a, a'\})$.
        Or $f(\{a, a'\})$ contient exactement un élément et $f(A \setminus \{a, a'\})$ contient au plus $N - 2$ éléments.
        Ainsi $f(A)$ contient au plus $N - 1$ éléments. Donc $f$ n'est pas surjective.
        Donc $f$ est injective.
        \item \textbf{$f$ injective $\implies$ $f$ bijective.} \color{red}À faire DM.\color{black} 
        \item \textbf{$f$ injective $\implies$ $f$ bijective.} $f$ est injective donc $f$ est surjective ainsi par definition $f$ est bijective.
    \end{itemize}
\end{demonstration}

\begin{proposition}
    Soit $N \in \N^*$ et $A$ un ensemble de $N$ éléments.
    L'ensemble de toutes les permutation de $A$ contient $N!$ éléments.
\end{proposition}

\begin{demonstration}
    Soit $A = \{1, \cdots N\}$ et $f: A \mapsto A$ une permutation.
    Le nombre de façon de choisir $f(1)$ est $N$, puis $f(2)$ est $N - 1$ et ainsi de suite. \color{red}Remplacer par une récurence.\color{black}
\end{demonstration}

\begin{remark}
    Une permutation peut être vu comme une relation d'order total.
\end{remark}

\begin{definition}
    Soit $n, m \in \N^*$. Le nombre d'arrangements de $m$ parmi $n$, noté $A^m_n$.
    Le nombre d'application injective $f: \{1, \cdots, m\} \to \{1, \cdots, n\}$
\end{definition}

\begin{proposition}
    \begin{itemize}
        \item Si $m \gt n$ alors $A^m_n = 0$.
        \item Si $m = n$ alors $A^m_n = n! = m!$.
    \end{itemize}
\end{proposition}

\begin{proposition}
    Si $n \geq m$ alors $A^m_n = n(n-1)(n-2)\cdots(n-m+1) = \dfrac{n!}{(n - m)!}$.
\end{proposition}

\end{document}
