\documentclass[a4paper, 12pt]{article}
\usepackage{amsmath, amssymb, amsthm, stmaryrd}
\usepackage{geometry}
\usepackage{mathrsfs}
\usepackage{pgfplots}
\usepackage{tcolorbox}
\geometry{hmargin=2.5cm, vmargin=2.5cm}

\renewcommand*{\today}{25 mars 2025}

\title{Discrete Et Geometrique | CM: 7}
\author{Par Lorenzo}
\date{\today}

\newtheorem{theorem}{Théorème}[section]
\newtheorem{definition}{Définition}[section]
\newtheorem{example}{Example}[section]
\newtheorem{remark}{Remarques}[section]
\newtheorem{lemme}{Lemme}[section]
\newtheorem{corollaire}{Corollaire}[section]

\newtheorem{_proposition}{Proposition}[section]
\newenvironment{proposition}[1][]{
    \begin{_proposition}[#1]~\par
    \vspace*{0.5em}
}{%
    \end{_proposition}%
}

\newtheorem{_proprietes}{Propriétés}[section]
\newenvironment{proprietes}[1][]{
        \begin{_proprietes}[#1]~\par
        \vspace*{0.5em}
}{%
        \end{_proprietes}%
}

\newenvironment{rdem}[1][]{
    \begin{tcolorbox}[colframe=black, colback=white!10, sharp corners]
        #1
}{%
    \end{tcolorbox}
     
}

\newtheorem{_demonstration}{Démonstration}[section]
\newenvironment{demonstration}[1][]{
    \begin{_demonstration}[#1]~\par
    \vspace*{0.5em}
}{%
    \end{_demonstration}%
    \qed%
}

\newtheorem*{_demonstration*}{Démonstration}
\newenvironment{demonstration*}[1][]{
    \begin{_demonstration*}[#1]~\par
    \vspace*{0.5em}
}{%
    \end{_demonstration*}%
    \qed%
}

\newenvironment{ldefinition}{
    \begin{definition}~\par
    \vspace*{0.5em}
    \begin{enumerate}
}{
        \end{enumerate}
        \end{definition}
}

\newenvironment{lexample}{
    \begin{example}~\par
    \vspace*{0.5em}
    \begin{enumerate}
}{
        \end{enumerate}
        \end{example}
}

\newtheorem{_methode}{Méthode}[section]
\newenvironment{methode}{
    \begin{_methode}~\par
    \vspace*{0.5em}
}{
        \end{_methode}
}

\def\N{\mathbb{N}}
\def\Z{\mathbb{Z}}
\def\Q{\mathbb{Q}}
\def\R{\mathbb{R}}
\def\C{\mathbb{C}}
\def\K{\mathbb{K}}
\def\k{\Bbbk}

\def\un{(u_n)_{n \in \N}}
\def\xn#1{(#1_n)_{n \in \N}}

\def\o{\overline}
\def\eps{\varepsilon}

% \funcdef{name}{domain}{codomain}{variable}{expression}
% name: Name of the function (e.g. f)
% domain: Domain of the function (e.g. \mathbb{R})
% codomain: Codomain of the function (e.g. \mathbb{R})
% variable: Variables of the function (e.g. x)
% expression: Expression of the function (e.g. x^2)
\newcommand{\funcdef}[5]{%
    #1 :
    \begin{cases}
        #2 \rightarrow #3 \\
        #4 \mapsto #5
    \end{cases}
}

\newcommand{\lt}{\ensuremath <}
\newcommand{\gt}{\ensuremath >}

\begin{document}

\maketitle

\subsection{Variables aléatoires}

\begin{definition}
    $X$ une variable aléatoire réelle c'est à dire $X: \Omega \mapsto \R$
\end{definition}

\begin{definition}
    Soient $X$ une variable aléatoire et $A$ un événement en termes de $X$.
    C'est à dire de forme $\{ w \in \Omega | X(w) \in S\}$ avec $S \subset \R$.

    Notation correcte:
    \begin{itemize}
        \item $\{X \in S\} = \{w \in \Omega | X(w) \in S\}$
        \item $\{X = 3\} = \{w \in \Omega | X(w) \in \{3\}\}$
        \item $\{X \geq 7\} = \{w \in \Omega | X(w) \in [7, +\infty[\}$
    \end{itemize}
\end{definition}

\begin{remark}
    Tout ensemble d'événements résultant d'une variable aléatoire est d'ordre multiplie de quelque chose (dans un lancé de dé, des multiple de 6)
\end{remark}

\begin{remark}
    Premières questions à se poser:
    \begin{itemize}
        \item $P(X  \in S) = ?$
    \end{itemize}
\end{remark}

\begin{definition}
    Soit $X$ une variable aléatoire. La loi de $X$ (notée $P_X$)
    est une fonction $P_X: \mathscr{P}(\R) \to [0, 1]$

    définie par $P_X(S) = P(\{X \in S\})$ avec $S \subset \R$
\end{definition}

\begin{proposition}
    Etant donné une variable aléatoire $X$ d'un univers fini $\Omega$.
    Soit $S, S' \subset \Omega$.

    Supposons $S \cap X(\Omega) = S' \cap X(\Omega)$ alors $P_X(S) = P_X(S')$
\end{proposition}

\begin{demonstration}
    \begin{align*}
        P_X(S) = P(\{X \in S\}) &= P(\{w \in \Omega | X(w) \in S\}) \\
        &= P(\{w \in \Omega | X(w) \in S \cap X(\Omega)\}) \\
        &= P(\{w \in \Omega | X(w) \in S' \cap X(\Omega)\}) \\
        &= \dots = P_X(S')
    \end{align*}
\end{demonstration}

\begin{proposition}
    Soit $X$ une variable aléatoire d'un univers fini $\Omega$.
    Soit $S \subset \R$.

    Alors $P_X(S) = \sum\limits_{x \in S \cap X(\Omega)} P_X(\{x\})$
\end{proposition}

\begin{demonstration}
    \begin{align*}
        P_X(S) &= P(\{w \in \Omega | X(w) \in S\}) \\
        &= P(\{w \in \Omega | X(w) \in S \cap X(\Omega)\}) \\
        &= P(\bigcup\limits_{x \in S \cap X(\Omega)} \{w \in \Omega | X(w) = x\}) \\
        &= \sum\limits_{x \in S \cap X(\Omega)} P(\{w \in \Omega | X(w) = x\}) \\
    \end{align*}
\end{demonstration}

\begin{remark}
    Pour définir la loi d'une variable aléatoire, il suffit de donner $P(\{X = x\}) = P_X(\{x\})$ pour tout $x \in X(\Omega)$
\end{remark}

\begin{proposition}
    Soit $X$ une variable aléatoire d'un univers fini $\Omega$.

    $\sum\limits_{x \in X(\Omega)} P_X(\{x\}) = 1$
\end{proposition}

\begin{demonstration}
    \begin{align*}
        \sum\limits_{x \in X(\Omega)} P_X(\{x\}) &= \sum\limits_{x \in X(\Omega) \cap \R} P(\{x\}) \\
        &= P_X(\R) \\
        &= P(\{X \in \R\}) \\
        &= P(\Omega) = 1
    \end{align*}
\end{demonstration}

\begin{proposition}
    Soit $X$ une variable aléatoire.

    Soit $S, S' \subset \R$ tels que $S \cap S' = \emptyset$.

    $P_X(S \cup S') = P_X(S) + P_X(S')$
\end{proposition}

\begin{demonstration}
    \begin{align*}
        P_X(S \cup S') &= P(\{w \in \Omega | X(w) \in S\} \cup\{w \in \Omega | X(w) \in S\}) \\
        &= P(\{w \in \Omega | X(w) \in S\}) + P(\{w \in \Omega | X(w) \in S'\}) \\
        &= P_X(S) + P_X(S')
    \end{align*}
\end{demonstration}

\end{document}