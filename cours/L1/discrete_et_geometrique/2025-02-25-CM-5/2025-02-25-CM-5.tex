\documentclass[a4paper, 12pt]{article}
\usepackage{amsmath, amssymb, amsthm, stmaryrd}
\usepackage{geometry}
\usepackage{pgfplots}
\usepackage{tcolorbox}
\geometry{hmargin=2.5cm, vmargin=2.5cm}

\renewcommand*{\today}{25 février 2025}

\title{Discrete Et Geometrique | CM: 5}
\author{Par Lorenzo}
\date{\today}

\newtheorem{theorem}{Théorème}[section]
\newtheorem{definition}{Définition}[section]
\newtheorem{example}{Example}[section]
\newtheorem{remark}{Remarques}[section]
\newtheorem{lemme}{Lemme}[section]
\newtheorem{corollaire}{Corollaire}[section]

\newtheorem{_proposition}{Proposition}[section]
\newenvironment{proposition}[1][]{
    \begin{_proposition}[#1]~\par
    \vspace*{0.5em}
}{%
    \end{_proposition}%
}

\newtheorem{_proprietes}{Propriétés}[section]
\newenvironment{proprietes}[1][]{
        \begin{_proprietes}[#1]~\par
        \vspace*{0.5em}
}{%
        \end{_proprietes}%
}

\newenvironment{rdem}[1][]{
    \begin{tcolorbox}[colframe=black, colback=white!10, sharp corners]
        #1
}{%
    \end{tcolorbox}
     
}

\newtheorem{_demonstration}{Démonstration}[section]
\newenvironment{demonstration}[1][]{
    \begin{_demonstration}[#1]~\par
    \vspace*{0.5em}
}{%
    \end{_demonstration}%
    \qed%
}

\newtheorem*{_demonstration*}{Démonstration}
\newenvironment{demonstration*}[1][]{
    \begin{_demonstration*}[#1]~\par
    \vspace*{0.5em}
}{%
    \end{_demonstration*}%
    \qed%
}

\newenvironment{ldefinition}{
    \begin{definition}~\par
    \vspace*{0.5em}
    \begin{enumerate}
}{
        \end{enumerate}
        \end{definition}
}

\newenvironment{lexample}{
    \begin{example}~\par
    \vspace*{0.5em}
    \begin{enumerate}
}{
        \end{enumerate}
        \end{example}
}

\newtheorem{_methode}{Méthode}[section]
\newenvironment{methode}{
    \begin{_methode}~\par
    \vspace*{0.5em}
}{
        \end{_methode}
}

\def\N{\mathbb{N}}
\def\Z{\mathbb{Z}}
\def\Q{\mathbb{Q}}
\def\R{\mathbb{R}}
\def\C{\mathbb{C}}
\def\K{\mathbb{K}}
\def\k{\Bbbk}

\def\un{(u_n)_{n \in \N}}
\def\xn#1{(#1_n)_{n \in \N}}

\def\o{\overline}
\def\eps{\varepsilon}

% \funcdef{name}{domain}{codomain}{variable}{expression}
% name: Name of the function (e.g. f)
% domain: Domain of the function (e.g. \mathbb{R})
% codomain: Codomain of the function (e.g. \mathbb{R})
% variable: Variables of the function (e.g. x)
% expression: Expression of the function (e.g. x^2)
\newcommand{\funcdef}[5]{%
    #1 :
    \begin{cases}
        #2 \rightarrow #3 \\
        #4 \mapsto #5
    \end{cases}
}

\newcommand{\lt}{\ensuremath <}
\newcommand{\gt}{\ensuremath >}

\begin{document}

\maketitle

\begin{proposition}
    Soit $A, B \subset \Omega$ alors $P(A \cup B) = P(A) + P(B) - P(A \cap B)$.
\end{proposition}

\begin{demonstration}
    $A \cup B = (A\backslash B) \cup (B\backslash A) \cup (A \cap B)$
    
    et $(A\backslash B) \cap (B\backslash A) = \emptyset$
    
    Ainsi (d'après la proposition d'union disjointe)
    $P(A \cup B) = P(A\backslash B) + P((B\backslash A) \cup (A \cap B))$


    D'autre part $A = (A\backslash B) \cup (A \cap B)$ \quad (1)

    donc $P(A) = P(A\backslash B) + P(A \cap B)$ \quad (2)

    de même $P(B) = P(B\backslash A) + P(A \cap B)$ \quad (3)
\end{demonstration}

\begin{proposition}
    Soit $A_1, A_2, \ldots, A_n \subset \Omega$ et $k \neq j \implies A_k \cap A_j = \emptyset$

    Alors
    \begin{align*}
        P(\bigcup_{k=1}^n A_k) = \sum_{k=1}^n P(A_k)
    \end{align*}
\end{proposition}

\begin{demonstration}
    Par récurrence:

        si $n=2$, c'est la proposition précédente.

        Supposons $P\left(\bigcup\limits_{k=1}^{n} A_k\right) = \sum_{k=1}^{n} P(A_k)$.

        si $k \neq j \implies A_k \cap A_j = \emptyset$

        Soit $A_{n+1} \subset \Omega$ tel que $A_{n+1} \cap A_k = \emptyset$ pour tout $k \in \{1, \dots, n\}$

        Alors
        \begin{align*}
            P\left(\bigcup_{k=1}^{n+1} A_k\right) &= P\left(\left(\bigcup_{k=1}^{n} A_k\right) \cup A_{n+1}\right) \\
            &= \sum_{k=1}^{n} P(A_k) + P(A_{n+1}) \\
            &= \sum_{k=1}^{n+1} P(A_k)
        \end{align*}
\end{demonstration}

\begin{corollaire}
    Soit $A \subset \Omega$ alors $P(A) = \sum\limits_{w \in A} P(\{w\})$
\end{corollaire}

\begin{remark}
    pour connaitre $P(A)$ pour tout $A \subset \Omega$, il suffit de connaître $P(\{w\}))$ pour tout $w \in \Omega$.
\end{remark}

\begin{definition}
    D'où viennent les valeurs de $P$

    \begin{enumerate}
        \item Peut être donné
        \item Hypothèse d'équiprobabilité
        
            Les probabilités de tout les singletons de l'univers sont égales.
            
            D'autre part, $\sum\limits_{k=1}^n P(\{w_k\}) = 1$ donc $P(\{w_k\}) = \frac{1}{n}$.
    \end{enumerate}
\end{definition}

\section{Probabilité conditionnelle}

\begin{definition}
    Soit un univers $\Omega$ et une probabilité $P: \mathcal{P}(\Omega) \to [0, 1]$.

    Soit $A \subset \Omega$ un événement de probabilité non nulle.

    Par tout événement $B \subset \Omega$ introduisons $P_A(B)$ - probabilité conditionnelle de $B$ sachant $A$.

    $P_A(B) = \frac{P(A \cap B)}{P(A)}$
\end{definition}

\begin{remark}
    La notation $P(B | A)$ est aussi acceptée, mais ne sera pas utilisée pour éviter de penser que $B | A$ est un événement.
\end{remark}

\begin{proposition}
    Dans les condition de la définition avec $A \neq \emptyset$, on a:
    \begin{enumerate}
        \item $P_A(\Omega) = 1$
        \item $P_A(B_1 \cup B_2) = P_A(B_1) + P_A(B_2)$ si $B_1 \cap B_2 = \emptyset$
        \item $P_A(\bigcup\limits_{k=1}^n B_k) = \sum\limits_{k=1}^n P_A(B_k)$ si $k \neq j \implies B_k \cap B_j = \emptyset$
    \end{enumerate}
\end{proposition}

\begin{demonstration}
    \begin{enumerate}
        \item $P_A(\Omega) = \frac{P(A \cap \Omega)}{P(A)} = \frac{P(A)}{P(A)} = 1$
        \item $P_A(B_1 \cup B_2) = \frac{P(A \cap (B_1 \cup B_2))}{P(A)} = \frac{P((A \cap B_1) \cup (A \cap B_2))}{P(A)} = \frac{P(A \cap B_1) + P(A \cap B_2)}{P(A)} = \frac{P(A \cap B_1)}{P(A)} + \frac{P(A \cap B_2)}{P(A)} = P_A(B_1) + P_A(B_2)$
        \item Par récurrence:
        
            si $n=2$, c'est la proposition précédente.

            Supposons $P_A\left(\bigcup\limits_{k=1}^{n} B_k\right) = \sum_{k=1}^{n} P_A(B_k)$.

            \begin{align*}
                P_A\left(\bigcup_{k=1}^{n+1} B_k\right) &= P_A\left(\left(\bigcup_{k=1}^{n} B_k\right) \cup B_{n+1}\right) \\
                &= P_A\left(\bigcup_{k=1}^{n} B_k\right) + P_A(B_{n+1}) \\
                &= \sum_{k=1}^{n} P_A(B_k) + P_A(B_{n+1}) \\
                &= \sum_{k=1}^{n+1} P_A(B_k)
            \end{align*}
    \end{enumerate}
\end{demonstration}

\begin{theorem}[formule des probabilités totales]
    Soit l'univers $\Omega$ et la probabilité $P: \mathcal{P}(\Omega) \to [0, 1]$.

    Soit $A_1, A_2, \ldots, A_n \subset \Omega$ tels que $k \neq j \implies A_k \cap A_j = \emptyset$ et $\bigcup\limits_{k=1}^n A_k = \Omega$ (partition de $\Omega$).
    Soit $B$ un événement.

    Alors $P(B) = \sum\limits_{k=1}^n P_{A_k}(B)P(A_k)$
\end{theorem}

\begin{demonstration}
    $B = \bigcup\limits_{k=1}^n (B \cap A_k)$

    Alors

    $P(B) = \sum\limits_{k=1}^n P(B \cap A_k) = \sum\limits_{k=1}^n P_{A_k}(B)P(A_k)$
\end{demonstration}

\end{document}
