\documentclass[a4paper, 12pt]{article}
\usepackage{amsmath, amssymb, amsthm, stmaryrd}
\usepackage{geometry}
\usepackage{pgfplots}
\usepackage{tcolorbox}
\geometry{hmargin=2.5cm, vmargin=2.5cm}

\renewcommand*{\today}{18 mars 2025}

\title{Discrete Et Geometrique | CM: 6}
\author{Par Lorenzo}
\date{\today}

\newtheorem{theorem}{Théorème}[section]
\newtheorem{definition}{Définition}[section]
\newtheorem{example}{Example}[section]
\newtheorem{remark}{Remarques}[section]
\newtheorem{lemme}{Lemme}[section]
\newtheorem{corollaire}{Corollaire}[section]

\newtheorem{_proposition}{Proposition}[section]
\newenvironment{proposition}[1][]{
    \begin{_proposition}[#1]~\par
    \vspace*{0.5em}
}{%
    \end{_proposition}%
}

\newtheorem{_proprietes}{Propriétés}[section]
\newenvironment{proprietes}[1][]{
        \begin{_proprietes}[#1]~\par
        \vspace*{0.5em}
}{%
        \end{_proprietes}%
}

\newenvironment{rdem}[1][]{
    \begin{tcolorbox}[colframe=black, colback=white!10, sharp corners]
        #1
}{%
    \end{tcolorbox}
     
}

\newtheorem{_demonstration}{Démonstration}[section]
\newenvironment{demonstration}[1][]{
    \begin{_demonstration}[#1]~\par
    \vspace*{0.5em}
}{%
    \end{_demonstration}%
    \qed%
}

\newtheorem*{_demonstration*}{Démonstration}
\newenvironment{demonstration*}[1][]{
    \begin{_demonstration*}[#1]~\par
    \vspace*{0.5em}
}{%
    \end{_demonstration*}%
    \qed%
}

\newenvironment{ldefinition}{
    \begin{definition}~\par
    \vspace*{0.5em}
    \begin{enumerate}
}{
        \end{enumerate}
        \end{definition}
}

\newenvironment{lexample}{
    \begin{example}~\par
    \vspace*{0.5em}
    \begin{enumerate}
}{
        \end{enumerate}
        \end{example}
}

\newtheorem{_methode}{Méthode}[section]
\newenvironment{methode}{
    \begin{_methode}~\par
    \vspace*{0.5em}
}{
        \end{_methode}
}

\def\N{\mathbb{N}}
\def\Z{\mathbb{Z}}
\def\Q{\mathbb{Q}}
\def\R{\mathbb{R}}
\def\C{\mathbb{C}}
\def\K{\mathbb{K}}
\def\k{\Bbbk}

\def\un{(u_n)_{n \in \N}}
\def\xn#1{(#1_n)_{n \in \N}}

\def\o{\overline}
\def\eps{\varepsilon}

% \funcdef{name}{domain}{codomain}{variable}{expression}
% name: Name of the function (e.g. f)
% domain: Domain of the function (e.g. \mathbb{R})
% codomain: Codomain of the function (e.g. \mathbb{R})
% variable: Variables of the function (e.g. x)
% expression: Expression of the function (e.g. x^2)
\newcommand{\funcdef}[5]{%
    #1 :
    \begin{cases}
        #2 \rightarrow #3 \\
        #4 \mapsto #5
    \end{cases}
}

\newcommand{\lt}{\ensuremath <}
\newcommand{\gt}{\ensuremath >}

\begin{document}

\maketitle

Soient $A, B_1, B_2 \subset \Omega$ tels que $B_1 \cap B_2 = \emptyset$. On a

\begin{align*}
    P_A(B_1 \cup B_2) = P_A(B_1) + P_A(B_2)
\end{align*}

\begin{proposition}[formule de Bayes]
Soient $\{A_1, \dots, A_N\} \subset \Omega$ (une partition de $\Omega$) et $B \subset \Omega$ tels que $P(B) > 0$. On a

Alors $\forall j \in \llbracket 1, N \rrbracket$,

$P_B(A_j) = \dfrac{P_{A_j}P(A_j)}{\sum_{k=1}^{N}P_{A_k}(B)P(A_k)}$
    
\end{proposition}

\begin{demonstration}
    \begin{align*}
        P_B(A_j) = \frac{P(A_j \cap B)}{P(B)} = \frac{P_{A_j}(B)P(A_j)}{P(B)} = \frac{P_{A_j}(B)P(A_j)}{P(B)} = \dfrac{P_{A_j}P(A_j)}{\sum_{k=1}^{N}P_{A_k}(B)P(A_k)}
    \end{align*}
\end{demonstration}

\subsection{Evénements indépendants}

\begin{definition}
    Soit $A, B \subset \Omega$.
    
    On dit que $A$ et $B$ sont indépendants si $P(A \cap B) = P(A)P(B)$.
\end{definition}

\begin{remark}
    Si $P(A) \neq 0$ alors $A$ et $B$ sont indépendants si et seulement si $P_{A}(B) = P(B)$.
\end{remark}

\begin{proposition}
    Soit $A \subset \Omega$.
    \begin{itemize}
        \item $A$ et $\Omega$ sont indépendants.
        \item $A$ et $\emptyset$ sont indépendants.
    \end{itemize}
\end{proposition}

\begin{demonstration}
    Je laisse la démonstration à faire par le lecteur.
\end{demonstration}

\begin{proposition}
    Soit $A \subset \Omega$.

    Alors $A$ et $A$ sont indépendants si et seulement si $P(A) = 0$ ou $P(A) = 1$.
    C'est-à-dire si et seulement si $A$ est un événement certain ($\Omega$) ou un événement impossible ($\emptyset$).
\end{proposition}

\begin{proposition}
    Soit $A, B \subset \Omega$ et $A \subset B$.

    Alors $P(A) \leq P(B)$.
\end{proposition}

\begin{demonstration}
    On peut définir $B = A \cup (B\backslash A)$, ainsi

    $P(B) = P(A) + P(B\backslash A)$ (on voit que $P(B\backslash A) \geq 0$).

    Donc $P(B) \geq P(A)$
\end{demonstration}

\begin{proposition}
    Soient $A, B \subset \Omega$ et $A \subset B$

    Alors $A$ et $B$ indépendants si et seulement si $A = \emptyset$ ou $B = \Omega$
\end{proposition}

\begin{demonstration}
    $\impliedby$: est déjà connu.

    $\implies$: $P(A \cap B) = P(A)P(B) \iff P(A) = P(A)P(B) \iff P(A) = 0 \lor P(B) = 1$
\end{demonstration}

\begin{definition}(cas de 3 événements)
    Soient $A, B, C \subset \Omega$

    $A, B, C$ sont indépendants si et seulement si
    \begin{itemize}
        \item $A, B$ sont indépendants
        \item $B, C$ sont indépendants
        \item $A, C$ sont indépendants
        \item $P(A \cap B \cap C) = P(A)P(B)P(C)$
    \end{itemize}
\end{definition}

\begin{remark}
    En général la dernière n'est pas impliqué par les autres.
\end{remark}

\begin{definition}(cas d'un nombre fini d'événement)
    Soient $A_1, \dots, A_N \subset \Omega$.

    $A_1, \dots, A_N$ indépendants si et seulement si
    \begin{enumerate}
        \item si $N = 2$: $P(A_1 \cap A_2) = P(A_1)P(A_2)$
        \item si $N \gt 2$:
            $\forall k \in \{1, \dots, N\}$
            $\{A_j | j \in \{1, \dots, N\}\backslash\{k\}\}$ sont indépendants

            et $P(\Cap_{k=1}^N A_k = \prod_{k=1}^NP(A_k))$
    \end{enumerate}
\end{definition}

\end{document}
