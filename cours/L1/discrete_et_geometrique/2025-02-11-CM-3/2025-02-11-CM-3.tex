\documentclass[a4paper, 12pt]{article}
\usepackage{amsmath, amssymb, amsthm, stmaryrd}
\usepackage{geometry}
\usepackage{pgfplots}
\usepackage{tcolorbox}
\geometry{hmargin=2.5cm, vmargin=2.5cm}

\renewcommand*{\today}{11 février 2025}

\title{Discrete Et Geometrique | CM: 3}
\author{Par Lorenzo}
\date{\today}

\newtheorem{theorem}{Théorème}[section]
\newtheorem{definition}{Définition}[section]
\newtheorem{example}{Example}[section]
\newtheorem{remark}{Remarques}[section]
\newtheorem{lemme}{Lemme}[section]
\newtheorem{corollaire}{Corollaire}[section]

\newtheorem{_proposition}{Proposition}[section]
\newenvironment{proposition}[1][]{
    \begin{_proposition}[#1]~\par
    \vspace*{0.5em}
}{%
    \end{_proposition}%
}

\newtheorem{_proprietes}{Propriétés}[section]
\newenvironment{proprietes}[1][]{
        \begin{_proprietes}[#1]~\par
        \vspace*{0.5em}
}{%
        \end{_proprietes}%
}

\newenvironment{rdem}[1][]{
    \begin{tcolorbox}[colframe=black, colback=white!10, sharp corners]
        #1
}{%
    \end{tcolorbox}
     
}

\newtheorem{_demonstration}{Démonstration}[section]
\newenvironment{demonstration}[1][]{
    \begin{_demonstration}[#1]~\par
    \vspace*{0.5em}
}{%
    \end{_demonstration}%
    \qed%
}

\newtheorem*{_demonstration*}{Démonstration}
\newenvironment{demonstration*}[1][]{
    \begin{_demonstration*}[#1]~\par
    \vspace*{0.5em}
}{%
    \end{_demonstration*}%
    \qed%
}

\newenvironment{ldefinition}{
    \begin{definition}~\par
    \vspace*{0.5em}
    \begin{enumerate}
}{
        \end{enumerate}
        \end{definition}
}

\newenvironment{lexample}{
    \begin{example}~\par
    \vspace*{0.5em}
    \begin{enumerate}
}{
        \end{enumerate}
        \end{example}
}

\newtheorem{_methode}{Méthode}[section]
\newenvironment{methode}{
    \begin{_methode}~\par
    \vspace*{0.5em}
}{
        \end{_methode}
}

\def\N{\mathbb{N}}
\def\Z{\mathbb{Z}}
\def\Q{\mathbb{Q}}
\def\R{\mathbb{R}}
\def\C{\mathbb{C}}
\def\K{\mathbb{K}}
\def\k{\Bbbk}

\def\un{(u_n)_{n \in \N}}
\def\xn#1{(#1_n)_{n \in \N}}

\def\o{\overline}
\def\eps{\varepsilon}

% \funcdef{name}{domain}{codomain}{variable}{expression}
% name: Name of the function (e.g. f)
% domain: Domain of the function (e.g. \mathbb{R})
% codomain: Codomain of the function (e.g. \mathbb{R})
% variable: Variables of the function (e.g. x)
% expression: Expression of the function (e.g. x^2)
\newcommand{\funcdef}[5]{%
    #1 :
    \begin{cases}
        #2 \rightarrow #3 \\
        #4 \mapsto #5
    \end{cases}
}

\newcommand{\lt}{\ensuremath <}
\newcommand{\gt}{\ensuremath >}

\begin{document}

\maketitle

\begin{theorem}[Formule du binôme de Newton]\

    Soit $x, y \in \C, n \in \N^*$.

    $(x + y)^n = \sum_{k=0}^n C^k_n x^k y^{n-k}$.
\end{theorem}

\begin{demonstration}
    Par récurrence.
    Soit $P_n: (x + y)^n = \sum_{k=0}^n C^k_n x^k y^{n-k}$.

    \noindent\underline{Initialisation} : $n = 1$.

    $C_1^0x^0y^1 + C_1^1x^1y^1 = x + y = (x + y)^1$.

    \noindent\underline{Hérédité} : Supposons $P_n$ vraie pour un certain $n \in \N^*$. Montrons $P_{n+1}$.

    \begin{align*}
        (x + y)^{n+1} &= (x + y)(x + y)^n \\
        &= (x + y)\sum_{k=0}^n C^k_n x^k y^{n-k} \\
        &= \sum_{k=0}^n C^k_n x^{k+1} y^{n-k} + \sum_{k=0}^n C^k_n x^k y^{n-k+1} \\
        &= \sum_{k=1}^{n+1} C^{k-1}_n x^k y^{n-k+1} + \sum_{k=0}^n C^k_n x^k y^{n-k+1} \\
        &= \sum_{k=1}^{n} C^{k-1}_n x^k y^{n-k+1} + C^n_n x^{n+1} y^{0} + \sum_{k=1}^{n} C^k_n x^k y^{n-k+1} + C^0_n x^0 y^{n+1} \\
        &= \sum_{k=1}^{n} C^{k-1}_n x^k y^{n-k+1} + x^{n+1} + \sum_{k=1}^{n} C^k_n x^k y^{n-k+1} + y^{n+1} \\
        &= x^{n+1} + y^{n+1} + \sum_{k=1}^n [(C^{k-1}_n + C^k_n) x^k y^{(n+1)-k}] \\
        &= x^{n+1} + y^{n+1} + \sum_{k=1}^n C^{k}_{n+1} x^k y^{(n+1)-k} \\
        &= \sum_{k=0}^{n+1} C^k_{n+1} x^k y^{(n+1)-k}
    \end{align*}

    Donc $P_{n+1}$ est vraie.

    Par le principe de récurrence, $P_n$ est vraie pour tout $n \in \N^*$.

\end{demonstration}

\begin{remark}
    Idée de la démonstration combinatorienne:
    \begin{itemize}
        \item $(x + y)^n$ il y a $2^n$ termes.
        \item $x^n + x^{n-1}y \times \text{ nombre de façon de choisir y} = C^1_n = x^{n-2}y^{2} \times \text{ nombre de façon de choisir 2 y} = C^2_n \cdots$
    \end{itemize}
\end{remark}

\begin{corollaire}
    \begin{itemize}
        \item $\sum^n_{k=0} C^k_n = 2^n$
        \item $\sum^n_{k=0} (-1)^k C^k_n = 0$
    \end{itemize}
\end{corollaire}

\begin{demonstration}
    $0 = (-1 +1)^n = \sum^n_{k=0}C^k_n(-1)^k1^{n-k}$
\end{demonstration}

\begin{example}
    Calculons $\sum^n_{k=0} k C^k_n =: S_n$

    \begin{align*}
        S_n = \sum^n_{k=0} k C^k_n &= \sum^n_{k=1} k C^k_n \\
        &= \sum^n_{k=1} k \frac{n!}{k!(n-k)!} \\
        &= \sum^n_{k=1} \frac{n!}{(k-1)!(n-k)!} \\
        &= \sum^n_{k=1} \frac{n!}{((n-1)-(k-1))!(k-1)!} \\
        &= \sum^n_{k=1} \frac{n(n-1)!}{((n-1)-(k-1))!(k-1)!} \\
        &= n \sum^n_{k=1} C^{k-1}_{n-1} \\
        &= n \times 2^{n-1}
    \end{align*}
\end{example}

\begin{example}
    \begin{align*}
        ((x + 1)^n)' = (\sum_{k=0}^n x^k)' \\
        &= (C^0_n x^0)' + (\sum_{k=1}^n C^k_n x^k)' \\
        &= 0 + \sum_{k=1}^n k C^k_n x^{k-1} \\
    \end{align*}
\end{example}

DM calculer $\sum^n_{k=0} k^2 C^k_n$
Indication: $k^2 = k(k-1) + k$

\begin{example}
    Packet de 32 cartes, nombres ($N$) de mains de 5 cartes contenant un roi et un carreau.

    $N = C^4_21 + C^1_3 \times C^1_7 \times C^3_21$
\end{example}

\end{document}
