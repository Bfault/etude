\documentclass[a4paper, 12pt]{article}
\usepackage{amsmath, amssymb, amsthm}
\usepackage{geometry}
\usepackage{tcolorbox}
\geometry{hmargin=2.5cm, vmargin=2.5cm}

\renewcommand*{\today}{03 octobre 2024}

\title{Science Decision | CM: 5}
\author{Par Lorenzo}
\date{\today}

\newtheorem{theorem}{Théorème}[section]
\newtheorem{definition}{Définition}[section]
\newtheorem{example}{Example}[section]
\newtheorem{remark}{Remarques}[section]
\newtheorem{lemme}{Lemme}[section]
\newtheorem{corollaire}{Corollaire}[section]

\newtheorem{_proposition}{Proposition}[section]
\newenvironment{proposition}[1][]{
    \begin{_proposition}[#1]~\par
    \vspace*{0.5em}
}{%
    \end{_proposition}%
}

\newtheorem{_proprietes}{Propriétés}[section]
\newenvironment{proprietes}[1][]{
        \begin{_proprietes}[#1]~\par
        \vspace*{0.5em}
}{%
        \end{_proprietes}%
}

\newenvironment{rdem}[1][]{
    \begin{tcolorbox}[colframe=black, colback=white!10, sharp corners]
        #1
}{%
    \end{tcolorbox}
     
}

\newtheorem{_demonstration}{Démonstration}[section]
\newenvironment{demonstration}[1][]{
    \begin{_demonstration}[#1]~\par
    \vspace*{0.5em}
}{%
    \end{_demonstration}%
    \qed%
}

\newtheorem*{_demonstration*}{Démonstration}
\newenvironment{demonstration*}[1][]{
    \begin{_demonstration*}[#1]~\par
    \vspace*{0.5em}
}{%
    \end{_demonstration*}%
    \qed%
}

\newenvironment{ldefinition}{
    \begin{definition}~\par
    \vspace*{0.5em}
    \begin{enumerate}
}{
        \end{enumerate}
        \end{definition}
}

\newenvironment{lexample}{
    \begin{example}~\par
    \vspace*{0.5em}
    \begin{enumerate}
}{
        \end{enumerate}
        \end{example}
}

\newtheorem{_methode}{Méthode}[section]
\newenvironment{methode}{
    \begin{_methode}~\par
    \vspace*{0.5em}
}{
        \end{_methode}
}

\def\N{\mathbb{N}}
\def\Z{\mathbb{Z}}
\def\Q{\mathbb{Q}}
\def\R{\mathbb{R}}
\def\C{\mathbb{C}}
\def\K{\mathbb{K}}
\def\k{\Bbbk}

\def\un{(u_n)_{n \in \N}}
\def\xn#1{(#1_n)_{n \in \N}}

\def\o{\overline}
\def\eps{\varepsilon}

% \funcdef{name}{domain}{codomain}{variable}{expression}
% name: Name of the function (e.g. f)
% domain: Domain of the function (e.g. \mathbb{R})
% codomain: Codomain of the function (e.g. \mathbb{R})
% variable: Variables of the function (e.g. x)
% expression: Expression of the function (e.g. x^2)
\newcommand{\funcdef}[5]{%
    #1 :
    \begin{cases}
        #2 \rightarrow #3 \\
        #4 \mapsto #5
    \end{cases}
}

\newcommand{\lt}{\ensuremath <}
\newcommand{\gt}{\ensuremath >}

\begin{document}

\maketitle

Soit $x \in X$ l'ensemble $\{y \in X \mid xRy\}$ est appelé classe d'équivalence
de x notée $C_x$.

\begin{example}
    "=" sur $\N$

    \item $\bullet$ $\forall a \in \N, a = a$ (Réflexive)
    \item $\bullet$ $\forall a, b \in \N, a = b \implies b = a$ (Symétrique)
    \item $\bullet$ $\forall a, b, c \in \N, a = b \land b = c \implies a = c$ (Transitive)

    "=" est une relation d'équivalence sur $\N$

    $C_2 = \{y \in \N, 2 = y\} = \{2\}$
\end{example}

$\{C_x \mid x \in X\}$ est l'ensemble quotient de X par R noté X/R.

\begin{proposition}
    Soit R une relation d'équivalence sur X, X/R forme une partition de X,
    i.e.

    \item $\bullet$ $\forall x, y \in X, C_x \cap C_y = \emptyset \text{ ou } C_x = C_y$
    \item $\bullet$ $X = \bigcup_{x \in X} C_x$
\end{proposition}

\begin{demonstration}
    Nous allons montrer que

    $\forall x, y \in X, \; xRy \implies C_x = C_y$

    $\forall x, y \in X, \; \neg xRy \implies C_x \cap C_y = \emptyset$

    Soient $x, y \in X$

    so $xRy$ soit $z \in C_x$ alors xRz
    %todo
\end{demonstration}

\begin{remark}
    Pour une relation binaire il est toujours vrai que $\forall x, y \in X, xRy \lor \neg xRy$
\end{remark}

%todo

\end{document}
