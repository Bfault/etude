\documentclass[a4paper, 12pt]{article}
\usepackage{amsmath, amssymb, amsthm}
\usepackage{geometry}
\usepackage{tcolorbox}
\geometry{hmargin=2.5cm, vmargin=2.5cm}

\renewcommand*{\today}{13 septembre 2024}

\title{Science Decision | CM: 2}
\author{Par Lorenzo}
\date{\today}

\newtheorem{theorem}{Théorème}[section]
\newtheorem{definition}{Définition}[section]
\newtheorem{example}{Example}[section]
\newtheorem{remark}{Remarques}[section]
\newtheorem{lemme}{Lemme}[section]
\newtheorem{corollaire}{Corollaire}[section]

\newtheorem{_proposition}{Proposition}[section]
\newenvironment{proposition}[1][]{
    \begin{_proposition}[#1]~\par
    \vspace*{0.5em}
}{%
    \end{_proposition}%
}

\newtheorem{_proprietes}{Propriétés}[section]
\newenvironment{proprietes}[1][]{
        \begin{_proprietes}[#1]~\par
        \vspace*{0.5em}
}{%
        \end{_proprietes}%
}

\newenvironment{rdem}[1][]{
    \begin{tcolorbox}[colframe=black, colback=white!10, sharp corners]
        #1
}{%
    \end{tcolorbox}
     
}

\newtheorem{_demonstration}{Démonstration}[section]
\newenvironment{demonstration}[1][]{
    \begin{_demonstration}[#1]~\par
    \vspace*{0.5em}
}{%
    \end{_demonstration}%
    \qed%
}

\newtheorem*{_demonstration*}{Démonstration}
\newenvironment{demonstration*}[1][]{
    \begin{_demonstration*}[#1]~\par
    \vspace*{0.5em}
}{%
    \end{_demonstration*}%
    \qed%
}

\newenvironment{ldefinition}{
    \begin{definition}~\par
    \vspace*{0.5em}
    \begin{enumerate}
}{
        \end{enumerate}
        \end{definition}
}

\newenvironment{lexample}{
    \begin{example}~\par
    \vspace*{0.5em}
    \begin{enumerate}
}{
        \end{enumerate}
        \end{example}
}

\newtheorem{_methode}{Méthode}[section]
\newenvironment{methode}{
    \begin{_methode}~\par
    \vspace*{0.5em}
}{
        \end{_methode}
}

\def\N{\mathbb{N}}
\def\Z{\mathbb{Z}}
\def\Q{\mathbb{Q}}
\def\R{\mathbb{R}}
\def\C{\mathbb{C}}
\def\K{\mathbb{K}}
\def\k{\Bbbk}

\def\un{(u_n)_{n \in \N}}
\def\xn#1{(#1_n)_{n \in \N}}

\def\o{\overline}
\def\eps{\varepsilon}

% \funcdef{name}{domain}{codomain}{variable}{expression}
% name: Name of the function (e.g. f)
% domain: Domain of the function (e.g. \mathbb{R})
% codomain: Codomain of the function (e.g. \mathbb{R})
% variable: Variables of the function (e.g. x)
% expression: Expression of the function (e.g. x^2)
\newcommand{\funcdef}[5]{%
    #1 :
    \begin{cases}
        #2 \rightarrow #3 \\
        #4 \mapsto #5
    \end{cases}
}

\newcommand{\lt}{\ensuremath <}
\newcommand{\gt}{\ensuremath >}

\begin{document}

\maketitle


\section{Relations binaires}

\begin{definition}
    Une relation binaire R sur un ensemble X est un sous-ensemble de paires ordonnées $(x, y) \in X^2$,
    on simplifie la notation par $xRy$ (resp. $\neg xRy$) pour $(x, y) \in R$ (resp. $(x, y) \notin R$).
\end{definition}

\vspace{1em}

\begin{proprietes}
\item {
    \textbf{réflexive} si
    \begin{align*}
        \forall x \in X, \; xRx
    \end{align*}
}
\item {
    \textbf{irréflexive} si
    \begin{align*}
        \forall x \in X, \; \neg(xRx)
    \end{align*}
}
\item {
    \textbf{symétrique} si
    \begin{align*}
        \forall x, y \in X, \; xRy \implies yRx
    \end{align*}
}
\item {
    \textbf{asymétrique} si
    \begin{align*}
        \forall x, y \in X, \; xRy \implies \neg(yRx)
    \end{align*}
}
\item {
    \textbf{antisymétrique} si
    \begin{align*}
        \forall x, y \in X, \; xRy \land yRx \implies x = y
    \end{align*}
}
\item {
    \textbf{transitive} si
    \begin{align*}
        \forall x, y, z \in X, \; xRy \land yRz \implies xRz
    \end{align*}
}
\item {
    \textbf{négativement transitive} si
    \begin{align*}
        \forall x, y, z \in X, \; \neg(xRy) \land \neg(yRz) \implies \neg(xRz)
    \end{align*}
}\item {
    \textbf{complète (ou totale)} si
    \begin{align*}
        \forall x, y \in X, \; xRy \lor yRx
    \end{align*}
}
\end{proprietes}

\begin{remark}
    la notation $xRy$ peut être remplacé par $(x, y) \in R$, par exemple pour la réflexivité, ($\forall x \in X, \; (x, x) \in R$).
\end{remark}

\vspace{2em}

Une relation qui satisfait certaines propriétés peut porter un nom.

\begin{ldefinition}
\item Une \textbf{relation d'équivalence} si elle est réfléxive, symétrique et transitive.
\item Un \textbf{préordre (ou quasi ordre)} si elle est réflexive et transitive.
\item Un \textbf{ordre faible (ou préordre total)} si elle est transitive et complète.
\item Un \textbf{ordre faible strict} si elle est asymétrique et négativement transitive.
\item Un \textbf{ordre partiel} si elle est réflexive, antisymétrique et transitive.
\item Un \textbf{ordre partiel (ou ordre, ordre linéaire, chaîne)} si elle est antisymétrique, transitive et complète.
\end{ldefinition}

\begin{lexample}
\item $\R$ est totalement ordonnées par $\geq$ et est appelé l'ordre naturel sur $\R$.
\item $\N$ avec $\gt$ est un ordre faible strict.
\item Deux entiers relatifs x et y sont congrus modulo $p \in \N$, s'il existe $k \in \N$ tel que $x = y + kp$, ce que l'on note $x \equiv y[p]$.
    La relation de congruence modulo sur $\Z$ est une relation d'équivalence.
\end{lexample}

\end{document}
