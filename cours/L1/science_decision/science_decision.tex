\documentclass[a4paper, 12pt]{article}
\usepackage{amsmath, amssymb, amsthm, stmaryrd}
\usepackage{geometry}
\usepackage{tcolorbox}
\usepackage{pgfplots}
\usepackage{hyperref}

\geometry{hmargin=2.5cm, vmargin=2.5cm}

\renewcommand*{\today}{06 December 2024}

\title{Science Decision}
\author{Par Lorenzo}
\date{\today}

\newtheorem{theorem}{Théorème}[section]
\newtheorem{definition}{Définition}[section]
\newtheorem{example}{Example}[section]
\newtheorem{remark}{Remarques}[section]
\newtheorem{lemme}{Lemme}[section]
\newtheorem{corollaire}{Corollaire}[section]

\newtheorem{_proposition}{Proposition}[section]
\newenvironment{proposition}[1][]{
    \begin{_proposition}[#1]~\par
    \vspace*{0.5em}
}{%
    \end{_proposition}%
}

\newtheorem{_proprietes}{Propriétés}[section]
\newenvironment{proprietes}[1][]{
        \begin{_proprietes}[#1]~\par
        \vspace*{0.5em}
}{%
        \end{_proprietes}%
}

\newenvironment{rdem}[1][]{
    \begin{tcolorbox}[colframe=black, colback=white!10, sharp corners]
        #1
}{%
    \end{tcolorbox}
     
}

\newtheorem{_demonstration}{Démonstration}[section]
\newenvironment{demonstration}[1][]{
    \begin{_demonstration}[#1]~\par
    \vspace*{0.5em}
}{%
    \end{_demonstration}%
    \qed%
}

\newtheorem*{_demonstration*}{Démonstration}
\newenvironment{demonstration*}[1][]{
    \begin{_demonstration*}[#1]~\par
    \vspace*{0.5em}
}{%
    \end{_demonstration*}%
    \qed%
}

\newenvironment{ldefinition}{
    \begin{definition}~\par
    \vspace*{0.5em}
    \begin{enumerate}
}{
        \end{enumerate}
        \end{definition}
}

\newenvironment{lexample}{
    \begin{example}~\par
    \vspace*{0.5em}
    \begin{enumerate}
}{
        \end{enumerate}
        \end{example}
}

\newtheorem{_methode}{Méthode}[section]
\newenvironment{methode}{
    \begin{_methode}~\par
    \vspace*{0.5em}
}{
        \end{_methode}
}

\def\N{\mathbb{N}}
\def\Z{\mathbb{Z}}
\def\Q{\mathbb{Q}}
\def\R{\mathbb{R}}
\def\C{\mathbb{C}}
\def\K{\mathbb{K}}
\def\k{\Bbbk}

\def\un{(u_n)_{n \in \N}}
\def\xn#1{(#1_n)_{n \in \N}}

\def\o{\overline}
\def\eps{\varepsilon}

% \funcdef{name}{domain}{codomain}{variable}{expression}
% name: Name of the function (e.g. f)
% domain: Domain of the function (e.g. \mathbb{R})
% codomain: Codomain of the function (e.g. \mathbb{R})
% variable: Variables of the function (e.g. x)
% expression: Expression of the function (e.g. x^2)
\newcommand{\funcdef}[5]{%
    #1 :
    \begin{cases}
        #2 \rightarrow #3 \\
        #4 \mapsto #5
    \end{cases}
}

\newcommand{\lt}{\ensuremath <}
\newcommand{\gt}{\ensuremath >}

\begin{document}

\maketitle

\tableofcontents


% Begin of 2024-09-12-CM-1.tex

Arrivé apres le premier CM (cours à venir)


% End of 2024-09-12-CM-1.tex

% Begin of 2024-09-13-CM-2.tex


\section{Relations binaires}

\begin{definition}
    Une relation binaire R sur un ensemble X est un sous-ensemble de paires ordonnées $(x, y) \in X^2$,
    on simplifie la notation par $xRy$ (resp. $\neg xRy$) pour $(x, y) \in R$ (resp. $(x, y) \notin R$).
\end{definition}

\vspace{1em}

\begin{proprietes}
\item {
    \textbf{réflexive} si
    \begin{align*}
        \forall x \in X, \; xRx
    \end{align*}
}
\item {
    \textbf{irréflexive} si
    \begin{align*}
        \forall x \in X, \; \neg(xRx)
    \end{align*}
}
\item {
    \textbf{symétrique} si
    \begin{align*}
        \forall x, y \in X, \; xRy \implies yRx
    \end{align*}
}
\item {
    \textbf{asymétrique} si
    \begin{align*}
        \forall x, y \in X, \; xRy \implies \neg(yRx)
    \end{align*}
}
\item {
    \textbf{antisymétrique} si
    \begin{align*}
        \forall x, y \in X, \; xRy \land yRx \implies x = y
    \end{align*}
}
\item {
    \textbf{transitive} si
    \begin{align*}
        \forall x, y, z \in X, \; xRy \land yRz \implies xRz
    \end{align*}
}
\item {
    \textbf{négativement transitive} si
    \begin{align*}
        \forall x, y, z \in X, \; \neg(xRy) \land \neg(yRz) \implies \neg(xRz)
    \end{align*}
}\item {
    \textbf{complète (ou totale)} si
    \begin{align*}
        \forall x, y \in X, \; xRy \lor yRx
    \end{align*}
}
\end{proprietes}

\begin{remark}
    la notation $xRy$ peut être remplacé par $(x, y) \in R$, par exemple pour la réflexivité, ($\forall x \in X, \; (x, x) \in R$).
\end{remark}

\vspace{2em}

Une relation qui satisfait certaines propriétés peut porter un nom.

\begin{definition}
\item Une \textbf{relation d'équivalence} si elle est réfléxive, symétrique et transitive.
\item Un \textbf{préordre (ou quasi ordre)} si elle est réflexive et transitive.
\item Un \textbf{ordre faible (ou préordre total)} si elle est transitive et complète.
\item Un \textbf{ordre faible strict} si elle est asymétrique et négativement transitive.
\item Un \textbf{ordre partiel} si elle est réflexive, antisymétrique et transitive.
\item Un \textbf{ordre partiel (ou ordre, ordre linéaire, chaîne)} si elle est antisymétrique, transitive et complète.
\end{definition}

\begin{example}
\item $\R$ est totalement ordonnées par $\geq$ et est appelé l'ordre naturel sur $\R$.
\item $\N$ avec $\gt$ est un ordre faible strict.
\item Deux entiers relatifs x et y sont congrus modulo $p \in \N$, s'il existe $k \in \N$ tel que $x = y + kp$, ce que l'on note $x \equiv y[p]$.
    La relation de congruence modulo sur $\Z$ est une relation d'équivalence.
\end{example}


% End of 2024-09-13-CM-2.tex

% Begin of 2024-09-19-CM-3.tex

\begin{proposition}
    Si R est asymétrique alors R est réflexive.
\end{proposition}

\noindent
Démontré trivialement par les définitions d'asymétrie et de réflexivité.

\begin{proposition}
    Si R est irréflexive et transitive alors R est asymétrique.
\end{proposition}

\begin{demonstration}
    On suppose que R est irréflexive, transitive et non asymétrique.

    \vspace{0.5em}

    La non asymétrie se traduit par
    \begin{align*}
        \neg(\forall x, y \in X, \; xRy \implies \neg(yRx)) \quad \equiv \quad \exists x, y \in X, \; xRy \land yRx 
    \end{align*}

    Ainsi avec la non asymétrie et la transitivité on arrive à
    \begin{align*}
        &\exists x, y \in X, \; xRy \land yRx \quad \land \quad \forall x, y, z \in X, \; xRy \land yRz \implies xRz \\
        &\equiv \quad \exists x, y \in X, \; xRy \land yRx \implies xRx
    \end{align*}

    Ce qui est absurde car ça contredit l'irréflexivité !!!

    \begin{rdem}
        Donc Si R est irréflexive et transitive alors R est asymétrique.
    \end{rdem}
\end{demonstration}

\begin{proposition}
    R est négativement transitive ssi
    \begin{align*}
        \forall x, y, z \in X, \; xRz \implies xRy \lor yRz
    \end{align*}
\end{proposition}

\begin{demonstration}
    Utilisons la contraposée du négativement transitive \par (Rappel la contraposée de $P \implies Q$ est $\neg Q \implies \neg P$),

    \begin{align*}
        &\forall x, y, z \in X, \; \neg(\neg(xRz)) \implies \neg(\neg xRy \land \neg yRz) \\
        &\equiv \quad \forall x, y, z \in X, \; xRz \implies xRy \lor yRz
    \end{align*}

    \begin{rdem}
        Ainsi R est négativement transitive ssi
        \begin{align*}
            \forall x, y, z \in X, \; xRz \implies xRy \lor yRz
        \end{align*}
    \end{rdem}
\end{demonstration}

\begin{proposition}
    Si R est complète alors R est réflexive 
\end{proposition}
Démontré facilement par définition en prenant un x et un $y = x$.


% End of 2024-09-19-CM-3.tex

% Begin of 2024-09-26-CM-4.tex

\subsection{Opérations sur les relations}

Puisque une relation R sur X est un sous ensemble de $X \times X$, on peut facilement utiliser
des opérations ensemblistes.

\begin{definition}
    Étant donné deux relation $R_1$ et $R_2$ sur un ensemble X.

    \item $\bullet$ la relation \textbf{complémentaire} de $R_1$, la relation binaire $R_1^c$ sur X telle que
    
    $\forall x, y \in X, \; x R_1^c y \text{ si } \neg(x R_1 y)$
    \item $\bullet$ la \textbf{réunion} de $R_1$ et $R_2$ est la relation binaire $R_1 \cup R_2$ sur X telle que
    
    $\forall x, y \in X, \; x R_1 \cup R_2 y \text{ si } x R_1 y \lor x R_2 y$
    \item $\bullet$ l'\textbf{intersection} de $R_1$ et $R_2$ est la relation binaire $R_1 \cap R_2$ telle que
    
    $\forall x, y \in X, \; x R_1 \cap R_2 y \text{ si } x R_1 y \land x R_2 y$
    \item $\bullet$ la relation $R_1$ est \textbf{compatible} avec $R_2$ si
    
    $\forall x, y \in X, \; x R_1 y \implies x R_2 y$ ou de manière équivalente $R_1 \subset R_2$
    \item $\bullet$ la relation \textbf{réciproque} (ou duale, inverse) de $R_1$,
    la relation binaire $R_1^{-1}$ sur X telle que
    
    $\forall x, y \in X, \; y R_1^{-1} x \text{ si } x R_1 y$
    \item $\bullet$ la \textbf{composée} de $R_1$ et $R_2$,
    la relation binaire $R_1 \circ R_2$ sur X telle que

    $\forall x, y \in X, \; x R_1 \circ R_2 y \text{ si }\exists z \in X, x R_2 z \land z R_1 y$
\end{definition}

\subsection{Relations d'équivalence}

\begin{proposition}
    L'intersection $R_1 \cap R_2$ de deux relations d'équivalences $R_1$ et $R_2$ sur un ensemble X est une relation d'équivalence.
\end{proposition}

\begin{demonstration}
    \item $\bullet$ \textbf{Réflexive} car $\forall x \in X, \; x R_1 x \land x R_2 x$, ainsi $x R_1 \cap R_2 x$ pour tout $x \in X$.
    \item $\bullet$ \textbf{Symétrie} car $\forall x, y \in X, \; (x R_1 y \land y R_1 x) \land (x R_2 y \land y R_2 x)$, ainsi $\forall x, y \in X, \; x R_1 y \land x R_2 y$ ce qui implique que $y R_1 x \land y R_2 x$
    soit $\forall x, y \in X, \; (x R_1 y \land y R_2 x) \land (x R_2 y \land y R_1 x)$.
    \item $\bullet$ \textbf{Transitive} car $\forall x, y \in X, x R_1 y \land x R_2 y \land y R_1 z \land y R_2 z \implies xR_1z \land xR_2z \implies x R_1 \cap R_2 z$
    
    \begin{rdem}
        $R_1 \cap R_2$ est Réflexive, Symétrique, Transitive donc c'est une relation d'équivalence.
    \end{rdem}
\end{demonstration}


% End of 2024-09-26-CM-4.tex

% Begin of 2024-10-03-CM-5.tex

Soit $x \in X$ l'ensemble $\{y \in X \mid xRy\}$ est appelé classe d'équivalence
de x notée $C_x$.

\begin{example}
    "=" sur $\N$

    \item $\bullet$ $\forall a \in \N, a = a$ (Réflexive)
    \item $\bullet$ $\forall a, b \in \N, a = b \implies b = a$ (Symétrique)
    \item $\bullet$ $\forall a, b, c \in \N, a = b \land b = c \implies a = c$ (Transitive)

    "=" est une relation d'équivalence sur $\N$

    $C_2 = \{y \in \N, 2 = y\} = \{2\}$
\end{example}

$\{C_x \mid x \in X\}$ est l'ensemble quotient de X par R noté X/R.

\begin{proposition}
    Soit R une relation d'équivalence sur X, X/R forme une partition de X,
    i.e.

    \item $\bullet$ $\forall x, y \in X, C_x \cap C_y = \emptyset \text{ ou } C_x = C_y$
    \item $\bullet$ $X = \bigcup_{x \in X} C_x$
\end{proposition}

\begin{demonstration}
    Nous allons montrer que

    $\forall x, y \in X, \; xRy \implies C_x = C_y$

    $\forall x, y \in X, \; \neg xRy \implies C_x \cap C_y = \emptyset$

    Soient $x, y \in X$

    so $xRy$ soit $z \in C_x$ alors xRz
    %todo
\end{demonstration}

\begin{remark}
    Pour une relation binaire il est toujours vrai que $\forall x, y \in X, xRy \lor \neg xRy$
\end{remark}

%todo


% End of 2024-10-03-CM-5.tex

% Begin of 2024-10-17-CM-6.tex

\subsection{Ordre faible et ordre total}

Soit R une relation binaire sur l'ensemble X.

\vspace{1em}

\noindent
On définit I et S sur X par

$\forall x \in X, y \in X, xIy \text{ si } xRy \land yRx$

$\forall x \in X, y \in X, xSy \text{ si } xRy \land \neg yRx$

\begin{example}
    $A = \{a, b, c\}$

    $R = \{(a, b), (b, a), (a, c), (b, c)\}$

    $I = \{(a, b), (b, a)\}$

    $S = \{(a, c), (b, c)\}$
\end{example}

\begin{proposition}
    si R est un ordre faible sur X, alors

    \item \textbf{1.} I est une relation d'équivalence
    \item \textbf{2.} S est irréflexive et transitive
\end{proposition}

\begin{demonstration}
    I relation d'équivalence:

    \item I reflexive
    
    Soit $x \in X, xIx \iff xRx \land xRX \iff xRx$ vrai car R est complète

    \item I symétrique
    
    Soient $x \in X, y \in X, xIy \implies xRy \land yRx \implies yRx \land xRy \implies yIx$

    \item I transitive
    
    Soient $x, y, z \in X, xIy \land yIz \implies xRy \land yRx \land yRz \land zRy \implies xRy \land yRz \land zRy \land yRx \implies xRz \land zRx \implies xIz$
\end{demonstration}

On définit $R^*$ sur $X/I$ par 

$\forall C_x \in X/I, C_y \in X/I, C_xR^*C_y \text{ lorsque } xRy$

$R^*$ sur X/I est la réduction (relation quotient) de R sur X

\begin{proposition}
    Si R est un ordre faible alors $R^*$ est un ordre total sur X/I
\end{proposition}

\begin{demonstration}
    \item $R^*$ antisymétrique
    
    Soient $C_x, C_y \in X/I, C_x R^* C_y \land C_y R^* C_x \implies C_x = C_y \implies xIy \implies y \in C_x \implies C_x = C_y$

    \item $R^*$ transitive
    
    Soient $C_x, C_y, C_z \in X/I$

    $C_x R^* C_y \land C_y R^* C_z \implies xRy \land yRz \implies xRz \implies C_x R^* C_z$


    \item $R^*$ complète
    
    $C_x, C_y \in X/I, C_xR^*C_y \lor C_yR^*C_x$
    car R complète
\end{demonstration}

\subsubsection{Irréflexive et transitive}

voir plus tard




% End of 2024-10-17-CM-6.tex

% Begin of 2024-10-24-CM-7.tex

\subsection{Ordre total, Ordre partiel}

R ordre partiel sur X

Soit $x \in X$, x est:

\begin{itemize}
    \item un élément maximal si $\forall y \in X\backslash\{x\}, \neg (yRx)$
    \item le plus grand élément si $\forall y \in X, xRy$
    \item un élément minimal si $\forall y \in X\backslash\{x\}, \neg (xRy)$
    \item le plus petit élément si $\forall y \in X, yRx$
\end{itemize}

\begin{proposition}
    Il y a au plus un plus grand (resp. petit) élément.
\end{proposition}

\begin{demonstration}
    Soit x et x' deux plus grand (resp. petit) éléments avec $x \neq x'$.

    Ainsi $\forall y \in X, xRy \text{ et } x'Ry \implies xRx' \text{ et } x'Rx \implies x = x'$

    Absurde car on a supposé $x \neq x'$
\end{demonstration}

Construction de diagramme de Hasse
\begin{itemize}
    \item si xRy : x au dessus de y
    \item et si x \textbf{couvre} y : il n'existe pas $z \in X\backslash\{x, y\}$ tel que $xRz \land zRy$
    \item alors il y a une arête qui relie x et y
\end{itemize}


% End of 2024-10-24-CM-7.tex

% Begin of 2024-11-07-CM-8.tex

\begin{definition}
    Une \textbf{chaîne} est un ensemble d'éléments de X totalement ordonné
\end{definition}

\begin{definition}
    Une \textbf{antichaîne} si $\forall x, xRy \lor yRx \implies x = y$
\end{definition}

Soit c le nombre minimal de chaînes pour partitionner X

Soit A une anti-chaîne de cardinal maximal a

\begin{remark}
    partition avec le plus grand nombre de chaînes : \{\{a\}, \{b\}, ..., \{g\}\}
\end{remark}

\begin{proposition}
    X ensemble partiellement ordonné par R.

    Le nombre d'éléments d'une antichaîne de cardianl maximal (a) est
    égale au nombre minimum de chaînes pour partitionner X.
\end{proposition}

\begin{demonstration}
    Preuve par récurrence sur $|x|$

    Init. $|x| = 1$

    X est une chaîne et une antichaîne a = 1

    on partitionne X en 1 chaîne c = a = 1

    here. Suppose que ça marche pour tous jusqu'a n

    2 cas:

    (a) si X contient une antichaîne de cardinal a contenant au moins un element non mimimal et au moins un element maximal

    (b) si X contient une antichaîne de cardinal a contenant que des elements maximaux ou minimaux

    (a) Soient $H = \{x \in X | \exists z \in A, xRz\}$ et $B = \{x \in X | \exists z \in A, zRx\}$

    du fait de (a) $\exists w \in A$ non maximal implique $\exists y \in X, yRw$ et $y \notin B$ donc $|B| \leq n - 1$
    donc B peut être partitionné en a chaînes
\end{demonstration}


% End of 2024-11-07-CM-8.tex

% Begin of 2024-11-14-CM-9.tex

Suite de démonstration

\begin{demonstration}

\end{demonstration}

\section{Préférence et utilité}

\subsection{Définition}

\begin{definition}
    Soit $(X, \gtrsim)$ une structure de préférence (voir chap 2)

    $\gtrsim$ est une relation binaire sur X
\end{definition}

On veut une application $f: (X \gtrsim) \rightarrow (\R, \geq)$ telle que $x \gtrsim y \iff f(x) \geq f(y)$

f est une fonction d'utilité.

Soient $f: (X, R_1) \rightarrow (Y, R_2)$, On dira que f est isotone si $\forall x, z \in X, xR_1 z \implies f(x)R_2 f(z)$

Un homomorphisme de $(X, R_1)$ vers $(Y, R_2)$ si $\forall x, z \in X, x R_1 z \iff f(x)R_2 f(z)$

\begin{remark}
    Homomorphisme implique isotone
\end{remark}


% End of 2024-11-14-CM-9.tex

% Begin of 2024-11-21-CM-10.tex


\subsection{Représentation de type $(X, \succeq) \rightarrow (\R, \geq)$: cas fini}

Soit $\succ$ sur X

\begin{proposition}
    Soit X fini, une codition nécessaire et suffisante (C.N.S) pour qu'existe une fonction

    $\exists f: (X, \succ) \rightarrow (\R, \gt)$ tel que $x \succ y \iff f(x) \gt f(y)$ si et seulement si $\succ$ est un ordre faible stricte.
\end{proposition}

\begin{demonstration}
    Preuve condition nécessaire, on montre $\neg Q \implies \neg P \equiv P \implies Q$

    \item Asymétrie: Soient $x, y \in X$ tel que $x \succ y \implies f(x) \lt f(y) \implies \neg f(x) \lt f(y) \implies \neg y \succ x$ donc asymétrique.

    \item Negativement transitive: Soient $x, y, z \in X$ tels que $\neg x \succ y \land \neg y \succ z \implies \neg f(x) \gt f(y) \land \neg (y) \gt f(z) \implies f(x) \leq f(y) \land f(y) \leq f(z) \implies f(x) \leq f(z) \implies \neg (x) \gt f(z) \implies \neg x \succ z$
    Finalement negativement transitive.

    Preuve condition suffisante, on montre $Q \implies P$, Soit $\succ$ un o.f.s sur X

    Soit $x \in X$ on définit:$\curlyvee x = \{y \in X \mid x \succ y\}$
    
    Soit $f(x) = Card(\curlyvee x)$ si $X = \Z$
\end{demonstration}



% End of 2024-11-21-CM-10.tex



\end{document}
