\documentclass[a4paper, 12pt]{article}
\usepackage{amsmath, amssymb, amsthm}
\usepackage{geometry}
\usepackage{tcolorbox}
\geometry{hmargin=2.5cm, vmargin=2.5cm}

\renewcommand*{\today}{19 septembre 2024}

\title{Science Decision | CM: 3}
\author{Par Lorenzo}
\date{\today}

\newtheorem{theorem}{Théorème}[section]
\newtheorem{definition}{Définition}[section]
\newtheorem{example}{Example}[section]
\newtheorem{remark}{Remarques}[section]
\newtheorem{lemme}{Lemme}[section]
\newtheorem{corollaire}{Corollaire}[section]

\newtheorem{_proposition}{Proposition}[section]
\newenvironment{proposition}[1][]{
    \begin{_proposition}[#1]~\par
    \vspace*{0.5em}
}{%
    \end{_proposition}%
}

\newtheorem{_proprietes}{Propriétés}[section]
\newenvironment{proprietes}[1][]{
        \begin{_proprietes}[#1]~\par
        \vspace*{0.5em}
}{%
        \end{_proprietes}%
}

\newenvironment{rdem}[1][]{
    \begin{tcolorbox}[colframe=black, colback=white!10, sharp corners]
        #1
}{%
    \end{tcolorbox}
     
}

\newtheorem{_demonstration}{Démonstration}[section]
\newenvironment{demonstration}[1][]{
    \begin{_demonstration}[#1]~\par
    \vspace*{0.5em}
}{%
    \end{_demonstration}%
    \qed%
}

\newtheorem*{_demonstration*}{Démonstration}
\newenvironment{demonstration*}[1][]{
    \begin{_demonstration*}[#1]~\par
    \vspace*{0.5em}
}{%
    \end{_demonstration*}%
    \qed%
}

\newenvironment{ldefinition}{
    \begin{definition}~\par
    \vspace*{0.5em}
    \begin{enumerate}
}{
        \end{enumerate}
        \end{definition}
}

\newenvironment{lexample}{
    \begin{example}~\par
    \vspace*{0.5em}
    \begin{enumerate}
}{
        \end{enumerate}
        \end{example}
}

\newtheorem{_methode}{Méthode}[section]
\newenvironment{methode}{
    \begin{_methode}~\par
    \vspace*{0.5em}
}{
        \end{_methode}
}

\def\N{\mathbb{N}}
\def\Z{\mathbb{Z}}
\def\Q{\mathbb{Q}}
\def\R{\mathbb{R}}
\def\C{\mathbb{C}}
\def\K{\mathbb{K}}
\def\k{\Bbbk}

\def\un{(u_n)_{n \in \N}}
\def\xn#1{(#1_n)_{n \in \N}}

\def\o{\overline}
\def\eps{\varepsilon}

% \funcdef{name}{domain}{codomain}{variable}{expression}
% name: Name of the function (e.g. f)
% domain: Domain of the function (e.g. \mathbb{R})
% codomain: Codomain of the function (e.g. \mathbb{R})
% variable: Variables of the function (e.g. x)
% expression: Expression of the function (e.g. x^2)
\newcommand{\funcdef}[5]{%
    #1 :
    \begin{cases}
        #2 \rightarrow #3 \\
        #4 \mapsto #5
    \end{cases}
}

\newcommand{\lt}{\ensuremath <}
\newcommand{\gt}{\ensuremath >}

\begin{document}

\maketitle

\begin{proposition}
    Si R est asymétrique alors R est réflexive.
\end{proposition}

\noindent
Démontré trivialement par les définitions d'asymétrie et de réflexivité.

\begin{proposition}
    Si R est irréflexive et transitive alors R est asymétrique.
\end{proposition}

\begin{demonstration}
    On suppose que R est irréflexive, transitive et non asymétrique.

    \vspace{0.5em}

    La non asymétrie se traduit par
    \begin{align*}
        \neg(\forall x, y \in X, \; xRy \implies \neg(yRx)) \quad \equiv \quad \exists x, y \in X, \; xRy \land yRx 
    \end{align*}

    Ainsi avec la non asymétrie et la transitivité on arrive à
    \begin{align*}
        &\exists x, y \in X, \; xRy \land yRx \quad \land \quad \forall x, y, z \in X, \; xRy \land yRz \implies xRz \\
        &\equiv \quad \exists x, y \in X, \; xRy \land yRx \implies xRx
    \end{align*}

    Ce qui est absurde car ça contredit l'irréflexivité !!!

    \begin{rdem}
        Donc Si R est irréflexive et transitive alors R est asymétrique.
    \end{rdem}
\end{demonstration}

\begin{proposition}
    R est négativement transitive ssi
    \begin{align*}
        \forall x, y, z \in X, \; xRz \implies xRy \lor yRz
    \end{align*}
\end{proposition}

\begin{demonstration}
    Utilisons la contraposée du négativement transitive \par (Rappel la contraposée de $P \implies Q$ est $\neg Q \implies \neg P$),

    \begin{align*}
        &\forall x, y, z \in X, \; \neg(\neg(xRz)) \implies \neg(\neg xRy \land \neg yRz) \\
        &\equiv \quad \forall x, y, z \in X, \; xRz \implies xRy \lor yRz
    \end{align*}

    \begin{rdem}
        Ainsi R est négativement transitive ssi
        \begin{align*}
            \forall x, y, z \in X, \; xRz \implies xRy \lor yRz
        \end{align*}
    \end{rdem}
\end{demonstration}

\begin{proposition}
    Si R est complète alors R est réflexive 
\end{proposition}
Démontré facilement par définition en prenant un x et un $y = x$.

\end{document}
