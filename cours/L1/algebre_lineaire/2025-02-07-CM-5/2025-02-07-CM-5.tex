\documentclass[a4paper, 12pt]{article}
\usepackage{amsmath, amssymb, amsthm, stmaryrd}
\usepackage{geometry}
\usepackage{pgfplots}
\usepackage{tcolorbox}
\geometry{hmargin=2.5cm, vmargin=2.5cm}

\renewcommand*{\today}{07 février 2025}

\title{Algebre Lineaire | CM: 5}
\author{Par Lorenzo}
\date{\today}

\newtheorem{theorem}{Théorème}[section]
\newtheorem{definition}{Définition}[section]
\newtheorem{example}{Example}[section]
\newtheorem{remark}{Remarques}[section]
\newtheorem{lemme}{Lemme}[section]
\newtheorem{corollaire}{Corollaire}[section]

\newtheorem{_proposition}{Proposition}[section]
\newenvironment{proposition}[1][]{
    \begin{_proposition}[#1]~\par
    \vspace*{0.5em}
}{%
    \end{_proposition}%
}

\newtheorem{_proprietes}{Propriétés}[section]
\newenvironment{proprietes}[1][]{
        \begin{_proprietes}[#1]~\par
        \vspace*{0.5em}
}{%
        \end{_proprietes}%
}

\newenvironment{rdem}[1][]{
    \begin{tcolorbox}[colframe=black, colback=white!10, sharp corners]
        #1
}{%
    \end{tcolorbox}
     
}

\newtheorem{_demonstration}{Démonstration}[section]
\newenvironment{demonstration}[1][]{
    \begin{_demonstration}[#1]~\par
    \vspace*{0.5em}
}{%
    \end{_demonstration}%
    \qed%
}

\newtheorem*{_demonstration*}{Démonstration}
\newenvironment{demonstration*}[1][]{
    \begin{_demonstration*}[#1]~\par
    \vspace*{0.5em}
}{%
    \end{_demonstration*}%
    \qed%
}

\newenvironment{ldefinition}{
    \begin{definition}~\par
    \vspace*{0.5em}
    \begin{enumerate}
}{
        \end{enumerate}
        \end{definition}
}

\newenvironment{lexample}{
    \begin{example}~\par
    \vspace*{0.5em}
    \begin{enumerate}
}{
        \end{enumerate}
        \end{example}
}

\newtheorem{_methode}{Méthode}[section]
\newenvironment{methode}{
    \begin{_methode}~\par
    \vspace*{0.5em}
}{
        \end{_methode}
}

\def\N{\mathbb{N}}
\def\Z{\mathbb{Z}}
\def\Q{\mathbb{Q}}
\def\R{\mathbb{R}}
\def\C{\mathbb{C}}
\def\K{\mathbb{K}}
\def\k{\Bbbk}

\def\un{(u_n)_{n \in \N}}
\def\xn#1{(#1_n)_{n \in \N}}

\def\o{\overline}
\def\eps{\varepsilon}

% \funcdef{name}{domain}{codomain}{variable}{expression}
% name: Name of the function (e.g. f)
% domain: Domain of the function (e.g. \mathbb{R})
% codomain: Codomain of the function (e.g. \mathbb{R})
% variable: Variables of the function (e.g. x)
% expression: Expression of the function (e.g. x^2)
\newcommand{\funcdef}[5]{%
    #1 :
    \begin{cases}
        #2 \rightarrow #3 \\
        #4 \mapsto #5
    \end{cases}
}

\newcommand{\lt}{\ensuremath <}
\newcommand{\gt}{\ensuremath >}

\begin{document}

\maketitle

\section{Espaces vectoriels}

\subsection{Espace vectoriel}

\begin{definition}
    Étant donné deux ensembles $E$ et $\K$, toute application de $\K \times E$ dans $E$ s'appelle loi de composition externe sur $E$ (à domaine opérateur $\K$).
\end{definition}

\begin{definition}
    On dit qu'un ensemble $E$ est un espace vectoriel sur un corps $\K$ s'il est muni d'une loi interne notée $+$
    et d'une loi externe notée $\bullet$ de $\K \times E$ dans $E$.
    
    $(\lambda, u) \mapsto \lambda \cdot u$ telles que:
    
    \begin{enumerate}
        \item $(E, +)$ est un groupe commutatif.
        \item $\forall (\lambda, \mu) \in \K^2, \forall (u, v) \in E^2$, on a:
        \begin{enumerate}
            \item $(\lambda + \mu) \bullet u = \lambda \bullet u + \mu u$
            \item $\lambda \bullet (u + v) = \lambda \bullet u + \lambda \bullet v$
            \item $\lambda \bullet (\mu \bullet u) = (\lambda \mu) \bullet u$
            \item $1 \bullet u = u$ ($1 \in \K$)
        \end{enumerate}
    \end{enumerate}

    Les éléments de $E$ sont appelés vecteurs et les éléments de $\K$ sont appelés scalaires.
    E est $\K$-espace vectoriel.
\end{definition}

\begin{definition}
    \begin{enumerate}
        \item La commutativé de $(E, +)$ découle des autres axiomes d'espace vectoriel.
        \item L'espace vectoriel nul est $E = \{0_e\}$.
    \end{enumerate}
\end{definition}

\end{document}
