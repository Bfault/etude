\documentclass[a4paper, 12pt]{article}
\usepackage{amsmath, amssymb, amsthm, stmaryrd}
\usepackage{geometry}
\usepackage{pgfplots}
\usepackage{tcolorbox}
\geometry{hmargin=2.5cm, vmargin=2.5cm}

\renewcommand*{\today}{24 janvier 2025}

\title{Algebre Lineaire | CM: 2}
\author{Par Lorenzo}
\date{\today}

\newtheorem{theorem}{Théorème}[section]
\newtheorem{definition}{Définition}[section]
\newtheorem{example}{Example}[section]
\newtheorem{remark}{Remarques}[section]
\newtheorem{lemme}{Lemme}[section]
\newtheorem{corollaire}{Corollaire}[section]

\newtheorem{_proposition}{Proposition}[section]
\newenvironment{proposition}[1][]{
    \begin{_proposition}[#1]~\par
    \vspace*{0.5em}
}{%
    \end{_proposition}%
}

\newtheorem{_proprietes}{Propriétés}[section]
\newenvironment{proprietes}[1][]{
        \begin{_proprietes}[#1]~\par
        \vspace*{0.5em}
}{%
        \end{_proprietes}%
}

\newenvironment{rdem}[1][]{
    \begin{tcolorbox}[colframe=black, colback=white!10, sharp corners]
        #1
}{%
    \end{tcolorbox}
     
}

\newtheorem{_demonstration}{Démonstration}[section]
\newenvironment{demonstration}[1][]{
    \begin{_demonstration}[#1]~\par
    \vspace*{0.5em}
}{%
    \end{_demonstration}%
    \qed%
}

\newtheorem*{_demonstration*}{Démonstration}
\newenvironment{demonstration*}[1][]{
    \begin{_demonstration*}[#1]~\par
    \vspace*{0.5em}
}{%
    \end{_demonstration*}%
    \qed%
}

\newenvironment{ldefinition}{
    \begin{definition}~\par
    \vspace*{0.5em}
    \begin{enumerate}
}{
        \end{enumerate}
        \end{definition}
}

\newenvironment{lexample}{
    \begin{example}~\par
    \vspace*{0.5em}
    \begin{enumerate}
}{
        \end{enumerate}
        \end{example}
}

\newtheorem{_methode}{Méthode}[section]
\newenvironment{methode}{
    \begin{_methode}~\par
    \vspace*{0.5em}
}{
        \end{_methode}
}

\def\N{\mathbb{N}}
\def\Z{\mathbb{Z}}
\def\Q{\mathbb{Q}}
\def\R{\mathbb{R}}
\def\C{\mathbb{C}}
\def\K{\mathbb{K}}
\def\k{\Bbbk}

\def\un{(u_n)_{n \in \N}}
\def\xn#1{(#1_n)_{n \in \N}}

\def\o{\overline}
\def\eps{\varepsilon}

% \funcdef{name}{domain}{codomain}{variable}{expression}
% name: Name of the function (e.g. f)
% domain: Domain of the function (e.g. \mathbb{R})
% codomain: Codomain of the function (e.g. \mathbb{R})
% variable: Variables of the function (e.g. x)
% expression: Expression of the function (e.g. x^2)
\newcommand{\funcdef}[5]{%
    #1 :
    \begin{cases}
        #2 \rightarrow #3 \\
        #4 \mapsto #5
    \end{cases}
}

\newcommand{\lt}{\ensuremath <}
\newcommand{\gt}{\ensuremath >}

\begin{document}

\maketitle

%todo

\begin{remark}
    \begin{itemize}.
        \item Le produit de deux matrices n'est pas commutatif.
        \item On peut multiplier deux matrices non nulls et obtenir une matrice nulle.
    \end{itemize}
    Ainsi on dit que l'ensemble $\mathcal{M}(\K)$ possède des diviseurs de zéro.
\end{remark}

\subsection{Partitionnement par blocs}

\begin{definition}
    Les sous matrices permettent un partitionnement par blocs.
    Soit $A \in \mathcal{M}_{m,n}(\K)$, on peut écrire $A$ sous la forme:
    $$
    A = \left(
    \begin{array}{c|c|c}
        A_{1,1} & \cdots & A_{1,p} \\
        \hline
        \vdots & \ddots & \vdots \\
        \hline
        A_{q,1} & \cdots & A_{q,p}
    \end{array}
    \right)
    $$
    où pour chaque $i \in \{1, \cdots r\}$, les matrice $A_{ik}$ ont le même nombre de lignes pour tout $k \in \{1, \cdots m\}$.
    %todo
\end{definition}

\subsection{Inverse et puissance}

\begin{definition}
    Soit $A \in \mathcal{M}_n(\K)$, on note $A^0 = Id_n$ et pour $k \in \N^*$, $A^k$ est défini par récurrence par : $A^k = AA^{k-1}$
\end{definition}

\begin{proprietes}
    Soit $A \in \mathcal{M}_n(\K)$ et $k \in \N^*$, alors $A^k = A \times A \times \cdots \times A$ ($k$ fois). où le facteur A intervient k fois.
\end{proprietes}

\begin{definition}
    Soit $A \in \mathcal{M}_n(\K)$, on dit que $A$ est inversible s'il existe $B \in \mathcal{M}_n(\K)$ tel que $AB = BA = Id_n$.
    Dans ce cas, on note $B = A^{-1}$.
    On note $\mathcal{GL}_n(\K)$ l'ensemble des matrices inversibles de $\mathcal{M}_n(\K)$.
\end{definition}

\begin{remark}
    La notation $\mathcal{GL}$ veut dire "Groupe Linéaire" et provient du fait que l'ensemble des matrices inversibles muni de la multiplication des matrices est un groupe.
\end{remark}

\begin{proposition}
    Soit $A \in \mathcal{M}_n(\K)$, si $A$ est inversible, il existe une unique matrice $B \in \mathcal{M}_n(\K)$ telle que $AB = BA = Id_n$.
    Cette matrice s'appelle l'inverse de $A$ et est notée $A^{-1}$.
\end{proposition}

\begin{demonstration}
    S'il existe deux matrices $B$ et $B'$ telles que $AB = BA = Id_n = AB' = B'A$ alors $B' = B'Id_n = B'(AB) = (B'A)B = Id_nB = B$
\end{demonstration}

\begin{proposition}
    Soit $A \in \mathcal{M}_n(\K)$, s'il existe une matrice $B \in \mathcal{M}_n(\K)$ telle que $AB = Id_n$, alors $A$ est inversible et $B = A^{-1}$.
    Il suffit donc de vérifier le produit d'un seul coté.
\end{proposition}

\begin{proposition}
    Soit $A, B \in \mathcal{M}_n(\K)$, si $AB$ est inversible alors $(AB)^{-1} = B^{-1}A^{-1}$.
\end{proposition}

\begin{demonstration}
    $(AB)(B^{-1}A^{-1}) = A(BB^{-1}A^{-1}) = AId_nA^{-1} = AA^{-1} = Id_n$
\end{demonstration}

\begin{proposition}
    Soit $A \in \mathcal{M}_n(\K), B \in \mathcal{M}_{n,p}(\K)$ possède une unique solution $X = A^{-1}B$
\end{proposition}

\begin{demonstration}
    $AX = B \implies A^{-1}(AX) = A^{-1}B \implies (A^{-1}A)X = A^{-1}B \implies X = A^{-1}B$
\end{demonstration}

\subsection{Système linéaire}

\begin{definition}
    Soit $n, p \in \N^*$. Un système linéaire de $n$ équations linéaire à $p$ inconnues (à coefficients dans $\K$) s'écrit:
    $$
    \begin{cases}
        a_{1,1}x_1 + a_{1,2}x_2 + \cdots + a_{1,p}x_p = b_1 \\
        a_{2,1}x_1 + a_{2,2}x_2 + \cdots + a_{2,p}x_p = b_2 \\
        \vdots \\
        a_{n,1}x_1 + a_{n,2}x_2 + \cdots + a_{n,p}x_p = b_n \\
    \end{cases}
    $$
    Les coefficients $a_{i,j}$ et $b_i$ pour $i \in \{1, \cdots, n\}$ et $j \in \{1, \cdots, p\}$ sont des éléments de $\K$.
    Les coefficients $x_1, \cdots, x_p$ sont les inconnues du système.

    On appelle solution du système linéaire tout $p$-uplet $(x_1, \cdots, x_p) \in \K^p$ tel que les équations du système sont vérifiées.
\end{definition}

%TODO

\begin{definition}
    Un système linéaire de n équations à n équations est régulier ou de Cramer, s'il possède une unique solution.
    Un système linéaire de n équations linéaires à p inconnues est dit compatible quand il a au moins une solution.
\end{definition}

\begin{definition}
    Deux système linéaire (L1) et (L2) à p inconnues $x_1, \cdots, x_p$ sont équivalents si $S(L1) = S(L2)$.
\end{definition}

\begin{definition}
    Soit $n, p \in \N^*$, $A \in \mathcal{M}_{n,p}(\K)$ et $\alpha, \lambda \in \K, \alpha \neq 0$
    \begin{enumerate}
        \item Notons A' la matrice obtenue à partir de A en multipliant par la $\alpha$ la qième ligne de A, $q \in \{1, \cdots, n\}$. Les autres lignes restant inchangées.
        \item Notons A'' (resp A''') la matrice obtenue à partir de A en échangeant les lignes q et k de A (resp. en ajotant à la qième ligne de A le produit de $\lambda$ de la kième ligne de A).
        $q, k \in \{1, \cdots, n\}$, $q \neq k, n \geq 2$, les autres lignes restant inchangées.
        On dit que les matrices A', A'', A''' se déduisent de A par opération élémentaire sur les lignes.
    \end{enumerate}
\end{definition}

%TODO

\end{document}
