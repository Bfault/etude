\documentclass[a4paper, 12pt]{article}
\usepackage{amsmath, amssymb, amsthm, stmaryrd}
\usepackage{geometry}
\usepackage{pgfplots}
\usepackage{tcolorbox}
\geometry{hmargin=2.5cm, vmargin=2.5cm}

\renewcommand*{\today}{29 janvier 2025}

\title{Algebre Lineaire | CM: 3}
\author{Par Lorenzo}
\date{\today}

\newtheorem{theorem}{Théorème}[section]
\newtheorem{definition}{Définition}[section]
\newtheorem{example}{Example}[section]
\newtheorem{remark}{Remarques}[section]
\newtheorem{lemme}{Lemme}[section]
\newtheorem{corollaire}{Corollaire}[section]

\newtheorem{_proposition}{Proposition}[section]
\newenvironment{proposition}[1][]{
    \begin{_proposition}[#1]~\par
    \vspace*{0.5em}
}{%
    \end{_proposition}%
}

\newtheorem{_proprietes}{Propriétés}[section]
\newenvironment{proprietes}[1][]{
        \begin{_proprietes}[#1]~\par
        \vspace*{0.5em}
}{%
        \end{_proprietes}%
}

\newenvironment{rdem}[1][]{
    \begin{tcolorbox}[colframe=black, colback=white!10, sharp corners]
        #1
}{%
    \end{tcolorbox}
     
}

\newtheorem{_demonstration}{Démonstration}[section]
\newenvironment{demonstration}[1][]{
    \begin{_demonstration}[#1]~\par
    \vspace*{0.5em}
}{%
    \end{_demonstration}%
    \qed%
}

\newtheorem*{_demonstration*}{Démonstration}
\newenvironment{demonstration*}[1][]{
    \begin{_demonstration*}[#1]~\par
    \vspace*{0.5em}
}{%
    \end{_demonstration*}%
    \qed%
}

\newenvironment{ldefinition}{
    \begin{definition}~\par
    \vspace*{0.5em}
    \begin{enumerate}
}{
        \end{enumerate}
        \end{definition}
}

\newenvironment{lexample}{
    \begin{example}~\par
    \vspace*{0.5em}
    \begin{enumerate}
}{
        \end{enumerate}
        \end{example}
}

\newtheorem{_methode}{Méthode}[section]
\newenvironment{methode}{
    \begin{_methode}~\par
    \vspace*{0.5em}
}{
        \end{_methode}
}

\def\N{\mathbb{N}}
\def\Z{\mathbb{Z}}
\def\Q{\mathbb{Q}}
\def\R{\mathbb{R}}
\def\C{\mathbb{C}}
\def\K{\mathbb{K}}
\def\k{\Bbbk}

\def\un{(u_n)_{n \in \N}}
\def\xn#1{(#1_n)_{n \in \N}}

\def\o{\overline}
\def\eps{\varepsilon}

% \funcdef{name}{domain}{codomain}{variable}{expression}
% name: Name of the function (e.g. f)
% domain: Domain of the function (e.g. \mathbb{R})
% codomain: Codomain of the function (e.g. \mathbb{R})
% variable: Variables of the function (e.g. x)
% expression: Expression of the function (e.g. x^2)
\newcommand{\funcdef}[5]{%
    #1 :
    \begin{cases}
        #2 \rightarrow #3 \\
        #4 \mapsto #5
    \end{cases}
}

\newcommand{\lt}{\ensuremath <}
\newcommand{\gt}{\ensuremath >}

\begin{document}

\maketitle

\begin{definition}
    Une matrice $A \in \mathcal{M}_{n,p}(\K)$ est dite sous forme échelonnée si:
    \begin{enumerate}
        \item toutes ses lignes non identiquement nulles sont situées au dessus de ses lignes identiquement nulles.
        \item chaque élément de tête d'une ligne (élément non nul le plus ç gauche d'une ligne non identiquement nulle) se trouve dans une colonne à droite de l'élément de tête de la ligne précédente.
    \end{enumerate}
\end{definition}

\begin{remark}
    La condition (2) implique que tous les éléments en dessous d'un élément de tête sont nuls.
\end{remark}

\begin{definition}[Échelonnement d'une matrice]
    Soit $A \in \mathcal{M}_{n,p}(\K)$. Il existe une matrice $E \in \mathcal{M}_{n}(\K)$, produit de matrice élémentaires, telle que la matrice $E \times A$ est échelonnée.
    Autrement dit, toute matrice est équivalente par rapport aux lignes à une matrice échelonnée.
\end{definition}

\begin{lemme}
    Soit $A \in \mathcal{M}_{n,1}(\K), n \geq 2$ et $i \in \{1, \cdots, n-1\}$, tels que l'un des coefficients $a_j$ pour indice $j \in \{i, \cdots, n\}$ soit non nul, alors
    il existe $E_A \in \mathcal{M}_n(\K)$ produit de matrice élémentaires.
    $a \in \K, a \neq 0$ tel que si:
    $$
    A = \left(
        \begin{array}{c}
            a_1 \\
            \vdots \\
            a_{i-1} \\
            a_i \\
            a_{i+1} \\
            \vdots \\
            a_n
        \end{array}
    \right)
    $$
    alors 
    $$
    E_A \times A = \left(
        \begin{array}{c}
            a_1 \\
            \vdots \\
            a_{i-1} \\
            a_i \\
            0 \\
            \vdots \\
            0
        \end{array}
    \right)
    $$

    %TODO
\end{lemme}

\begin{demonstration}
    Quitte à échanger la ligne i avec une ligne j en dessius (ce qui revient à faire le produit $E_{i,j} \times A$),
    on se ramène au cas $a = a_i \neq 0$, on fait les opérations:
    $I_j \leftarrow I_j - \frac{a_j}{a_i}I_i, j \in \{i + 1, \cdots, n\}$.

    %TODO
\end{demonstration}

%TODO jusqu'au rang

\end{document}
