\documentclass[a4paper, 12pt]{article}
\usepackage{amsmath, amssymb, amsthm}
\usepackage{geometry}
\usepackage{tcolorbox}
\geometry{hmargin=2.5cm, vmargin=2.5cm}

\renewcommand*{\today}{10 octobre 2024}

\title{Analyse | CM: 6}
\author{Par Lorenzo}
\date{\today}

\newtheorem{theorem}{Théorème}[section]
\newtheorem{definition}{Définition}[section]
\newtheorem{example}{Example}[section]
\newtheorem{remark}{Remarques}[section]
\newtheorem{lemme}{Lemme}[section]
\newtheorem{corollaire}{Corollaire}[section]

\newtheorem{_proposition}{Proposition}[section]
\newenvironment{proposition}[1][]{
    \begin{_proposition}[#1]~\par
    \vspace*{0.5em}
}{%
    \end{_proposition}%
}

\newtheorem{_proprietes}{Propriétés}[section]
\newenvironment{proprietes}[1][]{
        \begin{_proprietes}[#1]~\par
        \vspace*{0.5em}
}{%
        \end{_proprietes}%
}

\newenvironment{rdem}[1][]{
    \begin{tcolorbox}[colframe=black, colback=white!10, sharp corners]
        #1
}{%
    \end{tcolorbox}
     
}

\newtheorem{_demonstration}{Démonstration}[section]
\newenvironment{demonstration}[1][]{
    \begin{_demonstration}[#1]~\par
    \vspace*{0.5em}
}{%
    \end{_demonstration}%
    \qed%
}

\newtheorem*{_demonstration*}{Démonstration}
\newenvironment{demonstration*}[1][]{
    \begin{_demonstration*}[#1]~\par
    \vspace*{0.5em}
}{%
    \end{_demonstration*}%
    \qed%
}

\newenvironment{ldefinition}{
    \begin{definition}~\par
    \vspace*{0.5em}
    \begin{enumerate}
}{
        \end{enumerate}
        \end{definition}
}

\newenvironment{lexample}{
    \begin{example}~\par
    \vspace*{0.5em}
    \begin{enumerate}
}{
        \end{enumerate}
        \end{example}
}

\newtheorem{_methode}{Méthode}[section]
\newenvironment{methode}{
    \begin{_methode}~\par
    \vspace*{0.5em}
}{
        \end{_methode}
}

\def\N{\mathbb{N}}
\def\Z{\mathbb{Z}}
\def\Q{\mathbb{Q}}
\def\R{\mathbb{R}}
\def\C{\mathbb{C}}
\def\K{\mathbb{K}}
\def\k{\Bbbk}

\def\un{(u_n)_{n \in \N}}
\def\xn#1{(#1_n)_{n \in \N}}

\def\o{\overline}
\def\eps{\varepsilon}

% \funcdef{name}{domain}{codomain}{variable}{expression}
% name: Name of the function (e.g. f)
% domain: Domain of the function (e.g. \mathbb{R})
% codomain: Codomain of the function (e.g. \mathbb{R})
% variable: Variables of the function (e.g. x)
% expression: Expression of the function (e.g. x^2)
\newcommand{\funcdef}[5]{%
    #1 :
    \begin{cases}
        #2 \rightarrow #3 \\
        #4 \mapsto #5
    \end{cases}
}

\newcommand{\lt}{\ensuremath <}
\newcommand{\gt}{\ensuremath >}

\begin{document}

\maketitle

\begin{proprietes}
    Soient $\un$ et $\xn{v}$, tel que $lim_{n \to +\infty}v_n = +\infty$
    
    \item $$
    \lim_{n \to +\infty}\frac{1}{v_n} = 0
    $$

    \item Si $\un$ est minorée, alors
    $$
    \lim_{n \to +\infty}(u_n + v_n) = +\infty
    $$

    \item Si $\un$ est minorée par un réel strictement positif, alors
    $$
    \lim_{n \to +\infty}(u_n \times v_n) = +\infty
    $$

    \item Si $\lim_{n \to +\infty} u_n = 0$ et $\forall n \in \N, u_n \gt 0$, alors
    $$
    \lim_{n \to +\infty}\frac{1}{u_n} = +\infty
    $$
\end{proprietes}

\begin{theorem}
    Toute suite convergente est bornée.
\end{theorem}

\begin{demonstration}
    Soit une suite $\un$ qui converge vers l.

    $$
    \lim_{n \to +\infty}u_n = l \iff \forall \varepsilon \gt 0, \exists N \in \N, \forall n \in \N, n \geq N, |u_n - l| \lt \varepsilon
    $$

    On écrit $u_n = u_n - l + l$ ainsi $|u_n| = |(u_n - l) + l| \leq |u_n - l| + |l|$

    En outre $\forall n \geq N, |u_n| \leq \varepsilon + |l|$

    De plus $\forall n \lt N, |u_n| \leq \max(u_0, u_1, ..., u_{n-1})$

    \begin{rdem}
        Finalement $\forall n \in \N, |u_n| \leq \max(u_0, u_1, ..., u_{n-1}, \varepsilon + |l|)$
    \end{rdem}
\end{demonstration}

\begin{corollaire}
    Si la suite $\un$ est bornée et $\lim_{n \to \infty} v_n = 0$ alors $\lim_{n \to \infty} (u_n \times v_n) = 0$
\end{corollaire}

\subsubsection{Formes indéterminées}

On parle de formes indéterminées, lorsque à priori on ne peut rien dire sur la limite.

Il s'agit de limite de type:

\begin{itemize}
    \item $+\infty -\infty$
    \item $0 \times \infty$
    \item $\dfrac{\infty}{\infty}$, $\dfrac{0}{0}$, $a^\infty$
\end{itemize}

Dans ce cas il faut étudier plus précisement la suite.

Par exemple en utilisant les croissances comparées

\subsubsection{Quelques innégalités}

\begin{proprietes}
    \item Soient $\un$ et $\xn{v}$ deux suites convergentes telles que $\forall n \in \N, u_n \leq v_n$,
    $$
    \lim_{n \to +\infty}u_n \leq \lim_{n \to +\infty}v_n
    $$
    \item Soient $\un$ et $\xn{v}$ telles que $\forall n \in \N, u_n \leq v_n$,
    $$
    \lim_{n \to +\infty}u_n = +\infty \implies \lim_{n \to +\infty}v_n = +\infty
    $$
\end{proprietes}

\end{document}
