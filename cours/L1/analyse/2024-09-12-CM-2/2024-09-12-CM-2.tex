\documentclass[a4paper, 12pt]{article}
\usepackage{amsmath, amssymb, amsthm}
\usepackage{geometry}
\usepackage{tcolorbox}
\geometry{hmargin=2.5cm, vmargin=2.5cm}

\renewcommand*{\today}{12 septembre 2024}

\title{Analyse | CM: 2}
\author{Par Lorenzo}
\date{\today}

\newtheorem{theorem}{Théorème}[section]
\newtheorem{definition}{Définition}[section]
\newtheorem{example}{Example}[section]
\newtheorem{remark}{Remarques}[section]
\newtheorem{lemme}{Lemme}[section]
\newtheorem{corollaire}{Corollaire}[section]

\newtheorem{_proposition}{Proposition}[section]
\newenvironment{proposition}[1][]{
    \begin{_proposition}[#1]~\par
    \vspace*{0.5em}
}{%
    \end{_proposition}%
}

\newtheorem{_proprietes}{Propriétés}[section]
\newenvironment{proprietes}[1][]{
        \begin{_proprietes}[#1]~\par
        \vspace*{0.5em}
}{%
        \end{_proprietes}%
}

\newenvironment{rdem}[1][]{
    \begin{tcolorbox}[colframe=black, colback=white!10, sharp corners]
        #1
}{%
    \end{tcolorbox}
     
}

\newtheorem{_demonstration}{Démonstration}[section]
\newenvironment{demonstration}[1][]{
    \begin{_demonstration}[#1]~\par
    \vspace*{0.5em}
}{%
    \end{_demonstration}%
    \qed%
}

\newtheorem*{_demonstration*}{Démonstration}
\newenvironment{demonstration*}[1][]{
    \begin{_demonstration*}[#1]~\par
    \vspace*{0.5em}
}{%
    \end{_demonstration*}%
    \qed%
}

\newenvironment{ldefinition}{
    \begin{definition}~\par
    \vspace*{0.5em}
    \begin{enumerate}
}{
        \end{enumerate}
        \end{definition}
}

\newenvironment{lexample}{
    \begin{example}~\par
    \vspace*{0.5em}
    \begin{enumerate}
}{
        \end{enumerate}
        \end{example}
}

\newtheorem{_methode}{Méthode}[section]
\newenvironment{methode}{
    \begin{_methode}~\par
    \vspace*{0.5em}
}{
        \end{_methode}
}

\def\N{\mathbb{N}}
\def\Z{\mathbb{Z}}
\def\Q{\mathbb{Q}}
\def\R{\mathbb{R}}
\def\C{\mathbb{C}}
\def\K{\mathbb{K}}
\def\k{\Bbbk}

\def\un{(u_n)_{n \in \N}}
\def\xn#1{(#1_n)_{n \in \N}}

\def\o{\overline}
\def\eps{\varepsilon}

% \funcdef{name}{domain}{codomain}{variable}{expression}
% name: Name of the function (e.g. f)
% domain: Domain of the function (e.g. \mathbb{R})
% codomain: Codomain of the function (e.g. \mathbb{R})
% variable: Variables of the function (e.g. x)
% expression: Expression of the function (e.g. x^2)
\newcommand{\funcdef}[5]{%
    #1 :
    \begin{cases}
        #2 \rightarrow #3 \\
        #4 \mapsto #5
    \end{cases}
}

\newcommand{\lt}{\ensuremath <}
\newcommand{\gt}{\ensuremath >}

\begin{document}

\maketitle

\section{$\sqrt{2}$ n'est pas un rationnel}

Il existe des nombres qu'on ne peut pas écrire sous la forme d'une fraction de deux entiers, on les nomme les irrationnels.
Ils apparaissent naturellement (e.g. La diagonale d'un carré de longeur 1).

\begin{proposition} \label{propal}
    Pour la suite nous avons besoin de démontrer que 
    \begin{align*}
        p \in \Z, \; 2 \mid p^2 \implies 2 \mid p
    \end{align*}

    si $p^2$ est pair alors $p$ est pair

\end{proposition}

\begin{demonstration}
    Supposons que le carré d'un nombre impair est pair:

    \begin{align*}
        \forall k \in \Z, \; p = 2k + 1 \quad \iff \quad p^2 =&\; (2k + 1)^2 \\
         =&\; (2k)^2 + 2*2k*1 + 1^2 \\
         =&\; 4k^2 + 4k + 1 \\
         =&\; 2(2k^2 + 2k) + 1
    \end{align*}

    
    Ici $p^2$ est impair, Absurde $p^2$ ne peut pas être à la fois pair et impair!
    \begin{rdem}
        Donc si le carré d'un entier relatif $(p^2)$ est pair p est aussi pair.
    \end{rdem}

\end{demonstration}

\begin{proposition}
    $\sqrt{2} \in \R \backslash \Q$ ($\sqrt{2}$ est irrationnel).
\end{proposition}

\begin{demonstration}
    Supposons que $\sqrt{2}$ est un rationnel, c'est à dire,
    
    \begin{align*}
        \exists \; p \in \Z, q \in \N^*, \; pgcd(p, q) = 1, \quad \sqrt{2}=\dfrac{p}{q}
    \end{align*}

    Ici, $pgcd(p, q) = 1$ signifie que $p$ et $q$ sont premiers entre eux. Cette condition est importante pour assurer que la fraction est irréductible.


    \begin{align*}
        \sqrt{2} = \dfrac{p}{q} \quad \iff& \quad 2 = \dfrac{p^2}{q^2} \\
        \iff& \quad p^2 = 2q^2
    \end{align*}

    \vspace{0.5em}
    %todo check pourquoi \ref{propal} fait de la demer
    Ici on peut voir que $p^2$ est pair et grace à \textbf{(la proposition 1.1)} on sait que $p$ l'est aussi.
    Donc on peut réecrire $p$ par $\exists k \in \Z, \; 2k = p$
    et ainsi remplacer $p^2$.

    \nobreak

    \begin{align*}
        2q^2 = p^2 \quad \iff& \quad 2q^2 = (2k)^2 \\
        \iff& \quad q^2 = \dfrac{4p^2}{2} \\
        \iff& \quad q^2 = 2p^2
    \end{align*}

    Donc $q^2$ est pair et $q$ également.

    Finalement 2 divise p et q est absurde car p et q sont premier entre eux, ils ne peuvent pas être tout les deux multiple de 2.
    
    \begin{rdem}
        L'hypothèse de départ $\sqrt{2} \in \R \backslash \Q$ est fausse, ainsi $\sqrt{2}$ est irrationnel.
    \end{rdem}

\end{demonstration}

\section{Propriété de $\R$}

On nomme $\R$ l'ensemble des nombres réels, il contient $\N$, $\Z$, $\Q$ et $\R \backslash \Q$.

$\overline{\R} = \R \cup \{-\infty; +\infty\}$ (les réels et l'infini)

\subsection{Règle de calcul}

Soient a, b et c des nombres réels quelconques.
On note $+$ l'addition et $\times$ la multiplication.

\vspace*{1em}

\begin{proprietes}
    \item l'associativité.
        \begin{flalign*}
            &(a + b) + c = a + (b + c)&& \\
            &(a \times b) \times c = a \times (b \times c)&&
        \end{flalign*}
    \item l'élement neutre.
        \begin{flalign*}
            &\exists e \in \R, \forall x \in \R, \; e + x = x + e = x&& \\
            &\exists e' \in \R, \forall x \in \R, \; e' \times x = x \times e' = x&&
        \end{flalign*}
        Pour l'addition et la multiplication dans $\R$, leur élément neutre est 0 et 1 respectivement.
        e et e' seront pour la suite, les élements neutres de l'addition et de la multiplication respectivement.
    \item l'élement inverse.
        \begin{flalign*}
            &\exists i \in \R, \forall x \in \R, \; i + a = a + i = e&& \\
            &\exists i \in \R^*, \forall x \in \R^*, \; i \times a = a \times i = e'&&
        \end{flalign*}
    \item commutativité.
        \begin{flalign*}
            &a + b = b + a&& \\
            &a \times b = b \times a&&
        \end{flalign*}
    \item distributivité de la multiplication par rapport à l'addition.
    \begin{flalign*}
        &(a + b) \times c = c \times (a + b) = ac + bc&& \\
    \end{flalign*}
\end{proprietes}

\begin{remark}
    $(\R, +, *)$ est un corps abélien/commutatif
\end{remark}

\subsection{L'ordre sur $\R$}

Soient a, b et c des nombres réels quelconques.

\begin{proprietes}
    \item réfléxivité.
        \begin{flalign*}
            a \leq a
        \end{flalign*}
    \item l'antisymmétrie.
        \begin{flalign*}
            a \leq b \text{ et } b \leq a \implies a = b
        \end{flalign*}
    \item transitivité.
        \begin{flalign*}
            a \leq b \text{ et } b \leq a \implies a = b
        \end{flalign*}
    \item comparabilité
    
        quelque soit a et b dans $\R$, on a toujours $a \leq b$ or $b \leq a$
\end{proprietes}

\noindent
Les propriétés 1, 2 et 3 signifie que $leq$ est une relation d'ordre,
ajouté la 4, on parle de relation d'ordre totale.

\begin{remark}
    On définit la relation d'ordre supérieur ou égale ($\geq$) par \break $a \geq b \iff b \leq a$
\end{remark}

\begin{remark}
    On définit la relation d'ordre strictement inférieur ($\lt$) par \break $a \lt b \iff a \leq b \text{ et } a \neq b$
\end{remark}

\begin{remark}
    On définit la relation d'ordre strictement supérieur ($\gt$) par \break $a \gt b \iff a \geq b \text{ et } a \neq b$
\end{remark}

\begin{remark}
    $\lt$ et $\gt$ ne vérifient pas 1 et 2, ils ne sont donc pas des relations d'ordres mais des relations d'ordres stricts.
\end{remark}

\begin{proprietes}
    \item $a \leq b \implies a + c \leq b + c$
    \item $a \leq b \text{ et } c \geq 0 \implies ac \leq bc$
    \item $ab = 0 \implies a = 0 \text{ ou } b = 0$
\end{proprietes}

On définit le maximum entre deux réels comme
\begin{equation*}
    max(a, b) =
    \begin{cases}
        a, & \text{si } b \leq a \\
        b, & \text{sinon}
    \end{cases}
\end{equation*}

\end{document}
