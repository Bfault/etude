\documentclass[a4paper, 12pt]{article}
\usepackage{amsmath, amssymb, amsthm}
\usepackage{geometry}
\usepackage{pgfplots}
\usepackage{tcolorbox}
\geometry{hmargin=2.5cm, vmargin=2.5cm}

\renewcommand*{\today}{14 novembre 2024}

\title{Analyse | CM: 9}
\author{Par Lorenzo}
\date{\today}

\newtheorem{theorem}{Théorème}[section]
\newtheorem{definition}{Définition}[section]
\newtheorem{example}{Example}[section]
\newtheorem{remark}{Remarques}[section]
\newtheorem{lemme}{Lemme}[section]
\newtheorem{corollaire}{Corollaire}[section]

\newtheorem{_proposition}{Proposition}[section]
\newenvironment{proposition}[1][]{
    \begin{_proposition}[#1]~\par
    \vspace*{0.5em}
}{%
    \end{_proposition}%
}

\newtheorem{_proprietes}{Propriétés}[section]
\newenvironment{proprietes}[1][]{
        \begin{_proprietes}[#1]~\par
        \vspace*{0.5em}
}{%
        \end{_proprietes}%
}

\newenvironment{rdem}[1][]{
    \begin{tcolorbox}[colframe=black, colback=white!10, sharp corners]
        #1
}{%
    \end{tcolorbox}
     
}

\newtheorem{_demonstration}{Démonstration}[section]
\newenvironment{demonstration}[1][]{
    \begin{_demonstration}[#1]~\par
    \vspace*{0.5em}
}{%
    \end{_demonstration}%
    \qed%
}

\newtheorem*{_demonstration*}{Démonstration}
\newenvironment{demonstration*}[1][]{
    \begin{_demonstration*}[#1]~\par
    \vspace*{0.5em}
}{%
    \end{_demonstration*}%
    \qed%
}

\newenvironment{ldefinition}{
    \begin{definition}~\par
    \vspace*{0.5em}
    \begin{enumerate}
}{
        \end{enumerate}
        \end{definition}
}

\newenvironment{lexample}{
    \begin{example}~\par
    \vspace*{0.5em}
    \begin{enumerate}
}{
        \end{enumerate}
        \end{example}
}

\newtheorem{_methode}{Méthode}[section]
\newenvironment{methode}{
    \begin{_methode}~\par
    \vspace*{0.5em}
}{
        \end{_methode}
}

\def\N{\mathbb{N}}
\def\Z{\mathbb{Z}}
\def\Q{\mathbb{Q}}
\def\R{\mathbb{R}}
\def\C{\mathbb{C}}
\def\K{\mathbb{K}}
\def\k{\Bbbk}

\def\un{(u_n)_{n \in \N}}
\def\xn#1{(#1_n)_{n \in \N}}

\def\o{\overline}
\def\eps{\varepsilon}

% \funcdef{name}{domain}{codomain}{variable}{expression}
% name: Name of the function (e.g. f)
% domain: Domain of the function (e.g. \mathbb{R})
% codomain: Codomain of the function (e.g. \mathbb{R})
% variable: Variables of the function (e.g. x)
% expression: Expression of the function (e.g. x^2)
\newcommand{\funcdef}[5]{%
    #1 :
    \begin{cases}
        #2 \rightarrow #3 \\
        #4 \mapsto #5
    \end{cases}
}

\newcommand{\lt}{\ensuremath <}
\newcommand{\gt}{\ensuremath >}

\begin{document}

\maketitle

\subsection{Limites d'une fonction}

\subsubsection{Définition}

\begin{definition}
    On dit qu'une fonction f admet une limite $l \in \R$ en $x_0$ si $\forall \eps \gt 0, \exists \delta \gt 0, |x - x_0| \lt \delta \implies |f(x) - l| \lt \eps$

    On note $\lim_{x \to x_0} f(x) = l$
\end{definition}

\begin{remark}
    \item On peut remplacer $\lt$ par $\leq$
    \item L'ordre est important, $\delta$ dépend de $\eps$
\end{remark}

\begin{definition}
    Soient f définie sur un intervalle I de $\R$, et $x_0 \in \R$ dans I ou aux extrémités de I.

    \item On dit que f admet par limite $+\infty$ en $x_0$ si $\forall M \gt 0, \exists \delta \gt 0, |x - x_0| \lt \delta \implies f(x) \geq M$
    \item On dit que f admet par limite $-\infty$ en $x_0$ si $\forall M \gt 0, \exists \delta \gt 0, |x - x_0| \lt \delta \implies f(x) \leq -M$
\end{definition}

\begin{definition}
    On dit que f admet une limite $l \in \R$ en $+\infty$ si $\forall \eps \gt 0, \exists n \gt 0, x \gt n \implies |f(x) - l| \lt \eps$
\end{definition}

\begin{definition}
    f admet une limite en $+\infty$ en $+\infty$ si $\forall M \gt 0, \exists m \gt 0, x \gt m \implies f(x) \gt M$
\end{definition}

% faire les graphs

\begin{definition}
    On appelle limite à droite en $x_0$ de f, la limite de f en $x_0$ restreinte aux valeurs $x \gt x_0$ et on note
    $\lim_{x \to x_0^+} f(x) = \lim_{\substack{x \to x_0 \\ x \gt x_0}}f(x)$
\end{definition}

\begin{remark}
    \item Si $x \gt x_0, |x - x_0| = x-x_0$ et $|x - x_0| \lt \delta$ devient $x_0 \lt x \lt x_0 + \delta$
    \item Si $x \lt x_0, |x - x_0| = -(x-x_0)$ et $|x - x_0| \lt \delta$ devient $x_0-\delta \lt x \lt x_0$
\end{remark}

\begin{proposition}
    \item Si f admet une limite en $x_0$ alors f admet une limite en $x_0^+$ et en $x_0^-$ et les limites coincident.
    \item Si une fonction admet une limite à gauche et une limite à droite en $x_0$ et qu'elles sont égales, alors f admet cette même limite en $x_0$.
\end{proposition}

\begin{demonstration}
    À faire (juste les définitions)
\end{demonstration}

\subsubsection{Propriétés}

\begin{theorem}
    Si f admet une limite, elle est unique.
\end{theorem}

\begin{demonstration}
    Pareil que pour les suite (supposer deux limites différentes puis absurde)
\end{demonstration}

\begin{corollaire}
    Si la limite à gauche est différente de la limite à droite, alors f n'admet pas de limite.
\end{corollaire}

\subsubsection{Règles de calcul}

Notons $\lim_{x \to x_0}f(x) = l, \lim_{x \to x_0}g(x) = l'$

\begin{itemize}
    \item $\forall \lambda \in \R, \lim_{x \to x_0}\lambda f(x) = \lambda l$
    \item $\lim_{x \to x_0}(f + g)(x) = l + l'$
    \item $\lim_{x \to x_0}(f \times g)(x) = l \times l'$
    \item Si $l \neq 0, \lim_{x \to x_0}\dfrac{g(x)}{f(x)} = \dfrac{l'}{l}$
    \item Si $f \leq g$ alors $l \leq l'$
\end{itemize}

\begin{remark}
    Si $f \lt g$ alors $l \leq l'$
\end{remark}

\begin{theorem}
    Théoreme des gendarmes

    Si $f \leq g \leq h$ alors $\lim_{x \to x_0}f(x) \leq \lim_{x \to x_0}g(x) \leq \lim_{x \to x_0}h(x)$
\end{theorem}

\begin{proprietes}
    On note $g \circ f$ la composition des fonctions f et g définie par $(g \circ f)(x) = g(f(x))$
\end{proprietes}

\subsection{Continitué en 1 point}

\begin{definition}
    On dit que f est continue en $x_0$ si f admet une limite en $x_0$ et cette limite vaut $f(x_0)$ autrement dit $\forall \eps \gt 0, \exists \delta \gt 0, |x - x_0| \lt \delta \implies |f(x) - f(x_0)| \lt \eps$
\end{definition}

\subsubsection{Règles de calcul}

Soient $f: I \to \R$ et $g: I \to \R$ continue dans $x_0$.

\begin{itemize}
    \item $\forall \lambda \in \R$, $\lambda f$ est continue en $x_0$
    \item $f + g$ est continue en $x_0$
    \item $f \times g$ est continue en $x_0$
    \item Si $f(x) \neq 0$ alors $\dfrac{g}{f}$ est continue en $x_0$
\end{itemize}

\begin{proposition}
    Soient $f: I \to \R$ et $g: J \to \R$ avec $f(I) \subset J$.

    Si f est continue en $x_0$ et g aussi alors $g \circ f$ est continue en $x_0$.

    $\lim_{x \to x_0}(g \circ f)(x) = (g \circ f)(x_0)$
\end{proposition}

\subsection{Prolongement par continuitué}

\begin{definition}
    Soit f une fonction définie sur l'intervalle I privé de $x_0$
    $f: I\\\{x_0\} \rightarrow \R$. On dit que f est prolongeable par continuité par
    continuité en $x_0$ so f admet une limite finie l en $x_0$.

    On note $\over{~}f: I \rightarrow \R$ le prolongement défini par

    $$
    \tilde{f} (x) =
    \begin{cases}
        f(x) si x \neq x_0 \\
        l si x = x_0
    \end{cases}
    $$
\end{definition}
%todo

\subsection{Continuité sur un intervalle}

\begin{definition}
    Soit $f: I \rightarrow \R$ est dite continue sur l'intervalle I si elle est continue en
    tout point de I.
\end{definition}

\subsection{Théoreme des valeurs intermédiaires}

\begin{theorem}
    Soit $f: [a, b] \rightarrow \R$ une fonction continue sur le segment [a, b].

    Alors pour toute valeurs y comprise entre f(a) et f(b), il existe un $c \in [a, b]$ tel que y = f(c)
\end{theorem}

\begin{demonstration}
    Comme f est continue sur [a, b], f est continue en tout point $c \in [a, b]$
    Autrement dit $\lim_{x \to c}(x) = f(c)$.

    Supposons que $f(a) \leq f(b)$. Alors $y \in [f(a), f(b)]$ signifie $f(a) \leq y \leq f(b) \iff f(a) \leq \lim_{x \to c}f(x) \leq f(b)$
\end{demonstration}

\begin{corollaire}
    Si $f: [a, b] \rightarrow \R$ continue et $f(a)f(b) \lt 0$ alors $\exists c \in ]a, b[$ tel que $f(c) = 0$
\end{corollaire}

\begin{corollaire}
    Si f est continue sur un intervalle I alors $f(I) = \{y = f(x) | x \in I\}$ est aussi un intervalle.
\end{corollaire}

\begin{remark}
    \item c n'est pas forcément unique.
    \item en général $f([a, b]) \neq [f(a), f(b)]$
\end{remark}

\end{document}
