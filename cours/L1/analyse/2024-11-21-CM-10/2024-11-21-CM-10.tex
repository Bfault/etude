\documentclass[a4paper, 12pt]{article}
\usepackage{amsmath, amssymb, amsthm}
\usepackage{geometry}
\usepackage{pgfplots}
\usepackage{tcolorbox}
\geometry{hmargin=2.5cm, vmargin=2.5cm}

\renewcommand*{\today}{21 novembre 2024}

\title{Analyse | CM: 10}
\author{Par Lorenzo}
\date{\today}

\newtheorem{theorem}{Théorème}[section]
\newtheorem{definition}{Définition}[section]
\newtheorem{example}{Example}[section]
\newtheorem{remark}{Remarques}[section]
\newtheorem{lemme}{Lemme}[section]
\newtheorem{corollaire}{Corollaire}[section]

\newtheorem{_proposition}{Proposition}[section]
\newenvironment{proposition}[1][]{
    \begin{_proposition}[#1]~\par
    \vspace*{0.5em}
}{%
    \end{_proposition}%
}

\newtheorem{_proprietes}{Propriétés}[section]
\newenvironment{proprietes}[1][]{
        \begin{_proprietes}[#1]~\par
        \vspace*{0.5em}
}{%
        \end{_proprietes}%
}

\newenvironment{rdem}[1][]{
    \begin{tcolorbox}[colframe=black, colback=white!10, sharp corners]
        #1
}{%
    \end{tcolorbox}
     
}

\newtheorem{_demonstration}{Démonstration}[section]
\newenvironment{demonstration}[1][]{
    \begin{_demonstration}[#1]~\par
    \vspace*{0.5em}
}{%
    \end{_demonstration}%
    \qed%
}

\newtheorem*{_demonstration*}{Démonstration}
\newenvironment{demonstration*}[1][]{
    \begin{_demonstration*}[#1]~\par
    \vspace*{0.5em}
}{%
    \end{_demonstration*}%
    \qed%
}

\newenvironment{ldefinition}{
    \begin{definition}~\par
    \vspace*{0.5em}
    \begin{enumerate}
}{
        \end{enumerate}
        \end{definition}
}

\newenvironment{lexample}{
    \begin{example}~\par
    \vspace*{0.5em}
    \begin{enumerate}
}{
        \end{enumerate}
        \end{example}
}

\newtheorem{_methode}{Méthode}[section]
\newenvironment{methode}{
    \begin{_methode}~\par
    \vspace*{0.5em}
}{
        \end{_methode}
}

\def\N{\mathbb{N}}
\def\Z{\mathbb{Z}}
\def\Q{\mathbb{Q}}
\def\R{\mathbb{R}}
\def\C{\mathbb{C}}
\def\K{\mathbb{K}}
\def\k{\Bbbk}

\def\un{(u_n)_{n \in \N}}
\def\xn#1{(#1_n)_{n \in \N}}

\def\o{\overline}
\def\eps{\varepsilon}

% \funcdef{name}{domain}{codomain}{variable}{expression}
% name: Name of the function (e.g. f)
% domain: Domain of the function (e.g. \mathbb{R})
% codomain: Codomain of the function (e.g. \mathbb{R})
% variable: Variables of the function (e.g. x)
% expression: Expression of the function (e.g. x^2)
\newcommand{\funcdef}[5]{%
    #1 :
    \begin{cases}
        #2 \rightarrow #3 \\
        #4 \mapsto #5
    \end{cases}
}

\newcommand{\lt}{\ensuremath <}
\newcommand{\gt}{\ensuremath >}

\begin{document}

\maketitle

\subsubsection{Fonction continue sur un segment}

\begin{theorem}
    Soit f une fonction continue sur un segment (un intervalle fermé et borné).
    Alors f est bornée et atteint ses bornes.

    Autrement dit, si $f: [a, b] \rightarrow \R$ alors $f([a, b]) = [m, M]$ avec $m = \min_{x \in [a, b]}f(x)$ et $M = \max_{x \in [a, b]}f(x)$
\end{theorem}

\begin{demonstration}
    Par un intervalle I, on sait d'apres le TVI que f(I) est un intervalle. Montrons que $m = inf(f(I))$
    et $M = sup(f(I))$ puis que m et M appartiennent à f(I).

    Vérifions que f est bornée. Supposons que f n'est pas majorée, c'est à dire
    $\forall A \gt 0, \exists x_0 \in I, f(x_0) \gt A$, ou $\lim_{x \to x_0}f(x) = +\infty$.
    Mais f est continue, donc $\lim_{x \to x_0}f(x) = f(x_0) \lt +\infty$, Absurde.

    ** skip du cas minorée.

    Donc f est bornée, l'ensemble f(I) est borné et admet une borne supérieure M et une borne inférieure m.

    Vérifions que $M \in f(I)$. Supposons que $M \notin f(I)$, c'est à dire que $\forall x \in I, f(x) \lt M$.

    On étudie $g(x) = \dfrac{1}{M - f(x)}$ qui est bien définie car $f(x) \neq M$, et g est bornée.

    Par définition de la borne supérieure, il existe une suite $(u_n)_{n \in \N}$ qui converge vers M avec
    $\forall n \in \N, u_n \in f(I)$, D'apres le TVI, il existe une suite $(c_n)_{n \in \N}$ de I tel que $u_n = f(C_n) \rightarrow_{n \to +\infty} M$.

    Mais $g(c_n) = \dfrac{1}{M - f(c_n)} \rightarrow_{n \to +\infty} + \infty$ ce qui contredit le fait que g est bornée.

    Finalement $M \in f(I)$
\end{demonstration}

\subsubsection{Suite définie par une fonction}

Soit f une fonction continue. On définit une suite récurrente $(u_n)_{n \in \N}$ par $\begin{cases}u_0 \in \R \\ u_{n+1} = f(u_n)\end{cases}$

c'est à dire $u_1 = f(u_0), u_2 = f(u_1) = f(f(u_0)) = f \circ f(u_0)$

\begin{theorem}
    Si f est continue, et si la suite $\un$ converge vers l, alors l est le point fixe de f,
    autrement dit $f(l) = l$
\end{theorem}

\begin{demonstration}
    $u_{n+1} = f(u_n)$ qui donne quand $n \to +\infty$ alors $l = f(l)$
\end{demonstration}

\begin{proprietes}
    \item Si f est continue et croissante sur [a, b], alors la suite $\un$ est monotone et converge vers $l = f(l)$.
    \item Si f est continue et décroissante sur [a, b], alors la sous-suite $(u_{2n})$ converge vers une limite $l_1$ solution
    de $l_1 = f \circ f(l_1)$ et la sous suite $(u_{2n + 1})$ converge vers une limite $l_2$ solution de $l_2 = f \circ f(l_2)$.
\end{proprietes}

\begin{demonstration}
    **Voir TD
\end{demonstration}

\subsection{Théorème de la bijection}

\subsubsection{Injection, surjection et bijection}

\begin{definition}
    Soit f une fonction de A dans B, deux partie de $\R$, $f: A \subset \R \rightarrow B \subset \R$.

    \item f est \textbf{injective} si $\forall x, x' \in A, f(x) = f(x') \implies x = x'$
    \item f est \textbf{sujrective} si $\forall y \in B, \exists x \in A, y = f(x)$
    \item f est \textbf{bijective} si f est injective et sujrective, c'est à dire $\forall y \in B, \exists! x \in A, y = f(x)$
\end{definition}

\begin{theorem}
    Si $f: A \rightarrow B$ est bijective, alors il existe une application $g: B \rightarrow A$
    telle que $f \circ g = Id_B$ et $g \circ f = Id_A$.

    On note $g = f^{-1}$ l'application \textbf{réciproque} de f (qui est aussi une bijection).
\end{theorem}

\subsubsection{Fonctions monotones}

\begin{theorem}
    Soit $f: I \rightarrow \R$, où I est un intervalle de $\R$, continue et strictement monotone. Alors
    \item f est une bijection de l'intervalle I dans l'intervalle f(I).
    \item La fonction réciproque $f^{-1}: f(I) \rightarrow I$ est continue et strictement monotone avec le
    même sens de variation que f.
\end{theorem}

\begin{demonstration}
    Supposons que f strictement croissante.

    Soit $x \neq x'$ avec $f(x) = f(x')$ alors $\begin{cases}\text{soit } x \lt x' \text{ et } f(x) \lt f(x') \\ \text{soit } x \gt x' \text{ et } f(x) \gt f(x')\end{cases}$

    Car f strictement croissante, ce qui contredit le fait que $f(x) = f(x')$. Finalement $x = x'$

    De plus, il est surjective car l'image d'un intervalle par une fonction continue est un intervalle $f(I) = \{y = f(x); x \in I\}$.

    On conclut que f est injective et sujrective alors elle est bijective.
\end{demonstration}

\subsection{Fonctions usuelles inverses}

\subsubsection{Logarithme et exponentielle}

\begin{definition}
    Il existe une unique fonction notée $ln : ]0, +\infty[ \rightarrow \R$ tell que
    \item $ln(a \times b) = ln(a) + ln(b)$
    \item $ln(\dfrac{1}{a}) = -ln(a)$
    \item $ln(a^n) = n\times ln(a)$
    
    \item On appelle cette fonction logarithme népérien caractérisée par $ln(e) = 1$.
    On définit le logarithme de e à base a comme $log_a$ comme $log_a(x) = \dfrac{ln(x)}{ln(a)}$ ou $log(a) = 1$
\end{definition}

\begin{proprietes}
    La fonction ln est continue et strictement croissante sur $]0, +\infty[$ avec
    $\forall x \gt 0, (ln (x))' = \dfrac{1}{x}$, elle définit une bijection de $]0, +\infty[$ dans $\R$

    \item $\lim_{x \to 0^+} ln(x) = - \infty$ et $\lim_{x \to +\infty} ln(x) = +\infty$
    \item ln(1) = 0
\end{proprietes}

\begin{definition}
    La fonction réciproque du logarithme népérien s'appelle exponentielle notée exp(x) ou $e^x: \R \rightarrow ]0, +\infty[$
\end{definition}

\begin{proprietes}
    En écrivant $f \circ f^{-1} = Id_{\R}$ et $f^{-1} \circ f = Id_{]0, +\infty[}$, il vient
    \item $\forall x \in \R, ln(exp(x)) = x$ et $\forall y \in ]0, +\infty[, exp(ln(y))=y$
    \item $exp(a + b) = exp(a) exp(b)$
    \item $exp: \R \rightarrow ]0, +\infty[$ est continue et strictement croissante.
\end{proprietes}

\begin{definition}
    On appelle la fonction puissance de $a \gt 0$ comme $a^x = exp(x ln(a))$
\end{definition}

\end{document}
