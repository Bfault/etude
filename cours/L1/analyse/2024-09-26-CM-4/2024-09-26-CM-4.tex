\documentclass[a4paper, 12pt]{article}
\usepackage{amsmath, amssymb, amsthm}
\usepackage{geometry}
\usepackage{tcolorbox}
\geometry{hmargin=2.5cm, vmargin=2.5cm}

\renewcommand*{\today}{26 septembre 2024}

\title{Analyse | CM: 4}
\author{Par Lorenzo}
\date{\today}

\newtheorem{theorem}{Théorème}[section]
\newtheorem{definition}{Définition}[section]
\newtheorem{example}{Example}[section]
\newtheorem{remark}{Remarques}[section]
\newtheorem{lemme}{Lemme}[section]
\newtheorem{corollaire}{Corollaire}[section]

\newtheorem{_proposition}{Proposition}[section]
\newenvironment{proposition}[1][]{
    \begin{_proposition}[#1]~\par
    \vspace*{0.5em}
}{%
    \end{_proposition}%
}

\newtheorem{_proprietes}{Propriétés}[section]
\newenvironment{proprietes}[1][]{
        \begin{_proprietes}[#1]~\par
        \vspace*{0.5em}
}{%
        \end{_proprietes}%
}

\newenvironment{rdem}[1][]{
    \begin{tcolorbox}[colframe=black, colback=white!10, sharp corners]
        #1
}{%
    \end{tcolorbox}
     
}

\newtheorem{_demonstration}{Démonstration}[section]
\newenvironment{demonstration}[1][]{
    \begin{_demonstration}[#1]~\par
    \vspace*{0.5em}
}{%
    \end{_demonstration}%
    \qed%
}

\newtheorem*{_demonstration*}{Démonstration}
\newenvironment{demonstration*}[1][]{
    \begin{_demonstration*}[#1]~\par
    \vspace*{0.5em}
}{%
    \end{_demonstration*}%
    \qed%
}

\newenvironment{ldefinition}{
    \begin{definition}~\par
    \vspace*{0.5em}
    \begin{enumerate}
}{
        \end{enumerate}
        \end{definition}
}

\newenvironment{lexample}{
    \begin{example}~\par
    \vspace*{0.5em}
    \begin{enumerate}
}{
        \end{enumerate}
        \end{example}
}

\newtheorem{_methode}{Méthode}[section]
\newenvironment{methode}{
    \begin{_methode}~\par
    \vspace*{0.5em}
}{
        \end{_methode}
}

\def\N{\mathbb{N}}
\def\Z{\mathbb{Z}}
\def\Q{\mathbb{Q}}
\def\R{\mathbb{R}}
\def\C{\mathbb{C}}
\def\K{\mathbb{K}}
\def\k{\Bbbk}

\def\un{(u_n)_{n \in \N}}
\def\xn#1{(#1_n)_{n \in \N}}

\def\o{\overline}
\def\eps{\varepsilon}

% \funcdef{name}{domain}{codomain}{variable}{expression}
% name: Name of the function (e.g. f)
% domain: Domain of the function (e.g. \mathbb{R})
% codomain: Codomain of the function (e.g. \mathbb{R})
% variable: Variables of the function (e.g. x)
% expression: Expression of the function (e.g. x^2)
\newcommand{\funcdef}[5]{%
    #1 :
    \begin{cases}
        #2 \rightarrow #3 \\
        #4 \mapsto #5
    \end{cases}
}

\newcommand{\lt}{\ensuremath <}
\newcommand{\gt}{\ensuremath >}

\begin{document}

\maketitle

\subsection{Majorants et minorants}

\begin{definition}
    Soit A une partie non vide de $\R$, un réel M est dit majorant de A si il vérifie $\forall x \in A, \; M \geq x$
\end{definition}

\begin{definition}
    Soit A une partie non vide de $\R$, un réel m est dit minorant de A si il vérifie $\forall x \in A, \; m \leq x$
\end{definition}

\begin{remark}
    Le majorant et minorant n'appartiennent pas forcément à l'ensemble A.
\end{remark}

\begin{definition}
    Si un majorant (resp. minorant) de A existe, on dit que A est majorée (resp. minorée). On dit que A est bornée si A est majorée et minorée.
\end{definition}

\subsection{Bornes supérieures et bornes inférieures}

\begin{definition}
    Soit A un partie non vide de $\R$.

    \item \textbf{1.} On dit que M est la borne supérieure de A, si M est un majorant de A et que M est le plus petit des majorants.
        Si il existe on note M = sup A.
    \item \textbf{2.} On dit que m est la borne inférieure de A, si m est un minorant de A et que m est le plus grand des minorant.
        Si il existe on note m = inf A.
\end{definition}

\begin{remark}
    sup A et inf A n'appartiennent pas forcément à A. Mais si ils appartiennent à l'ensemble ils deviennent max A et min A respectivement.
\end{remark}

\begin{example}
    Posons $A := [0, 1[$,

    $min A = 0$ et $max A$ n'existe pas.

    les minorants de A sont $]-\infty, 0]$ et les majorants de A sont $[1, +\infty[$
    
    $inf A = 0$ et $sup A = 1$
\end{example}

\begin{proposition}
    Soit A une partie non vide de $\R$ et majorée. La borne supérieure est l'unique réel sup A, tel que
    
    \item (i)$\forall x \in A, \; x \leq sup A$ et
    \item (ii) $\forall y \in \R, \; y \lt sup A \implies (\exists x \in A, y \lt x)$
\end{proposition}

\begin{demonstration}
    Montrons que sup A vérifie (i) et (ii).

    \begin{rdem}
        Comme sup A est un majorant, elle vérifie (i)
    \end{rdem}

    Posons $y \lt sup A$, comme sup A est le plus petit des majorants, y ne peut pas être un majorant de A.
    
    \begin{rdem}
        Donc $\exists x \in A, \; y < x$
    \end{rdem}

    Soit M un réel qui vérifie (i) et (ii), supposons que M n'est pas le plus petit des majorants.
    Il existe un autre majorant y, tel que $y \lt M$.

    \begin{rdem}
        Mais d'après (ii) $\exists x \in A, \; y < x$, donc y n'est pas un majorant de A.
    \end{rdem}
\end{demonstration}

\begin{theorem}
    Toute partie non vide de $\R$ majorée admet une borne supérieure.
\end{theorem}

\begin{theorem}
    Toute partie non vide de $\R$ minorée admet une borne inférieure.
\end{theorem}

\begin{proposition}
    Soit A une partie non vide majorée de $\R$. La borne supérieure est l'unique réel sup A, tel que

    \item (i) sup A est un majorant de A
    \item (ii) il existe une suite $(x_n)_{n \in \N}$ d'éléments de A qui converge vers sup A.
\end{proposition}

\section{Les suites}

\subsection{Définition d'une suite}

\begin{definition}
    Une suite est l'application
    \begin{align*}
        u: \N &\longrightarrow \R\\
        n &\longmapsto u(n)
    \end{align*}
    Pour $n \in \N$, on note $u(n)$, ou plus souvent $u_n$, le n-ième terme de la suite.
    On écrit~$(u_n)_{n \in \N}$
\end{definition}

\end{document}
