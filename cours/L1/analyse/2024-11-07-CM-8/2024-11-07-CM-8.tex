\documentclass[a4paper, 12pt]{article}
\usepackage{amsmath, amssymb, amsthm}
\usepackage{geometry}
\usepackage{tcolorbox}
\geometry{hmargin=2.5cm, vmargin=2.5cm}

\renewcommand*{\today}{07 novembre 2024}

\title{Analyse | CM: 8}
\author{Par Lorenzo}
\date{\today}

\newtheorem{theorem}{Théorème}[section]
\newtheorem{definition}{Définition}[section]
\newtheorem{example}{Example}[section]
\newtheorem{remark}{Remarques}[section]
\newtheorem{lemme}{Lemme}[section]
\newtheorem{corollaire}{Corollaire}[section]

\newtheorem{_proposition}{Proposition}[section]
\newenvironment{proposition}[1][]{
    \begin{_proposition}[#1]~\par
    \vspace*{0.5em}
}{%
    \end{_proposition}%
}

\newtheorem{_proprietes}{Propriétés}[section]
\newenvironment{proprietes}[1][]{
        \begin{_proprietes}[#1]~\par
        \vspace*{0.5em}
}{%
        \end{_proprietes}%
}

\newenvironment{rdem}[1][]{
    \begin{tcolorbox}[colframe=black, colback=white!10, sharp corners]
        #1
}{%
    \end{tcolorbox}
     
}

\newtheorem{_demonstration}{Démonstration}[section]
\newenvironment{demonstration}[1][]{
    \begin{_demonstration}[#1]~\par
    \vspace*{0.5em}
}{%
    \end{_demonstration}%
    \qed%
}

\newtheorem*{_demonstration*}{Démonstration}
\newenvironment{demonstration*}[1][]{
    \begin{_demonstration*}[#1]~\par
    \vspace*{0.5em}
}{%
    \end{_demonstration*}%
    \qed%
}

\newenvironment{ldefinition}{
    \begin{definition}~\par
    \vspace*{0.5em}
    \begin{enumerate}
}{
        \end{enumerate}
        \end{definition}
}

\newenvironment{lexample}{
    \begin{example}~\par
    \vspace*{0.5em}
    \begin{enumerate}
}{
        \end{enumerate}
        \end{example}
}

\newtheorem{_methode}{Méthode}[section]
\newenvironment{methode}{
    \begin{_methode}~\par
    \vspace*{0.5em}
}{
        \end{_methode}
}

\def\N{\mathbb{N}}
\def\Z{\mathbb{Z}}
\def\Q{\mathbb{Q}}
\def\R{\mathbb{R}}
\def\C{\mathbb{C}}
\def\K{\mathbb{K}}
\def\k{\Bbbk}

\def\un{(u_n)_{n \in \N}}
\def\xn#1{(#1_n)_{n \in \N}}

\def\o{\overline}
\def\eps{\varepsilon}

% \funcdef{name}{domain}{codomain}{variable}{expression}
% name: Name of the function (e.g. f)
% domain: Domain of the function (e.g. \mathbb{R})
% codomain: Codomain of the function (e.g. \mathbb{R})
% variable: Variables of the function (e.g. x)
% expression: Expression of the function (e.g. x^2)
\newcommand{\funcdef}[5]{%
    #1 :
    \begin{cases}
        #2 \rightarrow #3 \\
        #4 \mapsto #5
    \end{cases}
}

\newcommand{\lt}{\ensuremath <}
\newcommand{\gt}{\ensuremath >}

\begin{document}

\maketitle

\subsection{Suites adjacentes}

\begin{definition}
    Soient $\un$ et $\xn{v}$ deux suites. Elles sont adjacentes si.

    \item (i) $(u_n)$ est croissante, $v_n$ décroissante
    \item (ii) $\forall n \in N, u_n \leq v_n$
    \item (iii) $\lim_{n \to +\infty} (v_n - u_n) = 0$
\end{definition}

\begin{theorem}
    Si $\un$ et $\xn{v}$ sont deux suites adjacentes, alors elles convergent vers la même limite.
\end{theorem}

\begin{demonstration}
    Une suite $\un$ croissante et une suite $\xn{v}$ décroissante.

    Ainsi $\un$ est majorée par $v_0$ donc elle converge vers une limite $l_1$
    et $\xn{v}$ est minorée par $u_0$ donc elle converge vers une limite $l_2$

    comme $\lim_{n \to +\infty}(v_n - u_n) = 0 \implies l_2 - l_1 = 0 \implies l_2 = l_1$
\end{demonstration}

\subsection{Les Sous-suites}

\begin{definition}
    Soit $\un$. Une sous-suite ou suite extraite est une suite
    $(u_{\phi(n)})_{n \in \N}$ où

    $$
    \phi: \substack{\N \longmapsto \N \\ n \longmapsto \phi(n)}
    $$
    est une fonction croissante.
\end{definition}

\begin{proposition}
    Si la suite $\un$ converge vers l, alors toute suite extraite convergent vers l.
\end{proposition}

\begin{demonstration}
    $\lim_{n \to +\infty}u_n = l \iff \forall \eps \gt 0, \exists N \in \N, \forall n \geq N, |u_n - l| \lt \eps$

    Comme $\phi$ est croissante, en particulier si $n \geq N$ alors $\phi(n) \geq \phi(N)$ et $|u_{\phi(n)}-l| \lt \eps$.

    Autrement dit, $\lim_{n \to +\infty} u_{\phi(n)} = l$
\end{demonstration}

\begin{corollaire}
    Si il existe une sous-suite qui diverge, ou deux sous-suites qui convergent vers
    deux limites différentes, alors la suite diverge.
\end{corollaire}

\begin{theorem}
    Le théorème de Bolzano-Weierstrass dit que toute suite bornée admet au moins une sous-suite qui converge.
\end{theorem}

\begin{demonstration}
    On procède par dichotomie. Comme la suite est bornée, on peut supposer qu'elle
    prend ses valeurs dans l'intervalle [a, b]

    On pose $a_0 = a$, $b_0 = b$ et $\phi(0) = 0$
    La suite $\un$ a une infinité de valeurs dans $[a, \dfrac{a+b}{2}]$ ou $[\dfrac{a+b}{2}, b]$.

    On note $[a_1, b_1]$ cet intervalle
    $a_0 = a_1$ et
    
    $a_1 = a$ si $(u_n)$ a une infinité de valeurs dans $[a_1, \dfrac{a+b}{2}]$ sinon $a_1 = \dfrac{a+b}{2}$

    $b_1 = \dfrac{a+b}{2}$ si $(u_n)$ a une infinité de valeurs dans $[\dfrac{a+b}{2}, b_1]$ sinon $b_1 = b$

    On peut ensuite construire un intervalle $[a_n, b_n]$ de longeur $\dfrac{b - a}{2^n}$
    et un entier $\phi(n) \geq \phi(n-1)$ avec $u_{\phi(n)} \in [a_n, b_n]$.

    Par construction la suite $\xn{a}$ est croissante et la suite $\xn{b}$ est décroissante,
    et $a_n \leq b_n$

    De plus $\lim_{n \to +\infty}b_n - a_n = \lim_{n \to +\infty} \dfrac{b-a}{2^n} = 0$.

    Donc $xn{a}$ et $xn{b}$ sont adjacentes, elles convergent vers la même limite l.

    Mais $u_{\phi(n)} \in [a_n, b_n]$, ou encore $a_n \leq u_{\phi(n)} \leq b_n$

    et d'après le théorème des gendarmes, $\lim_{n \to +\infty}u_{\phi(n)} = l$
\end{demonstration}

\section{Etude de fonctions}

\subsection{Notion de fonction}

\begin{definition}
    Une fonction d'une variable réelle à valeurs réelles est une application $f: \substack{U \longmapsto \R \\ x \longmapsto f(x)}$ où U est une partie de $\R$ appelée ensemble de définition de f.
\end{definition}

Le graphe $\Gamma$ est la partie du plan $\R^2$ défini par $\Gamma = \{(x, f(x)); x \in U\}$.
Pour $x \in U$, f(x) est l'image de x par f.

\subsection{Opérations sur les fonctions}

Soient $f: U \longmapsto \R$ et $g: U \longmapsto \R$ définies sur le même domaine U.

On définit la somme de deux fonctions $h = f + g$ comme
$$
\forall x \in U, h(x) = (f + g)(x) = f(x) + g(x)
$$

On définit le produit de 2 fonctions $h = f \times g$ comme
$$
\forall x \in U, h(x) = (f \times g)(x) = f(x) \times g(x)
$$

\begin{remark}
    La multiplication par un scalaire $\lambda \in \R$ est définie comme,\par $\forall x \in U, (\lambda f)(x) = \lambda f(x)$
\end{remark}

\subsection{Fonction monotone, bornée}

\begin{definition}
    Soient $f: U \longmapsto \R$ et $g: U \longmapsto \R$

    \item 1. $f \leq g$ si $\forall x \in U, f(x) \leq g(x)$
    \item 2. $f \geq 0$ si $\forall x \in U, f(x) \geq 0$
    \item 3. f est constante si $\exists C \in \R$ tel que $\forall x \in U, f(x) = C$
\end{definition}

\begin{definition}
    \item 1. la fonction f est croissante si $\forall x, y \in U, x \leq y \implies f(x) \leq f(y)$.
    \item 2. la fonction f est strictement croissante si $\forall x, y \in U, x \lt y \implies f(x) \lt f(y)$.
    \item 3. la fonction f est décroissante si $\forall x, y \in U, x \leq y \implies f(x) \geq f(y)$.
    \item 4. la fonction f est strictement décroissante si $\forall x, y \in U, x \lt y \implies f(x) \gt f(y)$.
    \item 5. f est monotone si elle est croissante ou décroissante.
\end{definition}

\begin{definition}
    \item 1. On dit que f est majorée si $\exists M \in \R, \forall x \in U, f(x) \leq M$.
    \item 2. On dit que f est minorée si $\exists m \in \R, \forall x \in U, f(x) \geq m$.
    \item 3. On dit que f est bornée si elle est majorée et minorée, ou encore si\par\noindent $\exists M \in \R, \forall x \in U, |f(x)| \leq M$
\end{definition}

\subsection{Parité et périodicité}

\begin{definition}
    Soit I un intervalle symétrique par rapport à 0 ($I = ]-a; a[$) et $f: I \longmapsto \R$.

    \item 1. On dit que f est paire si $\forall x \in I, f(-x) = f(x)$
    \item 2. On dit que f est impaire si $\forall x \in I, f(-x) = -f(x)$
\end{definition}

\begin{definition}
    Soient $f: \R \longmapsto \R$ et T un nombre réel strictement positif.

    La fonction f est périodique de période T si $\forall x \in \R, f(x + T) = f(x)$
\end{definition}
\end{document}
