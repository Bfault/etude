\documentclass[a4paper, 12pt]{article}
\usepackage{amsmath, amssymb, amsthm, stmaryrd}
\usepackage{geometry}
\usepackage{pgfplots}
\usepackage{tcolorbox}
\geometry{hmargin=2.5cm, vmargin=2.5cm}

\renewcommand*{\today}{05 décembre 2024}

\title{Analyse | CM: 12}
\author{Par Lorenzo}
\date{\today}

\newtheorem{theorem}{Théorème}[section]
\newtheorem{definition}{Définition}[section]
\newtheorem{example}{Example}[section]
\newtheorem{remark}{Remarques}[section]
\newtheorem{lemme}{Lemme}[section]
\newtheorem{corollaire}{Corollaire}[section]

\newtheorem{_proposition}{Proposition}[section]
\newenvironment{proposition}[1][]{
    \begin{_proposition}[#1]~\par
    \vspace*{0.5em}
}{%
    \end{_proposition}%
}

\newtheorem{_proprietes}{Propriétés}[section]
\newenvironment{proprietes}[1][]{
        \begin{_proprietes}[#1]~\par
        \vspace*{0.5em}
}{%
        \end{_proprietes}%
}

\newenvironment{rdem}[1][]{
    \begin{tcolorbox}[colframe=black, colback=white!10, sharp corners]
        #1
}{%
    \end{tcolorbox}
     
}

\newtheorem{_demonstration}{Démonstration}[section]
\newenvironment{demonstration}[1][]{
    \begin{_demonstration}[#1]~\par
    \vspace*{0.5em}
}{%
    \end{_demonstration}%
    \qed%
}

\newtheorem*{_demonstration*}{Démonstration}
\newenvironment{demonstration*}[1][]{
    \begin{_demonstration*}[#1]~\par
    \vspace*{0.5em}
}{%
    \end{_demonstration*}%
    \qed%
}

\newenvironment{ldefinition}{
    \begin{definition}~\par
    \vspace*{0.5em}
    \begin{enumerate}
}{
        \end{enumerate}
        \end{definition}
}

\newenvironment{lexample}{
    \begin{example}~\par
    \vspace*{0.5em}
    \begin{enumerate}
}{
        \end{enumerate}
        \end{example}
}

\newtheorem{_methode}{Méthode}[section]
\newenvironment{methode}{
    \begin{_methode}~\par
    \vspace*{0.5em}
}{
        \end{_methode}
}

\def\N{\mathbb{N}}
\def\Z{\mathbb{Z}}
\def\Q{\mathbb{Q}}
\def\R{\mathbb{R}}
\def\C{\mathbb{C}}
\def\K{\mathbb{K}}
\def\k{\Bbbk}

\def\un{(u_n)_{n \in \N}}
\def\xn#1{(#1_n)_{n \in \N}}

\def\o{\overline}
\def\eps{\varepsilon}

% \funcdef{name}{domain}{codomain}{variable}{expression}
% name: Name of the function (e.g. f)
% domain: Domain of the function (e.g. \mathbb{R})
% codomain: Codomain of the function (e.g. \mathbb{R})
% variable: Variables of the function (e.g. x)
% expression: Expression of the function (e.g. x^2)
\newcommand{\funcdef}[5]{%
    #1 :
    \begin{cases}
        #2 \rightarrow #3 \\
        #4 \mapsto #5
    \end{cases}
}

\newcommand{\lt}{\ensuremath <}
\newcommand{\gt}{\ensuremath >}

\begin{document}

\maketitle

%verifier les sections
\section{Dérivée d'une fonction}

% retrouver sur internet
equation $y = ax + b = (x - x_0)f'(x_0) + f(x_0) = f'(x_0)x + (f(x_0) - x_0f'(x_0))$

\begin{definition}
    Soit $f: I \R \R$, où I est un interval ouvert de $\R$.
    Soit $x_0 \in I$, on dit que f est dérivable en $x_0$ si le taux d'accroissements
    $\dfrac{f(x)-f(x_0)}{x - x_0}$ admet une limite lorsque x tend vers $x_0$,
    et on la note $\lim_{x \to x_0}\dfrac{f(x) - f(x_0)}{x - x_0} = f'(x_0)$.

    $$
    \forall \eps \gt 0, \exists \delta \gt 0 \text{ alors } |\dfrac{f(x) - f(x_0)}{x - x_0} - f'(x_0)| \lt \eps \iff |f(x) - f(x_0) - (x - x_0)f'(x_0)| \lt \eps |x - x_0| \iff |f(x) - (f(x_0) + (x - x_0)f'(x_0))| \lt \eps |x - x_0|
    $$
\end{definition}

\begin{definition}
    La fonction est dérivable sur I si f est dérivable en tout point x de I.
    On note la fonction $f': \substack{I \R \R \\ x \mapsto f'(x)}$ (parfois $\dfrac{df}{dx}$)
\end{definition}

\begin{proposition}
    \begin{enumerate}
        \item f est dérivable en $x_0$ si et seulement si $\lim_{h \to 0}\dfrac{f(x_0 + h)- f(x_0)}{h}$ existe et est finie.
        \item f est dérivable en $x_0$ ssi il existe un nombre réel $f'(x_0)$ et une fonction $\eps: I \R \R$ avec
        $\eps(x) \to 0$ %mettre une petit x tend vers x_0 sous la flèche
        tel que $f(x) = f(x_0) + f'(x_0)(x - x_0) + \eps(x)(x-x_0)$
    \end{enumerate}
\end{proposition}
%regarder les approximation d'ordre 0, 1 et 2

\begin{demonstration}
    On pose $x = x_0 + h \Rightarrow x - x_0 = h$
\end{demonstration}

\begin{definition}
    La droite qui passe par les points $(x_0, f(x_0))$ et $(x, f(x))$ admet par coefficient directeur $\dfrac{f(x) - f(x_0)}{x - x_0}$.

    A la limite $x \rightarrow x_0$ on trouve le coefficient directeur de la tangente qui vaut $f'(x_0)$ et l'équation de la tangente au point
    $(x_0, f(x_0))$ est donné par $y = (x - x_0)f'(x_0) + f(x_0)$.
\end{definition}

\subsection{Dérivabilité et continuité}

\begin{proposition}
    \begin{enumerate}
        \item Si f est dérivable en $x_0$, alors f est continue en $x_0$.
        \item Si f est dérivable sur I, alors f est continue sur I.
        \item Si f est dérivable et f' est continue, on dit que f est de class $\phi^1$. %check si c'est bien phi
    \end{enumerate}
\end{proposition}

\begin{demonstration}
    Comme f dérivable en $x_0, \exists \eps: I \rightarrow \R$ avec $\eps(x) \to 0$ %pareil x tend vers x_0
    tel que $f(x) = f(x_0) + (x - x_0)f'(x_0) + \eps(x)(x-x_0)$
    et on veut montrer que $\lim_{x \to x_0}f(x) = f(x_0)$

    Ici $\lim_{x \to x_0}f(x) = \lim_{x \to x_0}f(x_0)+(x - x_0)f'(x_0) + \eps(x)(x - x_0)$
\end{demonstration}

\begin{remark}
    \begin{enumerate}
        \item Si f n'est pas continue, alors f n'est pas dérivable (contraposée).
        \item La réciproque est fausse en général (Exemple la fonction $|x|$ en x = 0)
    \end{enumerate}
\end{remark}

\subsection{Calcul de dérivée}

\subsubsection{Règle de calcul}

\begin{proprietes}
    Soient $f: I \rightarrow \R$ et $g: I \rightarrow \R$ dérivable sur I.
    Alors $\forall x \in I$, on a
    \begin{enumerate}
        \item $(f + g)'(x) = f'(x) + g'(x)$
        \item $\forall \lambda \in \R (\lambda f)'(x) = \lambda f'(x)$
        \item $(fg)'(x) = f'(x)g(x) + f(x)g'(x)$
        \item Si $g(x) \neq 0, (\dfrac{f}{g})'(x) = \dfrac{f'(x)g(x) - f(x)g'(x)}{g(x)^2}$
    \end{enumerate}
\end{proprietes}

\begin{demonstration}
    Pour 1 et 2, on utilise la définition de la dérivée.
    $$
    \dfrac{(\lambda f + g)(x) - (\lambda f + g)(x_0)}{x - x_0} \\
    = \dfrac{\lambda f(x) + g(x) - \lambda f(x_0) - g(x_0)}{x - x_0} \\
    = \dfrac{(\lambda f(x) - \lambda f(x_0)) + (g(x) - g(x_0))}{x - x_0} \\
    = \lambda (\dfrac{f(x) - f(x_0)}{x - x_0}) + \dfrac{g(x) - g(x_0)}{x - x_0}
    $$
    %overfull here
    
    Pour 3, on cherche
    $(fg)'(x_0) = \lim_{x \to x_0}\dfrac{fg(x) - fg(x_0)}{x - x_0}$.
    
    $$
    \dfrac{(fg)(x) - (fg)(x_0)}{x - x_0} = \dfrac{f(x)g(x) - f(x_0)g(x_0) + f(x_0)g(x) - f(x_0)g(x_0)}{x - x_0} \\
    = %TODO
    $$
    
    Flemme
\end{demonstration}

\subsubsection{Dérivée de fonctions usuelles}

% A revoir
\begin{methode}
    \begin{enumerate}
        \item $f(x) = x^n \Rightarrow f'(x) = nx^{n-1}$
        \item $f(x) = \dfrac{1}{x} \Rightarrow f'(x) = -\dfrac{1}{x^2}$
        \item $f(x) = \sqrt{x} \Rightarrow f'(x) = \dfrac{1}{2\sqrt{x}}$
        \item $f(x) = e^x \Rightarrow f'(x) = e^x$
        \item $f(x) = a^x \Rightarrow f'(x) = a^x\ln(a)$
        \item $f(x) = \ln(x) \Rightarrow f'(x) = \dfrac{1}{x}$
        \item $f(x) = \log_a(x) \Rightarrow f'(x) = \dfrac{1}{x\ln(a)}$
        \item $f(x) = \sin(x) \Rightarrow f'(x) = \cos(x)$
        \item $f(x) = \cos(x) \Rightarrow f'(x) = -\sin(x)$
        \item $f(x) = \tan(x) \Rightarrow f'(x) = \dfrac{1}{\cos^2(x)}$
    \end{enumerate}
\end{methode}

\subsubsection{Composition}

\begin{proposition}
    Si f dérivable en x et g dérivable en f(x), alors $g \circ f$ est dérivable en x et $(g \circ f)'(x) = g'(f(x))f'(x)$.
\end{proposition}

\begin{demonstration}
    À faire
\end{demonstration}

\begin{corollaire}
    $f: I \rightarrow J$ bijective et dérivable et $f^{-1}: J \rightarrow I$ sa réciproque.
    Si f' ne s'annule pas, alors $f^{-1}$ est dérivable et $(f^{-1})'(x) = \dfrac{1}{f'(f^{-1}(x))}$, $\forall x \in I$
\end{corollaire}

\begin{demonstration}
    À faire
\end{demonstration}

\subsubsection{Dérivées succesives}

Par récurrence, on définit la dérivée n-ième, notée $f^{(n)}$, comme
$f^{(0)} = f, f^{(1)} = f', f^{(n+1)} = (f^{(n)})'$.

Si les dérivées jusqu'à l'ordre n sont définies, on dit que f est de classe $\phi^n$.

\begin{definition}[Formule de Leibniz]
    $(fg)^{(n)}(x) = (f(x)g(x))^{(n)}$
\end{definition}

\begin{definition}[binome de Newton]
    $(a + b)^n = \sum_{k=0}^{n}\binom{n}{k}a^{n-k}b^k$
    À faire
\end{definition}



\end{document}
