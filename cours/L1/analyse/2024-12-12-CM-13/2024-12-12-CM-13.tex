\documentclass[a4paper, 12pt]{article}
\usepackage{amsmath, amssymb, amsthm, stmaryrd}
\usepackage{geometry}
\usepackage{pgfplots}
\usepackage{tcolorbox}
\geometry{hmargin=2.5cm, vmargin=2.5cm}

\renewcommand*{\today}{12 décembre 2024}

\title{Analyse | CM: 13}
\author{Par Lorenzo}
\date{\today}

\newtheorem{theorem}{Théorème}[section]
\newtheorem{definition}{Définition}[section]
\newtheorem{example}{Example}[section]
\newtheorem{remark}{Remarques}[section]
\newtheorem{lemme}{Lemme}[section]
\newtheorem{corollaire}{Corollaire}[section]

\newtheorem{_proposition}{Proposition}[section]
\newenvironment{proposition}[1][]{
    \begin{_proposition}[#1]~\par
    \vspace*{0.5em}
}{%
    \end{_proposition}%
}

\newtheorem{_proprietes}{Propriétés}[section]
\newenvironment{proprietes}[1][]{
        \begin{_proprietes}[#1]~\par
        \vspace*{0.5em}
}{%
        \end{_proprietes}%
}

\newenvironment{rdem}[1][]{
    \begin{tcolorbox}[colframe=black, colback=white!10, sharp corners]
        #1
}{%
    \end{tcolorbox}
     
}

\newtheorem{_demonstration}{Démonstration}[section]
\newenvironment{demonstration}[1][]{
    \begin{_demonstration}[#1]~\par
    \vspace*{0.5em}
}{%
    \end{_demonstration}%
    \qed%
}

\newtheorem*{_demonstration*}{Démonstration}
\newenvironment{demonstration*}[1][]{
    \begin{_demonstration*}[#1]~\par
    \vspace*{0.5em}
}{%
    \end{_demonstration*}%
    \qed%
}

\newenvironment{ldefinition}{
    \begin{definition}~\par
    \vspace*{0.5em}
    \begin{enumerate}
}{
        \end{enumerate}
        \end{definition}
}

\newenvironment{lexample}{
    \begin{example}~\par
    \vspace*{0.5em}
    \begin{enumerate}
}{
        \end{enumerate}
        \end{example}
}

\newtheorem{_methode}{Méthode}[section]
\newenvironment{methode}{
    \begin{_methode}~\par
    \vspace*{0.5em}
}{
        \end{_methode}
}

\def\N{\mathbb{N}}
\def\Z{\mathbb{Z}}
\def\Q{\mathbb{Q}}
\def\R{\mathbb{R}}
\def\C{\mathbb{C}}
\def\K{\mathbb{K}}
\def\k{\Bbbk}

\def\un{(u_n)_{n \in \N}}
\def\xn#1{(#1_n)_{n \in \N}}

\def\o{\overline}
\def\eps{\varepsilon}

% \funcdef{name}{domain}{codomain}{variable}{expression}
% name: Name of the function (e.g. f)
% domain: Domain of the function (e.g. \mathbb{R})
% codomain: Codomain of the function (e.g. \mathbb{R})
% variable: Variables of the function (e.g. x)
% expression: Expression of the function (e.g. x^2)
\newcommand{\funcdef}[5]{%
    #1 :
    \begin{cases}
        #2 \rightarrow #3 \\
        #4 \mapsto #5
    \end{cases}
}

\newcommand{\lt}{\ensuremath <}
\newcommand{\gt}{\ensuremath >}

\begin{document}

\maketitle

\subsection{Problèmes d'extrema}

\subsubsection{Extremum local}

\begin{definition}
    Soit $f: I \rightarrow \R$. On dit que f admet un minimum local en $x_0$
    (respectivement un maxumum) si il existe un intervalle J qui contient $x_0$ (on parle de voisinage)
    
    $\forall x \in J \cap I, f(x_0) \leq f(x)$ (resp $f(x_0) \geq f(x)$)
    
    On dit que $x_0$ est extremum si c'est un minimum ou un maximum.
    Si l'inégalité est vrai pour tout x de I, on dit que $x_0$ est un extremum global.
\end{definition}

% faire dessins si pas flemme

\begin{theorem}
    Soit I un intervalle ouvert et $f: I \rightarrow \R$ dérivable.
    Si f admet un extremum local en $x_0$, alors $f'(x_0) = 0$. ($x_0$ est appelé point critique)
\end{theorem}

\begin{remark}
    \begin{enumerate}
        \item La réciproque est fausse.
        \item Au point critique la tangente à la courbe est une droite horizontale.
    \end{enumerate}
\end{remark}

\begin{demonstration}
    Supposons que $x_0$ est un maximum local de f, c'est-à-dire $\forall x \in I \cap J, f(x) \leq f(x_0)$ Ainsi,
    pour $x \lt x_0$, $\dfrac{f(x) - f(x_0)}{x - x_0} \geq 0$ et pour $x \gt x_0$, $\dfrac{f(x) - f(x_0)}{x - x_0} \leq 0$.
    Comme f est dérivable, $\lim_{\substack{x \to x_0 \\ x \lt x_0}} \dfrac{f(x) - f(x_0)}{x - x_0} = \lim_{\substack{x \to x_0 \\ x \gt x_0}} \dfrac{f(x) - f(x_0)}{x - x_0} = f'(x_0)$
    Donc $f'(x_0) = 0$.
\end{demonstration}

\begin{theorem}[de Kolle]
    Soit $f: [a, b] \rightarrow \R$ continue sur $[a, b]$ et dérivable sur $]a, b[$.
    telle que $f(a) = f(b)$. Alors il existe $c \in ]a, b[$ tel que $f'(c) = 0$.
\end{theorem}

\begin{demonstration}
    On cherche $c \in ]a, b[$ extremmum local de f.
    Si f est constante, alors $f'(x) = 0$ pour tout $x \in ]a, b[$.
    Sinon, f n'est pas constante $\exists x_0 \in ]a, b[, f(x_0) \neq f(a) \neq f(b)$, f est continue sur $[a, b]$ donc f est bornée et admet un maximum en un point $c \in [a, b]$.
    Par définition du maximum, $f(c) \geq f(x_0) \gt f(a) = f(b)$.
    Et par continuité de f, $c \neq a \text{ et } b$.
    %A revoir
\end{demonstration}

\subsubsection{Accroissements finis}

\begin{theorem}[des accroissements finis (TAF)]
    Soit $f: [a, b] \rightarrow \R$ continue et dérivable sur $]a, b[$.
    Il existe $c \in ]a, b[$ tel que $f(b) - f(a) = f'(c)(b - a)$.
\end{theorem}

\begin{remark}
    $\lim_{b \to a} \dfrac{f(b) - f(a)}{b - a} = f'(a)$ (Pas le TAF)
\end{remark}

\begin{demonstration}
    On applique le théorème de Rolle à la fonction $g(x) = f(x) - \dfrac{f(b) - f(a)}{b - a}(x - a)$.

    On a $g(a) = f(a)$ et $g(b) = f(a)$, donc $g(a) = g(b)$ et g continue sur $[a, b]$ et dérivable sur $]a, b[$
    comme somme de fonctions continues et dérivables.

    Donc d'après le théorème de Rolle, il existe $c \in ]a, b[$ tel que \par $g'(c) = f'(c) - \dfrac{f(b) - f(a)}{b - a} = 0$.
\end{demonstration}

\begin{corollaire}
    \begin{enumerate}
        \item $\forall x \in ]a, b[, f'(x) \geq 0$ alors f est croissante.
        \item $\forall x \in ]a, b[, f'(x) \leq 0$ alors f est décroissante.
        \item $\forall x \in ]a, b[, f'(x) = 0$ alors f est constante.
        \item $\forall x \in ]a, b[, f'(x) \gt 0$ alors f est strictement croissante.
        \item $\forall x \in ]a, b[, f'(x) \lt 0$ alors f est strictement décroissante.
    \end{enumerate}
\end{corollaire}

\begin{demonstration}
    Soient $x, y \in ]a, b[, x \leq y$ alors d'après le TAF, $\exists c \in ]x, y[$ tel que $f(x) - f(y) = f'(c)(x - y) \leq 0$.
    Donc $f(x) \leq f(y)$, f est croissante.
\end{demonstration}

\begin{corollaire}[inégalité des accroissements finis]
    Soit $f: I \rightarrow \R$ dérivable sur l'intervalle I ouvert. Si il existe une constante
    $M \gt 0$ telle que $\forall x \in I, |f'(x)| \leq M$, alors $\forall x, y \in I, |f(x) - f(y)| \leq M|x - y|$.
\end{corollaire}

\begin{demonstration}
    Soient $x, y \in I, x \leq y$ alors d'après le TAF, $\exists c \in ]x, y[$ tel que $f(x) - f(y) = f'(c)(x - y)$.
    Donc $|f(x) - f(y)| = |f'(c)||x - y| \leq M|x - y|$.
\end{demonstration}

\begin{corollaire}[règle de l'Hospital]
    Soit $f, g: I \rightarrow \R$ dérivables en $x_0 \in I$. On suppose que 
    \begin{enumerate}
        \item $f(x_0) = g(x_0) = 0$
        \item $\forall x \in I$ et $x \neq x_0, g(x) \neq 0$
    \end{enumerate}
    Si $\lim_{x \to x_0} \dfrac{f'(x)}{g'(x)} = l$ alors $\lim_{x \to x_0} \dfrac{f(x)}{g(x)} = l$.
\end{corollaire}

\begin{demonstration}
    On applique le théorème de Rolle à la fonction $h(x) = g(a)f(x) - f(a)g(x)$.
    Pour $a \in I, a \lt x_0$
    $h'(x) = g(a)f'(x) - f(a)g'(x)$ et $h(a) = h(x_0) = 0$.
    Ainsi $\exists c \in ]a, x_0[$ tel que $h'(c) = 0$.
    $g(a)f'(c) - f(a)g'(c) = 0 \iff g(a)f'(c) = f(a)g'(c) \iff \dfrac{f(a)}{g(a)} = \dfrac{f'(c)}{g'(c)}$.
    En faisant $a \to x_0$ on a aussi $c \to x_0$ car $c \in ]a, x_0[$.
\end{demonstration}
\end{document}
