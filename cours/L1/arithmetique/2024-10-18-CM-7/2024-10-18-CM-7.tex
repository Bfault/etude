\documentclass[a4paper, 12pt]{article}
\usepackage{amsmath, amssymb, amsthm}
\usepackage{geometry}
\usepackage{tcolorbox}
\geometry{hmargin=2.5cm, vmargin=2.5cm}

\renewcommand*{\today}{18 octobre 2024}

\title{Arithmétique | CM: 7}
\author{Par Lorenzo}
\date{\today}

\newtheorem{theorem}{Théorème}[section]
\newtheorem{definition}{Définition}[section]
\newtheorem{example}{Example}[section]
\newtheorem{remark}{Remarques}[section]
\newtheorem{lemme}{Lemme}[section]
\newtheorem{corollaire}{Corollaire}[section]

\newtheorem{_proposition}{Proposition}[section]
\newenvironment{proposition}[1][]{
    \begin{_proposition}[#1]~\par
    \vspace*{0.5em}
}{%
    \end{_proposition}%
}

\newtheorem{_proprietes}{Propriétés}[section]
\newenvironment{proprietes}[1][]{
        \begin{_proprietes}[#1]~\par
        \vspace*{0.5em}
}{%
        \end{_proprietes}%
}

\newenvironment{rdem}[1][]{
    \begin{tcolorbox}[colframe=black, colback=white!10, sharp corners]
        #1
}{%
    \end{tcolorbox}
     
}

\newtheorem{_demonstration}{Démonstration}[section]
\newenvironment{demonstration}[1][]{
    \begin{_demonstration}[#1]~\par
    \vspace*{0.5em}
}{%
    \end{_demonstration}%
    \qed%
}

\newtheorem*{_demonstration*}{Démonstration}
\newenvironment{demonstration*}[1][]{
    \begin{_demonstration*}[#1]~\par
    \vspace*{0.5em}
}{%
    \end{_demonstration*}%
    \qed%
}

\newenvironment{ldefinition}{
    \begin{definition}~\par
    \vspace*{0.5em}
    \begin{enumerate}
}{
        \end{enumerate}
        \end{definition}
}

\newenvironment{lexample}{
    \begin{example}~\par
    \vspace*{0.5em}
    \begin{enumerate}
}{
        \end{enumerate}
        \end{example}
}

\newtheorem{_methode}{Méthode}[section]
\newenvironment{methode}{
    \begin{_methode}~\par
    \vspace*{0.5em}
}{
        \end{_methode}
}

\def\N{\mathbb{N}}
\def\Z{\mathbb{Z}}
\def\Q{\mathbb{Q}}
\def\R{\mathbb{R}}
\def\C{\mathbb{C}}
\def\K{\mathbb{K}}
\def\k{\Bbbk}

\def\un{(u_n)_{n \in \N}}
\def\xn#1{(#1_n)_{n \in \N}}

\def\o{\overline}
\def\eps{\varepsilon}

% \funcdef{name}{domain}{codomain}{variable}{expression}
% name: Name of the function (e.g. f)
% domain: Domain of the function (e.g. \mathbb{R})
% codomain: Codomain of the function (e.g. \mathbb{R})
% variable: Variables of the function (e.g. x)
% expression: Expression of the function (e.g. x^2)
\newcommand{\funcdef}[5]{%
    #1 :
    \begin{cases}
        #2 \rightarrow #3 \\
        #4 \mapsto #5
    \end{cases}
}

\newcommand{\lt}{\ensuremath <}
\newcommand{\gt}{\ensuremath >}

\begin{document}

\maketitle

%todo

\begin{definition}
    On note $\Z/n\Z$ l'ensemble des classes de congruences.

    $\Z/n\Z = \{\overline{0}, \overline{1}, ..., \overline{n-1}\}$ est un ensemble fini à n élements.
\end{definition}

\begin{proposition}
    Soient a, a', b, b' dans $\Z$ tels que $a \equiv a' [n]$ et $b \equiv b' [n]$
    Alors $a + b \equiv a' + b' [n]$
\end{proposition}

\begin{demonstration}
    $(a - a') = kn$ et $(b - b') = k'n$

    $(a + b) - (a' + b') = a - a' + b - b' = kn + k'n = (k + k')n$

    Donc $a + b \equiv a' + b' [n]$
\end{demonstration}

\begin{definition}
    Soient $a, b \in \Z$. On pose dans $\Z/n\Z: \overline{a} + \overline{b} = \overline{a + b}$ et $\overline{a} \times \overline{b} = \overline{a \times b}$
\end{definition}

\begin{proposition}
    $(\Z/n\Z, +, \times)$ est un anneau commutatif.

    $\overline{0}$ est l'élement neutre pour l'addition et $\overline{1}$ est l'élement neutre pour la multiplication.
\end{proposition}

\begin{demonstration}
    %todo
\end{demonstration}

On peut faire des tables d'addition et de multiplication dans $\Z/n\Z$

\begin{example}
    %todo
\end{example}

\begin{lemme}
    Soient a et b dans $\Z$ tels que $a \equiv b [n]$.

    Pour tout $p \in \N^*, a^p \equiv b^p [n]$
\end{lemme}

\begin{demonstration}
    Dans $\Z/n\Z$ on veut montrer que $\o{a^p} = \o{b^p}$

    Or $\o{a^p} = \o{a \times ... \times a} = \o{a} \times ... \times \o{a} =$
    %todo
\end{demonstration}

\begin{remark}
    En revanche on n'a pas $p \equiv q [n] \implies a^p \equiv a^q [n]$
\end{remark}

\begin{theorem}
    $\{\Z/n\Z, +, \times\}$ est un corps si et seulement si n est premier.
\end{theorem}

\begin{demonstration}
    Dire que $\o{a}$ est inversible dans $\Z/n\Z$ c'est dire
    qu'il existe $\o{u}$ tel que $\o{a}\o{u} = \o{1} \iff \exists u \in \Z, \exists k \in \Z, au = 1 + kn \iff \exists u \in \Z, \exists k' \in \Z, au + k'n = 1$
    Cette équation a des solutions si n et m sont premier entre eux (bezout)
    %todo
\end{demonstration}

\end{document}
