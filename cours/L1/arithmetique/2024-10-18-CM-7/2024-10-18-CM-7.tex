\documentclass[a4paper, 12pt]{article}
\usepackage{amsmath, amssymb, amsthm}
\usepackage{geometry}
\usepackage{tcolorbox}
\geometry{hmargin=2.5cm, vmargin=2.5cm}

\renewcommand*{\today}{18 octobre 2024}

\title{Arithmétique | CM: 7}
\author{Par Lorenzo}
\date{\today}

\newtheorem{theorem}{Théorème}[section]
\newtheorem{definition}{Définition}[section]
\newtheorem{example}{Example}[section]
\newtheorem{remark}{Remarques}[section]
\newtheorem{lemme}{Lemme}[section]
\newtheorem{corollaire}{Corollaire}[section]

\newtheorem{_proposition}{Proposition}[section]
\newenvironment{proposition}[1][]{
    \begin{_proposition}[#1]~\par
    \vspace*{0.5em}
}{%
    \end{_proposition}%
}

\newtheorem{_proprietes}{Propriétés}[section]
\newenvironment{proprietes}[1][]{
        \begin{_proprietes}[#1]~\par
        \vspace*{0.5em}
}{%
        \end{_proprietes}%
}

\newenvironment{rdem}[1][]{
    \begin{tcolorbox}[colframe=black, colback=white!10, sharp corners]
        #1
}{%
    \end{tcolorbox}
     
}

\newtheorem{_demonstration}{Démonstration}[section]
\newenvironment{demonstration}[1][]{
    \begin{_demonstration}[#1]~\par
    \vspace*{0.5em}
}{%
    \end{_demonstration}%
    \qed%
}

\newtheorem*{_demonstration*}{Démonstration}
\newenvironment{demonstration*}[1][]{
    \begin{_demonstration*}[#1]~\par
    \vspace*{0.5em}
}{%
    \end{_demonstration*}%
    \qed%
}

\newenvironment{ldefinition}{
    \begin{definition}~\par
    \vspace*{0.5em}
    \begin{enumerate}
}{
        \end{enumerate}
        \end{definition}
}

\newenvironment{lexample}{
    \begin{example}~\par
    \vspace*{0.5em}
    \begin{enumerate}
}{
        \end{enumerate}
        \end{example}
}

\newtheorem{_methode}{Méthode}[section]
\newenvironment{methode}{
    \begin{_methode}~\par
    \vspace*{0.5em}
}{
        \end{_methode}
}

\def\N{\mathbb{N}}
\def\Z{\mathbb{Z}}
\def\Q{\mathbb{Q}}
\def\R{\mathbb{R}}
\def\C{\mathbb{C}}
\def\K{\mathbb{K}}
\def\k{\Bbbk}

\def\un{(u_n)_{n \in \N}}
\def\xn#1{(#1_n)_{n \in \N}}

\def\o{\overline}
\def\eps{\varepsilon}

% \funcdef{name}{domain}{codomain}{variable}{expression}
% name: Name of the function (e.g. f)
% domain: Domain of the function (e.g. \mathbb{R})
% codomain: Codomain of the function (e.g. \mathbb{R})
% variable: Variables of the function (e.g. x)
% expression: Expression of the function (e.g. x^2)
\newcommand{\funcdef}[5]{%
    #1 :
    \begin{cases}
        #2 \rightarrow #3 \\
        #4 \mapsto #5
    \end{cases}
}

\newcommand{\lt}{\ensuremath <}
\newcommand{\gt}{\ensuremath >}

\begin{document}

\maketitle

\section{Arithmétique modulaire : $(\Z/n\Z)$}

\begin{definition}
    Soient $a, b \in \Z$. On dit que a et b sont \textbf{congrus modulo} n si $a - b \in n\Z$.
    
    On note alors $a \equiv b [n]$ ou encore $a \equiv b \mod n$.
\end{definition}

\begin{proposition}
    \begin{enumerate}
        \item On a $a \equiv b [n] \iff \exists k \in \Z, a = b + kn$.
        
        On note $\o{b} := \{b + nk \mid k \in \Z\} = \{a \in \Z \mid a \equiv b [n]\}$. On l'appelle la \textbf{classe de congruence}.
    
        \item Supposons que $a = nq + r$ soit la division euclidienne de a par n. Alors $\o{a} = \o{r}$.
        \item Il y a exactement n classes de congruence distinctes : les $\o{r}$, pour $r \in \{0, 1, ..., n-1\}$. Elles sont disjointes 2 à 2.
    \end{enumerate}
\end{proposition}

\begin{definition}
    On note $\Z/n\Z$ l'ensemble des classes de congruences.

    $\Z/n\Z = \{\o{0}, \o{1}, ..., \o{n-1}\}$ est un ensemble fini à n élements.
\end{definition}

\begin{demonstration}
    À faire
\end{demonstration}

\begin{remark}
    La congruence est un relation d'équivalence ainsi les classes congruences sont les classes d'équivalences pour la relation de congruence.

    Ainsi $\Z/n\Z$ se réinterprète comme $\Z/R$ avec $xRy \iff x \equiv y [n]$
\end{remark}

\subsection{L'anneau $(\Z/n\Z, +, \times)$}

\begin{proposition}
    Soient $a, a', b, b' \in \Z$ tels que $a \equiv a' [n]$ et $b \equiv b' [n]$
    Alors $a + b \equiv a' + b' [n]$
\end{proposition}

\begin{demonstration}
    \item $(a - a') = kn$ et $(b - b') = k'n$ avec $k, k' \in \Z$

    \item $(a + b) - (a' + b') = a - a' + b - b' = kn + k'n = (k + k')n$

    \begin{rdem}
        \item Donc $a + b \equiv a' + b' [n]$
    \end{rdem}
\end{demonstration}

\begin{definition}
    Soient $a, b \in \Z$. On pose dans $\Z/n\Z: \overline{a} + \overline{b} = \overline{a + b}$ et $\overline{a} \times \overline{b} = \overline{a \times b}$
\end{definition}

\begin{proposition}
    $(\Z/n\Z, +, \times)$ est un anneau commutatif unitaire.

    $\o{0}$ est l'élement neutre pour l'addition et $\o{1}$ est l'élement neutre pour la multiplication.
\end{proposition}

\begin{demonstration}
    À faire
\end{demonstration}


\begin{example}
    On peut faire des tables d'addition et de multiplication dans $\Z/n\Z$.
    Par exemple la table de multiplication de $\Z/3\Z$
    \begin{table}[ht]
        \begin{tabular}{|l|l|l|l|}
        \hline
        $\times$  & $\o{0}$ & $\o{1}$ & $\o{2}$ \\
        $\o{0}$ & $\o{0}$ & $\o{0}$ & $\o{0}$ \\
        $\o{1}$ & $\o{0}$ & $\o{1}$ & $\o{2}$ \\
        $\o{2}$ & $\o{0}$ & $\o{2}$ & $\o{1}$ \\ \hline
        \end{tabular}
    \end{table}
\end{example}

\begin{lemme}
    Soient a et b dans $\Z$ tels que $a \equiv b [n]$.

    Pour tout $p \in \N^*, a^p \equiv b^p [n]$
\end{lemme}

\begin{demonstration}
    À faire
\end{demonstration}

\begin{remark}
    En revanche on n'a pas $p \equiv q [n] \implies a^p \equiv a^q [n]$
\end{remark}

\begin{theorem}
    $\{\Z/n\Z, +, \times\}$ est un corps si et seulement si n est premier.
\end{theorem}

\begin{demonstration}
    À faire
\end{demonstration}

\end{document}
