\documentclass[a4paper, 12pt]{article}
\usepackage{amsmath, amssymb, amsthm}
\usepackage{geometry}
\usepackage{tcolorbox}
\geometry{hmargin=2.5cm, vmargin=2.5cm}

\renewcommand*{\today}{04 octobre 2024}

\title{Arithmetique | CM: 5}
\author{Par Lorenzo}
\date{\today}

\newtheorem{theorem}{Théorème}[section]
\newtheorem{definition}{Définition}[section]
\newtheorem{example}{Example}[section]
\newtheorem{remark}{Remarques}[section]
\newtheorem{lemme}{Lemme}[section]
\newtheorem{corollaire}{Corollaire}[section]

\newtheorem{_proposition}{Proposition}[section]
\newenvironment{proposition}[1][]{
    \begin{_proposition}[#1]~\par
    \vspace*{0.5em}
}{%
    \end{_proposition}%
}

\newtheorem{_proprietes}{Propriétés}[section]
\newenvironment{proprietes}[1][]{
        \begin{_proprietes}[#1]~\par
        \vspace*{0.5em}
}{%
        \end{_proprietes}%
}

\newenvironment{rdem}[1][]{
    \begin{tcolorbox}[colframe=black, colback=white!10, sharp corners]
        #1
}{%
    \end{tcolorbox}
     
}

\newtheorem{_demonstration}{Démonstration}[section]
\newenvironment{demonstration}[1][]{
    \begin{_demonstration}[#1]~\par
    \vspace*{0.5em}
}{%
    \end{_demonstration}%
    \qed%
}

\newtheorem*{_demonstration*}{Démonstration}
\newenvironment{demonstration*}[1][]{
    \begin{_demonstration*}[#1]~\par
    \vspace*{0.5em}
}{%
    \end{_demonstration*}%
    \qed%
}

\newenvironment{ldefinition}{
    \begin{definition}~\par
    \vspace*{0.5em}
    \begin{enumerate}
}{
        \end{enumerate}
        \end{definition}
}

\newenvironment{lexample}{
    \begin{example}~\par
    \vspace*{0.5em}
    \begin{enumerate}
}{
        \end{enumerate}
        \end{example}
}

\newtheorem{_methode}{Méthode}[section]
\newenvironment{methode}{
    \begin{_methode}~\par
    \vspace*{0.5em}
}{
        \end{_methode}
}

\def\N{\mathbb{N}}
\def\Z{\mathbb{Z}}
\def\Q{\mathbb{Q}}
\def\R{\mathbb{R}}
\def\C{\mathbb{C}}
\def\K{\mathbb{K}}
\def\k{\Bbbk}

\def\un{(u_n)_{n \in \N}}
\def\xn#1{(#1_n)_{n \in \N}}

\def\o{\overline}
\def\eps{\varepsilon}

% \funcdef{name}{domain}{codomain}{variable}{expression}
% name: Name of the function (e.g. f)
% domain: Domain of the function (e.g. \mathbb{R})
% codomain: Codomain of the function (e.g. \mathbb{R})
% variable: Variables of the function (e.g. x)
% expression: Expression of the function (e.g. x^2)
\newcommand{\funcdef}[5]{%
    #1 :
    \begin{cases}
        #2 \rightarrow #3 \\
        #4 \mapsto #5
    \end{cases}
}

\newcommand{\lt}{\ensuremath <}
\newcommand{\gt}{\ensuremath >}

\begin{document}

\maketitle

\section{Arithmétique avancée dans $\Z$}

\subsection{Bézout, Gauss}

\begin{proposition}[Bézout]
    Soient $a, b \in \Z^*$. Il existe $u, v \in \Z$ tels que

    $au + bv = PGCD(a, b)$
\end{proposition}

\begin{methode}
    Pour trouver une relation de Bezout, il suffit de remonter l'algorithme d'Euclide. Que l'on
    appelle \textbf{l'algorithme d'Euclide étendu}.

    \begin{enumerate}
        \item Faire l'algorithme d'Euclide
        \item Réecrire le reste avec les autres valeurs
    \end{enumerate}
\end{methode}

\begin{lemme}
    Les sous-groupes de $\Z$ sont les $n\Z := \{nk \mid k \in \Z\}$ avec $n \in \Z$
\end{lemme}

\begin{demonstration}
    \begin{itemize}
        \item $n\Z$ sous groupe de $(\Z, +)$ (cf TD1)
        \item Soit H un sous groupe de $(\Z, +)$ alors $0 \in H$
        \begin{itemize}
            \item Si $H = \{0\}$ alors $H = 0\Z$
            \item Sinon il existe un x non nuls dans H, alors $(-x) \in H$ "A completer"
        \end{itemize}
    \end{itemize}
\end{demonstration}

\begin{corollaire}
    Soient $a, b \in \Z$ alors $a\Z + b\Z = \{au + bv | u, v \in \Z\} = \delta \Z$ où $\delta = \mathrm{PGCD}(a, b)$
\end{corollaire}

\begin{demonstration}
    Soient $u, v \in \Z$ et $c = au + bv$. Comme $\delta a$ et $\delta b$ alors $\delta c$.

    Réciproquement, soit $c \in \delta\Z$, il existe un c' dans $\Z$ tel que $c = \delta c'$. Par Bézout, il existe
    $u', v' \in \Z$ tels que $au' + bv' = \delta$, en multipliant par c' on a $au'c' + bv'c' = \delta c' = c$. Il suffit alors
    de poser $u = u'c'$ et $v = v'c'$.

    On dit alors que le sous groupe \textbf{engendré par} a et b coïncide avec le sous groupe engendré par leurs PGCD.
\end{demonstration}

\begin{proposition}[Gauss]
    Soient $n, a, b \in \Z^*$ tels que $n | ab$ et $\mathrm{PGCD}(n, a) = 1$. Alors $n|b$.
\end{proposition}

\begin{demonstration}
    Par Bezout, il existe u et v tels que $nu + av = 1$. Donc $nub + abv = b$. De $ab = nk$ (pour un $k \in \Z$),
    on déduit $n(ub + kv) = b$. Donc $n | b$.
\end{demonstration}

\subsection{Unicité de la décomposition en facteurs premiers}

\begin{lemme}\
    \begin{enumerate}
        \item Soient $a, b, c \in \Z^*$
        \begin{equation*}
            \left.
            \begin{array}{l}
            \mathrm{PGCD}(c, a) = 1 \\
            \mathrm{PGCD}(c, b) = 1
            \end{array}
            \right\}
            \implies \mathrm{PGCD}(c, ab) = 1
        \end{equation*}
        \item Soient p un nombre premier et $a, b \in \Z^*$
        \begin{enumerate}
            \item On a $\mathrm{PGCD(a, p) = 1}$ ou $p | a$
            \item On a $[p|ab \implies (p|a ou p|b)]$
        \end{enumerate}
    \end{enumerate}
\end{lemme}

\begin{demonstration}
    À faire
\end{demonstration}

\begin{proposition}
    Une décomposition en facteurs premier est unique à l'ordre des facteurs près.
\end{proposition}

\begin{demonstration}
    À faire
\end{demonstration}

\subsection{Résolution des équations diophantiennes}

Soient $a, b \in \Z^*$ et $c \in \Z$.

On cherche à résoudre l'équation suivante d'inconnues entères u, v

$$
au + bv = c
$$

\begin{methode}
    \begin{enumerate}
        \item Posant $\delta = \mathrm{PGCD(a, b)}$, on a $a = \delta a'$, $b = \delta b'$ et $c = \delta c'$ avec $a', b', c' \in \Z$
        on a donc
        
        $$
        a'u + b'v = c'
        $$
        
        Soit $d = \mathrm{PGCD(a', b')}$ alors $d\delta$ est un diviseur commun à $a = \delta a'$ et $b = \delta b'$.
        Par maximalité du diviseur commun $\delta$, on a $d = 1$. Donc a' et b' sont premier entre eux.
        \item Bézout nous fournit une solution à l'équation $a'u + b'v = 1$, qu'il suffit de multiplier
        par c' pour avoir une solution particulière $(u_0, v_0)$.
        \item Soit $(u, v) \in \Z^2$ une solution. $a'u + b'v = c'$ et $a'u_0 + b'v_0 = c'$ donc $a'(u - u_0) + b'(v - v_0) = 0$.
        On a $\mathrm{PGCD(a', b')}= 1$ donc, d'après Gauss, $a' | (v - v_0)$. Donc $\exists k \in \Z$ tel que $v - v_0 = ka' \implies v = v_0 + ka'$ donc $u = u_0 - b'k$.
        \item L'ensemble des solutions est donc contenu dans $\{u_0 - b'k, v_0 + a'k \mid k \in \Z\}$
    \end{enumerate}
\end{methode}

\end{document}
