\documentclass[a4paper, 12pt]{article}
\usepackage{amsmath, amssymb, amsthm}
\usepackage{geometry}
\usepackage{tcolorbox}
\geometry{hmargin=2.5cm, vmargin=2.5cm}

\renewcommand*{\today}{13 septembre 2024}

\title{Arithmetique | CM: 2}
\author{Par Lorenzo}
\date{\today}

\newtheorem{theorem}{Théorème}[section]
\newtheorem{definition}{Définition}[section]
\newtheorem{example}{Example}[section]
\newtheorem{remark}{Remarques}[section]
\newtheorem{lemme}{Lemme}[section]
\newtheorem{corollaire}{Corollaire}[section]

\newtheorem{_proposition}{Proposition}[section]
\newenvironment{proposition}[1][]{
    \begin{_proposition}[#1]~\par
    \vspace*{0.5em}
}{%
    \end{_proposition}%
}

\newtheorem{_proprietes}{Propriétés}[section]
\newenvironment{proprietes}[1][]{
        \begin{_proprietes}[#1]~\par
        \vspace*{0.5em}
}{%
        \end{_proprietes}%
}

\newenvironment{rdem}[1][]{
    \begin{tcolorbox}[colframe=black, colback=white!10, sharp corners]
        #1
}{%
    \end{tcolorbox}
     
}

\newtheorem{_demonstration}{Démonstration}[section]
\newenvironment{demonstration}[1][]{
    \begin{_demonstration}[#1]~\par
    \vspace*{0.5em}
}{%
    \end{_demonstration}%
    \qed%
}

\newtheorem*{_demonstration*}{Démonstration}
\newenvironment{demonstration*}[1][]{
    \begin{_demonstration*}[#1]~\par
    \vspace*{0.5em}
}{%
    \end{_demonstration*}%
    \qed%
}

\newenvironment{ldefinition}{
    \begin{definition}~\par
    \vspace*{0.5em}
    \begin{enumerate}
}{
        \end{enumerate}
        \end{definition}
}

\newenvironment{lexample}{
    \begin{example}~\par
    \vspace*{0.5em}
    \begin{enumerate}
}{
        \end{enumerate}
        \end{example}
}

\newtheorem{_methode}{Méthode}[section]
\newenvironment{methode}{
    \begin{_methode}~\par
    \vspace*{0.5em}
}{
        \end{_methode}
}

\def\N{\mathbb{N}}
\def\Z{\mathbb{Z}}
\def\Q{\mathbb{Q}}
\def\R{\mathbb{R}}
\def\C{\mathbb{C}}
\def\K{\mathbb{K}}
\def\k{\Bbbk}

\def\un{(u_n)_{n \in \N}}
\def\xn#1{(#1_n)_{n \in \N}}

\def\o{\overline}
\def\eps{\varepsilon}

% \funcdef{name}{domain}{codomain}{variable}{expression}
% name: Name of the function (e.g. f)
% domain: Domain of the function (e.g. \mathbb{R})
% codomain: Codomain of the function (e.g. \mathbb{R})
% variable: Variables of the function (e.g. x)
% expression: Expression of the function (e.g. x^2)
\newcommand{\funcdef}[5]{%
    #1 :
    \begin{cases}
        #2 \rightarrow #3 \\
        #4 \mapsto #5
    \end{cases}
}

\newcommand{\lt}{\ensuremath <}
\newcommand{\gt}{\ensuremath >}

\begin{document}

\maketitle

\section{Structures algébriques}

\subsection{Lois de compositions internes}

\begin{definition}
    Soit E un ensemble. On appelle \textbf{loi de composition interne} (l.c.i) sur E une opération binaire.

    On parle d'application $E \times E \rightarrow E$
\end{definition}

\begin{definition}
    Soit $*$ une l.c.i sur E. On dit que $*$ est

    \item \textbf{associative} si $\forall x, y, z \in E, \; x * (y * z) = (x * y) * z$
    \item \textbf{commutative} si $\forall x, y \in E, \; x * y = x * y$
    \item \textbf{identitaire} (a un \textbf{élement neutre} $e \in E$) si $\forall x \in E, x * e = e * x = x$ 
\end{definition}

\subsection{Groupes}

\begin{definition}
    Soit G un ensemble et $*$ une l.c.i sur G. On dit que $(G, *)$ est un \textbf{groupe} lorsque les axiomes suivants sont vérifiés.

    \begin{itemize}
        \item $*$ est associative
        \item $*$ admet un élement neutre $e \in G$
        \item $\forall x \in G, \exists x' \in G$ tel que $x * x' = x' * x = e$ (on dit que x' est l'élement inverse ou symétrique de x pour $*$)
    \end{itemize}
\end{definition}

\begin{remark}
    Si de plus $*$ est commutative, alors le groupe est dit \textbf{abélien} (ou commutatif).
\end{remark}

\begin{example}
    Si X est un ensemble, notons $Bij(X)$, l'ensemble des application de X dans X admettant une application réciproque
    $$
    \forall\; f\!:\!X \rightarrow X,\; \exists\; g\!:\!X \rightarrow X,\; g \circ f = f \circ g = \mathrm{Id}_{X}:
    \begin{cases}
        X\! &\rightarrow X \\
        x\! &\mapsto x
    \end{cases}
    $$
    Ainsi $(Bij(X), \circ)$ est un groupe.
\end{example}

\begin{proposition}
    Si $(G, *)$ est un groupe alors

    \item \textbf{(a)} L'élement neutre de G est unique
    \item \textbf{(b)} Chaque $x \in G$ admet un unique élement inverse
    \item \textbf{(c)} Si $x, y, z \in G$ tel que $x * y = z * y$ alors $x * y$ (indépendament de l'ordre)
\end{proposition}

\begin{demonstration}
    \begin{rdem}
        \textbf{(a)} Soient e, e' des élements neutres de G par *, $e * e' = e' * e = e = e'$
    \end{rdem}
    \begin{rdem}
        \textbf{(b)} Soient x', x'' des élements inverse de $x \in G$,\par $x' = x' * e = x' * (x * x'') = (x' * x) * x'' = e * x'' = x''$
    \end{rdem}
    \begin{rdem}
        \textbf{(c)} Posons $x^{-1} * (x * y) = x^{-1} * (x * z) \implies (x^{-1} * x) * y = (x^{-1} * x) * z \implies e * y = e * z \implies y = z$
    \end{rdem}
\end{demonstration}

\begin{remark}
    Lorsqu'il n'y a pas d'ambiguïtés, l'inverse d'un élement x sera noté $x^{-1}$. Notons que $(x^{-1})^{-1} = x$
\end{remark}

\begin{definition}
    Soit (G, *) un groupe. Soit $H \subset G$, on dit que H est un \textbf{sous-groupe} de G lorsque les condtions suivantes sont vérifiées.
    \begin{itemize}
        \item $\forall x, y \in H, \; x * y \in H$. On dit que H est stable par $*$
        \item Muni de $*$, H est un groupe
    \end{itemize}
\end{definition}

\begin{proposition}
    Soit (G, $*$) un groupe et $H \subset G$. Les conditions suivantes sont équivalentes.

    \item \textbf{(a)} H est un sous groupe de G
    \item \textbf{(b)} $H \neq \emptyset$, H est stable par $*$ et par passage au symétrique ($\forall x \in H, \; x^{-1} \in H$)
    \item \textbf{(b)} $H \neq \emptyset$ et $\forall x, y \in H, \; x * y^{-1} \in H$
\end{proposition}

\begin{demonstration}
    \item $\bullet$ Démontrons que $(a) \implies (b)$.
    
        \item $\diamond$ H est un sous groupe donc doit admettre un élement neutre ($e_H$) donc $H \neq \emptyset$.
        Montrons que $e_H = e_G$, on a $e_H * e_H = e_H = e_G + e_H = e_G$.
        
        \vspace{0.5em}
        
        \item $\diamond$ La stabilité par * fait partie de la définition de sous groupe.
        
        \vspace{0.5em}
        
        \item $\diamond$ Soit $x \in H$, soit s' son symétrique dans H. x' est aussi un symétrique dans G. Dans G par unicité du symétrique $x^{-1} = x' \in H$.

    \item $\bullet$ Démontrons que $(b) \implies (c)$.
    
    \item $\diamond$ Soient $x, y \in H$. Alors $y^{-1} \in H$ et encore par $x * x^{-1} \in H$.
    
    \item $\bullet$ Démontrons que $(c) \implies (a)$.

    \item $\diamond$ l'associativité est montré par $\forall x, y, z \in H, x, y, z \in G, x * (y * z) = (x * y) * z$
    \item $\diamond$ l'élement neutre par $\exists x \in H, e = x * x^{-1} \in G$, ainsi $\forall x \in H, x \in G$
    \item $\diamond$ l'élement inverse par $x \in H$, prenons $y = e$, ainsi $x^{-1} * e = x^{-1}$, ici $x^{-1}$ est le symétrique de x dans H.
    \item $\diamond$ la stabilité par * dans H par $x, y \in H$, posons $z = y^{-1}$, ainsi $x * y = x * z^{-1} \in H$.
    
    \begin{rdem}
        Finalement par implication circulaire nous avons démontré que
        
        $(a) \iff (b) \iff (c)$
    \end{rdem}

\end{demonstration}

\end{document}
