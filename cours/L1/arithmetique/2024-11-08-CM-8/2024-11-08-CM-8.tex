\documentclass[a4paper, 12pt]{article}
\usepackage{amsmath, amssymb, amsthm}
\usepackage{geometry}
\usepackage{tcolorbox}
\geometry{hmargin=2.5cm, vmargin=2.5cm}

\renewcommand*{\today}{08 novembre 2024}

\title{Arithmetique | CM: 8}
\author{Par Lorenzo}
\date{\today}

\newtheorem{theorem}{Théorème}[section]
\newtheorem{definition}{Définition}[section]
\newtheorem{example}{Example}[section]
\newtheorem{remark}{Remarques}[section]
\newtheorem{lemme}{Lemme}[section]
\newtheorem{corollaire}{Corollaire}[section]

\newtheorem{_proposition}{Proposition}[section]
\newenvironment{proposition}[1][]{
    \begin{_proposition}[#1]~\par
    \vspace*{0.5em}
}{%
    \end{_proposition}%
}

\newtheorem{_proprietes}{Propriétés}[section]
\newenvironment{proprietes}[1][]{
        \begin{_proprietes}[#1]~\par
        \vspace*{0.5em}
}{%
        \end{_proprietes}%
}

\newenvironment{rdem}[1][]{
    \begin{tcolorbox}[colframe=black, colback=white!10, sharp corners]
        #1
}{%
    \end{tcolorbox}
     
}

\newtheorem{_demonstration}{Démonstration}[section]
\newenvironment{demonstration}[1][]{
    \begin{_demonstration}[#1]~\par
    \vspace*{0.5em}
}{%
    \end{_demonstration}%
    \qed%
}

\newtheorem*{_demonstration*}{Démonstration}
\newenvironment{demonstration*}[1][]{
    \begin{_demonstration*}[#1]~\par
    \vspace*{0.5em}
}{%
    \end{_demonstration*}%
    \qed%
}

\newenvironment{ldefinition}{
    \begin{definition}~\par
    \vspace*{0.5em}
    \begin{enumerate}
}{
        \end{enumerate}
        \end{definition}
}

\newenvironment{lexample}{
    \begin{example}~\par
    \vspace*{0.5em}
    \begin{enumerate}
}{
        \end{enumerate}
        \end{example}
}

\newtheorem{_methode}{Méthode}[section]
\newenvironment{methode}{
    \begin{_methode}~\par
    \vspace*{0.5em}
}{
        \end{_methode}
}

\def\N{\mathbb{N}}
\def\Z{\mathbb{Z}}
\def\Q{\mathbb{Q}}
\def\R{\mathbb{R}}
\def\C{\mathbb{C}}
\def\K{\mathbb{K}}
\def\k{\Bbbk}

\def\un{(u_n)_{n \in \N}}
\def\xn#1{(#1_n)_{n \in \N}}

\def\o{\overline}
\def\eps{\varepsilon}

% \funcdef{name}{domain}{codomain}{variable}{expression}
% name: Name of the function (e.g. f)
% domain: Domain of the function (e.g. \mathbb{R})
% codomain: Codomain of the function (e.g. \mathbb{R})
% variable: Variables of the function (e.g. x)
% expression: Expression of the function (e.g. x^2)
\newcommand{\funcdef}[5]{%
    #1 :
    \begin{cases}
        #2 \rightarrow #3 \\
        #4 \mapsto #5
    \end{cases}
}

\newcommand{\lt}{\ensuremath <}
\newcommand{\gt}{\ensuremath >}

\begin{document}

\maketitle

\begin{theorem}
    Soient $n_1, n_2, ..., n_k \in \N^*$, tels que $\forall i, n_i \geq 2$, avec les $n_i$ deux à deux premiers entre eux.
    Alors pour tous $a_1, ..., a_k \in \Z$, il existe $x \in \Z$, unique modulo
    $n := \Pi n_i$, tel que

    $$
        \forall i \in [|1, k|], x \equiv a_i mod n_i
    $$
    %todo

    Plus formellement, on a une application bijective,

    $$
    \{ \Z/n\Z \to (\Z/n_1\Z) \times ... \times (\Z/n_k\Z)
    \{ x mod n \mapsto (x mod n_1, ..., x mod n_k)
    $$
    %todo
\end{theorem}

\begin{demonstration}
    Montrons deja que

    $PGCD(\Pi_{i=1}^{k-1} n_i, n_k) = 1$

    Soit p un facteur premier de $\Pi_{i=1}^{k-1} n_i$
    Alors p divise l'un des $n_i$.

    Comme $n_i$ et $n_k$ sont premier entre eux p ne divise pas nk.

    Donc $\Pi_{i=1}^{k-1} n_i$ et $n_k$ n'ont pas de facteur premier en commun : leurs PGCD est 1.

    De même pour $i \in [|1;k|]$ $PGCD(\Pi_{i\neq j} n_j, n_i) = 1$.

    Ainsi on pose une relation de Bezout

    $$
        (\Pi_{i\neq j} n_j) u_i + n_i v_i = 1
    $$

    Soit $x_i := (\Pi_{j \neq i} n_j)u_i$

    Alors $x_i \equiv \{ 0 mod n_j si j \neq i \{ 1 mod n_i$
    %todo

    On pose $x = \Sigma_{i=1}^k a_i x_i$ alors $x \equiv a_i mod n_i$

    Si $y = x + qn$ alors $y = x + q \Pi_{j=1}^k n_j = x + q(\Pi_{j=1}^k n_j) n_i \equiv x mod n_i \equiv x_i mod n_i$

    En particulier l'application $\phi$ est bien définie

    D'après la première partie, $\phi$ est surjective.

    Il nous reste à démontrer l'injectivité qui est equivalente à l'unicité modulo n.

    Regardons les cardinaux $Card(\Z/n\Z) = n$

    $Card(\Z/n_1\Z \times ... \times \Z/n_k\Z) = n_1 \times ... \times n_k = n$

    Ainsi $\phi$ est injective
\end{demonstration}

\begin{remark}
    $\phi$ est un "isomorphisme" d'anneau
\end{remark}

Pour $k = z$

$$
\{ x \equiv a_1 mod n_1      \{ x =  a_1 + k_1 n_1
\{ x \equiv a_2 mod n_2 \iff \{ x = a_2 + k_2 n_2
$$
%todo
% replace also phi by varphi

Alors $a_1 + k_1 n_1 = a_2 + k_2 n_2 \iff k_1 n_1 - k_2 n_2 = a_2 - a_1$

c'est une équation diophotienne qu'on sait résoudre

Ensuite, il suffit de poser $x = a_1 + k_1 n_1$

\section{Polynômes et Fractions rationnelles}

\begin{definition}
    Un \textbf{polynôme à coefficient dans} $\k$: une suite $A = \xn{a}$
    telle que $\exists N \in \N, \forall n \gt N, a_n = 0$.

    On écrira souvent $A = a_0 + a_1X + a_2X^2 + ... + a_NX^N = \Sigma_{i=0}^N a_iX^i = \Sigma_{i \in \N} a_iX^i = \Sigma a_iX^i$

    \item $\k[X]$ = \{polynômes à coefficients dans $\k$\}
    \item \textbf{polynôme nul} : tous les coefficients sont nuls.
    \item \textbf{polynôme constant} : $\forall i \gt 0, a_i = 0$ ($A = cX^0 = c$ où $c \in \k$)
    \item \textbf{monôme} : polynôme de la forme %TODO
\end{definition}

Symbole de Kronecker $\delta_{i,j} = 1$ si i = j sinon 0

%TODO jusqu'a proposition anneau comm unitaire

\end{document}
