\documentclass[a4paper, 12pt]{article}
\usepackage{amsmath, amssymb, amsthm}
\usepackage{geometry}
\usepackage{tcolorbox}
\geometry{hmargin=2.5cm, vmargin=2.5cm}

\renewcommand*{\today}{17 octobre 2024}

\title{Science Decision | CM: 6}
\author{Par Lorenzo}
\date{\today}

\newtheorem{theorem}{Théorème}[section]
\newtheorem{definition}{Définition}[section]
\newtheorem{example}{Example}[section]
\newtheorem{remark}{Remarques}[section]
\newtheorem{lemme}{Lemme}[section]
\newtheorem{corollaire}{Corollaire}[section]

\newtheorem{_proposition}{Proposition}[section]
\newenvironment{proposition}[1][]{
    \begin{_proposition}[#1]~\par
    \vspace*{0.5em}
}{%
    \end{_proposition}%
}

\newtheorem{_proprietes}{Propriétés}[section]
\newenvironment{proprietes}[1][]{
        \begin{_proprietes}[#1]~\par
        \vspace*{0.5em}
}{%
        \end{_proprietes}%
}

\newenvironment{rdem}[1][]{
    \begin{tcolorbox}[colframe=black, colback=white!10, sharp corners]
        #1
}{%
    \end{tcolorbox}
     
}

\newtheorem{_demonstration}{Démonstration}[section]
\newenvironment{demonstration}[1][]{
    \begin{_demonstration}[#1]~\par
    \vspace*{0.5em}
}{%
    \end{_demonstration}%
    \qed%
}

\newtheorem*{_demonstration*}{Démonstration}
\newenvironment{demonstration*}[1][]{
    \begin{_demonstration*}[#1]~\par
    \vspace*{0.5em}
}{%
    \end{_demonstration*}%
    \qed%
}

\newenvironment{ldefinition}{
    \begin{definition}~\par
    \vspace*{0.5em}
    \begin{enumerate}
}{
        \end{enumerate}
        \end{definition}
}

\newenvironment{lexample}{
    \begin{example}~\par
    \vspace*{0.5em}
    \begin{enumerate}
}{
        \end{enumerate}
        \end{example}
}

\newtheorem{_methode}{Méthode}[section]
\newenvironment{methode}{
    \begin{_methode}~\par
    \vspace*{0.5em}
}{
        \end{_methode}
}

\def\N{\mathbb{N}}
\def\Z{\mathbb{Z}}
\def\Q{\mathbb{Q}}
\def\R{\mathbb{R}}
\def\C{\mathbb{C}}
\def\K{\mathbb{K}}
\def\k{\Bbbk}

\def\un{(u_n)_{n \in \N}}
\def\xn#1{(#1_n)_{n \in \N}}

\def\o{\overline}
\def\eps{\varepsilon}

% \funcdef{name}{domain}{codomain}{variable}{expression}
% name: Name of the function (e.g. f)
% domain: Domain of the function (e.g. \mathbb{R})
% codomain: Codomain of the function (e.g. \mathbb{R})
% variable: Variables of the function (e.g. x)
% expression: Expression of the function (e.g. x^2)
\newcommand{\funcdef}[5]{%
    #1 :
    \begin{cases}
        #2 \rightarrow #3 \\
        #4 \mapsto #5
    \end{cases}
}

\newcommand{\lt}{\ensuremath <}
\newcommand{\gt}{\ensuremath >}

\begin{document}

\maketitle

\subsection{Ordre faible et ordre total}

Soit R une relation binaire sur l'ensemble X.

\vspace{1em}

\noindent
On définit I et S sur X par

$\forall x \in X, y \in X, xIy \text{ si } xRy \land yRx$

$\forall x \in X, y \in X, xSy \text{ si } xRy \land \neg yRx$

\begin{example}
    $A = \{a, b, c\}$

    $R = \{(a, b), (b, a), (a, c), (b, c)\}$

    $I = \{(a, b), (b, a)\}$

    $S = \{(a, c), (b, c)\}$
\end{example}

\begin{proposition}
    si R est un ordre faible sur X, alors

    \item \textbf{1.} I est une relation d'équivalence
    \item \textbf{2.} S est irréflexive et transitive
\end{proposition}

\begin{demonstration}
    I relation d'équivalence:

    \item I reflexive
    
    Soit $x \in X, xIx \iff xRx \land xRX \iff xRx$ vrai car R est complète

    \item I symétrique
    
    Soient $x \in X, y \in X, xIy \implies xRy \land yRx \implies yRx \land xRy \implies yIx$

    \item I transitive
    
    Soient $x, y, z \in X, xIy \land yIz \implies xRy \land yRx \land yRz \land zRy \implies xRy \land yRz \land zRy \land yRx \implies xRz \land zRx \implies xIz$
\end{demonstration}

On définit $R^*$ sur $X/I$ par 

$\forall C_x \in X/I, C_y \in X/I, C_xR^*C_y \text{ lorsque } xRy$

$R^*$ sur X/I est la réduction (relation quotient) de R sur X

\begin{proposition}
    Si R est un ordre faible alors $R^*$ est un ordre total sur X/I
\end{proposition}

\begin{demonstration}
    \item $R^*$ antisymétrique
    
    Soient $C_x, C_y \in X/I, C_x R^* C_y \land C_y R^* C_x \implies C_x = C_y \implies xIy \implies y \in C_x \implies C_x = C_y$

    \item $R^*$ transitive
    
    Soient $C_x, C_y, C_z \in X/I$

    $C_x R^* C_y \land C_y R^* C_z \implies xRy \land yRz \implies xRz \implies C_x R^* C_z$


    \item $R^*$ complète
    
    $C_x, C_y \in X/I, C_xR^*C_y \lor C_yR^*C_x$
    car R complète
\end{demonstration}

\subsubsection{Irréflexive et transitive}

voir plus tard



\end{document}
