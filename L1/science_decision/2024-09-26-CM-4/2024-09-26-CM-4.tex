\documentclass[a4paper, 12pt]{article}
\usepackage{amsmath, amssymb, amsthm}
\usepackage{geometry}
\usepackage{tcolorbox}
\geometry{hmargin=2.5cm, vmargin=2.5cm}

\renewcommand*{\today}{26 septembre 2024}

\title{Science Decision | CM: 4}
\author{Par Lorenzo}
\date{\today}

\newtheorem{theorem}{Théorème}[section]
\newtheorem{definition}{Définition}[section]
\newtheorem{example}{Example}[section]
\newtheorem{remark}{Remarques}[section]
\newtheorem{lemme}{Lemme}[section]
\newtheorem{corollaire}{Corollaire}[section]

\newtheorem{_proposition}{Proposition}[section]
\newenvironment{proposition}[1][]{
    \begin{_proposition}[#1]~\par
    \vspace*{0.5em}
}{%
    \end{_proposition}%
}

\newtheorem{_proprietes}{Propriétés}[section]
\newenvironment{proprietes}[1][]{
        \begin{_proprietes}[#1]~\par
        \vspace*{0.5em}
}{%
        \end{_proprietes}%
}

\newenvironment{rdem}[1][]{
    \begin{tcolorbox}[colframe=black, colback=white!10, sharp corners]
        #1
}{%
    \end{tcolorbox}
     
}

\newtheorem{_demonstration}{Démonstration}[section]
\newenvironment{demonstration}[1][]{
    \begin{_demonstration}[#1]~\par
    \vspace*{0.5em}
}{%
    \end{_demonstration}%
    \qed%
}

\newtheorem*{_demonstration*}{Démonstration}
\newenvironment{demonstration*}[1][]{
    \begin{_demonstration*}[#1]~\par
    \vspace*{0.5em}
}{%
    \end{_demonstration*}%
    \qed%
}

\newenvironment{ldefinition}{
    \begin{definition}~\par
    \vspace*{0.5em}
    \begin{enumerate}
}{
        \end{enumerate}
        \end{definition}
}

\newenvironment{lexample}{
    \begin{example}~\par
    \vspace*{0.5em}
    \begin{enumerate}
}{
        \end{enumerate}
        \end{example}
}

\newtheorem{_methode}{Méthode}[section]
\newenvironment{methode}{
    \begin{_methode}~\par
    \vspace*{0.5em}
}{
        \end{_methode}
}

\def\N{\mathbb{N}}
\def\Z{\mathbb{Z}}
\def\Q{\mathbb{Q}}
\def\R{\mathbb{R}}
\def\C{\mathbb{C}}
\def\K{\mathbb{K}}
\def\k{\Bbbk}

\def\un{(u_n)_{n \in \N}}
\def\xn#1{(#1_n)_{n \in \N}}

\def\o{\overline}
\def\eps{\varepsilon}

% \funcdef{name}{domain}{codomain}{variable}{expression}
% name: Name of the function (e.g. f)
% domain: Domain of the function (e.g. \mathbb{R})
% codomain: Codomain of the function (e.g. \mathbb{R})
% variable: Variables of the function (e.g. x)
% expression: Expression of the function (e.g. x^2)
\newcommand{\funcdef}[5]{%
    #1 :
    \begin{cases}
        #2 \rightarrow #3 \\
        #4 \mapsto #5
    \end{cases}
}

\newcommand{\lt}{\ensuremath <}
\newcommand{\gt}{\ensuremath >}

\begin{document}

\maketitle

\subsection{Opérations sur les relations}

Puisque une relation R sur X est un sous ensemble de $X \times X$, on peut facilement utiliser
des opérations ensemblistes.

\begin{definition}
    Étant donné deux relation $R_1$ et $R_2$ sur un ensemble X.

    \item $\bullet$ la relation \textbf{complémentaire} de $R_1$, la relation binaire $R_1^c$ sur X telle que
    
    $\forall x, y \in X, \; x R_1^c y \text{ si } \neg(x R_1 y)$
    \item $\bullet$ la \textbf{réunion} de $R_1$ et $R_2$ est la relation binaire $R_1 \cup R_2$ sur X telle que
    
    $\forall x, y \in X, \; x R_1 \cup R_2 y \text{ si } x R_1 y \lor x R_2 y$
    \item $\bullet$ l'\textbf{intersection} de $R_1$ et $R_2$ est la relation binaire $R_1 \cap R_2$ telle que
    
    $\forall x, y \in X, \; x R_1 \cap R_2 y \text{ si } x R_1 y \land x R_2 y$
    \item $\bullet$ la relation $R_1$ est \textbf{compatible} avec $R_2$ si
    
    $\forall x, y \in X, \; x R_1 y \implies x R_2 y$ ou de manière équivalente $R_1 \subset R_2$
    \item $\bullet$ la relation \textbf{réciproque} (ou duale, inverse) de $R_1$,
    la relation binaire $R_1^{-1}$ sur X telle que
    
    $\forall x, y \in X, \; y R_1^{-1} x \text{ si } x R_1 y$
    \item $\bullet$ la \textbf{composée} de $R_1$ et $R_2$,
    la relation binaire $R_1 \circ R_2$ sur X telle que

    $\forall x, y \in X, \; x R_1 \circ R_2 y \text{ si }\exists z \in X, x R_2 z \land z R_1 y$
\end{definition}

\subsection{Relations d'équivalence}

\begin{proposition}
    L'intersection $R_1 \cap R_2$ de deux relations d'équivalences $R_1$ et $R_2$ sur un ensemble X est une relation d'équivalence.
\end{proposition}

\begin{demonstration}
    \item $\bullet$ \textbf{Réflexive} car $\forall x \in X, \; x R_1 x \land x R_2 x$, ainsi $x R_1 \cap R_2 x$ pour tout $x \in X$.
    \item $\bullet$ \textbf{Symétrie} car $\forall x, y \in X, \; (x R_1 y \land y R_1 x) \land (x R_2 y \land y R_2 x)$, ainsi $\forall x, y \in X, \; x R_1 y \land x R_2 y$ ce qui implique que $y R_1 x \land y R_2 x$
    soit $\forall x, y \in X, \; (x R_1 y \land y R_2 x) \land (x R_2 y \land y R_1 x)$.
    \item $\bullet$ \textbf{Transitive} car $\forall x, y \in X, x R_1 y \land x R_2 y \land y R_1 z \land y R_2 z \implies xR_1z \land xR_2z \implies x R_1 \cap R_2 z$
    
    \begin{rdem}
        $R_1 \cap R_2$ est Réflexive, Symétrique, Transitive donc c'est une relation d'équivalence.
    \end{rdem}
\end{demonstration}

\end{document}
