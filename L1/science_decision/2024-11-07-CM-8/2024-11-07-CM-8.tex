\documentclass[a4paper, 12pt]{article}
\usepackage{amsmath, amssymb, amsthm}
\usepackage{geometry}
\usepackage{tcolorbox}
\geometry{hmargin=2.5cm, vmargin=2.5cm}

\renewcommand*{\today}{07 novembre 2024}

\title{Science Decision | CM: 8}
\author{Par Lorenzo}
\date{\today}

\newtheorem{theorem}{Théorème}[section]
\newtheorem{definition}{Définition}[section]
\newtheorem{example}{Example}[section]
\newtheorem{remark}{Remarques}[section]
\newtheorem{lemme}{Lemme}[section]
\newtheorem{corollaire}{Corollaire}[section]

\newtheorem{_proposition}{Proposition}[section]
\newenvironment{proposition}[1][]{
    \begin{_proposition}[#1]~\par
    \vspace*{0.5em}
}{%
    \end{_proposition}%
}

\newtheorem{_proprietes}{Propriétés}[section]
\newenvironment{proprietes}[1][]{
        \begin{_proprietes}[#1]~\par
        \vspace*{0.5em}
}{%
        \end{_proprietes}%
}

\newenvironment{rdem}[1][]{
    \begin{tcolorbox}[colframe=black, colback=white!10, sharp corners]
        #1
}{%
    \end{tcolorbox}
     
}

\newtheorem{_demonstration}{Démonstration}[section]
\newenvironment{demonstration}[1][]{
    \begin{_demonstration}[#1]~\par
    \vspace*{0.5em}
}{%
    \end{_demonstration}%
    \qed%
}

\newtheorem*{_demonstration*}{Démonstration}
\newenvironment{demonstration*}[1][]{
    \begin{_demonstration*}[#1]~\par
    \vspace*{0.5em}
}{%
    \end{_demonstration*}%
    \qed%
}

\newenvironment{ldefinition}{
    \begin{definition}~\par
    \vspace*{0.5em}
    \begin{enumerate}
}{
        \end{enumerate}
        \end{definition}
}

\newenvironment{lexample}{
    \begin{example}~\par
    \vspace*{0.5em}
    \begin{enumerate}
}{
        \end{enumerate}
        \end{example}
}

\newtheorem{_methode}{Méthode}[section]
\newenvironment{methode}{
    \begin{_methode}~\par
    \vspace*{0.5em}
}{
        \end{_methode}
}

\def\N{\mathbb{N}}
\def\Z{\mathbb{Z}}
\def\Q{\mathbb{Q}}
\def\R{\mathbb{R}}
\def\C{\mathbb{C}}
\def\K{\mathbb{K}}
\def\k{\Bbbk}

\def\un{(u_n)_{n \in \N}}
\def\xn#1{(#1_n)_{n \in \N}}

\def\o{\overline}
\def\eps{\varepsilon}

% \funcdef{name}{domain}{codomain}{variable}{expression}
% name: Name of the function (e.g. f)
% domain: Domain of the function (e.g. \mathbb{R})
% codomain: Codomain of the function (e.g. \mathbb{R})
% variable: Variables of the function (e.g. x)
% expression: Expression of the function (e.g. x^2)
\newcommand{\funcdef}[5]{%
    #1 :
    \begin{cases}
        #2 \rightarrow #3 \\
        #4 \mapsto #5
    \end{cases}
}

\newcommand{\lt}{\ensuremath <}
\newcommand{\gt}{\ensuremath >}

\begin{document}

\maketitle

\begin{definition}
    Une \textbf{chaîne} est un ensemble d'éléments de X totalement ordonné
\end{definition}

\begin{definition}
    Une \textbf{antichaîne} si $\forall x, xRy \lor yRx \implies x = y$
\end{definition}

Soit c le nombre minimal de chaînes pour partitionner X

Soit A une anti-chaîne de cardinal maximal a

\begin{remark}
    partition avec le plus grand nombre de chaînes : \{\{a\}, \{b\}, ..., \{g\}\}
\end{remark}

\begin{proposition}
    X ensemble partiellement ordonné par R.

    Le nombre d'éléments d'une antichaîne de cardianl maximal (a) est
    égale au nombre minimum de chaînes pour partitionner X.
\end{proposition}

\begin{demonstration}
    Preuve par récurrence sur $|x|$

    Init. $|x| = 1$

    X est une chaîne et une antichaîne a = 1

    on partitionne X en 1 chaîne c = a = 1

    here. Suppose que ça marche pour tous jusqu'a n

    2 cas:

    (a) si X contient une antichaîne de cardinal a contenant au moins un element non mimimal et au moins un element maximal

    (b) si X contient une antichaîne de cardinal a contenant que des elements maximaux ou minimaux

    (a) Soient $H = \{x \in X | \exists z \in A, xRz\}$ et $B = \{x \in X | \exists z \in A, zRx\}$

    du fait de (a) $\exists w \in A$ non maximal implique $\exists y \in X, yRw$ et $y \notin B$ donc $|B| \leq n - 1$
    donc B peut être partitionné en a chaînes
\end{demonstration}

\end{document}
