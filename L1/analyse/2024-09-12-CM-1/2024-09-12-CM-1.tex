\documentclass[a4paper, 12pt]{article}
\usepackage{amsmath, amssymb, amsthm}
\usepackage{geometry}
\usepackage{tcolorbox}
\geometry{hmargin=2.5cm, vmargin=2.5cm}

\renewcommand*{\today}{12 septembre 2024}

\title{Analyse | CM: 1}
\author{Par Lorenzo}
\date{\today}

\newtheorem{theorem}{Théorème}[section]
\newtheorem{definition}{Définition}[section]
\newtheorem{example}{Example}[section]
\newtheorem{remark}{Remarques}[section]
\newtheorem{lemme}{Lemme}[section]
\newtheorem{corollaire}{Corollaire}[section]

\newtheorem{_proposition}{Proposition}[section]
\newenvironment{proposition}[1][]{
    \begin{_proposition}[#1]~\par
    \vspace*{0.5em}
}{%
    \end{_proposition}%
}

\newtheorem{_proprietes}{Propriétés}[section]
\newenvironment{proprietes}[1][]{
        \begin{_proprietes}[#1]~\par
        \vspace*{0.5em}
}{%
        \end{_proprietes}%
}

\newenvironment{rdem}[1][]{
    \begin{tcolorbox}[colframe=black, colback=white!10, sharp corners]
        #1
}{%
    \end{tcolorbox}
     
}

\newtheorem{_demonstration}{Démonstration}[section]
\newenvironment{demonstration}[1][]{
    \begin{_demonstration}[#1]~\par
    \vspace*{0.5em}
}{%
    \end{_demonstration}%
    \qed%
}

\newtheorem*{_demonstration*}{Démonstration}
\newenvironment{demonstration*}[1][]{
    \begin{_demonstration*}[#1]~\par
    \vspace*{0.5em}
}{%
    \end{_demonstration*}%
    \qed%
}

\newenvironment{ldefinition}{
    \begin{definition}~\par
    \vspace*{0.5em}
    \begin{enumerate}
}{
        \end{enumerate}
        \end{definition}
}

\newenvironment{lexample}{
    \begin{example}~\par
    \vspace*{0.5em}
    \begin{enumerate}
}{
        \end{enumerate}
        \end{example}
}

\newtheorem{_methode}{Méthode}[section]
\newenvironment{methode}{
    \begin{_methode}~\par
    \vspace*{0.5em}
}{
        \end{_methode}
}

\def\N{\mathbb{N}}
\def\Z{\mathbb{Z}}
\def\Q{\mathbb{Q}}
\def\R{\mathbb{R}}
\def\C{\mathbb{C}}
\def\K{\mathbb{K}}
\def\k{\Bbbk}

\def\un{(u_n)_{n \in \N}}
\def\xn#1{(#1_n)_{n \in \N}}

\def\o{\overline}
\def\eps{\varepsilon}

% \funcdef{name}{domain}{codomain}{variable}{expression}
% name: Name of the function (e.g. f)
% domain: Domain of the function (e.g. \mathbb{R})
% codomain: Codomain of the function (e.g. \mathbb{R})
% variable: Variables of the function (e.g. x)
% expression: Expression of the function (e.g. x^2)
\newcommand{\funcdef}[5]{%
    #1 :
    \begin{cases}
        #2 \rightarrow #3 \\
        #4 \mapsto #5
    \end{cases}
}

\newcommand{\lt}{\ensuremath <}
\newcommand{\gt}{\ensuremath >}

\begin{document}

\maketitle

\section{L'ensemble des nombres rationnels $\Q$}

\subsection{Ecriture décimale}

\begin{definition}
    On definit l'ensemble des nombres rationnels $\Q$ par \break
    $\Q = \{\dfrac{p}{q} \mid p \in \Z, q \in \N^*\}$,
    optionellement $pgcd(p, q) = 1$ peut être rajouté dans la définition.
    Cela ajoute le fait que p et q sont premier entre eux et donc $\dfrac{p}{q}$ un fraction irréductible.
    (rappel: $\N^*=\N\backslash\{0\}$, i.e. $\N$ privé de 0).
\end{definition}

\begin{remark}
    Les nombres décimaux sont des nombres de la forme \break $\dfrac{p}{10^n}$ avec $p \in \Z, n \in \N$
    (e.g. $1.234 = \dfrac{1234}{10^3}$).
\end{remark}

\begin{proposition}
    Un nombre est rationnel si et seulement si il admet une écriture décimale finie ou périodique.
\end{proposition}

\begin{demonstration}
    Démontrons que si un nombre à une partie décimale finie ou périodique, alors il est rationnel.

    \vspace{1em}

    a) partie décimale finie

    \vspace{0.5em}

    1. Supposons que x soit un nombre réel avec une partie décimale finie.
    x peut s'écrire sous la forme:
    \begin{align*}
        x = a.b_1b_2...b_n
    \end{align*}
    Ou $a \in \Z$ est la partie entière et $b_1b_2...b_n$ avec $n \in \N$ représente les chiffres de la partie décimale finie.
    
    \vspace{0.5em}

    2. On peut écrire x comme:
    \begin{align*}
        x = a + \dfrac{b_1b_2...b_n}{10^n}
    \end{align*}

    Ici, a est un entier et $\dfrac{b_1b_2...b_n}{10^n}$ un nombre rationnel.

    \begin{rdem}
        La somme d'un entier (nombre rationnel) et d'un nombre rationnel est un nombre rationnel.
    \end{rdem}

    \vspace{1em}

    b) partie décimale périodique

    \vspace{0.5em}

    1. Supposons que x soit un nombre réel avec une partie décimale périodique.
    x peut s'écrire sous la forme:
    \begin{align*}
        x = a.b_1b_2...b_n\overline{c_1c_2...c_m}
    \end{align*}
    où $a \in \Z$ est la partie entière, $b_1b_2...b_n$ sont les chiffres non répétitifs initiaux, et
    $\overline{c_1c_2...c_m}$ est le bloc périodique.

    \vspace{0.5em}

    2. Pour simplifier la démonstration, on peut isoler la partie périodique. Posons:
    \begin{align*}
        y = 0.\overline{c_1c_2...c_m}
    \end{align*}
    \vspace{0.5em}
    
    3. On multiplie y par $10^m$, où m est la longeur de la période.
    \begin{align*}
        &10^m y = c_1c_2...c_m + y \\
        \iff& 10^m y - y = c_1c_2...c_m \\
        \iff& y(10^m - 1) = c_1c_2...c_m \\
        \iff& y = \dfrac{c_1c_2...c_m}{10^m - 1} \\
    \end{align*}
    y est donc rationnel.

    \vspace{0.5em}

    4. On peut réexprimer x
    \begin{align*}
        x = a + \dfrac{b_1b_2...b_n}{10^n} + y
    \end{align*}

    \begin{rdem}
        La somme de nombre rationnel donne un nombre rationnel.
    \end{rdem}
    
    \begin{rdem}
        Ainsi si un nombre à une partie décimale finie ou périodique, alors il est rationnel.
    \end{rdem}


\end{demonstration}

\begin{demonstration}
    Démontrons que si un nombre est rationnel, alors sa partie décimale est finie ou périodique.

    Supposons que x est un nombre rationnel ainsi il s'écrit sous la forme
    $x = \dfrac{p}{q}$ avec $p \in \Z \text{ et } q \in \N^*$

    Lorsque qu'on effectue la division euclidienne $\dfrac{p}{q}$ deux cas se présentent:

    La division se termine par un nombre fini d'étapes.

    La division ne se termine pas et répète une séquence de chiffres.

    \begin{rdem}
        Ainsi si un nombre est rationnel, alors sa partie décimale est finie ou périodique.
    \end{rdem}

\end{demonstration}

\begin{example}
    Prenons $x = 12.34202320232023...$

    \vspace{1em}

    Etape 1: faire apparaitre la partie périodique à la virgule.
    Ici on multiplie par 100
    \begin{flalign}
        100x = 1\,234.202320232023...&&
    \end{flalign}
    Etape 2: on décale d'une période vers la gauche.
    Ici la période est de longeur 4 donc on multiplie par 10 000.
    \begin{flalign}
        10\,000 \times 100x = 12\,342\,023.20232023...&&
    \end{flalign}
    Etape 3: on soustrait (2) par (1) pour que la partie décimale s'annule.
    \begin{flalign}
        &10\,000 \times 100x - 100x = 12\,342\,023 - 1\,234&& \\
        \iff &999\,900x = 12\,340\,789&& \\
        \iff &x = \dfrac{12\,340\,789}{999\,900}&& \\
    \end{flalign}
\end{example}

\end{document}
