\documentclass[a4paper, 12pt]{article}
\usepackage{amsmath, amssymb, amsthm}
\usepackage{geometry}
\usepackage{tcolorbox}
\geometry{hmargin=2.5cm, vmargin=2.5cm}

\renewcommand*{\today}{27 septembre 2024}

\title{Ensemble Complex | CM: 4}
\author{Par Lorenzo}
\date{\today}

\newtheorem{theorem}{Théorème}[section]
\newtheorem{definition}{Définition}[section]
\newtheorem{example}{Example}[section]
\newtheorem{remark}{Remarques}[section]
\newtheorem{lemme}{Lemme}[section]
\newtheorem{corollaire}{Corollaire}[section]

\newtheorem{_proposition}{Proposition}[section]
\newenvironment{proposition}[1][]{
    \begin{_proposition}[#1]~\par
    \vspace*{0.5em}
}{%
    \end{_proposition}%
}

\newtheorem{_proprietes}{Propriétés}[section]
\newenvironment{proprietes}[1][]{
        \begin{_proprietes}[#1]~\par
        \vspace*{0.5em}
}{%
        \end{_proprietes}%
}

\newenvironment{rdem}[1][]{
    \begin{tcolorbox}[colframe=black, colback=white!10, sharp corners]
        #1
}{%
    \end{tcolorbox}
     
}

\newtheorem{_demonstration}{Démonstration}[section]
\newenvironment{demonstration}[1][]{
    \begin{_demonstration}[#1]~\par
    \vspace*{0.5em}
}{%
    \end{_demonstration}%
    \qed%
}

\newtheorem*{_demonstration*}{Démonstration}
\newenvironment{demonstration*}[1][]{
    \begin{_demonstration*}[#1]~\par
    \vspace*{0.5em}
}{%
    \end{_demonstration*}%
    \qed%
}

\newenvironment{ldefinition}{
    \begin{definition}~\par
    \vspace*{0.5em}
    \begin{enumerate}
}{
        \end{enumerate}
        \end{definition}
}

\newenvironment{lexample}{
    \begin{example}~\par
    \vspace*{0.5em}
    \begin{enumerate}
}{
        \end{enumerate}
        \end{example}
}

\newtheorem{_methode}{Méthode}[section]
\newenvironment{methode}{
    \begin{_methode}~\par
    \vspace*{0.5em}
}{
        \end{_methode}
}

\def\N{\mathbb{N}}
\def\Z{\mathbb{Z}}
\def\Q{\mathbb{Q}}
\def\R{\mathbb{R}}
\def\C{\mathbb{C}}
\def\K{\mathbb{K}}
\def\k{\Bbbk}

\def\un{(u_n)_{n \in \N}}
\def\xn#1{(#1_n)_{n \in \N}}

\def\o{\overline}
\def\eps{\varepsilon}

% \funcdef{name}{domain}{codomain}{variable}{expression}
% name: Name of the function (e.g. f)
% domain: Domain of the function (e.g. \mathbb{R})
% codomain: Codomain of the function (e.g. \mathbb{R})
% variable: Variables of the function (e.g. x)
% expression: Expression of the function (e.g. x^2)
\newcommand{\funcdef}[5]{%
    #1 :
    \begin{cases}
        #2 \rightarrow #3 \\
        #4 \mapsto #5
    \end{cases}
}

\newcommand{\lt}{\ensuremath <}
\newcommand{\gt}{\ensuremath >}

\begin{document}

\maketitle

\section{Théorie des ensembles}

\subsection{Opérations sur les ensembles}

\begin{definition}
    Un ensemble est une collection d'éléments. Il est défini par la connaissance de ses éléments.

    Soit A un ensemble $a \in A$ signifie que a appartient à A. On dit alors que a est un élément de A.
\end{definition}

\begin{remark}
    La définition d'un ensemble peut se faire des façon suivante:

    \item $\bullet$ liste exaustive ({1, 2, 3})
    \item $\bullet$ paramétrique ($\{2x + 1 \mid x \in \N\}$)
    \item $\bullet$ inplicite ($\{x \in \R \mid x(x+1) \gt 0\}$)
\end{remark}

\begin{remark}
    Dans un ensemble l'ordre et la répétition n'a pas son importance.
\end{remark}

\begin{definition}
    Soient A et B deux ensembles. On dit que A est un sous-ensemble de B lorsque $\forall x \in A, x \in B$, on note plus $A \subset B$.

    Soit A un ensemble fini, le cardinal de A est le nombre d'éléments de A, noté $card A$.

    Un ensemble avec un seul élément est un singleton.

    Un ensemble qui ne contient aucun éléments est appelé l'ensemble vide (noté $\emptyset$ ou \{\}),
    c'est un sous ensemble de tout les ensembles.
\end{definition}

\begin{remark}
    Un quantificateur universelle sur l'ensemble vide est automatiquement vérifié.
    (e.g. $\forall x \in \emptyset, P(x)$)
\end{remark}

\begin{definition}
    Soient A, B des parties d'un ensemble E.

    La réunion de A et de B, notée $A \cup B$ est la partie de E dont les éléments sont éléments de A ou de B.

    $A \cup B = \{x \in E, x \in A \lor x \in B\}$
\end{definition}

\begin{definition}
    Soient A, B des parties d'un ensemble E.

    L'intersection de A et de B, notée $A \cap B$ est la partie de E dont les éléments sont éléments de A et de B.

    $A \cap B = \{x \in E, x \in A \land x \in B\}$
\end{definition}

\begin{remark}
    La réunion n'est pas un ou exclusive.
\end{remark}

\begin{remark}
    $A \cup B$ est le plus petit ensemble contenant A et B
\end{remark}

\begin{remark}
    $A \cap B$ est le plus grand ensemble contenu dans A et B
\end{remark}

\begin{remark}
    Comme un élement peut seulement être ou ne pas être dans un ensemble, on peut faire une disjonction de cas.
\end{remark}

\begin{definition}
    Soient A, B deux sous ensemble d'un ensemble E.

    \item $\bullet$ A et B sont dits disjoints si $A \cap B = \emptyset$
    \item $\bullet$ Le complémentaire de A dans E est la partie de E dont les éléments sont tous les éléments de E qui ne sont pas dans A.
    On le note $E\backslash A = \{x \in E \mid x \notin A\}$. Autres notations: $C_E A$ ou $A^C$
    \item $\bullet$ La différence symétrique de A et B, notée $A \Delta B := (A \backslash B) \cup (B \backslash A)$
\end{definition}

%todo

\begin{definition}
    Soit I un ensemble, Soient $(A_i)_{i \in I}$ des sous ensembles d'une ensemble E.

    L'intersection des $A_i$ est $\Cap_{i \in I} A_i := \{x \in E, \forall i \in I, x \in A_i\}$
    
    L'union des $A_i$ est $\Cup_{i \in I} A_i := \{x \in E, \exists i \in I, x \in A_i\}$

    Par convention: si  $I = \emptyset$ alors $\Cup_{i \in I} A_i := 0$ et $I = \emptyset$ alors $\Cap_{i \in I} A_i := E$
\end{definition}

\begin{definition}
    Soient A, B deux sous ensembles non vides de E.

    A et B sont complémentaires dans E ou forment une partition de E si
    $E = A \cup B$ et $A \cap B = \emptyset$
    %todo A_i partition de E
\end{definition}

\begin{remark}
    Le non complémentaire vient du fait qu'une autre définition soit $A = E\backslash B \iff B = E\backslash A$
\end{remark}

Soit E un ensemble. On note $P(E)$ l'ensemble des parties de E.

\begin{remark}
    Il est équivalent d'écrire $A \subset E$ ou $A \in P(E)$
\end{remark}

\begin{remark}
    Pour tout ensemble E, on a $\emptyset \in P(E)$ et $E \in P(E)$
\end{remark}

\begin{theorem}
    Lorsque $card(E) = n$ avec $n \in \N$ alors $card(P(E)) = 2^n$
\end{theorem}

\begin{demonstration}
\textbf{Initialisation:} $card(E) = 0 \implies E = \emptyset$ alors $P(E) = \{\emptyset\}$ donc $card(P(E)) = 1 = 2^0$ 

\textbf{Hérédité:} Soit E de cardinal $n \geq 1$. Soit $a \in E$, $F = E\backslash \{a\}$

$card(F) = n - 1$

Les parties de E sont les X et les $X \cup \{a\}$ où $X \in P(F)$

Ainsi card(P(E)) = card(P(F)) + card(P(F))
%todo
\end{demonstration}

\begin{definition}
    Soient E et F deux ensembles.

    Le produit cartésien de E par F est l'ensemble $E \times F = \{(x, y) \mid x \in E \land y \in F\}$
\end{definition}

\begin{remark}
    Attention ce n'est pas commutatif, $E \times F \neq F \times E$
\end{remark}

\begin{definition}
    Soient $E_1, E_2, ..., E_n$ des ensembles.

    $E_1 \times E_2 \times ... \times E_n = \{(x_1, x_2, ..., x_n), \forall i \in \{1, 2, ..., n\}, x_i \in E_i\}$

    $(x_1, x_2, ..., x_n)$ est appelé un n-uplet.
    %todo E^n
\end{definition}

\end{document}
