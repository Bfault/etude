\documentclass[a4paper, 12pt]{article}
\usepackage{amsmath, amssymb, amsthm}
\usepackage{geometry}
\usepackage{tcolorbox}
\geometry{hmargin=2.5cm, vmargin=2.5cm}

\renewcommand*{\today}{18 octobre 2024}

\title{Ensemble Complex | CM: 7}
\author{Par Lorenzo}
\date{\today}

\newtheorem{theorem}{Théorème}[section]
\newtheorem{definition}{Définition}[section]
\newtheorem{example}{Example}[section]
\newtheorem{remark}{Remarques}[section]
\newtheorem{lemme}{Lemme}[section]
\newtheorem{corollaire}{Corollaire}[section]

\newtheorem{_proposition}{Proposition}[section]
\newenvironment{proposition}[1][]{
    \begin{_proposition}[#1]~\par
    \vspace*{0.5em}
}{%
    \end{_proposition}%
}

\newtheorem{_proprietes}{Propriétés}[section]
\newenvironment{proprietes}[1][]{
        \begin{_proprietes}[#1]~\par
        \vspace*{0.5em}
}{%
        \end{_proprietes}%
}

\newenvironment{rdem}[1][]{
    \begin{tcolorbox}[colframe=black, colback=white!10, sharp corners]
        #1
}{%
    \end{tcolorbox}
     
}

\newtheorem{_demonstration}{Démonstration}[section]
\newenvironment{demonstration}[1][]{
    \begin{_demonstration}[#1]~\par
    \vspace*{0.5em}
}{%
    \end{_demonstration}%
    \qed%
}

\newtheorem*{_demonstration*}{Démonstration}
\newenvironment{demonstration*}[1][]{
    \begin{_demonstration*}[#1]~\par
    \vspace*{0.5em}
}{%
    \end{_demonstration*}%
    \qed%
}

\newenvironment{ldefinition}{
    \begin{definition}~\par
    \vspace*{0.5em}
    \begin{enumerate}
}{
        \end{enumerate}
        \end{definition}
}

\newenvironment{lexample}{
    \begin{example}~\par
    \vspace*{0.5em}
    \begin{enumerate}
}{
        \end{enumerate}
        \end{example}
}

\newtheorem{_methode}{Méthode}[section]
\newenvironment{methode}{
    \begin{_methode}~\par
    \vspace*{0.5em}
}{
        \end{_methode}
}

\def\N{\mathbb{N}}
\def\Z{\mathbb{Z}}
\def\Q{\mathbb{Q}}
\def\R{\mathbb{R}}
\def\C{\mathbb{C}}
\def\K{\mathbb{K}}
\def\k{\Bbbk}

\def\un{(u_n)_{n \in \N}}
\def\xn#1{(#1_n)_{n \in \N}}

\def\o{\overline}
\def\eps{\varepsilon}

% \funcdef{name}{domain}{codomain}{variable}{expression}
% name: Name of the function (e.g. f)
% domain: Domain of the function (e.g. \mathbb{R})
% codomain: Codomain of the function (e.g. \mathbb{R})
% variable: Variables of the function (e.g. x)
% expression: Expression of the function (e.g. x^2)
\newcommand{\funcdef}[5]{%
    #1 :
    \begin{cases}
        #2 \rightarrow #3 \\
        #4 \mapsto #5
    \end{cases}
}

\newcommand{\lt}{\ensuremath <}
\newcommand{\gt}{\ensuremath >}

\begin{document}

\maketitle

\begin{definition}
    Soient E et F deux ensembles arbitraires On dit que E et F ont même cardinal
    s'il existe une bijection entre E et F.

    Soit E un ensemble. On dit que E est dénombrable s'il existe une injection de E dans~$\N$
\end{definition}

\begin{proposition}
    $\N^*, \Z, \N^2, \Q, \N^n$ sont dénombrables.

    $\R, P(\N)$ ne sont pas dénombrables

    $P(\N)$ et $\R$ ont même cardinal.

    %todo
\end{proposition}

\section{Relation et permutations}

\begin{definition}
    Soit E un ensemble non vide. Une relation binaire R sur E est la donnée d'une application
    $E \times E \to {\text{Vrai}, \text{Faux}}$.

    On dit que x est en relation avec y lorsque l'image de (x, y) par l'application
    est "Vrai" et on note alors xRy
\end{definition}

\begin{remark}
    %todo
\end{remark}

\begin{definition}
    Soit E un ensemble non vide et R une relation binaire sur E. On dit que R est

    \item \textbf{réflexive} lorsque $\forall x \in E, xRx$
    \item \textbf{symétrique} lorsque $\forall(x, y) \in E^2, xRy \iff yRx$
    \item \textbf{antisymétrique} lorsque $\forall (x, y) \in E^2, (xRy \land yRx) \implies x = y$
    \item \textbf{transitive} lorsque $\forall (x, y, z) \in E^3, xRy \land yRz \implies xRz$
\end{definition}

\begin{definition}
    Soit R une relation sur un ensemble E. On dit que R est une relation d'équivalence si elle est
    réflexive, symétrique et transitive.
\end{definition}

\begin{definition}
    Soit $E \neq \emptyset$ un ensemble muni d'une rel. d'équivalence R.

    Soit $x \in E$.

    On appelle classe d'équivalence modulo R de x et on note $\o{x}$ l'ensemble $\{x \in E, xRy\}$.
\end{definition}

\begin{theorem}
    L'ensemble des classes d'équivalence de E modulo R forme une partition de E.
\end{theorem}

\begin{demonstration}
%todo
\end{demonstration}

\begin{definition}
    L'ensemble des classes d'équivalence de E modulo R s'appelle l'ensemble quotient
    de E par R. On le note E/R.
\end{definition}

\begin{definition}
    Soit R une relation sur un ensemble E.

    On dit que R est une relation d'ordre si elle est réflexive, antisymétrique et transitive.
    Notée souvent $\preccurlyeq$ cursif. On dit que $(E, \preccurlyeq)$ est un ensemble ordonné.

    Relation d'ordre totale lorsque R est complète ($\forall x, y \in E, (xRy \lor yRx)$)
\end{definition}

\begin{definition}
    Soit $(E, \preccurlyeq)$ ensemble ordonné. Soit $A \in P(E)$. On dit que

    \item A admet un minimum lorsque
    
    $\exists a_0 \in A, \forall A, a_0 \preccurlyeq a$ On note $min(A) := a_0$

    \item A admet un maximum lorsque
    
    $\exists a_0 \in A, \forall a \in A, a \preccurlyeq a_0$ On note $max(A) := a_0$

    \item A est minoré lorsque
    
    $\exists m \in E, \forall a \in A, m \preccurlyeq a$

    %todo
\end{definition}

\begin{remark}
    Si A admet un minimum (resp. un maximum) alors A est minoré (resp. majoré)
\end{remark}

\begin{remark}
    Si A admet un minimum (resp. maximum), il est unique.
\end{remark}

\begin{proposition}
    Soit E un ensemble muni d'une relation d'ordre totale $\preccurlyeq$.
    Soit $A \in P(E)$ un ensemble fini non-vide. Alors A admet un minimum et un maximum.
\end{proposition}

\end{document}
