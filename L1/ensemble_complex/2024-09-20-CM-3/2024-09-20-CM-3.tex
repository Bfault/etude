\documentclass[a4paper, 12pt]{article}
\usepackage{amsmath, amssymb, amsthm}
\usepackage{geometry}
\usepackage{tcolorbox}
\geometry{hmargin=2.5cm, vmargin=2.5cm}

\renewcommand*{\today}{20 septembre 2024}

\title{Ensemble Complex | CM: 3}
\author{Par Lorenzo}
\date{\today}

\newtheorem{theorem}{Théorème}[section]
\newtheorem{definition}{Définition}[section]
\newtheorem{example}{Example}[section]
\newtheorem{remark}{Remarques}[section]
\newtheorem{lemme}{Lemme}[section]
\newtheorem{corollaire}{Corollaire}[section]

\newtheorem{_proposition}{Proposition}[section]
\newenvironment{proposition}[1][]{
    \begin{_proposition}[#1]~\par
    \vspace*{0.5em}
}{%
    \end{_proposition}%
}

\newtheorem{_proprietes}{Propriétés}[section]
\newenvironment{proprietes}[1][]{
        \begin{_proprietes}[#1]~\par
        \vspace*{0.5em}
}{%
        \end{_proprietes}%
}

\newenvironment{rdem}[1][]{
    \begin{tcolorbox}[colframe=black, colback=white!10, sharp corners]
        #1
}{%
    \end{tcolorbox}
     
}

\newtheorem{_demonstration}{Démonstration}[section]
\newenvironment{demonstration}[1][]{
    \begin{_demonstration}[#1]~\par
    \vspace*{0.5em}
}{%
    \end{_demonstration}%
    \qed%
}

\newtheorem*{_demonstration*}{Démonstration}
\newenvironment{demonstration*}[1][]{
    \begin{_demonstration*}[#1]~\par
    \vspace*{0.5em}
}{%
    \end{_demonstration*}%
    \qed%
}

\newenvironment{ldefinition}{
    \begin{definition}~\par
    \vspace*{0.5em}
    \begin{enumerate}
}{
        \end{enumerate}
        \end{definition}
}

\newenvironment{lexample}{
    \begin{example}~\par
    \vspace*{0.5em}
    \begin{enumerate}
}{
        \end{enumerate}
        \end{example}
}

\newtheorem{_methode}{Méthode}[section]
\newenvironment{methode}{
    \begin{_methode}~\par
    \vspace*{0.5em}
}{
        \end{_methode}
}

\def\N{\mathbb{N}}
\def\Z{\mathbb{Z}}
\def\Q{\mathbb{Q}}
\def\R{\mathbb{R}}
\def\C{\mathbb{C}}
\def\K{\mathbb{K}}
\def\k{\Bbbk}

\def\un{(u_n)_{n \in \N}}
\def\xn#1{(#1_n)_{n \in \N}}

\def\o{\overline}
\def\eps{\varepsilon}

% \funcdef{name}{domain}{codomain}{variable}{expression}
% name: Name of the function (e.g. f)
% domain: Domain of the function (e.g. \mathbb{R})
% codomain: Codomain of the function (e.g. \mathbb{R})
% variable: Variables of the function (e.g. x)
% expression: Expression of the function (e.g. x^2)
\newcommand{\funcdef}[5]{%
    #1 :
    \begin{cases}
        #2 \rightarrow #3 \\
        #4 \mapsto #5
    \end{cases}
}

\newcommand{\lt}{\ensuremath <}
\newcommand{\gt}{\ensuremath >}

\begin{document}

\maketitle

\section{Logique avec quantificateurs}

Quand on utilise des quantificateurs il y a des règles à suivre:

\textbf{Règle numéro 1:} Toute lettre dans un énoncé doit être introduite par un quantificateur.

\textbf{Règle numéro 2:} Cette introduction doit se faire avant la première occurence de la variable.

\textbf{Règle numéro 3:} On doit toujours préciser à quel ensemble appartient la variable.


\begin{methode}
    Quand on veut montrer un énoncé universel ($\forall x \in X, P(x)$)

    \vspace{0.5em}

    \item 1) \textbf{"Soit $x \in X$, montrons P(x)."}
    \item 2) \textbf{raisonnement profond.}
    \item 3) \textbf{On montre P(x).}
\end{methode}

\begin{example}
    $\forall x \in \R, \dfrac{x}{x^2 + 1} \geq \dfrac{-1}{2}$

    \vspace{1em}

    Soit $x \in \R$
    \begin{align*}
        \dfrac{x}{x^2 + 1} \geq -\dfrac{1}{2} &\implies 2x \geq -(x^2 + 1) \\
        &\implies (x^2 + 2x + 1) \geq 0\\
        &\implies (x + 1)^2 \geq 0
    \end{align*}
\end{example}

Pour donner un nom à une quantité/un objet mathématique, on écrit:

\textbf{Posons A := ...}, \textbf{Notons A le ...} ou \textbf{Soit A := ...}.

\begin{methode}
    Quand on veut montrer qu'il existe x appartenant à A vérifiant P(x),

    Soit on a en tête un exemple d'élément x dans A vérifiant P(x)
    \item \textbf{Posons x = ...}
    \item \textbf{Vérifions $x \in A$}
    \item \textbf{Vérifions P(x)}
    
    Soit on essaye d'utiliser des théorèmes d'existence pour montrer qu'un tel x existe.
\end{methode}

\begin{remark}
    Les mêmes quantificateurs peuvent être intervertis mais pas quand ils sont différent (un $\forall$ avec un $\exists$).
\end{remark}

\section{Méthodes de démonstration 2}

\subsection{Unicité d'un objet}

Nous croiserons régulièrement des énoncés du type: "Il y a au plus un élément $x \in X$ vérifiant P(x)".

\begin{methode}
    Pour montrer qu'un ensemble X contient au plus un élément vérifiant une propriété P, on peut procédé ainsi.

    \item 1) \textbf{Soient x et x' deux élément de X vérifiant P, montrons x = x'}
    \item 2) \textbf{Raisonnement profond.}
    \item 3) \textbf{On en conclut l'unicité d'un élément vérifiant P}.
\end{methode}

\begin{remark}
    L'unicité ne veut pas dire qu'on a montré l'existence.
\end{remark}


\begin{example}
    Soit $n \in \N$ Montrer qu'il existe au plus un multiple de 10 dans $X = \{n, n+1, ..., n+5\}$

    \vspace{1em}

    \begin{demonstration}
        Soient $k, k' \in [0, 5]$ tel que $10 \mid n+k$ et $10 \mid n+k' \implies \exists p \in \Z, n + k = 10p$ et $\exists p' \in \Z, n + k' = 10p'$
    
        Par soustraction $(n + k) - (n + k') = 10m - 10m' \implies (k - k') = 10(m - m')$

        Or $-5 \leq (k - k') \leq 5$ et le seul multiple de 10 dans cette intervalle est 0.

        \begin{rdem}
            Donc $k = k'$ et $n + k = n + k'$. %TODO: à revoir
        \end{rdem}
    \end{demonstration}
\end{example}

\subsection{Analyse synthèse}

\begin{methode}
    Pour déterminer l'ensemble des éléments d'un ensemble E vérifiant une propriété P, on peut raisonner par analyse/synthèse.

    \vspace{1em}

    \item \textbf{Analyse:} soit $x \in E$. on suppose que x vérifie P.
    \item ... on regarde les forme possible de x.
    \item \textbf{Synthèse:} Posons x = ... les différente formes possibles trouvées.
    \item Vérifions que x vérifie P (et appartient bien à E).
\end{methode}

\begin{example}
    Trouvons les couples de nombres réels non-nuls (x, y), solutions du système
    \begin{equation*}
        (S)
        \begin{cases}
            xy = 2\\
            \dfrac{y}{x} = 2
        \end{cases}
    \end{equation*}

    \begin{demonstration*}
        \textbf{Analyse:} Soit $(x, y) \in \R^2$

        $xy \times \dfrac{y}{x} = 2 \times 2 \implies y^2 = 4 \implies y = 2 \lor y = -2$

        La ligne 1 (de S) donne $x = \dfrac{2}{y}$

        Donc les seuls couples possibles pour (x, y) sont (1, 2) et (-1, -2)

        \vspace{0.5em}

        \textbf{Synthèse:} On vérifie les deux couples trouvés.

        $1 \time 2 = 2$ et $\dfrac{2}{1} = 2$
        puis $-1 \times (-2) = 2$ et $\dfrac{-2}{-1} = 2$

        \begin{rdem}
            Donc (1, 2) et (-1, -2) sont l'ensemble des couples qui sont solutions de S.
        \end{rdem}
    \end{demonstration*}
\end{example}

\subsection{Définition de $\N$ par récurence}

\begin{definition}
    $\N$ est l'ensemble construit par

    \vspace{1em}

    \item $\N$ contient un élément noté 0.
    \item Chaque élément $n \in \N$ admet un unique successeur noté succ(n) = n + 1.
    \item $\forall x \in \N, [succ(x) \neq 0]$.
    \item $\forall x, y \in \N, [succ(x) = succ(y) \implies x = y]$.
    \item $\forall A \subset \N, [(0 \in A \land (n \in A \implies succ(n) \in A)) \implies A = \N]$
    (important pour la récurence).
\end{definition}

\begin{remark}
    Avec cette notation par récurence on peut définir $\sum$ par
    \begin{equation*}
        \sum_{i=1}^{n}a_i =
        \begin{cases}
            0 &\text{ si } n = 0\\
            (\sum_{i=1}^{n-1}a_i) + a_n &\text{ si } n \geq 1
        \end{cases}
    \end{equation*}
\end{remark}

\begin{methode}
    Pour montrer une propriété $P_n$ est vrai pour tout entier $n \geq n_0$.

    \item Donner explicitement la propriété $P_n$.
    \item \textbf{Initialisation:} On montre $P_{n_0}$.
    \item \textbf{Hérédité:} Soit $n \in \N, n \geq n_0$, tel que $P_n$ est vraie.
    \item Montrons que $P_{n+1}$.
\end{methode}

\begin{remark}
    Il peut arrivé qu'on ne puisse pas déduire $P_{n+1}$ de $P_n$ mais seulement $P_{n+2}$
    à partir de $P_{n+1}$ et $P_n$, on fait alors une récurence double.
\end{remark}

\begin{methode}
    Si $P_{n_0}$ et $P_{n_0+1}$ sont vraies et si $\forall n \in \N, n \geq n_0, (P_n \land P_{n+1} \implies P_{n+2})$
    
    Alors $\forall n \in \N, n \geq n_0, P_n$ est vrai.
\end{methode}

Il existe aussi une récurence forte.

\begin{methode}
    Si $P_{n_0}$ est vraie et si $\forall n \in \N, n \geq n_0, (\forall k \in \N, n_0 \leq k \leq n, P_k \implies P_{n+1})$
    
    Alors $\forall n \in \N, n \geq n_0, P_n$ est vrai.
\end{methode}

\end{document}
