\documentclass[a4paper, 12pt]{article}
\usepackage{amsmath, amssymb, amsthm}
\usepackage{geometry}
\usepackage{tcolorbox}
\geometry{hmargin=2.5cm, vmargin=2.5cm}

\renewcommand*{\today}{04 octobre 2024}

\title{Ensemble Complex | CM: 5}
\author{Par Lorenzo}
\date{\today}

\newtheorem{theorem}{Théorème}[section]
\newtheorem{definition}{Définition}[section]
\newtheorem{example}{Example}[section]
\newtheorem{remark}{Remarques}[section]
\newtheorem{lemme}{Lemme}[section]
\newtheorem{corollaire}{Corollaire}[section]

\newtheorem{_proposition}{Proposition}[section]
\newenvironment{proposition}[1][]{
    \begin{_proposition}[#1]~\par
    \vspace*{0.5em}
}{%
    \end{_proposition}%
}

\newtheorem{_proprietes}{Propriétés}[section]
\newenvironment{proprietes}[1][]{
        \begin{_proprietes}[#1]~\par
        \vspace*{0.5em}
}{%
        \end{_proprietes}%
}

\newenvironment{rdem}[1][]{
    \begin{tcolorbox}[colframe=black, colback=white!10, sharp corners]
        #1
}{%
    \end{tcolorbox}
     
}

\newtheorem{_demonstration}{Démonstration}[section]
\newenvironment{demonstration}[1][]{
    \begin{_demonstration}[#1]~\par
    \vspace*{0.5em}
}{%
    \end{_demonstration}%
    \qed%
}

\newtheorem*{_demonstration*}{Démonstration}
\newenvironment{demonstration*}[1][]{
    \begin{_demonstration*}[#1]~\par
    \vspace*{0.5em}
}{%
    \end{_demonstration*}%
    \qed%
}

\newenvironment{ldefinition}{
    \begin{definition}~\par
    \vspace*{0.5em}
    \begin{enumerate}
}{
        \end{enumerate}
        \end{definition}
}

\newenvironment{lexample}{
    \begin{example}~\par
    \vspace*{0.5em}
    \begin{enumerate}
}{
        \end{enumerate}
        \end{example}
}

\newtheorem{_methode}{Méthode}[section]
\newenvironment{methode}{
    \begin{_methode}~\par
    \vspace*{0.5em}
}{
        \end{_methode}
}

\def\N{\mathbb{N}}
\def\Z{\mathbb{Z}}
\def\Q{\mathbb{Q}}
\def\R{\mathbb{R}}
\def\C{\mathbb{C}}
\def\K{\mathbb{K}}
\def\k{\Bbbk}

\def\un{(u_n)_{n \in \N}}
\def\xn#1{(#1_n)_{n \in \N}}

\def\o{\overline}
\def\eps{\varepsilon}

% \funcdef{name}{domain}{codomain}{variable}{expression}
% name: Name of the function (e.g. f)
% domain: Domain of the function (e.g. \mathbb{R})
% codomain: Codomain of the function (e.g. \mathbb{R})
% variable: Variables of the function (e.g. x)
% expression: Expression of the function (e.g. x^2)
\newcommand{\funcdef}[5]{%
    #1 :
    \begin{cases}
        #2 \rightarrow #3 \\
        #4 \mapsto #5
    \end{cases}
}

\newcommand{\lt}{\ensuremath <}
\newcommand{\gt}{\ensuremath >}

\begin{document}

\maketitle

\begin{definition}
    Soient X, Y deux ensembles.

    \item $\bullet$ Une application de X dans Y est la donnée, pour tout point
    $x \in X$ d'un unique point $y \in Y$ associé à x.
    On dit que y est l'image de x par l'application.

    \item $\bullet$ Soit f est une application de X dans Y.
    \item $\diamond$ on note f(x) l'image de x par f
    \item $\diamond$ X est appelé espace de départ de f.
    \item $\diamond$ Y est appelé espace d'arrivée de f.
    \item %todo
\end{definition}

\begin{remark}
    Deux applications f et g sont égales si elles ont même ensemble de départ, même ensemble d'arrivée et si

    $\forall x \in X, \; f(x) = g(x)$
\end{remark}

\begin{remark}
    On ne change pas une application en modifiant la (les) variable(s) muette(s) permettant de la définir. Ainsi les applications suivantes sont égales:

    $f : \R^2 \to \R (x, y) \to x + y,     g : \R^2 \to \R , (x_1, x_2) \to x_1 + x_2$
    %todo you know
\end{remark}

\begin{remark}
    Pour toute fonction f, si $x_1 = x_2$ alors $f(x_1) = f(x_2)$, la réciproque est fausse en général.
\end{remark}

\begin{remark}
    Etant donnée une application $f: X \to Y   x \to f(x)$ et $X' \subset X$, on peut créer une application restreinte $f_{\mid X'}: X' \to Y    x \to f(x)$
\end{remark}

\begin{definition}
    Soient X, Y des ensembles.

    \item $\bullet$ On appelle graphe dans $X \times Y$ toute partie G de $X \times Y$
    telle que $\forall x \in X, \exists! y \in Y, (x, y) \in G$

    \item $\bullet$ Si f est une application de X dans Y, alors le graphe de f est
    $G_f := \{(x, y) \in X \times Y \mid y = f(x)\}$
\end{definition}

\begin{remark}
    Réciproquement si G est un graph %todo
\end{remark}

\begin{remark}
    Une fonction n'étant pas nécessairement définie sur tout l'ensemble de départ considéré (souvent $\R$).
\end{remark}

\begin{definition}
    Soit $f \in Y^X$

    \item $\bullet$ On dit que $x \in X$ est un antécédent de $y \in Y$ par f lorsque $f(x) = y$, i.e. lorsque y est l'image de x par f.
    \item $\bullet$ Si A est un sous ensemble de X, on appelle image de A l'ensemble
    $f(A) := \{f(a) \mid a \in A\} = \{y \in Y \mid \exists a \in A, f(a) = y\} \subset Y$
    \item $\bullet$ Si B est un sous ensemble de Y, on appelle image réciproque de B l'ensemble des antécédents d'éléments de B:
    $f^{-1}(B) := \{x \in X \mid f(x) \in B\} \subset X$
\end{definition}

\begin{remark}
    Attention à la notation, f(a) et f(A) ne sont pas de même nature.
    Pour tout $x \in X, f(\{x\}) = \{f(x)\}$
\end{remark}

\begin{theorem}
    Soit $f \in Y^X$

    \item $\bullet$ Pour toutes parties A et B de Y on a,
    
    \item $\diamond$ $A \subset B \implies f^{-1}(A) \subset f^{-1}(B)$

    \item $\diamond$ $f^{-1}(A \cup B) = f^{-1}(A) \cup f^{-1}(B)$
    
    \item $\diamond$ $f^{-1}(A \cap B) = f^{-1}(A) \cap f^{-1}(B)$
    
    \item $\bullet$ Pour toutes parties A et B de X, on a
    
    \item $\diamond$ $A \subset B \implies f(A) \subset f(B)$
    \item $\diamond$ $f(A \cup B) = f(A) \cup f(B)$
    \item $\diamond$ $f(A \cap B) \subset f(A) \cap f(B)$
\end{theorem}

\begin{demonstration}
    %todo
\end{demonstration}


\end{document}
