\documentclass[a4paper, 12pt]{article}
\usepackage{amsmath, amssymb, amsthm}
\usepackage{geometry}
\usepackage{tcolorbox}
\geometry{hmargin=2.5cm, vmargin=2.5cm}

\renewcommand*{\today}{27 septembre 2024}

\title{Arithmetique | CM: 4}
\author{Par Lorenzo}
\date{\today}

\newtheorem{theorem}{Théorème}[section]
\newtheorem{definition}{Définition}[section]
\newtheorem{example}{Example}[section]
\newtheorem{remark}{Remarques}[section]
\newtheorem{lemme}{Lemme}[section]
\newtheorem{corollaire}{Corollaire}[section]

\newtheorem{_proposition}{Proposition}[section]
\newenvironment{proposition}[1][]{
    \begin{_proposition}[#1]~\par
    \vspace*{0.5em}
}{%
    \end{_proposition}%
}

\newtheorem{_proprietes}{Propriétés}[section]
\newenvironment{proprietes}[1][]{
        \begin{_proprietes}[#1]~\par
        \vspace*{0.5em}
}{%
        \end{_proprietes}%
}

\newenvironment{rdem}[1][]{
    \begin{tcolorbox}[colframe=black, colback=white!10, sharp corners]
        #1
}{%
    \end{tcolorbox}
     
}

\newtheorem{_demonstration}{Démonstration}[section]
\newenvironment{demonstration}[1][]{
    \begin{_demonstration}[#1]~\par
    \vspace*{0.5em}
}{%
    \end{_demonstration}%
    \qed%
}

\newtheorem*{_demonstration*}{Démonstration}
\newenvironment{demonstration*}[1][]{
    \begin{_demonstration*}[#1]~\par
    \vspace*{0.5em}
}{%
    \end{_demonstration*}%
    \qed%
}

\newenvironment{ldefinition}{
    \begin{definition}~\par
    \vspace*{0.5em}
    \begin{enumerate}
}{
        \end{enumerate}
        \end{definition}
}

\newenvironment{lexample}{
    \begin{example}~\par
    \vspace*{0.5em}
    \begin{enumerate}
}{
        \end{enumerate}
        \end{example}
}

\newtheorem{_methode}{Méthode}[section]
\newenvironment{methode}{
    \begin{_methode}~\par
    \vspace*{0.5em}
}{
        \end{_methode}
}

\def\N{\mathbb{N}}
\def\Z{\mathbb{Z}}
\def\Q{\mathbb{Q}}
\def\R{\mathbb{R}}
\def\C{\mathbb{C}}
\def\K{\mathbb{K}}
\def\k{\Bbbk}

\def\un{(u_n)_{n \in \N}}
\def\xn#1{(#1_n)_{n \in \N}}

\def\o{\overline}
\def\eps{\varepsilon}

% \funcdef{name}{domain}{codomain}{variable}{expression}
% name: Name of the function (e.g. f)
% domain: Domain of the function (e.g. \mathbb{R})
% codomain: Codomain of the function (e.g. \mathbb{R})
% variable: Variables of the function (e.g. x)
% expression: Expression of the function (e.g. x^2)
\newcommand{\funcdef}[5]{%
    #1 :
    \begin{cases}
        #2 \rightarrow #3 \\
        #4 \mapsto #5
    \end{cases}
}

\newcommand{\lt}{\ensuremath <}
\newcommand{\gt}{\ensuremath >}

\begin{document}

\maketitle

\subsection{Arithmétique élémentaire dans $\Z$}

\begin{definition}
    Soient x et y dans $\Z$. On dit que x divise y s'il existe $k \in \Z$ tel que $y = kx$.
    La notation associée est $x \mid y$.
    x est un diviseur de y ou y est un multiple de x
\end{definition}

\begin{remark}
    tout entier relatif divise 0.

    0 divise uniquement 0.

    si x est un diviseur de y alors (-x) est un diviseur de y

    1 et -1 sont les diviseurs de tout entier relatifs.
    
    les diviseurs de 1 et -1 sont 1 et -1

    $\forall x, y \in \N^* \implies (x \mid y \implies x \leq y)$
\end{remark}

\begin{definition}
    On dit que $p \in \N, \; p \geq 2$ est un nombre premier si les seuls diviseurs positifs de p sont 1 et p.
\end{definition}

\begin{remark}
    Une autre définition est tout nombre qui a exactement 2 diviseurs.
\end{remark}

\begin{remark}
    Pour vérifier qu'un nombre est premier, on peut regarde pour chaque $\forall k \in \N, k \leq \sqrt{p}$ si k divise p.
\end{remark}

%si motivation parler du crible d'eratosthène

\begin{definition}
    Soit $n \in \Z^*$, on appelle décomposition en facteurs premiers de n une écriture de la forme

    $n = c multi p_i = c(p_1 \times ... \times p_k)$

    où $c \in {+- 1}, k \in \N, p_1,...,p_k$ sont premiers
\end{definition}

\begin{proposition}
    Tout $n \in \Z^*$ admet une décomposition en facteurs premier.
\end{proposition}

\begin{demonstration}
    Il suffit de le démontrer pour $n \in \N^*$ et $c = 1$ et pour les négatifs on se ramène à $\N^*$ en posant $c = -1$

    Démonstration par récurrence forte.

    \textbf{Initialisation:} $n = 1$, on pose $c = 1, k = 0$, c'est un produit vide.

    \textbf{Initialisation:} Soit $n \in \N^*, \forall d \leq n$, on ait une telle décomposition.
    Si n + 1 est premier, on pose $k = 1 \quad P_1 = n + 1$.
    Si n + 1 n'est pas premier il admet un diviseur $d \in [2, n]$.
    Par hypothèse de récurrence $d = c \times p_1 \times ... \times p_k$.
    De même $d' = \dfrac{n+1}{d} \in [2; n]$
    $d' = p_1' \times ... \times p_k'$.

    Donc $n + 1 = d \times d' = p_1 \times ... \times p_k \times p_1' \times ... \times p_k'$
\end{demonstration}

Corollaire: Tout entier $n \geq 2$ admet au moins un diviseur premier
%todo

\begin{proposition}
    L'ensemble des nombre premiers est infini.
\end{proposition}

\begin{demonstration}
    Supposons (par l'absurde) qu'il y ait un nombre fini de nombres premiers $p_1, ..., p_m$

    On pose $N = p_1 \times ... \times p_m + 1$

    Alors N admet un diviseur premier $p_i (i \in [i; m])$
    i.e. $N = p_i N' \implies N = multi p_j + 1 \implies p_i N' - p_i multi_{i \neg j} p_j = 1 \implies p_i (N' - multi_{j \neg i} p_j) = 1$
\end{demonstration}

\begin{theorem}
    Soient $a \in \Z, b \in \N*$.
    
    Alors il existe un unique couple $(q, r) \in \Z \times \N, a = bq + r$ avec $b \gt r \geq 0$
\end{theorem}

\begin{demonstration}
    \textbf{Existence:} Pour $a \in \N$, raisonnement par récurrence.

    \textbf{Initialisation:} $a = 0$: On pose $q = 0 \text{ et } r = 0 \implies 0 = b \times 0 + 0$

    \textbf{Hérédité:} Si $a = bq + r$ avec $(b \gt r \geq 0)$

    Alors $a + 1 = bq + (r + 1)$, C'est une division euclidienne lorsque $r+1 \lt b \implies r \lt l - 1$

    Lorsque $r = b-1$

    $a + 1 = bq + ((b - 1) + 1) = bq + b = b(q + 1) + 0$,
    C'est une division euclidienne.

    Si $a \lt 0$ alors $(-a) \gt 0$ Donc $\exists (q, r) \in \Z \times \N, -a = bq + r \implies a = b \times (-q) + (-r)$ avec $(b \gt r \geq 0)$

    Si r = 0, c'est une division euclidienne.

    Sinon $-b \lt -r \lt 0 \implies 0 \lt -r + b \lt b$

    Donc $a = b \times (-q) + (-r + b) - b = b \times (-q - 1) + (-r + b)$
    C'est un division euclidienne.

    \textbf{Unicité:} Si $a = bq + r$ et $a = bq' + r'$ avec $b \gt r, r' \geq 0$

    Par soustraction: $0 = b(q - q') + r - r' \implies r' - r = b(q - q')$

    $b - 1 \geq r' - r \geq -b - 1$
    Donc $r' - r = 0 \implies r = r'$

    Ainsi $bq + r = bq' + r' \implies bq = bq' \implies q = q'$
\end{demonstration}

\begin{definition}
    le \textbf{pgcd} de deux nombres $a, b \in \Z^*$ est le plus grand diviseur commun à a et b. Il est noté $PGCD(a, b)$ (ou encore $a \land b$)

    On dit que a et b sont \textbf{premiers entre eux} si PGCD(a, b) = 1.

    Le \textbf{ppcm} de deux nombres $a, b \in \Z^*$ est le plus petit multiple strictement positif commun à a et b. Il est noté PPCM(a, b) (ou encore $a \lor b$)
\end{definition}

\begin{proposition}
    $\forall a, b \in \Z^*, PGCD(a, b) \times PPCM(a, b) = |ab|$
\end{proposition}

\begin{demonstration}
    Si on remplace a et b par leurs valeurs absolues: $||a||b|| = |ab|$

    Les multiples et les diviseurs de |a| et de a sont les mêmes.

    Donc $PGCD(a, b) = PGCD(|a|, |b|)$ et $PPCM(a, b) = PPCM(|a|, |b|)$

    Ainsi il suffit de montrer le résultat pour $a, b \in \N^*$

    On pose $d = PGCD(a, b)$

    $\exists a', b' \in \N^*, a = da' \text{ et } b = db'$

    $\dfrac{ab}{d} = \dfrac{da'b}{d} = a'b$
    $\dfrac{ab}{d} = \dfrac{adb'}{d} = ab'$
    %todo
\end{demonstration}

\begin{methode}
L'algorithme d'Euclide:

Le PGCD peut se calculer avec l'algorithme d'Euclide:

\item \textbf{1.} (Eventuellement) remplacer a et b par $|a|$ et $|b|$
\item \textbf{2.} De manière récursive:
\item \textbf{2.1} Calculer la division euclidienne de a par b: $a = bq + r$
\item \textbf{2.2} Si $r \neq 0$: recommencer en remplcaçant (a, b) par (b, r) Sinon sortir de la récursion
\item \textbf{3.} Le pgcd est le dernier reste non-nul calculé.
\end{methode}

%todo

\begin{proposition}
    Si d est un diviseur commun à a et b alors $d \mid PGCD(a, b)$
\end{proposition}

Corollaire:

Le PGCD est aussi le plus grand diviseur commun au sens de la divisibilité.

\end{document}
