\documentclass[a4paper, 12pt]{article}
\usepackage{amsmath, amssymb, amsthm}
\usepackage{geometry}
\usepackage{tcolorbox}
\geometry{hmargin=2.5cm, vmargin=2.5cm}

\renewcommand*{\today}{20 septembre 2024}

\title{Arithmetique | CM: 3}
\author{Par Lorenzo}
\date{\today}

\newtheorem{theorem}{Théorème}[section]
\newtheorem{definition}{Définition}[section]
\newtheorem{example}{Example}[section]
\newtheorem{remark}{Remarques}[section]
\newtheorem{lemme}{Lemme}[section]
\newtheorem{corollaire}{Corollaire}[section]

\newtheorem{_proposition}{Proposition}[section]
\newenvironment{proposition}[1][]{
    \begin{_proposition}[#1]~\par
    \vspace*{0.5em}
}{%
    \end{_proposition}%
}

\newtheorem{_proprietes}{Propriétés}[section]
\newenvironment{proprietes}[1][]{
        \begin{_proprietes}[#1]~\par
        \vspace*{0.5em}
}{%
        \end{_proprietes}%
}

\newenvironment{rdem}[1][]{
    \begin{tcolorbox}[colframe=black, colback=white!10, sharp corners]
        #1
}{%
    \end{tcolorbox}
     
}

\newtheorem{_demonstration}{Démonstration}[section]
\newenvironment{demonstration}[1][]{
    \begin{_demonstration}[#1]~\par
    \vspace*{0.5em}
}{%
    \end{_demonstration}%
    \qed%
}

\newtheorem*{_demonstration*}{Démonstration}
\newenvironment{demonstration*}[1][]{
    \begin{_demonstration*}[#1]~\par
    \vspace*{0.5em}
}{%
    \end{_demonstration*}%
    \qed%
}

\newenvironment{ldefinition}{
    \begin{definition}~\par
    \vspace*{0.5em}
    \begin{enumerate}
}{
        \end{enumerate}
        \end{definition}
}

\newenvironment{lexample}{
    \begin{example}~\par
    \vspace*{0.5em}
    \begin{enumerate}
}{
        \end{enumerate}
        \end{example}
}

\newtheorem{_methode}{Méthode}[section]
\newenvironment{methode}{
    \begin{_methode}~\par
    \vspace*{0.5em}
}{
        \end{_methode}
}

\def\N{\mathbb{N}}
\def\Z{\mathbb{Z}}
\def\Q{\mathbb{Q}}
\def\R{\mathbb{R}}
\def\C{\mathbb{C}}
\def\K{\mathbb{K}}
\def\k{\Bbbk}

\def\un{(u_n)_{n \in \N}}
\def\xn#1{(#1_n)_{n \in \N}}

\def\o{\overline}
\def\eps{\varepsilon}

% \funcdef{name}{domain}{codomain}{variable}{expression}
% name: Name of the function (e.g. f)
% domain: Domain of the function (e.g. \mathbb{R})
% codomain: Codomain of the function (e.g. \mathbb{R})
% variable: Variables of the function (e.g. x)
% expression: Expression of the function (e.g. x^2)
\newcommand{\funcdef}[5]{%
    #1 :
    \begin{cases}
        #2 \rightarrow #3 \\
        #4 \mapsto #5
    \end{cases}
}

\newcommand{\lt}{\ensuremath <}
\newcommand{\gt}{\ensuremath >}

\begin{document}

\maketitle

\begin{definition}
    Soient $(G, *)$ et $(H, \square)$ deux groupes.

    On appelle morphisme de groupes toute application $f: G \rightarrow H$ vérifiant

    $\forall x, y \in G, f(x * y) = f(x) \square f(y)$
\end{definition}

\begin{proposition}
    Si $f: G \rightarrow H$ est un morphisme de groupe, alors $f(e_G) = e_H$
\end{proposition}

\begin{demonstration}
    \begin{flalign*}
        f(e_G)& = f(e_G * e_G) = f(e_G) \square f(e_G)&& \\
        f(e_G)& = f(e_G) \square e_H&& \\
        f(e_G)& \square f(e_G) = f(e_G) \square e_H \implies f(e_G) = e_H&&
    \end{flalign*}
\end{demonstration}

\begin{proposition}
    Si $f: G \rightarrow H$ est un morphisme de groupe, alors
    $\forall x \in G, f(x^{-1}) = f(x)^{-1}$
\end{proposition}

\begin{demonstration}
    \begin{flalign*}
        f(x^{-1}) = f(x^{-1}) \square f(x) \square f(x)^{-1} = f(x^{-1} * x) \square f(x)^{-1} = f(x)^{-1}&&
    \end{flalign*}
\end{demonstration}

\section{Anneaux et Corps}

\begin{definition}
    Un anneau est $(A, +, \times)$ où A est un ensemble, + et x sont deux l.c.i sur A vérifiant les axiomes suivants

    \item $\bullet$ (A, +) est un groupe abélien (on note $0_A$ sont élément neutre)

    \item $\bullet$ $\times$ est associative

    \item $\bullet$ $\times$ est distributive sur +
\end{definition}

\begin{remark}
    On dit que $(A, +, \times)$ est un anneau commutatif si, de plus $\times$ est commutative.

    Un élément $x \in A$ est dit inversible dans A lorsqu'il adment un symétrique pour $\times$.
\end{remark}

\begin{proposition}
    Soit $(A, +, x)$ un anneau alors

    $\forall x \in A, 0_A \times x = 0_A$
\end{proposition}

\begin{demonstration}
    \begin{flalign*}
        0_A \times x &= (0_A + 0_A) \times x&&\\
        &= 0_A \times x + 0_A \times x \implies 0_A = 0_A \times x \text{ (par soustraction de $0_A \times x$)}&&
    \end{flalign*}
\end{demonstration}

\begin{proposition}
    Soient $x, y, z \in A$, Si $x \times z = y \times z$ et z est inversible alors x = y
\end{proposition}

\begin{demonstration}
    \begin{align*}
        x \times z = y \times z &\implies (x \times z) \times z^{-1} = (y \times z) \times z^{-1}\\
        &\implies x \times (z \times z^{-1}) = y \times (z \times z^{-1})\\
        &\implies x \times 1_A = y \times 1_A \\
        &\implies x = y
    \end{align*}
\end{demonstration}

\begin{definition}
    Un corps est la donnée d'un triplet
    $(\Bbbk, +, \times)$ où $\Bbbk$ est un ensemble, + et $\times$ sont deux l.c.i sur $\Bbbk$ vérifiant les axiomes suivants:

    \item $\bullet$ $(\Bbbk, +, \times)$ est un anneau commutatif
    \item $\bullet$ $(\Bbbk^*, \times)$ est un groupe abélien (de neutre noté $1_\K$).
\end{definition}

\begin{remark}
    De manière équivalente, un corps est un anneau commutatif avec un élément neutre pour $\times$ où tout élément non-nul est inversible.
\end{remark}

\section{Arithmétique des entiers}

\subsection{Rappels sur $\N$ et $\Z$}

\textcolor{red}{À vérifier, certains théorèmes manque de consistance}

\begin{theorem}
    (propriétés de + et $\times$ sur $\N$)
    \item \textbf{(a)} + et $\times$ sont associative et commutative sur $\N$

    \item \textbf{(b)} 0 est élement neutre pour + tandis que 1 est neutre pour $\times$

    \item \textbf{(c)} Il y a une distributivité de $\times$ sur +

    \item \textbf{(d)} $\forall x, y, m \in \N, \; x + m = y + m \implies x = y$
\end{theorem}

\begin{theorem}
    (propriétés de $\leq$ sur $\N$)

    \item \textbf{1)} (relation d'ordre total) $\forall m, n, p \in \N$
    
    \item \textbf{(a)} $n \leq n$
    \item \textbf{(b)} $m \leq n \land n \leq m \iff m = n$
    \item \textbf{(c)} $m \leq n \land n \leq p \implies m \leq p$
    \item \textbf{(d)} $m \leq n \lor n \leq m$
    
    \item \textbf{2)} Les opérations + et $\times$ sont compatibles avec la relation d'ordre

        $\forall n, m, p \in \N, \; n \leq m \implies (n + p \leq m + p) \land (n \times p \leq m \times p)$

    \item \textbf{3)} $\forall n \in \N, \; 0 \leq n$
    \item \textbf{4)} $\forall n, m \in \N, \forall p \in \N^*, \; n \leq m \implies n \times p \leq m \times p$
\end{theorem}

\begin{theorem}
    \item \textbf{1.} Toute partie finie de $\N$ admet un plus grand élément.
    \item \textbf{2.} Toute partie non vide de $\N$ admet un plus petit élément.
    \item \textbf{3.} Toute partie non vide et majorée de $\N$ admet un plus grand élément.
    \item \textbf{4.} $\N$ n'admet pas de plus grand élément.
\end{theorem}

\begin{theorem}
    (propriétés de + et $\times$ sur $\Z$)
    \item \textbf{(a)} + et $\times$ sont associative et commutative sur $\Z$

    \item \textbf{(b)} 0 est élement neutre pour + tandis que 1 est neutre pour $\times$

    \item \textbf{(c)} Il y a une distributivité de $\times$ sur +

    \item \textbf{(d)} Tout $m \in \Z$ admet un symétrique (élément inverse), $-m \in \Z$ pour +
\end{theorem}

\begin{theorem}
    (propriétés de $\leq$ sur $\Z$)

    \item \textbf{1)} $\leq$ est une relation d'ordre totale sur $\Z$.
    
    \item \textbf{2)} Soient $n, m, p \in \Z$
    
    \item \textbf{(a)} $n \leq m \iff n + p \leq m + p$
    \item \textbf{(b)} $\forall p \in \Z_+^*, n \leq m \iff np \leq mp$
    \item \textbf{(c)} $\forall p \in \Z_-^*, n \leq m \iff mp \leq np$
    \item \textbf{(d)} $\forall p \in \Z^*, m = n \iff mp = np$
\end{theorem}

\end{document}
