\RequirePackage{amsthm}
\RequirePackage{amsmath}
\RequirePackage{amssymb}
\RequirePackage{tcolorbox}

\tcbuselibrary{theorems,skins,breakable}


\newtheorem{_definition}{Définition}[section]
\newenvironment{definition}[1][]{
    \begin{tcolorbox}[
        sharp corners,
        colback=white,
        colframe=black,
        leftrule=1mm,
        toprule=-1mm, bottomrule=-1mm, rightrule=-1mm,
        top=0mm, left=1mm
    ]
    \begin{_definition}#1~\par
}{%
    \end{_definition}%
    \end{tcolorbox}
}

\newtcbtheorem[number within=section]{propriete}{Propriétés}%
{
    enhanced,
    colframe=black,
    coltitle=black, fonttitle=\bfseries,
    top=5mm,
    attach boxed title to top left={yshift=-3mm, xshift=6mm},
    boxed title style={colback=white, arc=2mm, shadow={1mm}{-1mm}{0mm}{black}}
}{propri}

\newtcbtheorem[number within=section]{proposition}{Proposition}%
{
    enhanced,
    colframe=black,
    coltitle=black, fonttitle=\bfseries,
    top=5mm,
    attach boxed title to top left={yshift=-3mm, xshift=6mm},
    boxed title style={colback=white, arc=2mm, shadow={1mm}{-1mm}{0mm}{black}}
}{propal}

\newtcbtheorem[number within=section]{lemme}{Lemme}%
{
    enhanced,
    colframe=black,
    coltitle=black, fonttitle=\bfseries,
    top=5mm,
    attach boxed title to top left={yshift=-3mm, xshift=6mm},
    boxed title style={colback=white, arc=2mm, shadow={1mm}{-1mm}{0mm}{black}}
}{lemme}

\newtcbtheorem[number within=section]{corollaire}{Corollaire}%
{
    enhanced,
    colframe=black,
    coltitle=black, fonttitle=\bfseries,
    top=5mm,
    attach boxed title to top left={yshift=-3mm, xshift=6mm},
    boxed title style={colback=white, arc=2mm, shadow={1mm}{-1mm}{0mm}{black}}
}{corollaire}

\newtcbtheorem[number within=section]{theoreme}{Théorème}%
{
    enhanced,
    colframe=black,
    coltitle=black, fonttitle=\bfseries,
    top=5mm,
    attach boxed title to top left={yshift=-3mm, xshift=6mm},
    boxed title style={colback=white, arc=2mm, shadow={1mm}{-1mm}{0mm}{black}}
}{theoreme}

\newtheorem{_demonstration}{Démonstration}[section]
\newenvironment{demonstration}[1][]{
    \begin{_demonstration}[#1]~\par
}{%
    \end{_demonstration}%
    \qed%
}

\newtheoremstyle{exemplecompact}  % nom du style
  {0pt}    % espace au-dessus (avant)
  {3pt}    % espace au-dessous (après)
  {}        % corps de texte (police normale)
  {}        % indentation
  {\bfseries} % style du titre
  {.}       % ponctuation après le titre
  {0.5em}   % espace après le titre
  {}        % en-tête du théorème

\theoremstyle{exemplecompact}
\newtheorem{_exemple}{Exemple}[section]
\newenvironment{exemple}
  {\begin{quote}\begin{_exemple}}
  {\end{_exemple}\end{quote}}

\newtheoremstyle{remarquecompact}  % nom du style
  {0pt}    % espace au-dessus (avant)
  {3pt}    % espace au-dessous (après)
  {}        % corps de texte (police normale)
  {}        % indentation
  {\bfseries} % style du titre
  {.}       % ponctuation après le titre
  {0.5em}   % espace après le titre
  {}        % en-tête du théorème

\theoremstyle{remarquecompact}
\newtheorem{_remarque}{Remarque}[section]
\newenvironment{remarque}
  {\begin{quote}\begin{_remarque}}
  {\end{_remarque}\end{quote}}

\newenvironment{methode}[1][]{
    \begin{tcolorbox}[
        leftrule=-1mm, toprule=-1mm, bottomrule=-1mm, rightrule=-1mm,
    ]
    \textbf{Point méthode.}\space
}{
    \end{tcolorbox}
}

\newenvironment{explication}[1][]{
    \begin{tcolorbox}[
        leftrule=-1mm, toprule=-1mm, bottomrule=-1mm, rightrule=-1mm,
    ]
    \textbf{Explication.}\space
}{
    \end{tcolorbox}
}